\documentclass[a4paper,10.5pt,twoside]{book} 
\usepackage{hyperref,fancyhdr,graphicx,amssymb,amsmath,times,makeidx,listings,color} 
\graphicspath{{img/}} 
\pagestyle{fancyplain} 
\lhead[\fancyplain{}{\textbf\thepage}]{\fancyplain{}{\rightmark}} 
\rhead[\fancyplain{}{\leftmark}]{\fancyplain{}{\textbf\thepage}} 
\cfoot{} 
\setlength{\textwidth}{6in} 
\setlength{\parindent}{0pc} 
\setlength{\oddsidemargin}{15.5pt} 
\setlength{\evensidemargin}{15.5pt} 
\setcounter{tocdepth}{1} 
\sloppy{} 
\definecolor{ca}{rgb}{0.8,0,0} 
\definecolor{cb}{rgb}{0,0.2,0.6} 
\definecolor{cc}{rgb}{0,0.5,0} 
\definecolor{cd}{rgb}{0.6,0.6,0.6} 
\def\Ccr{\color{ca}} 
\def\Ccb{\color{cb}} 
\def\Ccg{\color{cc}} 
\def\Ccc{\color{cd}} 
\def\Ccn{\color{black}} 
\def\comma{\discretionary{,}{}{,}} 
\newcommand{\Ca}[1]{\textcolor{ca}{#1}} 
\newcommand{\Cb}[1]{\textcolor{cb}{#1}} 
\newcommand{\Cc}[1]{\textcolor{cc}{#1}} 
\newcommand{\Cd}[1]{\textcolor{cd}{#1}} 
\title{\fbox{\parbox{\textwidth}{\begin{center}\vspace*{2cm}\includegraphics[width=12cm]{gmic_banner.jpg}\\\vspace*{1cm}{\Huge \textbf{The Handbook}\\{\small Version 2.1.1}\\\vspace*{1cm}}\end{center}}}} 
\author{\Large \bf David Tschumperl\'e} 
\renewcommand\indexname{Index of commands} 
\makeindex 
\lstset{columns=fullflexible,basicstyle=\normalfont} 
\begin{document} 
\maketitle 
\tableofcontents 
\chapter*{Preamble} 
\section*{License} 
This document is distributed under the \textbf{GNU Free Documentation License}, version 1.3.\\ 
Read the full license terms at \url{http://www.gnu.org/licenses/fdl-1.3.txt}.\\~\\ 
An online version of this documentation is available at:\\\url{http://gmic.eu/reference.shtml}. 
\section*{Motivations} 
\Cc{G'MIC} is a full-featured open-source framework for image processing, providing several different user interfaces to 
convert/manipulate/filter/visualize generic image datasets, from 1d scalar signales to 3d+t sequences of multi-spectral volumetric images. 
Technically speaking, what it does is: 
\begin{itemize} 
\item Define a lightweight but powerful script language (the \Cc{G'MIC} language) dedicated to the design of image processing pipelines. 
\item Provide several user interfaces embedding the corresponding interpreter: 
\begin{itemize} 
\item A command-line executable '\texttt{gmic}', to use the \Cc{G'MIC} framework from a shell. 
In this setting, \Cc{G'MIC} may be seen as a direct (and friendly) competitor of the ImageMagick or GraphicsMagick software suites. 
\item A plug-in '\texttt{gmic\_gimp}', to bring \Cc{G'MIC} capabilities to the GIMP image retouching software. 
\item A web-service '\texttt{\Cc{G'MIC} Online}', to allow users applying image processing algorithms directly in a web brower. 
\item A Qt-based interface '\texttt{ZArt}', for real-time manipulation of webcam images. 
\item A C++ library '\texttt{libgmic}', to be linked with third-party applications. 
\end{itemize} 
\end{itemize} 
\Cc{G'MIC} is focused on the design of possibly complex pipelines for converting, manipulating, filtering and visualizing generic 1d/2d/3d multi-spectral image datasets. This includes of course color images, but also more complex data as image sequences or 3d(+t) volumetric float-valued datasets.\\ 
 
\Cc{G'MIC} is an open framework: the default language can be extended with custom \Cc{G'MIC}-written commands, defining thus new available image filters or effects. By the way, \Cc{G'MIC} already contains a substantial set of pre-defined image processing algorithms and pipelines (more than 1000).\\ 
 
\Cc{G'MIC} has been designed with portability in mind and runs on different platforms (Windows, Unix, MacOSX). It is distributed under the CeCILL license (GPL-compatible). Since 2008, it is developed in the Image Team of the GREYC laboratory, in Caen/France, by permanent researchers working in the field of image processing on a daily basis. 
\section*{Version} 
 
 \Ca{\textbf{gmic:} GREYC's Magic for Image Computing.}\\ 
 
        \Cb{Version \textbf{2.1.1}, Copyright (c) 2008-2017, David Tschumperl\'e}\\ 
        \Cb{(\url{http://gmic.eu})} 
\chapter{Usage} 
\small
\begin{lstlisting}[escapechar=§]
    §\aftergroup\Ccb§gmic [command1 [arg1_1,arg1_2,..]] .. [commandN [argN_1,argN_2,..]]§\aftergroup\Ccn§ 
 
    '§\aftergroup\Ccg§gmic§\aftergroup\Ccn§' is the open-source interpreter of the §\aftergroup\Ccg§G'MIC§\aftergroup\Ccn§ language, a script-based programming 
    language dedicated to the design of possibly complex image processing pipelines and operators. 
    It can be used to convert, manipulate, filter and visualize image datasets made of one 
    or several 1d/2d or 3d multi-spectral images. 
 
    This reference documentation describes all the technical rules governing the §\aftergroup\Ccg§G'MIC§\aftergroup\Ccn§ language. 
    As a starting point, you may want to visit our detailed tutorial pages, at: 
     §\aftergroup\Ccr§http://gmic.eu/tutorial/§\aftergroup\Ccn§
\end{lstlisting}
\normalsize
~\\\section{Overall context}
\small
\begin{lstlisting}[escapechar=§]
  - At any time, §\aftergroup\Ccg§G'MIC§\aftergroup\Ccn§ manages one list of numbered (and optionally named) pixel-based images, 
     entirely stored in computer memory (uncompressed). 
 
  - The first image of the list has indice '§\aftergroup\Ccg§0§\aftergroup\Ccn§' and is denoted by '§\aftergroup\Ccb§[0]§\aftergroup\Ccn§'. The second image of the 
     list is denoted by '§\aftergroup\Ccb§[1]§\aftergroup\Ccn§', the third by '§\aftergroup\Ccb§[2]§\aftergroup\Ccn§' and so on. 
 
  - Negative indices are treated in a periodic way: '§\aftergroup\Ccb§[-1]§\aftergroup\Ccn§' refers to the last image of the list, 
     '§\aftergroup\Ccb§[-2]§\aftergroup\Ccn§' to the penultimate one, etc. Thus, if the list has 4 images, '§\aftergroup\Ccb§[1]§\aftergroup\Ccn§' and '§\aftergroup\Ccb§[-3]§\aftergroup\Ccn§' both 
     designate the second image of the list. 
 
  - A named image may be also indicated by '§\aftergroup\Ccb§[name]§\aftergroup\Ccn§', if '§\aftergroup\Ccg§name§\aftergroup\Ccn§' uses the character set §\aftergroup\Ccg§[a-zA-Z0-9_]§\aftergroup\Ccn§ 
     and does not start with a number. Image names can be set or reassigned at any moment during 
     the processing pipeline (see command '§\aftergroup\Ccb§-name§\aftergroup\Ccn§' for this purpose). 
 
  - §\aftergroup\Ccg§G'MIC§\aftergroup\Ccn§ defines a set of various commands and substitution mechanisms to allow the design of 
     complex pipelines and operators managing this list of images, in a very flexible way: 
     You can insert or remove images in the list, rearrange image order, process images 
     (individually or grouped), merge image data together, display and output image files, etc. 
 
  - Such a pipeline can be then added as a new custom §\aftergroup\Ccg§G'MIC§\aftergroup\Ccn§ command (stored in a user 
     command file), so it can be re-used afterwards in another pipeline if necessary.
\end{lstlisting}
\normalsize
~\\\section{Image definition and terminology}
\small
\begin{lstlisting}[escapechar=§]
  - In §\aftergroup\Ccg§G'MIC§\aftergroup\Ccn§, each image is modeled as a 1d, 2d, 3d or 4d array of scalar values, uniformly 
     discretized on a rectangular/parallelepipedic domain. 
 
  - The four dimensions of this array are respectively denoted by: 
 
    . '§\aftergroup\Ccg§width§\aftergroup\Ccn§', the number of image columns (size along the §\aftergroup\Ccg§'x'-axis§\aftergroup\Ccn§). 
    . '§\aftergroup\Ccg§height§\aftergroup\Ccn§', the number of image rows (size along the §\aftergroup\Ccg§'y'-axis§\aftergroup\Ccn§). 
    . '§\aftergroup\Ccg§depth§\aftergroup\Ccn§', the number of image slices (size along the §\aftergroup\Ccg§'z'-axis§\aftergroup\Ccn§). 
        (the depth is equal to §\aftergroup\Ccg§1§\aftergroup\Ccn§ for usual color or grayscale 2d images). 
    . '§\aftergroup\Ccg§spectrum§\aftergroup\Ccn§', the number of image channels (size along the §\aftergroup\Ccg§'c'-axis§\aftergroup\Ccn§). 
        (the spectrum is respectively equal to §\aftergroup\Ccg§3§\aftergroup\Ccn§ and §\aftergroup\Ccg§4§\aftergroup\Ccn§ for usual §\aftergroup\Ccg§RGB§\aftergroup\Ccn§ and §\aftergroup\Ccg§RGBA§\aftergroup\Ccn§ color images). 
 
  - There are no hard limitations on the size of the image along each dimension. For instance, 
     the number of image slices or channels can be of arbitrary size within the limits of 
     the available memory. 
 
  - The §\aftergroup\Ccg§width, height§\aftergroup\Ccn§ and §\aftergroup\Ccg§depth§\aftergroup\Ccn§ of an image are considered as spatial dimensions, while the 
     §\aftergroup\Ccg§spectrum§\aftergroup\Ccn§ has a multi-spectral meaning. Thus, a 4d image in §\aftergroup\Ccg§G'MIC§\aftergroup\Ccn§ should be most often 
     regarded as a 3d dataset of multi-spectral voxels. Most of the §\aftergroup\Ccg§G'MIC§\aftergroup\Ccn§ commands will stick with 
     this idea (e.g. command '§\aftergroup\Ccn§' blurs images only along the spatial '§\aftergroup\Ccg§xyz§\aftergroup\Ccn§'-axes). 
 
  - §\aftergroup\Ccg§G'MIC§\aftergroup\Ccn§ stores all the image data as buffers of '§\aftergroup\Ccg§float§\aftergroup\Ccn§' values (32 bits, value range 
     §\aftergroup\Ccg§[-3.4E38,+3.4E38]§\aftergroup\Ccn§). It performs all its image processing operations with floating point 
     numbers. Each image pixel takes then 32bits/channel (except if double-precision buffers have 
     been enabled during the compilation of the software, in which case 64bits/channel can be the 
     default). 
 
  - Considering '§\aftergroup\Ccg§float§\aftergroup\Ccn§'-valued pixels ensure to keep the numerical precision when executing 
     image processing pipelines. For image input/output operations, you may want to prescribe the 
     image datatype to be different than '§\aftergroup\Ccg§float§\aftergroup\Ccn§' (like '§\aftergroup\Ccg§bool§\aftergroup\Ccn§', '§\aftergroup\Ccg§char§\aftergroup\Ccn§', '§\aftergroup\Ccg§int§\aftergroup\Ccn§', etc...). 
     This is possible by specifying it as a file option when using I/O commands. 
     (see section '§\aftergroup\Ccb§Input/output properties§\aftergroup\Ccn§' to learn more about file options).
\end{lstlisting}
\normalsize
~\\\section{Items of a processing pipeline}
\small
\begin{lstlisting}[escapechar=§]
  - In §\aftergroup\Ccg§G'MIC§\aftergroup\Ccn§, an image processing pipeline is described as a sequence of items separated by the 
     space character ' '. Such items are interpreted and executed from the left to the right. 
     For instance, the expression: 
 
       §\aftergroup\Ccb§filename.jpg blur 3,0 sharpen 10 resize 200%,200% output file_out.jpg§\aftergroup\Ccn§ 
 
     defines a valid pipeline composed of nine §\aftergroup\Ccg§G'MIC§\aftergroup\Ccn§ items. 
 
  - Each §\aftergroup\Ccg§G'MIC§\aftergroup\Ccn§ item is a string that is either a §\aftergroup\Ccg§command§\aftergroup\Ccn§, a list of command §\aftergroup\Ccg§arguments§\aftergroup\Ccn§, 
     a §\aftergroup\Ccg§filename§\aftergroup\Ccn§, or a special §\aftergroup\Ccg§input string§\aftergroup\Ccn§. 
 
  - Escape characters '§\aftergroup\Ccg§\§\aftergroup\Ccn§' and double quotes '§\aftergroup\Ccg§"§\aftergroup\Ccn§' can be used to define items containing spaces or 
     other special characters. For instance, the two strings '§\aftergroup\Ccb§single\ item§\aftergroup\Ccn§' and '§\aftergroup\Ccb§"single item"§\aftergroup\Ccn§' 
     both define the same single item, with a space in it.
\end{lstlisting}
\normalsize
~\\\section{Input data items}
\small
\begin{lstlisting}[escapechar=§]
  - If a specified §\aftergroup\Ccg§G'MIC§\aftergroup\Ccn§ item appears to be an existing filename, the corresponding image data 
     are loaded and inserted at the end of the image list (which is equivalent to the use of 
     ' filename§\aftergroup\Ccn§'). 
 
  - Special filenames '§\aftergroup\Ccb§-§\aftergroup\Ccn§' and '§\aftergroup\Ccb§-.ext§\aftergroup\Ccn§' stand for the standard input/output streams, optionally 
     forced to be in a specific '§\aftergroup\Ccg§ext§\aftergroup\Ccn§' file format (e.g. '§\aftergroup\Ccb§-.jpg§\aftergroup\Ccn§' or '§\aftergroup\Ccb§-.png§\aftergroup\Ccn§'). 
 
  - The following special input strings may be used as §\aftergroup\Ccg§G'MIC§\aftergroup\Ccn§ items to create and insert new 
     images with prescribed values, at the end of the image list: 
 
    . '§\aftergroup\Ccb§[selection]§\aftergroup\Ccn§' or '§\aftergroup\Ccb§[selection]xN§\aftergroup\Ccn§': Insert 1 or N copies of already existing images. 
       '§\aftergroup\Ccg§selection§\aftergroup\Ccn§' may represent one or several images 
       (see section '§\aftergroup\Ccb§Command items and selections§\aftergroup\Ccn§' to learn more about selections). 
 
    . '§\aftergroup\Ccb§width[%],_height[%],_depth[%],_spectrum[%],_values§\aftergroup\Ccn§': Insert a new image with specified 
       size and values (adding '§\aftergroup\Ccg§%§\aftergroup\Ccn§' to a dimension means 'percentage of the size along the same 
       axis, taken from the last image '§\aftergroup\Ccg§[-1]§\aftergroup\Ccn§''). Any specified dimension can be also written as 
       '§\aftergroup\Ccb§[image]§\aftergroup\Ccn§', and is then set to the size (along the same axis) of the existing specified image 
       §\aftergroup\Ccg§[image]§\aftergroup\Ccn§. '§\aftergroup\Ccg§values§\aftergroup\Ccn§' can be either a sequence of numbers separated by commas '§\aftergroup\Ccg§,§\aftergroup\Ccn§', or a 
       mathematical expression, as e.g. in input item '§\aftergroup\Ccb§256,256,1,3,if(c==0,x,if(c==1,y,0))§\aftergroup\Ccn§' which 
       creates a 256x256 RGB color image with a spatial shading on the red and green channels. 
       (see section '§\aftergroup\Ccb§Mathematical expressions§\aftergroup\Ccn§' to learn more about mathematical expressions). 
 
    . '§\aftergroup\Ccb§(v1,v2,..)§\aftergroup\Ccn§': Insert a new image from specified prescribed values. Value separator inside 
       parentheses can be '§\aftergroup\Ccg§,§\aftergroup\Ccn§' (column separator), '§\aftergroup\Ccg§;§\aftergroup\Ccn§' (row separator), '§\aftergroup\Ccg§/§\aftergroup\Ccn§' (slice separator) or 
       '§\aftergroup\Ccg§^§\aftergroup\Ccn§' (channel separator). For instance, expression '§\aftergroup\Ccb§(1,2,3;4,5,6;7,8,9)§\aftergroup\Ccn§' creates a 3x3 matrix 
       (scalar image), with values running from 1 to 9. 
 
    . '§\aftergroup\Ccb§0§\aftergroup\Ccn§': Insert a new '§\aftergroup\Ccg§empty§\aftergroup\Ccn§' image, containing no pixel data. Empty images are used only in rare 
       occasions. 
 
  - Input item '§\aftergroup\Ccb§name=value§\aftergroup\Ccn§' declares a new local or global variable '§\aftergroup\Ccg§name§\aftergroup\Ccn§', or assign a new value 
     to an existing variable. Variable names must use the character set §\aftergroup\Ccg§[a-zA-Z0-9_]§\aftergroup\Ccn§ and cannot 
     start with a number. 
 
  - A variable definition is always local to the current command except when it starts by the 
     underscore character '§\aftergroup\Ccg§_§\aftergroup\Ccn§'. In that case, it becomes also accessible by any command invoked 
     outside the current command scope (global variable). 
 
  - If a variable name starts with two underscores '§\aftergroup\Ccg§__§\aftergroup\Ccn§', the global variable is also shared among 
     different threads and can be read/set by commands running in parallel (see command '§\aftergroup\Ccb§parallel§\aftergroup\Ccn§' 
     for this purpose). Otherwise, it remains local to the thread that defined it. 
 
  - Numerical variables can be updated with the use of these special operators: 
     '§\aftergroup\Ccg§+=§\aftergroup\Ccn§' (addition), '§\aftergroup\Ccg§-=§\aftergroup\Ccn§' (subtraction), '§\aftergroup\Ccg§*=§\aftergroup\Ccn§' (multiplication), '§\aftergroup\Ccg§/=§\aftergroup\Ccn§' (division), '§\aftergroup\Ccg§%=§\aftergroup\Ccn§' (modulo), 
     '§\aftergroup\Ccg§&=§\aftergroup\Ccn§' (bitwise and), '§\aftergroup\Ccg§|=§\aftergroup\Ccn§' (bitwise or), '§\aftergroup\Ccg§^=§\aftergroup\Ccn§' (power), '§\aftergroup\Ccg§<<=§\aftergroup\Ccn§' and '§\aftergroup\Ccg§>>=§\aftergroup\Ccn§' (bitwise left and right 
     shifts). For instance, '§\aftergroup\Ccb§foo=1 foo+=3§\aftergroup\Ccn§'. 
 
  - Input item '§\aftergroup\Ccb§name.=string§\aftergroup\Ccn§' concatenates specified '§\aftergroup\Ccg§string§\aftergroup\Ccn§' to the end of variable '§\aftergroup\Ccg§name§\aftergroup\Ccn§'. 
 
  - Multiple variable assignments and updates are allowed, with expressions: 
     '§\aftergroup\Ccb§name1,name2,...,nameN=value§\aftergroup\Ccn§' or '§\aftergroup\Ccb§name1,name2,...,nameN=value1,value2,...,valueN§\aftergroup\Ccn§' 
     where assignment operator '§\aftergroup\Ccg§=§\aftergroup\Ccn§' can be replaced by one of the allowed operators 
     (e.g. '§\aftergroup\Ccg§+=§\aftergroup\Ccn§'). 
\end{lstlisting}
\normalsize
~\\\section{Command items and selections}
\small
\begin{lstlisting}[escapechar=§]
  - A §\aftergroup\Ccg§G'MIC§\aftergroup\Ccn§ item that is not a filename nor a special input string designates a §\aftergroup\Ccg§command§\aftergroup\Ccn§, 
     most of the time. Generally, commands perform image processing operations on one or several 
     available images of the list. 
 
  - Reccurent commands have two equivalent names (§\aftergroup\Ccg§regular§\aftergroup\Ccn§ and §\aftergroup\Ccg§short§\aftergroup\Ccn§). For instance, command names 
     '§\aftergroup\Ccb§resize§\aftergroup\Ccn§' and '§\aftergroup\Ccb§r§\aftergroup\Ccn§' refer to the same image resizing action. 
 
  - A §\aftergroup\Ccg§G'MIC§\aftergroup\Ccn§ command may have mandatory or optional §\aftergroup\Ccg§arguments§\aftergroup\Ccn§. Command arguments must be specified 
     in the next item on the command line. Commas '§\aftergroup\Ccg§,§\aftergroup\Ccn§' are used to separate multiple arguments of a 
     single command, when required. 
 
  - The execution of a §\aftergroup\Ccg§G'MIC§\aftergroup\Ccn§ command may be restricted only to a §\aftergroup\Ccg§subset§\aftergroup\Ccn§ of the image list, by 
     appending '§\aftergroup\Ccb§[selection]§\aftergroup\Ccn§' to the command name. Examples of valid syntaxes for '§\aftergroup\Ccg§selection§\aftergroup\Ccn§' are: 
 
    . '§\aftergroup\Ccb§command[-2]§\aftergroup\Ccn§': Apply command only on the penultimate image §\aftergroup\Ccg§[-2]§\aftergroup\Ccn§ of the list. 
    . '§\aftergroup\Ccb§command[0,1,3]§\aftergroup\Ccn§': Apply command only on images §\aftergroup\Ccg§[0],[1]§\aftergroup\Ccn§ and §\aftergroup\Ccg§[3]§\aftergroup\Ccn§. 
    . '§\aftergroup\Ccb§command[3-6]§\aftergroup\Ccn§': Apply command only on images §\aftergroup\Ccg§[3]§\aftergroup\Ccn§ to §\aftergroup\Ccg§[6]§\aftergroup\Ccn§ (i.e, §\aftergroup\Ccg§[3],[4],[5]§\aftergroup\Ccn§ and §\aftergroup\Ccg§[6]§\aftergroup\Ccn§). 
    . '§\aftergroup\Ccb§command[50%-100%]§\aftergroup\Ccn§': Apply command only on the second half of the image list. 
    . '§\aftergroup\Ccb§command[0,-4--1]§\aftergroup\Ccn§': Apply command only on the first image and the last four images. 
    . '§\aftergroup\Ccb§command[0-9:3]§\aftergroup\Ccn§': Apply command only on images §\aftergroup\Ccg§[0]§\aftergroup\Ccn§ to §\aftergroup\Ccg§[9]§\aftergroup\Ccn§, with a step of 3 
                          (i.e. on images §\aftergroup\Ccg§[0], [3], [6]§\aftergroup\Ccn§ and §\aftergroup\Ccg§[9]§\aftergroup\Ccn§). 
    . '§\aftergroup\Ccb§command[0--1:2]§\aftergroup\Ccn§': Apply command only on images of the list with even indices. 
    . '§\aftergroup\Ccb§command[0,2-4,50%--1]§\aftergroup\Ccn§': Apply command on images §\aftergroup\Ccg§[0],[2],[3],[4]§\aftergroup\Ccn§ and on the second half of 
                                 the image list. 
    . '§\aftergroup\Ccb§command[^0,1]§\aftergroup\Ccn§': Apply command on all images except the two first. 
    . '§\aftergroup\Ccb§command[name1,name2]§\aftergroup\Ccn§': Apply command on named images '§\aftergroup\Ccg§name1§\aftergroup\Ccn§' and '§\aftergroup\Ccg§name2§\aftergroup\Ccn§'. 
 
  - Indices in selections are always sorted in increasing order, and duplicate indices are 
     discarded. For instance, selections '§\aftergroup\Ccb§[3-1,1-3]§\aftergroup\Ccn§' and '§\aftergroup\Ccb§[1,1,1,3,2]§\aftergroup\Ccn§' are both equivalent to 
     '§\aftergroup\Ccb§[1-3]§\aftergroup\Ccn§'. If you want to repeat a single command multiple times on an image, use a 
     '§\aftergroup\Ccb§repeat..done§\aftergroup\Ccn§' loop instead. Inverting the order of images for a command is achieved by 
     explicitly inverting the order of the images in the list, with command '§\aftergroup\Ccb§reverse[selection]§\aftergroup\Ccn§'. 
 
  - Command selections '§\aftergroup\Ccb§[-1]§\aftergroup\Ccn§','§\aftergroup\Ccb§[-2]§\aftergroup\Ccn§' and '§\aftergroup\Ccb§[-3]§\aftergroup\Ccn§' are so often used that they have their own 
     shortcuts, respectively '§\aftergroup\Ccb§.§\aftergroup\Ccn§', '§\aftergroup\Ccb§..§\aftergroup\Ccn§' and '§\aftergroup\Ccb§...§\aftergroup\Ccn§'. For instance, command '§\aftergroup\Ccb§blur..§\aftergroup\Ccn§' is equivalent to 
     '§\aftergroup\Ccb§blur[-2]§\aftergroup\Ccn§'. These shortcuts work also when specifying command arguments. 
 
  - §\aftergroup\Ccg§G'MIC§\aftergroup\Ccn§ commands invoked without '§\aftergroup\Ccb§[selection]§\aftergroup\Ccn§' are applied on all images of the list, i.e. the 
     default selection is '§\aftergroup\Ccb§[0--1]§\aftergroup\Ccn§' (except for command '§\aftergroup\Ccb§input§\aftergroup\Ccn§' whose default selection is '§\aftergroup\Ccb§[-1]§\aftergroup\Ccn§'). 
 
  - Prepending a single hyphen '§\aftergroup\Ccg§-§\aftergroup\Ccn§' to a §\aftergroup\Ccg§G'MIC§\aftergroup\Ccn§ command has no effect. This may be useful to recognize 
     command items more easily in a one-liner pipeline (typically invoked from a shell). 
 
  - A §\aftergroup\Ccg§G'MIC§\aftergroup\Ccn§ command starting with a double hyphen '§\aftergroup\Ccg§--§\aftergroup\Ccn§' does not act 'in-place' but inserts its 
     result as one or several new images at the end of the image list. 
 
  - There are two different types of commands that can be run by the §\aftergroup\Ccg§G'MIC§\aftergroup\Ccn§ interpreter: 
 
    . §\aftergroup\Ccg§Native commands§\aftergroup\Ccn§, are the hard-coded functionalities in the interpreter core. They are thus 
       compiled as binary code and run fast, most of the time. Omitting an argument when invoking a 
       native command is not permitted, except if all following arguments are also omitted. 
       For instance, invoking '§\aftergroup\Ccb§plasma 10,,5§\aftergroup\Ccn§' is invalid but '§\aftergroup\Ccb§plasma 10§\aftergroup\Ccn§' is correct. 
    . §\aftergroup\Ccg§Custom commands§\aftergroup\Ccn§, are defined as §\aftergroup\Ccg§G'MIC§\aftergroup\Ccn§ pipelines of native or other custom commands. 
       They are interpreted by the §\aftergroup\Ccg§G'MIC§\aftergroup\Ccn§ interpreter, and thus run a bit slower than native commands. 
       Omitting arguments when invoking a custom command is permitted. For instance, expressions 
       '§\aftergroup\Ccb§flower ,,,100,,2§\aftergroup\Ccn§' or '§\aftergroup\Ccb§flower ,§\aftergroup\Ccn§' are correct. 
 
  - Most of the existing commands in §\aftergroup\Ccg§G'MIC§\aftergroup\Ccn§ are actually defined as §\aftergroup\Ccg§custom commands§\aftergroup\Ccn§. 
 
  - A user can easily add its own custom commands to the §\aftergroup\Ccg§G'MIC§\aftergroup\Ccn§ interpreter (see section 
     §\aftergroup\Ccb§'Adding custom commands§\aftergroup\Ccn§' for more details). New native commands cannot be added 
     (unless you modify the §\aftergroup\Ccg§G'MIC§\aftergroup\Ccn§ interpreter source code and recompile it).
\end{lstlisting}
\normalsize
~\\\section{Input/output properties}
\small
\begin{lstlisting}[escapechar=§]
  - §\aftergroup\Ccg§G'MIC§\aftergroup\Ccn§ is able to read/write most of the classical image file formats, including: 
 
    . 2d grayscale/color files: §\aftergroup\Ccb§.png, .jpeg, .gif, .pnm, .tif, .bmp, ...§\aftergroup\Ccn§ 
    . 3d volumetric files: §\aftergroup\Ccb§.dcm, .hdr, .nii, .pan, .inr, .pnk, ...§\aftergroup\Ccn§ 
    . video files: §\aftergroup\Ccb§.mpeg, .avi, .mov, .ogg, .flv, ...§\aftergroup\Ccn§ 
    . Generic ascii or binary data files: §\aftergroup\Ccb§.gmz, .cimg, .cimgz, .dlm, .asc, .pfm, .raw, .txt, .h.§\aftergroup\Ccn§ 
    . 3d object files: §\aftergroup\Ccb§.off.§\aftergroup\Ccn§ 
 
  - When dealing with color images, §\aftergroup\Ccg§G'MIC§\aftergroup\Ccn§ generally reads, writes and displays data using the usual 
     sRGB color space. 
 
  - §\aftergroup\Ccg§G'MIC§\aftergroup\Ccn§ is able to manage §\aftergroup\Ccg§3d objects§\aftergroup\Ccn§ that may be read from files or generated by §\aftergroup\Ccg§G'MIC§\aftergroup\Ccn§ commands. 
     A 3d object is stored as a one-column scalar image containing the object data, in the 
     following order: §\aftergroup\Ccg§{ magic_number; sizes; vertices; primitives; colors; opacities }§\aftergroup\Ccn§. 
     These 3d representations can be then processed as regular images. 
     (see command '§\aftergroup\Ccb§split3d§\aftergroup\Ccn§' for accessing each of these 3d object data separately). 
 
  - Be aware that usual file formats may be sometimes not adapted to store all the available image 
     data, since §\aftergroup\Ccg§G'MIC§\aftergroup\Ccn§ uses float-valued image buffers. For instance, saving an image that was 
     initially loaded as a 16bits/channel image, as a §\aftergroup\Ccb§.jpg§\aftergroup\Ccn§ file will result in a loss of 
     information. Use the §\aftergroup\Ccg§G'MIC§\aftergroup\Ccn§-specific file extensions §\aftergroup\Ccb§.cimgz§\aftergroup\Ccn§ or §\aftergroup\Ccb§.gmz§\aftergroup\Ccn§ to ensure that all data 
     precision are preserved when saving images. 
 
  - Sometimes, file options may/must be set for file formats: 
 
    . §\aftergroup\Ccg§Video files:§\aftergroup\Ccn§ Only sub-frames of an image sequence may be loaded, using the input expression 
       '§\aftergroup\Ccb§filename.ext,[first_frame[,last_frame[,step]]]§\aftergroup\Ccn§'. Set '§\aftergroup\Ccg§last_frame==-1§\aftergroup\Ccn§' to tell it must be 
       the last frame of the video. Set '§\aftergroup\Ccg§step§\aftergroup\Ccn§' to 0 to force an opened video file to be 
       opened/closed. Output framerate and codec can be also set by using the output expression 
       '§\aftergroup\Ccb§filename.avi,_fps,_codec,_keep_open={ 0 | 1 }§\aftergroup\Ccn§'. '§\aftergroup\Ccg§codec§\aftergroup\Ccn§' is a 4-char string 
       (see §\aftergroup\Ccr§http://www.fourcc.org/codecs.php§\aftergroup\Ccn§) or '§\aftergroup\Ccg§0§\aftergroup\Ccn§' for the default codec. '§\aftergroup\Ccg§keep_open§\aftergroup\Ccn§' tells if 
       the output video file must be kept open for appending new frames afterwards. 
 
    . §\aftergroup\Ccg§.cimg[z] files:§\aftergroup\Ccn§ Only crops and sub-images of .cimg files can be loaded, using the input 
      expressions '§\aftergroup\Ccb§filename.cimg,N0,N1§\aftergroup\Ccn§', '§\aftergroup\Ccb§filename.cimg,N0,N1,x0,x1§\aftergroup\Ccn§', 
      '§\aftergroup\Ccb§filename.cimg,N0,N1,x0,y0,x1,y1§\aftergroup\Ccn§', '§\aftergroup\Ccb§filename.cimg,N0,N1,x0,y0,z0,x1,y1,z1§\aftergroup\Ccn§' or 
      '§\aftergroup\Ccb§filename.cimg,N0,N1,x0,y0,z0,c0,x1,y1,z1,c1§\aftergroup\Ccn§'. 
      Specifying '§\aftergroup\Ccg§-1§\aftergroup\Ccn§' for one coordinates stands for the maximum possible value. Output expression 
      '§\aftergroup\Ccb§filename.cimg[z][,datatype]§\aftergroup\Ccn§' can be used to force the output pixel type. '§\aftergroup\Ccg§datatype§\aftergroup\Ccn§' can be 
      §\aftergroup\Ccg§{ uchar | char | ushort | short | uint | int | uint64 | int64 | float | double }§\aftergroup\Ccn§. 
 
    . §\aftergroup\Ccg§.raw binary files:§\aftergroup\Ccn§ Image dimensions and input pixel type may be specified when loading §\aftergroup\Ccg§.raw§\aftergroup\Ccn§ 
       files with input expresssion 
       '§\aftergroup\Ccb§filename.raw[,datatype][,width][,height[,depth[,dim[,offset]]]]]§\aftergroup\Ccn§'. If no dimensions are 
       specified, the resulting image is a one-column vector with maximum possible height. Pixel 
       type can also be specified with the output expression '§\aftergroup\Ccb§filename.raw[,datatype]§\aftergroup\Ccn§'. 
       '§\aftergroup\Ccg§datatype§\aftergroup\Ccn§' can be the same as for §\aftergroup\Ccg§.cimg[z]§\aftergroup\Ccn§ files. 
 
    . §\aftergroup\Ccg§.yuv files:§\aftergroup\Ccn§ Image dimensions must be specified, and only sub-frames of an image sequence may 
       be loaded, using the input expression 
       '§\aftergroup\Ccb§filename.yuv,width,height[,first_frame[,last_frame[,step]]]§\aftergroup\Ccn§'. 
 
    . §\aftergroup\Ccg§.tiff files:§\aftergroup\Ccn§ Only sub-images of multi-pages tiff files can be loaded, using the input 
       expression '§\aftergroup\Ccb§filename.tif,_first_frame,_last_frame,_step§\aftergroup\Ccn§'. 
       Output expression '§\aftergroup\Ccb§filename.tiff,_datatype,_compression,_force_multipage,_use_bigtiff§\aftergroup\Ccn§' can 
       be used to specify the output pixel type, as well as the compression method. 
       '§\aftergroup\Ccg§datatype§\aftergroup\Ccn§' can be the same as for §\aftergroup\Ccg§.cimg[z]§\aftergroup\Ccn§ files. '§\aftergroup\Ccg§compression§\aftergroup\Ccn§' can be 
       §\aftergroup\Ccg§{ none (default) | lzw | jpeg }§\aftergroup\Ccn§. '§\aftergroup\Ccg§force_multipage§\aftergroup\Ccn§ can be §\aftergroup\Ccg§{ 0=no (default) | 1=yes }§\aftergroup\Ccn§. 
       '§\aftergroup\Ccg§use_bigtiff§\aftergroup\Ccn§ can be §\aftergroup\Ccg§{ 0=no | 1=yes (default) }§\aftergroup\Ccn§. 
 
    . §\aftergroup\Ccg§.gif files:§\aftergroup\Ccn§ Animated gif files can be saved, using the input expression 
       '§\aftergroup\Ccb§filename.gif,fps>0,nb_loops§\aftergroup\Ccn§'. Specify '§\aftergroup\Ccg§nb_loops=0§\aftergroup\Ccn§' to get an infinite number of animation 
       loops (this is the default behavior). 
 
    . §\aftergroup\Ccg§.jpeg files:§\aftergroup\Ccn§ The output quality may be specified (in %), using the output expression 
       '§\aftergroup\Ccb§filename.jpg,30§\aftergroup\Ccn§' (here, to get a 30% quality output). '§\aftergroup\Ccg§100§\aftergroup\Ccn§' is the default. 
 
    . §\aftergroup\Ccg§.mnc files:§\aftergroup\Ccn§ The output header can set from another file, using the output expression 
       '§\aftergroup\Ccb§filename.mnc,header_template.mnc§\aftergroup\Ccn§'. 
 
    . §\aftergroup\Ccg§.pan, .cpp, .hpp, .c and .h files:§\aftergroup\Ccn§ The output datatype can be selected with output expression 
       '§\aftergroup\Ccb§filename[,datatype]§\aftergroup\Ccn§'. '§\aftergroup\Ccg§datatype§\aftergroup\Ccn§' can be the same as for §\aftergroup\Ccg§.cimg[z]§\aftergroup\Ccn§ files. 
 
    . §\aftergroup\Ccg§.gmic files:§\aftergroup\Ccn§ These filenames are assumed to be §\aftergroup\Ccg§G'MIC§\aftergroup\Ccn§ custom commands files. Loading such a 
       file will add the commands it defines to the interpreter. Debug information can be 
       enabled/disabled by the input expression '§\aftergroup\Ccb§filename.gmic[,add_debug_info={ 0 | 1 }]§\aftergroup\Ccn§'. 
 
    . Inserting '§\aftergroup\Ccb§ext:§\aftergroup\Ccn§' on the beginning of a filename (e.g. '§\aftergroup\Ccb§jpg:filename§\aftergroup\Ccn§') forces §\aftergroup\Ccg§G'MIC§\aftergroup\Ccn§ to 
       read/write the file as it would have been done if it had the specified extension '§\aftergroup\Ccg§.ext§\aftergroup\Ccn§'. 
 
  - Some input/output formats and options may not be supported, depending on the configuration 
     flags that have been set during the build of the §\aftergroup\Ccg§G'MIC§\aftergroup\Ccn§ software.
\end{lstlisting}
\normalsize
~\\\section{Substitution rules}
\small
\begin{lstlisting}[escapechar=§]
  - §\aftergroup\Ccg§G'MIC§\aftergroup\Ccn§ items containing '§\aftergroup\Ccg§$§\aftergroup\Ccn§' or '§\aftergroup\Ccg§{}§\aftergroup\Ccn§' are substituted before being interpreted. Use these 
     substituting expressions to access various data from the interpreter environment. 
 
  - '§\aftergroup\Ccb§$name§\aftergroup\Ccn§' and '§\aftergroup\Ccb§${name}§\aftergroup\Ccn§' are both substituted by the value of the specified named §\aftergroup\Ccg§variable§\aftergroup\Ccn§ 
     (set previously by the item '§\aftergroup\Ccb§name=value§\aftergroup\Ccn§'). If this variable has not been already set, the 
     expression is substituted by the highest positive §\aftergroup\Ccg§indice§\aftergroup\Ccn§ of the named image '§\aftergroup\Ccg§[name]§\aftergroup\Ccn§'. If no 
     image has this name, the expression is substituted by the value of the §\aftergroup\Ccg§OS environment variable§\aftergroup\Ccn§ 
     with same name (it may be thus an empty string). 
     The following reserved variables are predefined by the §\aftergroup\Ccg§G'MIC§\aftergroup\Ccn§ interpreter: 
 
       . '§\aftergroup\Ccb§$!§\aftergroup\Ccn§': The current number of images in the list. 
       . '§\aftergroup\Ccb§$>§\aftergroup\Ccn§' and '§\aftergroup\Ccb§$<§\aftergroup\Ccn§': The increasing/decreasing indice of the latest (currently running) 
          '§\aftergroup\Ccb§repeat...done§\aftergroup\Ccn§' loop. 
       . '§\aftergroup\Ccb§$/§\aftergroup\Ccn§': The current call stack. Stack items are separated by slashes '§\aftergroup\Ccg§/§\aftergroup\Ccn§'. 
       . '§\aftergroup\Ccb§$|§\aftergroup\Ccn§': The current value (expressed in seconds) of a millisecond precision timer. 
       . '§\aftergroup\Ccb§$^§\aftergroup\Ccn§': The current verbosity level. 
       . '§\aftergroup\Ccb§$_cpus§\aftergroup\Ccn§': The number of computation cores available on your machine. 
       . '§\aftergroup\Ccb§$_pid§\aftergroup\Ccn§': The current process identifier, as an integer. 
       . '§\aftergroup\Ccb§$_prerelease§\aftergroup\Ccn§': For pre-releases only, the date of the pre-release as '§\aftergroup\Ccg§mmddyy§\aftergroup\Ccn§'. 
          For stable releases, this variable is not defined. 
       . '§\aftergroup\Ccb§$_version§\aftergroup\Ccn§': A 3-digits number telling about the current version of the §\aftergroup\Ccg§G'MIC§\aftergroup\Ccn§ interpreter 
          (currently '§\aftergroup\Ccg§211§\aftergroup\Ccn§'). 
       . '§\aftergroup\Ccb§$_vt100§\aftergroup\Ccn§': Set to §\aftergroup\Ccg§1§\aftergroup\Ccn§ (default value) if colored text output is allowed on the console. 
       . '§\aftergroup\Ccb§$_path_rc§\aftergroup\Ccn§': The path to the §\aftergroup\Ccg§G'MIC§\aftergroup\Ccn§ folder used to store resources and configuration files 
         (its value is OS-dependent). 
       . '§\aftergroup\Ccb§$_path_user§\aftergroup\Ccn§': The path to the §\aftergroup\Ccg§G'MIC§\aftergroup\Ccn§ user file §\aftergroup\Ccg§.gmic§\aftergroup\Ccn§ or §\aftergroup\Ccg§user.gmic§\aftergroup\Ccn§ (its value is 
          OS-dependent). 
 
  - '§\aftergroup\Ccb§${"command line"}§\aftergroup\Ccn§' is substituted by the §\aftergroup\Ccg§status value§\aftergroup\Ccn§ set by the execution of the specified 
     command line (see command '§\aftergroup\Ccb§status§\aftergroup\Ccn§' to learn more about status). Expression '§\aftergroup\Ccb§${}§\aftergroup\Ccn§' thus stands 
     for the §\aftergroup\Ccg§current status value§\aftergroup\Ccn§. 
 
  - '§\aftergroup\Ccb§{``string}§\aftergroup\Ccn§' (starting with two backquotes) is substituted by a §\aftergroup\Ccg§double-quoted version§\aftergroup\Ccn§ of the 
     specified string. 
 
  - '§\aftergroup\Ccb§{/string}§\aftergroup\Ccn§' is substituted by the §\aftergroup\Ccg§escaped version§\aftergroup\Ccn§ of the specified string. 
 
  - '§\aftergroup\Ccb§{'string'}§\aftergroup\Ccn§' (between single quotes) is substituted by the §\aftergroup\Ccg§sequence of ascii codes§\aftergroup\Ccn§ that compose 
     the specified string, separated by commas '§\aftergroup\Ccg§,§\aftergroup\Ccn§'. For instance, item '§\aftergroup\Ccb§{'foo'}§\aftergroup\Ccn§' is substituted 
     by '§\aftergroup\Ccb§102,111,111§\aftergroup\Ccn§'. 
 
  - '§\aftergroup\Ccb§{image,feature}§\aftergroup\Ccn§' is substituted by a specific feature of the image §\aftergroup\Ccb§[image]§\aftergroup\Ccn§. '§\aftergroup\Ccg§image§\aftergroup\Ccn§' can be 
     either an image number or an image name. It can be also eluded, in which case, the last image 
     '§\aftergroup\Ccb§[-1]§\aftergroup\Ccn§' of the list is considered for the requested feature. 
     Specified '§\aftergroup\Ccg§feature§\aftergroup\Ccn§' can be one of: 
 
       . '§\aftergroup\Ccb§b§\aftergroup\Ccn§': The image basename (i.e. filename without the folder path nor extension). 
       . '§\aftergroup\Ccb§f§\aftergroup\Ccn§': The image folder name. 
       . '§\aftergroup\Ccb§n§\aftergroup\Ccn§': The image name or filename (if the image has been read from a file). 
       . '§\aftergroup\Ccb§t§\aftergroup\Ccn§': The text string from the image values regarded as ascii codes. 
       . '§\aftergroup\Ccb§x§\aftergroup\Ccn§': The image extension (i.e last characters after the last '.' in the image name). 
       . '§\aftergroup\Ccb§^§\aftergroup\Ccn§ : The sequence of all image values, separated by commas ','. 
       . '§\aftergroup\Ccb§@subset§\aftergroup\Ccn§': The sequence of image values corresponding to the specified subset, and 
          separated by commas ','. 
       . Any other '§\aftergroup\Ccb§feature§\aftergroup\Ccn§' is considered as a §\aftergroup\Ccg§mathematical expression§\aftergroup\Ccn§ associated to the image 
          §\aftergroup\Ccb§[image]§\aftergroup\Ccn§ and is substituted by the result of its evaluation (float value). For instance, 
          expression '§\aftergroup\Ccb§{0,w+h}§\aftergroup\Ccn§' is substituted by the sum of the width and height of the first image 
          (see section '§\aftergroup\Ccb§Mathematical expressions§\aftergroup\Ccn§' for more details). If a mathematical expression 
          starts with an underscore '§\aftergroup\Ccg§_§\aftergroup\Ccn§', the resulting value is truncated to a readable format. 
          For instance, item '§\aftergroup\Ccb§{_pi}§\aftergroup\Ccn§' is substituted by '§\aftergroup\Ccg§3.14159§\aftergroup\Ccn§' (while '§\aftergroup\Ccb§{pi}§\aftergroup\Ccn§' is substituted by 
          '§\aftergroup\Ccg§3.141592653589793§\aftergroup\Ccn§'). 
       . A '§\aftergroup\Ccb§feature§\aftergroup\Ccn§' delimited by backquotes is replaced by a string whose ascii codes correspond 
          to the list of values resulting from the evaluation of the specified mathematical 
          expression. For instance, item '§\aftergroup\Ccb§{`[102,111,111]`}§\aftergroup\Ccn§' is substituted by '§\aftergroup\Ccb§foo§\aftergroup\Ccn§' and item 
          '§\aftergroup\Ccb§{`vector8(65)`}§\aftergroup\Ccn§' by '§\aftergroup\Ccb§AAAAAAAA§\aftergroup\Ccn§'. 
 
  - '§\aftergroup\Ccb§{*}§\aftergroup\Ccn§' is substituted by the §\aftergroup\Ccg§visibility state§\aftergroup\Ccn§ of the instant display window §\aftergroup\Ccb§[0]§\aftergroup\Ccn§ (can be 
     §\aftergroup\Ccg§{ 0=closed | 1=visible }§\aftergroup\Ccn§). 
 
  - '§\aftergroup\Ccb§{*,feature}§\aftergroup\Ccn§' or '§\aftergroup\Ccb§{*indice,feature}§\aftergroup\Ccn§' is substituted by a specific feature of the instant 
     display window §\aftergroup\Ccb§#0§\aftergroup\Ccn§ (or §\aftergroup\Ccb§#indice§\aftergroup\Ccn§, if specified). Requested '§\aftergroup\Ccg§feature§\aftergroup\Ccn§' can be: 
 
       . '§\aftergroup\Ccb§w§\aftergroup\Ccn§': display width (i.e. width of the display area managed by the window). 
       . '§\aftergroup\Ccb§h§\aftergroup\Ccn§': display height (i.e. height of the display area managed by the window). 
       . '§\aftergroup\Ccb§wh§\aftergroup\Ccn§': display width x display height. 
       . '§\aftergroup\Ccb§d§\aftergroup\Ccn§': window width (i.e. width of the window widget). 
       . '§\aftergroup\Ccb§e§\aftergroup\Ccn§': window height (i.e. height of the window widget). 
       . '§\aftergroup\Ccb§de§\aftergroup\Ccn§': window width x window height. 
       . '§\aftergroup\Ccb§u§\aftergroup\Ccn§': screen width (actually independent on the window size). 
       .' §\aftergroup\Ccb§v§\aftergroup\Ccn§': screen height (actually independent on the window size). 
       . '§\aftergroup\Ccb§uv§\aftergroup\Ccn§': screen width x screen height. 
       . '§\aftergroup\Ccb§n§\aftergroup\Ccn§': current normalization type of the instant display. 
       . '§\aftergroup\Ccb§t§\aftergroup\Ccn§': window title of the instant display. 
       . '§\aftergroup\Ccb§x§\aftergroup\Ccn§': X-coordinate of the mouse position (or -1, if outside the display area). 
       . '§\aftergroup\Ccb§y§\aftergroup\Ccn§': Y-coordinate of the mouse position (or -1, if outside the display area). 
       . '§\aftergroup\Ccb§b§\aftergroup\Ccn§': state of the mouse buttons { 1=left-but. | 2=right-but. | 4=middle-but. }. 
       . '§\aftergroup\Ccb§o§\aftergroup\Ccn§': state of the mouse wheel. 
       . '§\aftergroup\Ccb§k§\aftergroup\Ccn§': decimal code of the pressed key if any, 0 otherwise. 
       . '§\aftergroup\Ccb§c§\aftergroup\Ccn§': boolean (0 or 1) telling if the instant display has been closed recently. 
       . '§\aftergroup\Ccb§r§\aftergroup\Ccn§': boolean telling if the instant display has been resized recently. 
       . '§\aftergroup\Ccb§m§\aftergroup\Ccn§': boolean telling if the instant display has been moved recently. 
       . Any other '§\aftergroup\Ccb§feature§\aftergroup\Ccn§' stands for a §\aftergroup\Ccg§keycode name§\aftergroup\Ccn§ (in capital letters), and is substituted by 
          a boolean describing the current key state §\aftergroup\Ccg§{ 0=pressed | 1=released }§\aftergroup\Ccn§. 
       . You can also prepend a hyphen '§\aftergroup\Ccb§-§\aftergroup\Ccn§' to a '§\aftergroup\Ccg§feature§\aftergroup\Ccn§' (that supports it) to flush the 
          corresponding event immediately after reading its state (works for keys, mouse and 
          window events). 
 
  - Item substitution is §\aftergroup\Ccg§never performed in items between double quotes§\aftergroup\Ccn§. One must break the quotes 
    to enable substitution if needed, as in §\aftergroup\Ccb§"3+8 kg = "{3+8}" kg"§\aftergroup\Ccn§. Using double quotes is then 
    a convenient way to disable the substitutions mechanism in items, when necessary. 
 
  - One can also disable the substitution mechanism on items outside double quotes, by escaping the 
     '§\aftergroup\Ccg§{§\aftergroup\Ccn§','§\aftergroup\Ccg§}§\aftergroup\Ccn§' or '§\aftergroup\Ccg§$§\aftergroup\Ccn§' characters, as in '§\aftergroup\Ccb§\{3+4\}\ doesn't\ evaluate§\aftergroup\Ccn§'.
\end{lstlisting}
\normalsize
~\\\section{Mathematical expressions}
\small
\begin{lstlisting}[escapechar=§]
  - §\aftergroup\Ccg§G'MIC§\aftergroup\Ccn§ has an embedded §\aftergroup\Ccg§mathematical parser§\aftergroup\Ccn§. It is used to evaluate (possibly complex) expressions 
     inside braces '§\aftergroup\Ccg§{}§\aftergroup\Ccn§', or formulas in commands that may take one as an argument (e.g. '§\aftergroup\Ccb§fill§\aftergroup\Ccn§'). 
 
  - When the context allows it, a formula is evaluated §\aftergroup\Ccg§for each pixel§\aftergroup\Ccn§ of the selected images 
     (e.g. '§\aftergroup\Ccb§fill§\aftergroup\Ccn§'). 
 
  - A math expression may return a §\aftergroup\Ccg§scalar§\aftergroup\Ccn§ or §\aftergroup\Ccg§vector§\aftergroup\Ccn§-valued result (with a fixed number of 
     components). 
 
  - The mathematical parser understands the following set of functions, operators and variables: 
 
    _ §\aftergroup\Ccg§Usual operators:§\aftergroup\Ccn§ §\aftergroup\Ccb§||§\aftergroup\Ccn§ (logical or), §\aftergroup\Ccb§&&§\aftergroup\Ccn§ (logical and), §\aftergroup\Ccb§|§\aftergroup\Ccn§ (bitwise or), §\aftergroup\Ccb§&§\aftergroup\Ccn§ (bitwise and), 
       §\aftergroup\Ccb§!=, ==, <=, >=, <, >, <<§\aftergroup\Ccn§ (left bitwise shift), §\aftergroup\Ccb§>>§\aftergroup\Ccn§ (right bitwise shift), §\aftergroup\Ccb§-, +, *, /, 
       %§\aftergroup\Ccn§ (modulo), §\aftergroup\Ccb§^§\aftergroup\Ccn§ (power), §\aftergroup\Ccb§!§\aftergroup\Ccn§ (logical not), §\aftergroup\Ccb§~§\aftergroup\Ccn§ (bitwise not), 
       §\aftergroup\Ccb§++§\aftergroup\Ccn§, §\aftergroup\Ccb§--§\aftergroup\Ccn§, §\aftergroup\Ccb§+=§\aftergroup\Ccn§, §\aftergroup\Ccb§-=§\aftergroup\Ccn§, §\aftergroup\Ccb§*=§\aftergroup\Ccn§, §\aftergroup\Ccb§/=§\aftergroup\Ccn§, §\aftergroup\Ccb§%=§\aftergroup\Ccn§, §\aftergroup\Ccb§&=§\aftergroup\Ccn§, §\aftergroup\Ccb§|=§\aftergroup\Ccn§, §\aftergroup\Ccb§^=§\aftergroup\Ccn§, §\aftergroup\Ccb§>>=§\aftergroup\Ccn§, §\aftergroup\Ccb§<<=§\aftergroup\Ccn§ (in-place operators). 
 
    _ §\aftergroup\Ccg§Usual math functions:§\aftergroup\Ccn§ §\aftergroup\Ccb§abs(), acos(), arg(), argkth(), argmax(), argmin(), asin(), atan(), 
       atan2(), bool(), cbrt(), ceil(), cos(), cosh(), cut(), exp(), fact(), fibo(), floor(), gauss(), int(), isval(), 
       isnan(), isinf(), isint(), isbool(), isfile(), isdir(), isin(), kth(), log(), log2(), log10(), 
       max(), mean(), med(), min(), narg(), prod(), rol()§\aftergroup\Ccn§ (left bit rotation), 
       §\aftergroup\Ccb§ror()§\aftergroup\Ccn§ (right bit rotation)§\aftergroup\Ccb§, round(), sign(), sin(), sinc(), sinh(), sqrt(), std(), 
       srand(_seed), sum(), tan(), tanh(), variance(), xor()§\aftergroup\Ccn§. 
 
       . '§\aftergroup\Ccb§atan2(y,x)§\aftergroup\Ccn§' is the version of '§\aftergroup\Ccb§atan()§\aftergroup\Ccn§' with two arguments '§\aftergroup\Ccg§y§\aftergroup\Ccn§' and '§\aftergroup\Ccg§x§\aftergroup\Ccn§' (as in C/C++). 
       . '§\aftergroup\Ccb§permut(k,n,with_order)§\aftergroup\Ccn§' computes the number of permutations of §\aftergroup\Ccg§k§\aftergroup\Ccn§ objects from a set of 
          §\aftergroup\Ccg§n§\aftergroup\Ccn§ objects. 
       . '§\aftergroup\Ccb§gauss(x,_sigma)§\aftergroup\Ccn§' returns '§\aftergroup\Ccg§exp(-x^2/(2*s^2))/sqrt(2*pi*sigma^2)'§\aftergroup\Ccn§. 
       . '§\aftergroup\Ccb§cut(value,min,max)§\aftergroup\Ccn§' returns value if it is in range §\aftergroup\Ccg§[min,max]§\aftergroup\Ccn§, or §\aftergroup\Ccg§min§\aftergroup\Ccn§ or §\aftergroup\Ccg§max§\aftergroup\Ccn§ otherwise. 
       . '§\aftergroup\Ccb§narg(a_1,...,a_N)§\aftergroup\Ccn§' returns the number of specified arguments (here, §\aftergroup\Ccg§N§\aftergroup\Ccn§). 
       . '§\aftergroup\Ccb§arg(i,a_1,..,a_N)§\aftergroup\Ccn§' returns the §\aftergroup\Ccg§ith§\aftergroup\Ccn§ argument §\aftergroup\Ccg§a_i§\aftergroup\Ccn§. 
       . '§\aftergroup\Ccb§isval()§\aftergroup\Ccn§', '§\aftergroup\Ccb§isnan()§\aftergroup\Ccn§', '§\aftergroup\Ccb§isinf()§\aftergroup\Ccn§', '§\aftergroup\Ccb§isint()§\aftergroup\Ccn§', '§\aftergroup\Ccb§isbool()§\aftergroup\Ccn§' test the type of the given 
          number or expression, and return §\aftergroup\Ccg§0 (false)§\aftergroup\Ccn§ or §\aftergroup\Ccg§1 (true)§\aftergroup\Ccn§. 
       . '§\aftergroup\Ccb§isfile()§\aftergroup\Ccn§' (resp. '§\aftergroup\Ccb§isdir()§\aftergroup\Ccn§') returns §\aftergroup\Ccg§0 (false)§\aftergroup\Ccn§ or §\aftergroup\Ccg§1 (true)§\aftergroup\Ccn§ whether its argument is a 
          path to an existing file (resp. to a directory) or not. 
       . '§\aftergroup\Ccb§isin(v,a_1,...,a_n)§\aftergroup\Ccn§' returns §\aftergroup\Ccg§0 (false)§\aftergroup\Ccn§ or §\aftergroup\Ccg§1 (true)§\aftergroup\Ccn§ whether the first value '§\aftergroup\Ccg§v§\aftergroup\Ccn§' appears 
          in the set of other values 'a_i'. 
       . '§\aftergroup\Ccb§argmin()§\aftergroup\Ccn§', '§\aftergroup\Ccb§argmax()§\aftergroup\Ccn§', '§\aftergroup\Ccb§kth()§\aftergroup\Ccn§', '§\aftergroup\Ccb§max()§\aftergroup\Ccn§', '§\aftergroup\Ccb§mean()§\aftergroup\Ccn§', '§\aftergroup\Ccb§med()§\aftergroup\Ccn§', '§\aftergroup\Ccb§min()§\aftergroup\Ccn§', '§\aftergroup\Ccb§std()§\aftergroup\Ccn§', '§\aftergroup\Ccb§sum()§\aftergroup\Ccn§' 
          and '§\aftergroup\Ccb§variance()§\aftergroup\Ccn§' can be called with an arbitrary number of scalar/vector arguments. 
       . '§\aftergroup\Ccb§round(value,rounding_value,direction)§\aftergroup\Ccn§' returns a rounded value. '§\aftergroup\Ccg§direction§\aftergroup\Ccn§' can be 
          §\aftergroup\Ccg§{ -1=to-lowest | 0=to-nearest | 1=to-highest }§\aftergroup\Ccn§. 
 
    _ §\aftergroup\Ccg§Variable names§\aftergroup\Ccn§ below are pre-defined. They can be overrided. 
 
       . '§\aftergroup\Ccb§l§\aftergroup\Ccn§': length of the associated list of images. 
       . '§\aftergroup\Ccb§w§\aftergroup\Ccn§': width of the associated image, if any (§\aftergroup\Ccg§0§\aftergroup\Ccn§ otherwise). 
       . '§\aftergroup\Ccb§h§\aftergroup\Ccn§': height of the associated image, if any (§\aftergroup\Ccg§0§\aftergroup\Ccn§ otherwise). 
       . '§\aftergroup\Ccb§d§\aftergroup\Ccn§': depth of the associated image, if any (§\aftergroup\Ccg§0§\aftergroup\Ccn§ otherwise). 
       . '§\aftergroup\Ccb§s§\aftergroup\Ccn§': spectrum of the associated image, if any (§\aftergroup\Ccg§0§\aftergroup\Ccn§ otherwise). 
       . '§\aftergroup\Ccb§r§\aftergroup\Ccn§': shared state of the associated image, if any (§\aftergroup\Ccg§0§\aftergroup\Ccn§ otherwise). 
       . '§\aftergroup\Ccb§wh§\aftergroup\Ccn§': shortcut for width x height. 
       . '§\aftergroup\Ccb§whd§\aftergroup\Ccn§': shortcut for width x height x depth. 
       . '§\aftergroup\Ccb§whds§\aftergroup\Ccn§': shortcut for width x height x depth x spectrum (i.e. number of image values). 
       . '§\aftergroup\Ccb§im§\aftergroup\Ccn§','§\aftergroup\Ccb§iM§\aftergroup\Ccn§','§\aftergroup\Ccb§ia§\aftergroup\Ccn§','§\aftergroup\Ccb§iv§\aftergroup\Ccn§','§\aftergroup\Ccb§is§\aftergroup\Ccn§','§\aftergroup\Ccb§ip§\aftergroup\Ccn§','§\aftergroup\Ccb§ic§\aftergroup\Ccn§': Respectively the minimum, maximum, average, 
          variance, sum, product and median value of the associated image, if any (§\aftergroup\Ccg§0§\aftergroup\Ccn§ otherwise). 
       . '§\aftergroup\Ccb§xm§\aftergroup\Ccn§','§\aftergroup\Ccb§ym§\aftergroup\Ccn§','§\aftergroup\Ccb§zm§\aftergroup\Ccn§','§\aftergroup\Ccb§cm§\aftergroup\Ccn§': The pixel coordinates of the minimum value in the associated 
          image, if any (§\aftergroup\Ccg§0§\aftergroup\Ccn§ otherwise). 
       . '§\aftergroup\Ccb§xM§\aftergroup\Ccn§','§\aftergroup\Ccb§yM§\aftergroup\Ccn§','§\aftergroup\Ccb§zM§\aftergroup\Ccn§','§\aftergroup\Ccb§cM§\aftergroup\Ccn§': The pixel coordinates of the maximum value in the associated 
          image, if any (§\aftergroup\Ccg§0§\aftergroup\Ccn§ otherwise). 
       . All these variables are considered as §\aftergroup\Ccg§constant values§\aftergroup\Ccn§ by the math parser (for optimization 
          purposes) which is indeed the case most of the time. Anyway, this might not be the case, 
          if function '§\aftergroup\Ccb§resize(#ind,..)§\aftergroup\Ccn§' is used in the math expression. 
          If so, it is safer to invoke functions '§\aftergroup\Ccb§l()§\aftergroup\Ccn§', '§\aftergroup\Ccb§w(_#ind)§\aftergroup\Ccn§', '§\aftergroup\Ccb§h(_#ind)§\aftergroup\Ccn§', ... '§\aftergroup\Ccb§cm(_#ind)§\aftergroup\Ccn§' 
          and '§\aftergroup\Ccb§cM(_#ind)§\aftergroup\Ccn§' instead of the corresponding named variables. 
       . '§\aftergroup\Ccb§i§\aftergroup\Ccn§': current processed pixel value (i.e. value located at §\aftergroup\Ccg§(x,y,z,c)§\aftergroup\Ccn§) in the associated 
          image, if any (§\aftergroup\Ccg§0§\aftergroup\Ccn§ otherwise). 
       . '§\aftergroup\Ccb§iN§\aftergroup\Ccn§': Nth channel value of current processed pixel (i.e. value located at §\aftergroup\Ccg§(x,y,z,N)§\aftergroup\Ccn§) in 
          the associated image, if any (§\aftergroup\Ccg§0§\aftergroup\Ccn§ otherwise). '§\aftergroup\Ccg§N§\aftergroup\Ccn§' must be an integer in range §\aftergroup\Ccg§[0,9]§\aftergroup\Ccn§. 
       . '§\aftergroup\Ccb§R§\aftergroup\Ccn§','§\aftergroup\Ccb§G§\aftergroup\Ccn§','§\aftergroup\Ccb§B§\aftergroup\Ccn§' and '§\aftergroup\Ccb§A§\aftergroup\Ccn§' are equivalent to '§\aftergroup\Ccb§i0§\aftergroup\Ccn§', '§\aftergroup\Ccb§i1§\aftergroup\Ccn§', '§\aftergroup\Ccb§i2§\aftergroup\Ccn§' and '§\aftergroup\Ccb§i3§\aftergroup\Ccn§' respectively. 
       . '§\aftergroup\Ccb§I§\aftergroup\Ccn§': current vector-valued processed pixel in the associated image, if any (§\aftergroup\Ccg§0§\aftergroup\Ccn§ otherwise). 
          The number of vector components is equal to the number of image channels 
          (e.g. §\aftergroup\Ccg§I = [ R,G,B ]§\aftergroup\Ccn§ for a §\aftergroup\Ccg§RGB§\aftergroup\Ccn§ image). 
       . You may add '§\aftergroup\Ccb§#ind§\aftergroup\Ccn§' to any of the variable name above to retrieve the information for any 
          numbered image §\aftergroup\Ccg§[ind]§\aftergroup\Ccn§ of the list (when this makes sense). For instance '§\aftergroup\Ccb§ia#0§\aftergroup\Ccn§' denotes the 
          average value of the first image of the list). 
       . '§\aftergroup\Ccb§x§\aftergroup\Ccn§': current processed column of the associated image, if any (§\aftergroup\Ccg§0§\aftergroup\Ccn§ otherwise). 
       . '§\aftergroup\Ccb§y§\aftergroup\Ccn§': current processed row of the associated image, if any (§\aftergroup\Ccg§0§\aftergroup\Ccn§ otherwise). 
       . '§\aftergroup\Ccb§z§\aftergroup\Ccn§': current processed slice of the associated image, if any (§\aftergroup\Ccg§0§\aftergroup\Ccn§ otherwise). 
       . '§\aftergroup\Ccb§c§\aftergroup\Ccn§': current processed channel of the associated image, if any (§\aftergroup\Ccg§0§\aftergroup\Ccn§ otherwise). 
       . '§\aftergroup\Ccb§t§\aftergroup\Ccn§': thread id when an expression is evaluated with multiple threads (§\aftergroup\Ccg§0§\aftergroup\Ccn§ means 
          'master thread'). 
       . '§\aftergroup\Ccb§e§\aftergroup\Ccn§': value of e, i.e. §\aftergroup\Ccg§2.71828...§\aftergroup\Ccn§ 
       . '§\aftergroup\Ccb§pi§\aftergroup\Ccn§': value of pi, i.e. §\aftergroup\Ccg§3.1415926...§\aftergroup\Ccn§ 
       . '§\aftergroup\Ccb§u§\aftergroup\Ccn§': a random value between §\aftergroup\Ccg§[0,1]§\aftergroup\Ccn§, following a uniform distribution. 
       . '§\aftergroup\Ccb§g§\aftergroup\Ccn§': a random value, following a gaussian distribution of variance 1 
          (roughly in §\aftergroup\Ccg§[-6,6]§\aftergroup\Ccn§). 
       . '§\aftergroup\Ccb§interpolation§\aftergroup\Ccn§': value of the default interpolation mode used when reading pixel values 
           with the pixel access operators (i.e. when the interpolation argument is not explicitly 
           specified, see below for more details on pixel access operators). Its initial default 
           value is §\aftergroup\Ccg§0§\aftergroup\Ccn§. 
       . '§\aftergroup\Ccb§boundary§\aftergroup\Ccn§': value of the default boundary conditions used when reading pixel values with 
           the pixel access operators (i.e. when the boundary condition argument is not explicitly 
           specified, see below for more details on pixel access operators). Its initial default 
           value is §\aftergroup\Ccg§0§\aftergroup\Ccn§. 
 
    _ §\aftergroup\Ccg§Vector calculus:§\aftergroup\Ccn§ Most operators are also able to work with vector-valued elements. 
 
       . '§\aftergroup\Ccb§[ a0,a1,...,aN ]§\aftergroup\Ccn§' defines a §\aftergroup\Ccg§(N+1)§\aftergroup\Ccn§-dimensional vector with scalar coefficients §\aftergroup\Ccg§ak§\aftergroup\Ccn§. 
       . '§\aftergroup\Ccb§vectorN(a0,a1,,...,)§\aftergroup\Ccn§' does the same, with the §\aftergroup\Ccg§ak§\aftergroup\Ccn§ being repeated periodically if only a 
          few are specified. 
       . In both previous expressions, the §\aftergroup\Ccg§ak§\aftergroup\Ccn§ can be vectors themselves, to be concatenated into a 
          single vector. 
       . The scalar element §\aftergroup\Ccg§ak§\aftergroup\Ccn§ of a vector §\aftergroup\Ccg§X§\aftergroup\Ccn§ is retrieved by '§\aftergroup\Ccb§X[k]§\aftergroup\Ccn§'. 
       . The sub-vector §\aftergroup\Ccg§[ X[p]...X[p+q-1] ]§\aftergroup\Ccn§ (of size §\aftergroup\Ccg§q§\aftergroup\Ccn§) of a vector §\aftergroup\Ccg§X§\aftergroup\Ccn§ is retrieved by '§\aftergroup\Ccb§X[p,q]§\aftergroup\Ccn§'. 
       . Equality/inequality comparisons between two vectors is done with operators '§\aftergroup\Ccb§==§\aftergroup\Ccn§' and '§\aftergroup\Ccb§!=§\aftergroup\Ccn§'. 
       . Some vector-specific functions can be used on vector values: 
         '§\aftergroup\Ccb§cross(X,Y)§\aftergroup\Ccn§' (cross product), '§\aftergroup\Ccb§dot(X,Y)§\aftergroup\Ccn§' (dot product), '§\aftergroup\Ccb§size(X)§\aftergroup\Ccn§' (vector dimension), 
         '§\aftergroup\Ccb§sort(X,_is_increasing,_chunk_size)§\aftergroup\Ccn§' (sorting values), '§\aftergroup\Ccb§reverse(A)§\aftergroup\Ccn§' (reverse order of 
         components), '§\aftergroup\Ccb§shift(A,_length,_boundary_conditions)§\aftergroup\Ccn§' and 
         '§\aftergroup\Ccb§same(A,B,_nb_vals,_is_case_sensitive)§\aftergroup\Ccn§' (vector equality test). 
       . Function '§\aftergroup\Ccb§normP(u1,...,un)§\aftergroup\Ccn§' computes the LP-norm of the specified vector 
          (§\aftergroup\Ccb§P§\aftergroup\Ccn§ being an §\aftergroup\Ccg§unsigned integer§\aftergroup\Ccn§ constant or '§\aftergroup\Ccg§inf§\aftergroup\Ccn§'). If §\aftergroup\Ccb§P§\aftergroup\Ccn§ is omitted, the L2 norm is used. 
       . Function '§\aftergroup\Ccb§resize(A,size,_interpolation,_boundary_conditions)§\aftergroup\Ccn§' returns a resized version of 
          a vector '§\aftergroup\Ccg§A§\aftergroup\Ccn§' with specified interpolation mode. '§\aftergroup\Ccg§interpolation'§\aftergroup\Ccn§ can be §\aftergroup\Ccg§{ -1=none 
          (memory content) | 0=none | 1=nearest | 2=average | 3=linear | 4=grid | 5=bicubic | 
          6=lanczos }§\aftergroup\Ccn§, and '§\aftergroup\Ccg§boundary_conditions'§\aftergroup\Ccn§  can be §\aftergroup\Ccg§{ 0=dirichlet | 1=neumann | 2=periodic | 3=mirror }§\aftergroup\Ccn§. 
       . Function '§\aftergroup\Ccb§find(A,B,_is_forward,_starting_indice)§\aftergroup\Ccn§' returns the index where sub-vector §\aftergroup\Ccg§B§\aftergroup\Ccn§ 
          appears in vector §\aftergroup\Ccg§A§\aftergroup\Ccn§, (or §\aftergroup\Ccg§-1§\aftergroup\Ccn§ if §\aftergroup\Ccg§B§\aftergroup\Ccn§ is not found in §\aftergroup\Ccg§A§\aftergroup\Ccn§). Argument §\aftergroup\Ccg§A§\aftergroup\Ccn§ can be also replaced by 
          an image indice §\aftergroup\Ccg§#ind§\aftergroup\Ccn§. 
       . A §\aftergroup\Ccg§2§\aftergroup\Ccn§-dimensional vector may be seen as a complex number and used in those particular 
          functions/operators: 
          '§\aftergroup\Ccb§**§\aftergroup\Ccn§' (complex multiplication), '§\aftergroup\Ccb§//§\aftergroup\Ccn§' (complex division), '§\aftergroup\Ccb§^^§\aftergroup\Ccn§' (complex exponentiation), 
          '§\aftergroup\Ccb§**=§\aftergroup\Ccn§' (complex self-multiplication), '§\aftergroup\Ccb§//=§\aftergroup\Ccn§' (complex self-division), '§\aftergroup\Ccb§^^=§\aftergroup\Ccn§' (complex 
          self-exponentiation), '§\aftergroup\Ccb§cabs()§\aftergroup\Ccn§' (complex modulus), '§\aftergroup\Ccb§carg()§\aftergroup\Ccn§' (complex argument), '§\aftergroup\Ccb§cconj()§\aftergroup\Ccn§' 
          (complex conjugate), '§\aftergroup\Ccb§cexp()§\aftergroup\Ccn§' (complex exponential) and '§\aftergroup\Ccb§clog()§\aftergroup\Ccn§' (complex logarithm). 
       . A §\aftergroup\Ccg§MN§\aftergroup\Ccn§-dimensional vector may be seen as a §\aftergroup\Ccg§M§\aftergroup\Ccn§ x §\aftergroup\Ccg§N§\aftergroup\Ccn§ matrix and used in those particular 
          functions/operators: 
          '§\aftergroup\Ccb§*§\aftergroup\Ccn§' (matrix-vector multiplication), '§\aftergroup\Ccb§det(A)§\aftergroup\Ccn§' (determinant), '§\aftergroup\Ccb§diag(V)§\aftergroup\Ccn§' (diagonal matrix 
          from a vector), '§\aftergroup\Ccb§eig(A)§\aftergroup\Ccn§' (eigenvalues/eigenvectors), '§\aftergroup\Ccb§eye(n)§\aftergroup\Ccn§' (n x n identity matrix), 
          '§\aftergroup\Ccb§inv(A)§\aftergroup\Ccn§' (matrix inverse), '§\aftergroup\Ccb§mul(A,B,_nb_colsB)§\aftergroup\Ccn§' (matrix-matrix multiplication), 
          '§\aftergroup\Ccb§pseudoinv(A,_nb_colsA)§\aftergroup\Ccn§', '§\aftergroup\Ccb§rot(u,v,w,angle)§\aftergroup\Ccn§' (3d rotation matrix), '§\aftergroup\Ccb§rot(angle)§\aftergroup\Ccn§' (2d 
          rotation matrix), '§\aftergroup\Ccb§solve(A,B,_nb_colsB)§\aftergroup\Ccn§' (least-square solver of linear system A.X = B), 
          '§\aftergroup\Ccb§svd(A,_nb_colsA)§\aftergroup\Ccn§' (singular value decomposition), '§\aftergroup\Ccb§trace(A)§\aftergroup\Ccn§' (matrix trace) and 
          '§\aftergroup\Ccb§transp(A,nb_colsA)§\aftergroup\Ccn§' (matrix transpose). Argument '§\aftergroup\Ccb§nb_colsB§\aftergroup\Ccn§' may be omitted if it is 
          equal to §\aftergroup\Ccg§1§\aftergroup\Ccn§. 
       . Specifying a vector-valued math expression as an argument of a command that operates on 
          image values (e.g. '§\aftergroup\Ccb§fill§\aftergroup\Ccn§') modifies the whole spectrum range of the processed image(s), 
          for each spatial coordinates §\aftergroup\Ccg§(x,y,z)§\aftergroup\Ccn§. The command does not loop over the §\aftergroup\Ccg§C§\aftergroup\Ccn§-axis in this 
          case. 
 
    _ §\aftergroup\Ccg§String manipulation:§\aftergroup\Ccn§ Character strings are defined and managed as vectors objects. 
       Dedicated functions and initializers to manage strings are 
 
       . §\aftergroup\Ccb§[ 'string' ]§\aftergroup\Ccn§ and §\aftergroup\Ccb§'string'§\aftergroup\Ccn§ define a vector whose values are the ascii codes of the 
          specified §\aftergroup\Ccg§character string§\aftergroup\Ccn§ (e.g. §\aftergroup\Ccb§'foo'§\aftergroup\Ccn§ is equal to §\aftergroup\Ccg§[ 102,111,111 ]§\aftergroup\Ccn§). 
       . §\aftergroup\Ccb§_'character'§\aftergroup\Ccn§ returns the (scalar) ascii code of the specified character (e.g. §\aftergroup\Ccb§_'A'§\aftergroup\Ccn§ is 
          equal to §\aftergroup\Ccg§65§\aftergroup\Ccn§). 
       . A special case happens for §\aftergroup\Ccg§empty§\aftergroup\Ccn§ strings: Values of both expressions §\aftergroup\Ccb§[ '' ]§\aftergroup\Ccn§ and §\aftergroup\Ccb§''§\aftergroup\Ccn§ are §\aftergroup\Ccg§0§\aftergroup\Ccn§. 
       . Functions '§\aftergroup\Ccb§lowercase()§\aftergroup\Ccn§' and '§\aftergroup\Ccb§uppercase()§\aftergroup\Ccn§' return string with all string characters 
          lowercased or uppercased. 
       . Function '§\aftergroup\Ccb§stov(str,_starting_indice,_is_strict)§\aftergroup\Ccn§' parses specified string '§\aftergroup\Ccb§str§\aftergroup\Ccn§' and returns the value 
          contained in it. 
       . Function '§\aftergroup\Ccb§vtos(expr,_nb_digits,_siz)§\aftergroup\Ccn§' returns a vector of size '§\aftergroup\Ccb§siz§\aftergroup\Ccn§' which contains 
          the ascii representation of values described by expression '§\aftergroup\Ccb§expr§\aftergroup\Ccn§'. 
          '§\aftergroup\Ccb§nb_digits§\aftergroup\Ccn§' can be §\aftergroup\Ccg§{ -1=auto-reduced | 0=all | >0=max number of digits }§\aftergroup\Ccn§. 
       . Function '§\aftergroup\Ccb§echo(str1,str2,...,strN)§\aftergroup\Ccn§' prints the concatenation of given string arguments 
          on the console. 
       . Function '§\aftergroup\Ccb§cats(str1,str2,...,strN,siz)§\aftergroup\Ccn§' returns the concatenation of given string arguments 
          as a new vector of size '§\aftergroup\Ccb§siz§\aftergroup\Ccn§'. 
 
    _ §\aftergroup\Ccg§Special operators§\aftergroup\Ccn§ can be used: 
 
       . '§\aftergroup\Ccb§;§\aftergroup\Ccn§': expression separator. The returned value is always the last encountered expression. 
          For instance expression '§\aftergroup\Ccb§1;2;pi§\aftergroup\Ccn§' is evaluated as '§\aftergroup\Ccb§pi§\aftergroup\Ccn§'. 
       . '§\aftergroup\Ccb§=§\aftergroup\Ccn§': variable assignment. Variables in mathematical parser can only refer to numerical 
          values (vectors or scalars). Variable names are case-sensitive. Use this operator in 
          conjunction with '§\aftergroup\Ccb§;§\aftergroup\Ccn§' to define more complex evaluable expressions, such as 
          '§\aftergroup\Ccb§t=cos(x);3*t^2+2*t+1§\aftergroup\Ccn§'. 
          These variables remain §\aftergroup\Ccg§local§\aftergroup\Ccn§ to the mathematical parser and cannot be accessed outside 
           the evaluated expression. 
       . Variables defined in math parser may have a §\aftergroup\Ccg§constant§\aftergroup\Ccn§ property, by specifying keyword §\aftergroup\Ccb§const§\aftergroup\Ccn§ 
          before the variable name (e.g. §\aftergroup\Ccb§const foo = pi/4;§\aftergroup\Ccn§). The value set to such a variable must 
          be indeed a §\aftergroup\Ccb§constant scalar§\aftergroup\Ccn§. Constant variables allows certain types of optimizations in 
          the math JIT compiler. 
 
    _ The following §\aftergroup\Ccg§specific functions§\aftergroup\Ccn§ are also defined: 
 
       . '§\aftergroup\Ccb§u(max)§\aftergroup\Ccn§' or '§\aftergroup\Ccb§u(min,max)§\aftergroup\Ccn§': return a random value between §\aftergroup\Ccg§[0,max]§\aftergroup\Ccn§ or §\aftergroup\Ccg§[min,max]§\aftergroup\Ccn§, following 
          a uniform distribution. 
       . '§\aftergroup\Ccb§i(_a,_b,_c,_d,_interpolation_type,_boundary_conditions)§\aftergroup\Ccn§': return the value of the pixel 
          located at position §\aftergroup\Ccg§(a,b,c,d)§\aftergroup\Ccn§ in the associated image, if any (§\aftergroup\Ccg§0§\aftergroup\Ccn§ otherwise). 
          '§\aftergroup\Ccg§interpolation_type§\aftergroup\Ccn§' can be §\aftergroup\Ccg§{ 0=nearest neighbor | other=linear }§\aftergroup\Ccn§. 
          '§\aftergroup\Ccg§boundary_conditions§\aftergroup\Ccn§' can be §\aftergroup\Ccg§{ 0=dirichlet | 1=neumann | 2=periodic | 3=mirror }§\aftergroup\Ccn§. 
          Omitted coordinates are replaced by their default values which are respectively 
          §\aftergroup\Ccb§x, y, z, c, interpolation§\aftergroup\Ccn§ and §\aftergroup\Ccb§boundary§\aftergroup\Ccn§. 
          For instance command '§\aftergroup\Ccb§fill 0.5*(i(x+1)-i(x-1))§\aftergroup\Ccn§' will estimate the X-derivative of an 
          image with a classical finite difference scheme. 
       . '§\aftergroup\Ccb§j(_dx,_dy,_dz,_dc,_interpolation_type,_boundary_conditions)§\aftergroup\Ccn§' does the same for the pixel 
          located at position §\aftergroup\Ccg§(x+dx,y+dy,z+dz,c+dc)§\aftergroup\Ccn§ (pixel access relative to the current 
          coordinates). 
       . '§\aftergroup\Ccb§i[offset,_boundary_conditions]§\aftergroup\Ccn§' returns the value of the pixel located at specified 
          '§\aftergroup\Ccg§offset§\aftergroup\Ccn§' in the associated image buffer (or §\aftergroup\Ccg§0§\aftergroup\Ccn§ if offset is out-of-bounds). 
       . '§\aftergroup\Ccb§j[offset,_boundary_conditions]§\aftergroup\Ccn§' does the same for an offset relative to the current pixel 
          coordinates §\aftergroup\Ccg§(x,y,z,c)§\aftergroup\Ccn§. 
       . '§\aftergroup\Ccb§i(#ind,_x,_y,_z,_c,_interpolation,_boundary_conditions)§\aftergroup\Ccn§', 
          '§\aftergroup\Ccb§j(#ind,_dx,_dy,_dz,_dc,_interpolation,_boundary_conditions)§\aftergroup\Ccn§', 
          '§\aftergroup\Ccb§i[#ind,offset,_boundary_conditions]§\aftergroup\Ccn§' and '§\aftergroup\Ccb§i[offset,_boundary_conditions]§\aftergroup\Ccn§' are similar expressions used to 
          access pixel values for any numbered image §\aftergroup\Ccg§[ind]§\aftergroup\Ccn§ of the list. 
       . '§\aftergroup\Ccb§I/J[offset,_boundary_conditions]§\aftergroup\Ccn§' and '§\aftergroup\Ccb§I/J(#ind,_x,_y,_z,_interpolation,_boundary_conditions)§\aftergroup\Ccn§' do 
          the same as '§\aftergroup\Ccb§i/j[offset,_boundary_conditions]§\aftergroup\Ccn§' and 
          '§\aftergroup\Ccb§i/j(#ind,_x,_y,_z,_c,_interpolation,_boundary_conditions)§\aftergroup\Ccn§' but return a vector instead of a scalar 
          (e.g. a vector §\aftergroup\Ccg§[ R,G,B ]§\aftergroup\Ccn§ for a pixel at §\aftergroup\Ccg§(a,b,c)§\aftergroup\Ccn§ in a color image). 
       . '§\aftergroup\Ccb§sort(#ind,_is_increasing,_axis)§\aftergroup\Ccn§' sorts the values in the specified image §\aftergroup\Ccg§[ind]§\aftergroup\Ccn§. 
       . '§\aftergroup\Ccb§crop(_#ind,_x,_y,_z,_c,_dx,_dy,_dz,_dc,_boundary_conditions)§\aftergroup\Ccn§' returns a vector whose values come 
          from the cropped region of image §\aftergroup\Ccg§[ind]§\aftergroup\Ccn§ (or from default image selected if '§\aftergroup\Ccb§ind§\aftergroup\Ccn§' is not 
          specified). Cropped region starts from point §\aftergroup\Ccg§(x,y,z,c)§\aftergroup\Ccn§ and has a size of 
          §\aftergroup\Ccg§dx x dy x dz x dc§\aftergroup\Ccn§. Arguments for coordinates and sizes can be omitted if they are not 
          ambiguous (e.g. '§\aftergroup\Ccb§crop(#ind,x,y,dx,dy)§\aftergroup\Ccn§' is a valid invokation of this function). 
       . '§\aftergroup\Ccb§draw(_#ind,S,x,y,z,c,dx,_dy,_dz,_dc,_opacity,_M,_max_M)§\aftergroup\Ccn§' draws a sprite §\aftergroup\Ccg§S§\aftergroup\Ccn§ in image §\aftergroup\Ccg§[ind]§\aftergroup\Ccn§ 
          (or in default image selected if '§\aftergroup\Ccb§ind§\aftergroup\Ccn§' is not specified) at coordinates §\aftergroup\Ccg§(x,y,z,c)§\aftergroup\Ccn§. 
          The size of the sprite §\aftergroup\Ccg§dx x dy x dz x dc§\aftergroup\Ccn§ must be specified. You can also specify a 
          corresponding opacity mask §\aftergroup\Ccg§M§\aftergroup\Ccn§ if its size matches §\aftergroup\Ccg§S§\aftergroup\Ccn§. 
       . '§\aftergroup\Ccb§resize(#ind,w,_h,_d,_s,_interp,_boundary_conditions,cx,_cy,_cz,_cc)§\aftergroup\Ccn§' resizes an image of the 
          associated list with specified dimension and interpolation method. When using this, 
          function, you should consider retrieving the (non-constant) image dimensions using the 
          dynamic functions '§\aftergroup\Ccb§w(_#ind)§\aftergroup\Ccn§', '§\aftergroup\Ccb§h(_#ind)§\aftergroup\Ccn§', '§\aftergroup\Ccb§d(_#ind)§\aftergroup\Ccn§', '§\aftergroup\Ccb§s(_#ind)§\aftergroup\Ccn§', '§\aftergroup\Ccb§wh(_#ind)§\aftergroup\Ccn§', 
          '§\aftergroup\Ccb§whd(_#ind)§\aftergroup\Ccn§' and '§\aftergroup\Ccb§whds(_#ind)§\aftergroup\Ccn§' instead of the corresponding constant variables. 
       . '§\aftergroup\Ccb§if(condition,expr_then,_expr_else)§\aftergroup\Ccn§': return value of '§\aftergroup\Ccb§expr_then§\aftergroup\Ccn§' or '§\aftergroup\Ccb§expr_else§\aftergroup\Ccn§', 
          depending on the value of '§\aftergroup\Ccb§condition§\aftergroup\Ccn§' §\aftergroup\Ccg§(0=false, other=true)§\aftergroup\Ccn§. '§\aftergroup\Ccb§expr_else§\aftergroup\Ccn§' can be omitted 
          in which case §\aftergroup\Ccg§0§\aftergroup\Ccn§ is returned if the condition does not hold. Using the ternary operator 
          '§\aftergroup\Ccb§condition?expr_then[:expr_else]§\aftergroup\Ccn§' gives an equivalent expression. 
          For instance, §\aftergroup\Ccg§G'MIC§\aftergroup\Ccn§ commands '§\aftergroup\Ccb§fill if(x%10==0,255,i)§\aftergroup\Ccn§' and '§\aftergroup\Ccb§fill x%10?i:255§\aftergroup\Ccn§' both draw 
          blank vertical lines on every 10th column of an image. 
       . '§\aftergroup\Ccb§dowhile(expression,_condition)§\aftergroup\Ccn§' repeats the evaluation of '§\aftergroup\Ccb§expression§\aftergroup\Ccn§' until '§\aftergroup\Ccb§condition§\aftergroup\Ccn§' 
          vanishes (or until '§\aftergroup\Ccb§expression§\aftergroup\Ccn§' vanishes if no '§\aftergroup\Ccb§condition§\aftergroup\Ccn§' is specified). For instance, 
          the expression: '§\aftergroup\Ccb§if(N<2,N,n=N-1;F0=0;F1=1;dowhile(F2=F0+F1;F0=F1;F1=F2,n=n-1))§\aftergroup\Ccn§' returns 
          the Nth value of the Fibonacci sequence, for §\aftergroup\Ccg§N>=0§\aftergroup\Ccn§ (e.g., §\aftergroup\Ccg§46368§\aftergroup\Ccn§ for §\aftergroup\Ccg§N=24§\aftergroup\Ccn§). 
          '§\aftergroup\Ccb§dowhile(expression,condition)§\aftergroup\Ccn§' always evaluates the specified expression at least once, 
          then check for the loop condition. When done, it returns the last value of '§\aftergroup\Ccb§expression§\aftergroup\Ccn§'. 
       . '§\aftergroup\Ccb§for(init,condition,_procedure,body)§\aftergroup\Ccn§' first evaluates the expression '§\aftergroup\Ccb§init§\aftergroup\Ccn§', then 
          iteratively evaluates '§\aftergroup\Ccb§body§\aftergroup\Ccn§' (followed by '§\aftergroup\Ccb§procedure§\aftergroup\Ccn§' if specified) while '§\aftergroup\Ccb§condition§\aftergroup\Ccn§' 
          is verified (i.e. not zero). It may happen that no iteration is done, in which case the 
          function returns §\aftergroup\Ccg§nan§\aftergroup\Ccn§. Otherwise, it returns the last value of '§\aftergroup\Ccb§body§\aftergroup\Ccn§'. 
          For instance, the expression: '§\aftergroup\Ccb§if(N<2,N,for(n=N;F0=0;F1=1,n=n-1,F2=F0+F1;F0=F1;F1=F2))§\aftergroup\Ccn§' 
          returns the §\aftergroup\Ccg§Nth§\aftergroup\Ccn§ value of the Fibonacci sequence, for §\aftergroup\Ccg§N>=0§\aftergroup\Ccn§ (e.g., §\aftergroup\Ccg§46368§\aftergroup\Ccn§ for §\aftergroup\Ccg§N=24§\aftergroup\Ccn§). 
       . '§\aftergroup\Ccb§whiledo(condition,expression)§\aftergroup\Ccn§' is exactly the same as '§\aftergroup\Ccb§for(init,condition,expression)§\aftergroup\Ccn§' 
          without the specification of an initializing expression. 
       . '§\aftergroup\Ccb§break()§\aftergroup\Ccn§' and '§\aftergroup\Ccb§continue()§\aftergroup\Ccn§' respectively breaks and continues the current running bloc 
          (loop, init or main environment). 
       . '§\aftergroup\Ccb§date(attr,path)§\aftergroup\Ccn§' returns the date attribute for the given 'path' (file or directory), 
          with '§\aftergroup\Ccg§attr§\aftergroup\Ccn§' being §\aftergroup\Ccg§{ 0=year | 1=month | 2=day | 3=day of week | 4=hour | 5=minute | 
          6=second }§\aftergroup\Ccn§. 
       . '§\aftergroup\Ccb§date(_attr)§\aftergroup\Ccn§ returns the specified attribute for the current (locale) date. 
       . '§\aftergroup\Ccb§print(expr1,expr2,...)§\aftergroup\Ccn§ or '§\aftergroup\Ccb§print(#ind)§\aftergroup\Ccn§ prints the value of the specified expressions 
          (or image information) on the console, and returns the value of the last expression 
          (or §\aftergroup\Ccg§nan§\aftergroup\Ccn§ in case of an image). Function '§\aftergroup\Ccb§prints(expr)§\aftergroup\Ccn§' also prints the string composed 
          of the ascii characters defined by the vector-valued expression (e.g. '§\aftergroup\Ccb§prints('Hello')§\aftergroup\Ccn§'). 
       . '§\aftergroup\Ccb§debug(expression)§\aftergroup\Ccn§ prints detailed debug information about the sequence of operations done 
          by the math parser to evaluate the expression (and returns its value). 
       . '§\aftergroup\Ccb§display(_X,_w,_h,_d,_s)§\aftergroup\Ccn§ or '§\aftergroup\Ccb§display(#ind)§\aftergroup\Ccn§ display the contents of the vector '§\aftergroup\Ccb§X§\aftergroup\Ccn§' 
          (or specified image) and wait for user events. if no arguments are provided, a memory 
          snapshot of the math parser environment is displayed instead. 
       . '§\aftergroup\Ccb§init(expression)§\aftergroup\Ccn§ and '§\aftergroup\Ccb§end(expression)§\aftergroup\Ccn§ evaluates the specified expressions only once, 
          respectively at the beginning and end of the evaluation procedure, and this, 
          even when multiple evaluations are required (e.g. in '§\aftergroup\Ccb§fill init(foo=0);++foo§\aftergroup\Ccn§'). 
       . '§\aftergroup\Ccb§copy(dest,src,_nb_elts,_inc_d,_inc_s,_opacity)§\aftergroup\Ccn§ copies an entire memory block of '§\aftergroup\Ccb§nb_elts§\aftergroup\Ccn§' 
          elements starting from a source value '§\aftergroup\Ccb§src§\aftergroup\Ccn§' to a specified destination '§\aftergroup\Ccb§dest§\aftergroup\Ccn§', with 
          increments defined by '§\aftergroup\Ccb§inc_d§\aftergroup\Ccn§' and '§\aftergroup\Ccb§inc_s§\aftergroup\Ccn§' respectively for the destination and source 
          pointers. 
       . '§\aftergroup\Ccb§unref(a,b,...)§\aftergroup\Ccn§ destroys references to the named variable given as arguments. 
       . '§\aftergroup\Ccb§breakpoint()§\aftergroup\Ccn§ inserts a possible computation breakpoint (not supported by the cli interface). 
       . '§\aftergroup\Ccb§_(expr)§\aftergroup\Ccn§ just ignores its arguments (mainly useful for debugging). 
       . '§\aftergroup\Ccb§ext('pipeline')§\aftergroup\Ccn§ executes the specified §\aftergroup\Ccg§G'MIC§\aftergroup\Ccn§ pipeline as if it was called outside 
          the currently evaluated expression. 
 
    - §\aftergroup\Ccg§User-defined macros:§\aftergroup\Ccn§ 
 
       . Custom macro functions can be defined in a math expression, using the assignment operator 
          '§\aftergroup\Ccb§=§\aftergroup\Ccn§', e.g. '§\aftergroup\Ccb§foo(x,y) = cos(x + y); result = foo(1,2) + foo(2,3)§\aftergroup\Ccn§'. 
       . Trying to override a built-in function (e.g. '§\aftergroup\Ccb§abs()§\aftergroup\Ccn§') has no effect. 
       . Overloading macros with different number of arguments is possible. Re-defining a 
          previously defined macro with the same number of arguments discards its previous 
          definition. 
       . Macro functions are indeed processed as §\aftergroup\Ccg§macros§\aftergroup\Ccn§ by the mathematical evaluator. You should 
          avoid invoking them with arguments that are themselves results of assignments or 
          self-operations. For instance, '§\aftergroup\Ccb§foo(x) = x + x; z = 0; foo(++z)§\aftergroup\Ccn§' returns '§\aftergroup\Ccg§4§\aftergroup\Ccn§' rather 
          than expected value '§\aftergroup\Ccg§2§\aftergroup\Ccn§'. 
       . When substituted, macro arguments are placed inside parentheses, except if a number sign 
          '§\aftergroup\Ccb§#§\aftergroup\Ccn§' is located just before or after the argument name. For instance, expression 
          '§\aftergroup\Ccb§foo(x,y) = x*y; foo(1+2,3)§\aftergroup\Ccn§' returns '§\aftergroup\Ccg§9§\aftergroup\Ccn§' (being substituted as '§\aftergroup\Ccb§(1+2)*(3)§\aftergroup\Ccn§'), while 
          expression '§\aftergroup\Ccb§foo(x,y) = x#*y#; foo(1+2,3)§\aftergroup\Ccn§' returns '§\aftergroup\Ccg§7§\aftergroup\Ccn§' (being substituted as '§\aftergroup\Ccb§1+2*3§\aftergroup\Ccn§'). 
       . Number signs appearing between macro arguments function actually count for '§\aftergroup\Ccb§empty§\aftergroup\Ccn§' 
          separators. They may be used to force the substitution of macro arguments in unusual 
          places, e.g. as in '§\aftergroup\Ccb§str(N) = ['I like N#'];§\aftergroup\Ccn§'. 
 
    - §\aftergroup\Ccg§Multi-threaded§\aftergroup\Ccn§ and §\aftergroup\Ccg§in-place§\aftergroup\Ccn§ evaluation: 
 
       . If your image data are large enough and you have several CPUs available, it is likely that 
          the math expression passed to a '§\aftergroup\Ccb§fill§\aftergroup\Ccn§' or '§\aftergroup\Ccb§input§\aftergroup\Ccn§' command is evaluated in parallel, 
          using multiple computation threads. 
       . Starting an expression with '§\aftergroup\Ccb§:§\aftergroup\Ccn§' or '§\aftergroup\Ccb§*§\aftergroup\Ccn§' forces the evaluations required for an image to be 
          run in parallel, even if the amount of data to process is small (beware, it may be slower 
          to evaluate in this case!). Specify '§\aftergroup\Ccb§:§\aftergroup\Ccn§' (instead of '§\aftergroup\Ccb§*§\aftergroup\Ccn§') to avoid possible image copy 
          done before evaluating the expression (this saves memory, but do this only if you are 
          sure this step is not required!) 
       . If the specified expression starts with '§\aftergroup\Ccb§>§\aftergroup\Ccn§' or '§\aftergroup\Ccb§<§\aftergroup\Ccn§', the pixel access operators 
          '§\aftergroup\Ccb§i(), i[], j()§\aftergroup\Ccn§' and '§\aftergroup\Ccb§j[]§\aftergroup\Ccn§' return values of the image being currently modified, 
          in forward ('§\aftergroup\Ccb§>§\aftergroup\Ccn§') or backward ('§\aftergroup\Ccb§<§\aftergroup\Ccn§') order. The multi-threading evaluation of the 
          expression is also disabled in this case. 
       . Function '§\aftergroup\Ccb§critical(operands)§\aftergroup\Ccn§' forces the execution of the given operands in a single thread at a 
          time. 
 
    _ Expressions '§\aftergroup\Ccb§i(_#ind,x,_y,_z,_c)=value§\aftergroup\Ccn§', '§\aftergroup\Ccb§j(_#ind,x,_y,_z,_c)=value§\aftergroup\Ccn§', '§\aftergroup\Ccb§i[_#ind,offset]=value§\aftergroup\Ccn§' 
       and '§\aftergroup\Ccb§j[_#ind,offset]=value§\aftergroup\Ccn§' set a pixel value at a different location than the running one 
       in the image §\aftergroup\Ccg§[ind]§\aftergroup\Ccn§ (or in the associated image if argument '§\aftergroup\Ccb§#ind§\aftergroup\Ccn§' is omitted), either with 
       global coordinates/offsets (with '§\aftergroup\Ccb§i(...)§\aftergroup\Ccn§' and '§\aftergroup\Ccb§i[...]§\aftergroup\Ccn§'), or relatively to the current 
       position §\aftergroup\Ccg§(x,y,z,c)§\aftergroup\Ccn§ (with '§\aftergroup\Ccb§j(...)§\aftergroup\Ccn§' and '§\aftergroup\Ccb§j[...]§\aftergroup\Ccn§'). These expressions always return '§\aftergroup\Ccb§value§\aftergroup\Ccn§'. 
 
  - The last image of the list is always associated to the evaluations of '§\aftergroup\Ccb§{expressions}§\aftergroup\Ccn§', 
     e.g. §\aftergroup\Ccg§G'MIC§\aftergroup\Ccn§ sequence '§\aftergroup\Ccb§256,128 fill {w}§\aftergroup\Ccn§' will create a 256x128 image filled with value 256.
\end{lstlisting}
\normalsize
~\\\section{Image and data viewers}
\small
\begin{lstlisting}[escapechar=§]
  - §\aftergroup\Ccg§G'MIC§\aftergroup\Ccn§ has some very handy embedded §\aftergroup\Ccg§visualization modules§\aftergroup\Ccn§, for 1d signals (command '§\aftergroup\Ccb§plot§\aftergroup\Ccn§'), 
     1d/2d/3d images (command '§\aftergroup\Ccb§display§\aftergroup\Ccn§') and 3d objects (command '§\aftergroup\Ccb§display3d§\aftergroup\Ccn§'). It manages 
     interactive views of the selected image data. 
 
  - The following keyboard shortcuts are available in the interactive viewers: 
 
    . §\aftergroup\Ccg§(mousewheel)§\aftergroup\Ccn§: Zoom in/out. 
    . §\aftergroup\Ccg§CTRL+D§\aftergroup\Ccn§: Increase window size. 
    . §\aftergroup\Ccg§CTRL+C§\aftergroup\Ccn§: Decrease window size. 
    . §\aftergroup\Ccg§CTRL+R§\aftergroup\Ccn§: Reset window size. 
    . §\aftergroup\Ccg§CTRL+W§\aftergroup\Ccn§: Close window. 
    . §\aftergroup\Ccg§CTRL+F§\aftergroup\Ccn§: Toggle fullscreen mode. 
    . §\aftergroup\Ccg§CTRL+S§\aftergroup\Ccn§: Save current window snapshot as numbered file 'gmic_xxxx.bmp'. 
    . §\aftergroup\Ccg§CTRL+O§\aftergroup\Ccn§: Save current instance of the viewed data, as numbered file 'gmic_xxxx.cimgz'. 
 
  - Shortcuts specific to the 1d/2d/3d image viewer (command '§\aftergroup\Ccb§display§\aftergroup\Ccn§') are: 
 
    . §\aftergroup\Ccg§CTRL+A§\aftergroup\Ccn§: Switch cursor mode. 
    . §\aftergroup\Ccg§CTRL+P§\aftergroup\Ccn§: Play z-stack of frames as a movie (for volumetric 3d images). 
    . §\aftergroup\Ccg§CTRL+V§\aftergroup\Ccn§: Show/hide 3D view (for volumetric 3d images). 
    . §\aftergroup\Ccg§CTRL+(mousewheel)§\aftergroup\Ccn§: Go up/down. 
    . §\aftergroup\Ccg§SHIFT+(mousewheel)§\aftergroup\Ccn§: Go left/right. 
    . §\aftergroup\Ccg§Numeric PAD§\aftergroup\Ccn§: Zoom in/out (+/-) and move through zoomed image (digits). 
    . §\aftergroup\Ccg§BACKSPACE§\aftergroup\Ccn§: Reset zoom scale. 
 
  - Shortcuts specific to the 3d object viewer (command '§\aftergroup\Ccb§display3d§\aftergroup\Ccn§') are: 
 
    . §\aftergroup\Ccg§(mouse)+(left mouse button)§\aftergroup\Ccn§: Rotate 3d object. 
    . §\aftergroup\Ccg§(mouse)+(right mouse button)§\aftergroup\Ccn§: Zoom 3d object. 
    . §\aftergroup\Ccg§(mouse)+(middle mouse button)§\aftergroup\Ccn§: Shift 3d object. 
    . §\aftergroup\Ccg§CTRL+F1 ... CTRL+F6§\aftergroup\Ccn§: Toggle between different 3d rendering modes. 
    . §\aftergroup\Ccg§CTRL+Z§\aftergroup\Ccn§: Enable/disable z-buffered rendering. 
    . §\aftergroup\Ccg§CTRL+A§\aftergroup\Ccn§: Show/hide 3d axes. 
    . §\aftergroup\Ccg§CTRL+G§\aftergroup\Ccn§: Save 3d object, as numbered file 'gmic_xxxx.off'. 
    . §\aftergroup\Ccg§CTRL+T§\aftergroup\Ccn§: Switch between single/double-sided 3d modes.
\end{lstlisting}
\normalsize
~\\\section{Adding custom commands}
\small
\begin{lstlisting}[escapechar=§]
  - New custom commands can be added by the user, through the use of §\aftergroup\Ccg§G'MIC§\aftergroup\Ccn§ §\aftergroup\Ccg§custom commands files§\aftergroup\Ccn§. 
 
  - A command file is a simple ascii text file, where each line starts either by 
     '§\aftergroup\Ccb§command_name: command_definition§\aftergroup\Ccn§' or '§\aftergroup\Ccb§command_definition (continuation)§\aftergroup\Ccn§'. 
 
  - At startup, §\aftergroup\Ccg§G'MIC§\aftergroup\Ccn§ automatically includes user's command file §\aftergroup\Ccg§$HOME/.gmic§\aftergroup\Ccn§ (on Unix) or 
     §\aftergroup\Ccg§%APPDATA%/user.gmic§\aftergroup\Ccn§ (on Windows). The CLI tool '§\aftergroup\Ccg§gmic§\aftergroup\Ccn§' automatically runs the command 
     '§\aftergroup\Ccb§cli_start§\aftergroup\Ccn§' if defined. 
 
  - Custom command names must use character set §\aftergroup\Ccg§[a-zA-Z0-9_]§\aftergroup\Ccn§ and cannot start with a number. 
 
  - Any '§\aftergroup\Ccb§ # comment§\aftergroup\Ccn§' expression found in a custom commands file is discarded by the §\aftergroup\Ccg§G'MIC§\aftergroup\Ccn§ parser, 
     wherever it is located in a line. 
 
  - In a custom command, the following §\aftergroup\Ccg§$-expressions§\aftergroup\Ccn§ are recognized and substituted: 
 
    . '§\aftergroup\Ccb§$*§\aftergroup\Ccn§' is substituted by a copy of the specified string of arguments. 
    . '§\aftergroup\Ccb§$"*"§\aftergroup\Ccn§' is substituted by a copy of the specified string of arguments, each being 
       double-quoted. 
    . '§\aftergroup\Ccb§$#§\aftergroup\Ccn§' is substituted by the maximum indice of known arguments (either specified by the user 
       or set to a default value in the custom command). 
    . '§\aftergroup\Ccb§§\aftergroup\Ccn§' is substituted by the list of selected image indices that have been specified during the 
       command invokation. 
    . '§\aftergroup\Ccb§$?§\aftergroup\Ccn§' is substituted by a printable version of '§\aftergroup\Ccb§$[]§\aftergroup\Ccn§' to be used in command descriptions. 
    . '§\aftergroup\Ccb§$i§\aftergroup\Ccn§' and '§\aftergroup\Ccb§${i}§\aftergroup\Ccn§' are both substituted by the §\aftergroup\Ccg§i^th§\aftergroup\Ccn§ specified argument. Negative indices such as 
       '§\aftergroup\Ccb§${-j}§\aftergroup\Ccn§' are allowed and refer to the §\aftergroup\Ccg§j^th§\aftergroup\Ccn§ latest argument. '§\aftergroup\Ccb§$0§\aftergroup\Ccn§' is substituted by the 
       custom command name. 
    . '§\aftergroup\Ccb§${i=default}§\aftergroup\Ccn§' is substituted by the value of §\aftergroup\Ccb§$i§\aftergroup\Ccn§ (if defined) or by its new value set to 
        '§\aftergroup\Ccg§default§\aftergroup\Ccn§' otherwise ('§\aftergroup\Ccg§default§\aftergroup\Ccn§' may be a $-expression as well). 
    . '§\aftergroup\Ccb§${subset}§\aftergroup\Ccn§' is substituted by the argument values (separated by commas '§\aftergroup\Ccg§,§\aftergroup\Ccn§') of a specified 
       argument subset. For instance expression '§\aftergroup\Ccb§${2--2}§\aftergroup\Ccn§' is substitued by all specified command 
       arguments except the first and the last one. Expression '§\aftergroup\Ccb§${^0}§\aftergroup\Ccn§' is then substituted by all 
       arguments of the invoked command (eq. to '§\aftergroup\Ccb§$*§\aftergroup\Ccn§' if all specified arguments have indeed a 
       value). 
    . '§\aftergroup\Ccb§$=var§\aftergroup\Ccn§' is substituted by the set of instructions that will assign each argument §\aftergroup\Ccb§$i§\aftergroup\Ccn§ to the 
       named variable '§\aftergroup\Ccb§var$i§\aftergroup\Ccn§' (for i in §\aftergroup\Ccg§[0...$#]§\aftergroup\Ccn§). This is particularly useful when a custom 
       command want to manage variable numbers of arguments. Variables names must use character set 
       §\aftergroup\Ccg§[a-zA-Z0-9_]§\aftergroup\Ccn§ and cannot start with a number. 
 
  - These particular §\aftergroup\Ccg§$-expressions§\aftergroup\Ccn§ for custom commands are §\aftergroup\Ccg§always substituted§\aftergroup\Ccn§, even in 
     double-quoted items or when the dollar sign '§\aftergroup\Ccg§$§\aftergroup\Ccn§' is escaped with a backslash '§\aftergroup\Ccg§\§\aftergroup\Ccn§'. To avoid 
     substitution, place an empty double quoted string just after the '§\aftergroup\Ccg§$§\aftergroup\Ccn§' (as in '§\aftergroup\Ccb§$""1§\aftergroup\Ccn§'). 
 
  - Specifying arguments may be skipped when invoking a custom command, by replacing them by commas 
     '§\aftergroup\Ccg§,§\aftergroup\Ccn§' as in expression '§\aftergroup\Ccb§flower ,,3§\aftergroup\Ccn§'. Omitted arguments are set to their default values, which 
     must be thus explicitly defined in the code of the corresponding custom command (using 
     default argument expressions as '§\aftergroup\Ccb§${1=default}§\aftergroup\Ccn§'). 
 
  - If one numbered argument required by a custom command misses a value, an error is thrown by the 
     §\aftergroup\Ccg§G'MIC§\aftergroup\Ccn§ interpreter.
\end{lstlisting}
\normalsize
~\\\section{List of commands}
\small
\begin{lstlisting}[escapechar=§]
   All available §\aftergroup\Ccg§G'MIC§\aftergroup\Ccn§ commands are listed below, classified by themes. When several choices of 
   command arguments are possible, they appear separated by '§\aftergroup\Ccg§|§\aftergroup\Ccn§'. An argument specified inside '§\aftergroup\Ccg§[]§\aftergroup\Ccn§' 
   or starting by '§\aftergroup\Ccg§_§\aftergroup\Ccn§' is optional except when standing for an existing image §\aftergroup\Ccb§[image]§\aftergroup\Ccn§, where '§\aftergroup\Ccg§image§\aftergroup\Ccn§' 
   can be either an indice number or an image name. In this case, the '§\aftergroup\Ccg§[]§\aftergroup\Ccn§' characters are mandatory 
   when writing the item. A command marked with '§\aftergroup\Ccg§(+)§\aftergroup\Ccn§' is one of the §\aftergroup\Ccg§native§\aftergroup\Ccn§ commands. Note also that 
   all images that serve as illustrations in this reference documentation are normalized in §\aftergroup\Ccg§[0,255]§\aftergroup\Ccn§ 
   before being displayed. You may need to do this explicitly (command '§\aftergroup\Ccb§normalize 0,255§\aftergroup\Ccn§') if you 
   want to save and view images with the same aspect than those illustrated in the example codes.
\end{lstlisting}
\normalsize

\chapter{List of commands}

\section{Global options}


\subsection{\emph{debug\index{debug}} (+)}\vspace*{-0.7em}
Activate debug mode.
~\\When activated{\comma} the G'MIC interpreter becomes very verbose and outputs additionnal log
messages about its internal state on the standard output (stdout).
~\\This option is useful for developers or to report possible bugs of the interpreter.


\subsection{\emph{help\index{help}} }\vspace*{-0.7em}
~\\\textbf{\Cb{Arguments: }}\begin{flushleft}
{\small \Cb{\hspace*{0.5cm}$\bullet$~~\texttt{command}}}~~~\\
{\small \Cb{\hspace*{0.5cm}$\bullet$~~\texttt{(no arg)}}}\end{flushleft}
Display help (optionally for specified command only) and exit.
~\\(\emph{eq. to} {\small \texttt{'h'}}).


\subsection{\emph{version\index{version}} }\vspace*{-0.7em}
Display current version number on stdout.

\section{Input/output}


\subsection{\emph{camera\index{camera}} (+)}\vspace*{-0.7em}
~\\\textbf{\Cb{Arguments: }}\begin{flushleft}
{\small \Cb{\hspace*{0.5cm}$\bullet$~~\texttt{\_camera\_index$>$=0{\comma}\_nb\_frames$>$0{\comma}\_skip\_frames$>$=0{\comma}\_capture\_width\-$>$=0{\comma}\_capture\_height$>$=0}}}\end{flushleft}
Insert one or several frames from specified camera.
~\\When 'nb\_frames==0'{\comma} the camera stream is released instead of capturing new images.
\begin{flushleft}\Cc{\textbf{Default values}:\\~\\\hspace*{0.5cm}{\small $\bullet$~~\texttt{'camera\_index=0' (default camera){\comma} 'nb\_frames=1'{\comma} 'skip\_frames=0'} and \texttt{'capture\_width=capture\_height=0' (default size).}}}\end{flushleft}


\subsection{\emph{clut\index{clut}} }\vspace*{-0.7em}
~\\\textbf{\Cb{Arguments: }}\begin{flushleft}
{\small \Cb{\hspace*{0.5cm}$\bullet$~~\texttt{"clut\_name"{\comma}\_resolution$>$0}}}\end{flushleft}
Insert one of the pre-defined CLUTs at the end of the image list.\textbackslash n
~\\'clut\_name' can be \{ 60's ~$|$~ 60's\_faded ~$|$~ 60's\_faded\_alt ~$|$~ agfa\_apx\_100 ~$|$~ agfa\_apx\_25 ~$|$~ agfa\_precisa\_100 ~$|$~ agfa\_ultra\_color\_100 ~$|$~ agfa\_vista\_200 ~$|$~ alien\_green ~$|$~ analogfx\_anno\_1870\_color ~$|$~ analogfx\_old\_style\_i ~$|$~ analogfx\_old\_style\_ii ~$|$~ analogfx\_old\_style\_iii ~$|$~ analogfx\_sepia\_color ~$|$~ analogfx\_soft\_sepia\_i ~$|$~ analogfx\_soft\_sepia\_ii ~$|$~ black\_and\_white ~$|$~ bleach\_bypass ~$|$~ blue\_mono ~$|$~ color\_rich ~$|$~ expired\_fade ~$|$~ expired\_polaroid ~$|$~ extreme ~$|$~ fade ~$|$~ faded ~$|$~ faded\_alt ~$|$~ faded\_analog ~$|$~ faded\_extreme ~$|$~ faded\_vivid ~$|$~ faux\_infrared ~$|$~ fuji3510\_constlclip ~$|$~ fuji3510\_constlmap ~$|$~ fuji3510\_cuspclip ~$|$~ fuji3513\_constlclip ~$|$~ fuji3513\_constlmap ~$|$~ fuji3513\_cuspclip ~$|$~ fuji\_160c ~$|$~ fuji\_160c\_+ ~$|$~ fuji\_160c\_++ ~$|$~ fuji\_160c\_- ~$|$~ fuji\_400h ~$|$~ fuji\_400h\_+ ~$|$~ fuji\_400h\_++ ~$|$~ fuji\_400h\_- ~$|$~ fuji\_800z ~$|$~ fuji\_800z\_+ ~$|$~ fuji\_800z\_++ ~$|$~ fuji\_800z\_- ~$|$~ fuji\_astia\_100f ~$|$~ fuji\_fp-100c ~$|$~ fuji\_fp-100c\_+ ~$|$~ fuji\_fp-100c\_++ ~$|$~ fuji\_fp-100c\_+++ ~$|$~ fuji\_fp-100c\_++\_alt ~$|$~ fuji\_fp-100c\_- ~$|$~ fuji\_fp-100c\_-- ~$|$~ fuji\_fp-100c\_cool ~$|$~ fuji\_fp-100c\_cool\_+ ~$|$~ fuji\_fp-100c\_cool\_++ ~$|$~ fuji\_fp-100c\_cool\_- ~$|$~ fuji\_fp-100c\_cool\_-- ~$|$~ fuji\_fp-100c\_negative ~$|$~ fuji\_fp-100c\_negative\_+ ~$|$~ fuji\_fp-100c\_negative\_++ ~$|$~ fuji\_fp-100c\_negative\_+++ ~$|$~ fuji\_fp-100c\_negative\_++\_alt ~$|$~ fuji\_fp-100c\_negative\_- ~$|$~ fuji\_fp-100c\_negative\_-- ~$|$~ fuji\_fp-3000b ~$|$~ fuji\_fp-3000b\_+ ~$|$~ fuji\_fp-3000b\_++ ~$|$~ fuji\_fp-3000b\_+++ ~$|$~ fuji\_fp-3000b\_- ~$|$~ fuji\_fp-3000b\_-- ~$|$~ fuji\_fp-3000b\_hc ~$|$~ fuji\_fp-3000b\_negative ~$|$~ fuji\_fp-3000b\_negative\_+ ~$|$~ fuji\_fp-3000b\_negative\_++ ~$|$~ fuji\_fp-3000b\_negative\_+++ ~$|$~ fuji\_fp-3000b\_negative\_- ~$|$~ fuji\_fp-3000b\_negative\_-- ~$|$~ fuji\_fp-3000b\_negative\_early ~$|$~ fuji\_fp\_100c ~$|$~ fuji\_ilford\_delta\_3200 ~$|$~ fuji\_ilford\_delta\_3200\_+ ~$|$~ fuji\_ilford\_delta\_3200\_++ ~$|$~ fuji\_ilford\_delta\_3200\_- ~$|$~ fuji\_ilford\_hp5 ~$|$~ fuji\_ilford\_hp5\_+ ~$|$~ fuji\_ilford\_hp5\_++ ~$|$~ fuji\_ilford\_hp5\_- ~$|$~ fuji\_neopan\_1600 ~$|$~ fuji\_neopan\_1600\_+ ~$|$~ fuji\_neopan\_1600\_++ ~$|$~ fuji\_neopan\_1600\_- ~$|$~ fuji\_neopan\_acros\_100 ~$|$~ fuji\_provia\_100f ~$|$~ fuji\_provia\_400f ~$|$~ fuji\_provia\_400x ~$|$~ fuji\_sensia\_100 ~$|$~ fuji\_superia\_100 ~$|$~ fuji\_superia\_100\_+ ~$|$~ fuji\_superia\_100\_++ ~$|$~ fuji\_superia\_100\_- ~$|$~ fuji\_superia\_1600 ~$|$~ fuji\_superia\_1600\_+ ~$|$~ fuji\_superia\_1600\_++ ~$|$~ fuji\_superia\_1600\_- ~$|$~ fuji\_superia\_200 ~$|$~ fuji\_superia\_200\_xpro ~$|$~ fuji\_superia\_400 ~$|$~ fuji\_superia\_400\_+ ~$|$~ fuji\_superia\_400\_++ ~$|$~ fuji\_superia\_400\_- ~$|$~ fuji\_superia\_800 ~$|$~ fuji\_superia\_800\_+ ~$|$~ fuji\_superia\_800\_++ ~$|$~ fuji\_superia\_800\_- ~$|$~ fuji\_superia\_hg\_1600 ~$|$~ fuji\_superia\_reala\_100 ~$|$~ fuji\_superia\_x-tra\_800 ~$|$~ fuji\_velvia\_50 ~$|$~ fuji\_xtrans\_ii\_astia\_v2 ~$|$~ fuji\_xtrans\_ii\_classic\_chrome\_v1 ~$|$~ fuji\_xtrans\_ii\_pro\_neg\_hi\_v2 ~$|$~ fuji\_xtrans\_ii\_pro\_neg\_std\_v2 ~$|$~ fuji\_xtrans\_ii\_provia\_v2 ~$|$~ fuji\_xtrans\_ii\_velvia\_v2 ~$|$~ generic\_fuji\_astia\_100 ~$|$~ generic\_fuji\_provia\_100 ~$|$~ generic\_fuji\_velvia\_100 ~$|$~ generic\_kodachrome\_64 ~$|$~ generic\_kodak\_ektachrome\_100\_vs ~$|$~ golden ~$|$~ golden\_bright ~$|$~ golden\_fade ~$|$~ golden\_mono ~$|$~ golden\_vibrant ~$|$~ goldfx\_bright\_spring\_breeze ~$|$~ goldfx\_bright\_summer\_heat ~$|$~ goldfx\_hot\_summer\_heat ~$|$~ goldfx\_perfect\_sunset\_01min ~$|$~ goldfx\_perfect\_sunset\_05min ~$|$~ goldfx\_perfect\_sunset\_10min ~$|$~ goldfx\_spring\_breeze ~$|$~ goldfx\_summer\_heat ~$|$~ green\_mono ~$|$~ hong\_kong ~$|$~ ilford\_delta\_100 ~$|$~ ilford\_delta\_3200 ~$|$~ ilford\_delta\_400 ~$|$~ ilford\_fp4\_plus\_125 ~$|$~ ilford\_hp5\_plus\_400 ~$|$~ ilford\_hps\_800 ~$|$~ ilford\_pan\_f\_plus\_50 ~$|$~ ilford\_xp2 ~$|$~ kodak2383\_constlclip ~$|$~ kodak2383\_constlmap ~$|$~ kodak2383\_cuspclip ~$|$~ kodak2393\_constlclip ~$|$~ kodak2393\_constlmap ~$|$~ kodak2393\_cuspclip ~$|$~ kodak\_bw\_400\_cn ~$|$~ kodak\_e-100\_gx\_ektachrome\_100 ~$|$~ kodak\_ektachrome\_100\_vs ~$|$~ kodak\_elite\_100\_xpro ~$|$~ kodak\_elite\_chrome\_200 ~$|$~ kodak\_elite\_chrome\_400 ~$|$~ kodak\_elite\_color\_200 ~$|$~ kodak\_elite\_color\_400 ~$|$~ kodak\_elite\_extracolor\_100 ~$|$~ kodak\_hie\_(hs\_infra) ~$|$~ kodak\_kodachrome\_200 ~$|$~ kodak\_kodachrome\_25 ~$|$~ kodak\_kodachrome\_64 ~$|$~ kodak\_portra\_160 ~$|$~ kodak\_portra\_160\_+ ~$|$~ kodak\_portra\_160\_++ ~$|$~ kodak\_portra\_160\_- ~$|$~ kodak\_portra\_160\_nc ~$|$~ kodak\_portra\_160\_nc\_+ ~$|$~ kodak\_portra\_160\_nc\_++ ~$|$~ kodak\_portra\_160\_nc\_- ~$|$~ kodak\_portra\_160\_vc ~$|$~ kodak\_portra\_160\_vc\_+ ~$|$~ kodak\_portra\_160\_vc\_++ ~$|$~ kodak\_portra\_160\_vc\_- ~$|$~ kodak\_portra\_400 ~$|$~ kodak\_portra\_400\_+ ~$|$~ kodak\_portra\_400\_++ ~$|$~ kodak\_portra\_400\_- ~$|$~ kodak\_portra\_400\_nc ~$|$~ kodak\_portra\_400\_nc\_+ ~$|$~ kodak\_portra\_400\_nc\_++ ~$|$~ kodak\_portra\_400\_nc\_- ~$|$~ kodak\_portra\_400\_uc ~$|$~ kodak\_portra\_400\_uc\_+ ~$|$~ kodak\_portra\_400\_uc\_++ ~$|$~ kodak\_portra\_400\_uc\_- ~$|$~ kodak\_portra\_400\_vc ~$|$~ kodak\_portra\_400\_vc\_+ ~$|$~ kodak\_portra\_400\_vc\_++ ~$|$~ kodak\_portra\_400\_vc\_- ~$|$~ kodak\_portra\_800 ~$|$~ kodak\_portra\_800\_+ ~$|$~ kodak\_portra\_800\_++ ~$|$~ kodak\_portra\_800\_- ~$|$~ kodak\_t-max\_100 ~$|$~ kodak\_t-max\_3200 ~$|$~ kodak\_t-max\_400 ~$|$~ kodak\_tmax\_3200 ~$|$~ kodak\_tmax\_3200\_+ ~$|$~ kodak\_tmax\_3200\_++ ~$|$~ kodak\_tmax\_3200\_- ~$|$~ kodak\_tri-x\_400 ~$|$~ kodak\_tri-x\_400\_+ ~$|$~ kodak\_tri-x\_400\_++ ~$|$~ kodak\_tri-x\_400\_- ~$|$~ light\_blown ~$|$~ lomo ~$|$~ lomography\_redscale\_100 ~$|$~ lomography\_x-pro\_slide\_200 ~$|$~ mono\_tinted ~$|$~ mute\_shift ~$|$~ muted\_fade ~$|$~ natural\_vivid ~$|$~ nostalgic ~$|$~ orange\_tone ~$|$~ pink\_fade ~$|$~ polaroid\_664 ~$|$~ polaroid\_665 ~$|$~ polaroid\_665\_+ ~$|$~ polaroid\_665\_++ ~$|$~ polaroid\_665\_- ~$|$~ polaroid\_665\_-- ~$|$~ polaroid\_665\_negative ~$|$~ polaroid\_665\_negative\_+ ~$|$~ polaroid\_665\_negative\_- ~$|$~ polaroid\_665\_negative\_hc ~$|$~ polaroid\_667 ~$|$~ polaroid\_669 ~$|$~ polaroid\_669\_+ ~$|$~ polaroid\_669\_++ ~$|$~ polaroid\_669\_+++ ~$|$~ polaroid\_669\_- ~$|$~ polaroid\_669\_-- ~$|$~ polaroid\_669\_cold ~$|$~ polaroid\_669\_cold\_+ ~$|$~ polaroid\_669\_cold\_- ~$|$~ polaroid\_669\_cold\_-- ~$|$~ polaroid\_672 ~$|$~ polaroid\_690 ~$|$~ polaroid\_690\_+ ~$|$~ polaroid\_690\_++ ~$|$~ polaroid\_690\_- ~$|$~ polaroid\_690\_-- ~$|$~ polaroid\_690\_cold ~$|$~ polaroid\_690\_cold\_+ ~$|$~ polaroid\_690\_cold\_++ ~$|$~ polaroid\_690\_cold\_- ~$|$~ polaroid\_690\_cold\_-- ~$|$~ polaroid\_690\_warm ~$|$~ polaroid\_690\_warm\_+ ~$|$~ polaroid\_690\_warm\_++ ~$|$~ polaroid\_690\_warm\_- ~$|$~ polaroid\_690\_warm\_-- ~$|$~ polaroid\_polachrome ~$|$~ polaroid\_px-100uv+\_cold ~$|$~ polaroid\_px-100uv+\_cold\_+ ~$|$~ polaroid\_px-100uv+\_cold\_++ ~$|$~ polaroid\_px-100uv+\_cold\_+++ ~$|$~ polaroid\_px-100uv+\_cold\_- ~$|$~ polaroid\_px-100uv+\_cold\_-- ~$|$~ polaroid\_px-100uv+\_warm ~$|$~ polaroid\_px-100uv+\_warm\_+ ~$|$~ polaroid\_px-100uv+\_warm\_++ ~$|$~ polaroid\_px-100uv+\_warm\_+++ ~$|$~ polaroid\_px-100uv+\_warm\_- ~$|$~ polaroid\_px-100uv+\_warm\_-- ~$|$~ polaroid\_px-680 ~$|$~ polaroid\_px-680\_+ ~$|$~ polaroid\_px-680\_++ ~$|$~ polaroid\_px-680\_- ~$|$~ polaroid\_px-680\_-- ~$|$~ polaroid\_px-680\_cold ~$|$~ polaroid\_px-680\_cold\_+ ~$|$~ polaroid\_px-680\_cold\_++ ~$|$~ polaroid\_px-680\_cold\_++\_alt ~$|$~ polaroid\_px-680\_cold\_- ~$|$~ polaroid\_px-680\_cold\_-- ~$|$~ polaroid\_px-680\_warm ~$|$~ polaroid\_px-680\_warm\_+ ~$|$~ polaroid\_px-680\_warm\_++ ~$|$~ polaroid\_px-680\_warm\_- ~$|$~ polaroid\_px-680\_warm\_-- ~$|$~ polaroid\_px-70 ~$|$~ polaroid\_px-70\_+ ~$|$~ polaroid\_px-70\_++ ~$|$~ polaroid\_px-70\_+++ ~$|$~ polaroid\_px-70\_- ~$|$~ polaroid\_px-70\_-- ~$|$~ polaroid\_px-70\_cold ~$|$~ polaroid\_px-70\_cold\_+ ~$|$~ polaroid\_px-70\_cold\_++ ~$|$~ polaroid\_px-70\_cold\_- ~$|$~ polaroid\_px-70\_cold\_-- ~$|$~ polaroid\_px-70\_warm ~$|$~ polaroid\_px-70\_warm\_+ ~$|$~ polaroid\_px-70\_warm\_++ ~$|$~ polaroid\_px-70\_warm\_- ~$|$~ polaroid\_px-70\_warm\_-- ~$|$~ polaroid\_time\_zero\_(expired) ~$|$~ polaroid\_time\_zero\_(expired)\_+ ~$|$~ polaroid\_time\_zero\_(expired)\_++ ~$|$~ polaroid\_time\_zero\_(expired)\_- ~$|$~ polaroid\_time\_zero\_(expired)\_-- ~$|$~ polaroid\_time\_zero\_(expired)\_--- ~$|$~ polaroid\_time\_zero\_(expired)\_cold ~$|$~ polaroid\_time\_zero\_(expired)\_cold\_- ~$|$~ polaroid\_time\_zero\_(expired)\_cold\_-- ~$|$~ polaroid\_time\_zero\_(expired)\_cold\_--- ~$|$~ purple ~$|$~ retro ~$|$~ rollei\_ir\_400 ~$|$~ rollei\_ortho\_25 ~$|$~ rollei\_retro\_100\_tonal ~$|$~ rollei\_retro\_80s ~$|$~ rotate\_muted ~$|$~ rotate\_vibrant ~$|$~ rotated ~$|$~ rotated\_crush ~$|$~ smooth\_cromeish ~$|$~ smooth\_fade ~$|$~ soft\_fade ~$|$~ solarized\_color ~$|$~ solarized\_color2 ~$|$~ summer ~$|$~ summer\_alt ~$|$~ sunny ~$|$~ sunny\_alt ~$|$~ sunny\_rich ~$|$~ sunny\_warm ~$|$~ super\_warm ~$|$~ super\_warm\_rich ~$|$~ sutro\_fx ~$|$~ technicalfx\_backlight\_filter ~$|$~ vibrant ~$|$~ vibrant\_alien ~$|$~ vibrant\_contrast ~$|$~ vibrant\_cromeish ~$|$~ vintage ~$|$~ vintage\_alt ~$|$~ vintage\_brighter ~$|$~ warm ~$|$~ warm\_highlight ~$|$~ warm\_yellow ~$|$~ zilverfx\_b\_w\_solarization ~$|$~ zilverfx\_infrared ~$|$~ zilverfx\_vintage\_b\_w \}.
\begin{center}\includegraphics[keepaspectratio=true,height=6cm,width=\textwidth]{img/gmic_stdlib1.jpg}\\
{\footnotesize \textbf{Example 1~:} \texttt{clut summer}}
\end{center}

\subsection{\emph{command\index{command}} (+)}\vspace*{-0.7em}
~\\\textbf{\Cb{Arguments: }}\begin{flushleft}
{\small \Cb{\hspace*{0.5cm}$\bullet$~~\texttt{\_add\_debug\_info=\{ 0 ~$|$~ 1 \}{\comma}\{ filename ~$|$~ http[s]://URL ~$|$~ "stri\-ng" \}}}}\end{flushleft}
Import G'MIC custom commands from specified file{\comma} URL or string.
~\\(\emph{eq. to} {\small \texttt{'m').\textbackslash n}}).
~\\Imported commands are available directly after the '-command' invocation.
\begin{flushleft}\Cc{\textbf{Default value}:\\~\\\hspace*{0.5cm}{\small $\bullet$~~\texttt{'add\_debug\_info=1'.}}}\end{flushleft}
\begin{center}\includegraphics[keepaspectratio=true,height=6cm,width=\textwidth]{img/gmic_stdlib2.jpg}\\
{\footnotesize \textbf{Example 2~:} \texttt{image.jpg command "foo : mirror y deform \$""1" --foo[0] 5 --foo[0] 15}}
\end{center}

\subsection{\emph{cursor\index{cursor}} (+)}\vspace*{-0.7em}
~\\\textbf{\Cb{Arguments: }}\begin{flushleft}
{\small \Cb{\hspace*{0.5cm}$\bullet$~~\texttt{\_mode = \{ 0=hide ~$|$~ 1=show \}}}}\end{flushleft}
Show or hide mouse cursor for selected instant display windows.
~\\Command selection (if any) stands for instant display window indices instead of image indices.
\begin{flushleft}\Cc{\textbf{Default value}:\\~\\\hspace*{0.5cm}{\small $\bullet$~~\texttt{'mode=1'.}}}\end{flushleft}


\subsection{\emph{display\index{display}} (+)}\vspace*{-0.7em}
~\\\textbf{\Cb{Arguments: }}\begin{flushleft}
{\small \Cb{\hspace*{0.5cm}$\bullet$~~\texttt{\_X$>$=0{\comma}\_Y$>$=0{\comma}\_Z$>$=0{\comma}\_exit\_on\_anykey=\{ 0 ~$|$~ 1 \}}}}\end{flushleft}
Display selected images in an interactive viewer (use the instant display window [0] if opened).
~\\(\emph{eq. to} {\small \texttt{'d').\textbackslash n}}).
~\\Arguments 'X'{\comma}'Y'{\comma}'Z' determine the initial selection view{\comma} for 3d volumetric images.
\begin{flushleft}\Cc{\textbf{Default value}:\\~\\\hspace*{0.5cm}{\small $\bullet$~~\texttt{'X=Y=Z=0'} and \texttt{'exit\_on\_anykey=0'.}}}\end{flushleft}

~\\
~\textbf{Tutorial page: }\\\url{http://gmic.eu/tutorial/\_display.shtml}


\subsection{\emph{display0\index{display0}} }\vspace*{-0.7em}
Display selected images without value normalization.
~\\(\emph{eq. to} {\small \texttt{'d0'}}).


\subsection{\emph{display3d\index{display3d}} (+)}\vspace*{-0.7em}
~\\\textbf{\Cb{Arguments: }}\begin{flushleft}
{\small \Cb{\hspace*{0.5cm}$\bullet$~~\texttt{\_[background\_image]{\comma}\_exit\_on\_anykey=\{ 0 ~$|$~ 1 \}}}}~~~\\
{\small \Cb{\hspace*{0.5cm}$\bullet$~~\texttt{\_exit\_on\_anykey=\{ 0 ~$|$~ 1 \}}}}\end{flushleft}
Display selected 3d objects in an interactive viewer (use the instant display window [0] if opened).
~\\(\emph{eq. to} {\small \texttt{'d3d'}}).
\begin{flushleft}\Cc{\textbf{Default values}:\\~\\\hspace*{0.5cm}{\small $\bullet$~~\texttt{'[background\_image]=(default)'} and \texttt{'exit\_on\_anykey=0'.}}}\end{flushleft}


\subsection{\emph{display\_array\index{display\_array}} }\vspace*{-0.7em}
~\\\textbf{\Cb{Arguments: }}\begin{flushleft}
{\small \Cb{\hspace*{0.5cm}$\bullet$~~\texttt{\_width$>$0{\comma}\_height$>$0}}}\end{flushleft}
Display images in interactive windows where pixel neighborhoods can be explored.
\begin{flushleft}\Cc{\textbf{Default values}:\\~\\\hspace*{0.5cm}{\small $\bullet$~~\texttt{'width=13'} and \texttt{'height=width'.}}}\end{flushleft}


\subsection{\emph{display\_fft\index{display\_fft}} }\vspace*{-0.7em}
Display fourier transform of selected images{\comma} with centered log-module and argument.
~\\(\emph{eq. to} {\small \texttt{'dfft'}}).
\begin{center}\includegraphics[keepaspectratio=true,height=6cm,width=\textwidth]{img/gmic_stdlib3.jpg}\\
{\footnotesize \textbf{Example 3~:} \texttt{image.jpg --display\_fft}}
\end{center}

\subsection{\emph{display\_graph\index{display\_graph}} }\vspace*{-0.7em}
~\\\textbf{\Cb{Arguments: }}\begin{flushleft}
{\small \Cb{\hspace*{0.5cm}$\bullet$~~\texttt{\_width$>$32{\comma}\_height$>$32{\comma}\_plot\_type{\comma}\_vertex\_type{\comma}\_xmin{\comma}\_xmax{\comma}\_ym\-in{\comma}\_ymax{\comma}\_xlabel{\comma}\_ylabel}}}\end{flushleft}
Render graph plot from selected image data.
~\\'plot\_type' can be \{ 0=none ~$|$~ 1=lines ~$|$~ 2=splines ~$|$~ 3=bar \}.
~\\'vertex\_type' can be \{ 0=none ~$|$~ 1=points ~$|$~ 2{\comma}3=crosses ~$|$~ 4{\comma}5=circles ~$|$~ 6{\comma}7=squares \}.
~\\'xmin'{\comma}'xmax'{\comma}'ymin'{\comma}'ymax' set the coordinates of the displayed xy-axes.
\begin{flushleft}\Cc{\textbf{Default values}:\\~\\\hspace*{0.5cm}{\small $\bullet$~~\texttt{'width=640'{\comma} 'height=480'{\comma} 'plot\_type=1'{\comma} 'vertex\_type=1'{\comma} 'xmin=xmax=ymin=ymax=0 (auto)'{\comma} 'xlabel="x-axis"'} and \texttt{'ylabel="y-axis"'.}}}\end{flushleft}
\begin{center}\includegraphics[keepaspectratio=true,height=6cm,width=\textwidth]{img/gmic_stdlib4.jpg}\\
{\footnotesize \textbf{Example 4~:} \texttt{128{\comma}1{\comma}1{\comma}1{\comma}'cos(x/10+u)' --display\_graph 400{\comma}300{\comma}3}}
\end{center}

\subsection{\emph{display\_histogram\index{display\_histogram}} }\vspace*{-0.7em}
~\\\textbf{\Cb{Arguments: }}\begin{flushleft}
{\small \Cb{\hspace*{0.5cm}$\bullet$~~\texttt{\_width$>$0{\comma}\_height$>$0{\comma}\_clusters$>$0{\comma}\_min\_value[\%]{\comma}\_max\_value[\%]{\comma}\_\-show\_axes=\{ 0 ~$|$~ 1 \}{\comma}\_expression.}}}\end{flushleft}
Render a channel-by-channel histogram.
~\\If selected image has several slices{\comma} the rendering is performed for all input slices.
~\\'expression' is a mathematical expression used to transform the histogram data for visualization purpose.
~\\(\emph{eq. to} {\small \texttt{'dh'}}).
\begin{flushleft}\Cc{\textbf{Default values}:\\~\\\hspace*{0.5cm}{\small $\bullet$~~\texttt{'width=512'{\comma} 'height=300'{\comma} 'clusters=256'{\comma} 'min\_value=0\%'{\comma} 'max\_value=100\%'{\comma} 'show\_axes=1'} and \texttt{'expression=i'.}}}\end{flushleft}
\begin{center}\includegraphics[keepaspectratio=true,height=6cm,width=\textwidth]{img/gmic_stdlib5.jpg}\\
{\footnotesize \textbf{Example 5~:} \texttt{image.jpg --display\_histogram 512{\comma}300}}
\end{center}

\subsection{\emph{display\_parametric\index{display\_parametric}} }\vspace*{-0.7em}
~\\\textbf{\Cb{Arguments: }}\begin{flushleft}
{\small \Cb{\hspace*{0.5cm}$\bullet$~~\texttt{\_width$>$0{\comma}\_height$>$0{\comma}\_outline\_opacity{\comma}\_vertex\_radius$>$=0{\comma}\_is\_an\-tialiased=\{ 0 ~$|$~ 1 \}{\comma}\_is\_decorated=\{ 0 ~$|$~ 1 \}{\comma}\_xlabel{\comma}\_ylabel}}}\end{flushleft}
Render 2d or 3d parametric curve or point clouds from selected image data.
~\\Curve points are defined as pixels of a 2 or 3-channel image.
~\\If the point image contains more than 3 channels{\comma} additional channels define the (R{\comma}G{\comma}B) color for each vertex.
~\\If 'outline\_opacity$>$1'{\comma} the outline is colored according to the specified vertex colors and 'outline\_opacity-1' is used
as the actual drawing opacity.
\begin{flushleft}\Cc{\textbf{Default values}:\\~\\\hspace*{0.5cm}{\small $\bullet$~~\texttt{'width=512'{\comma} 'height=width'{\comma} 'outline\_opacity=3'{\comma} 'vertex\_radius=0'{\comma} 'is\_antialiased=1'{\comma} 'is\_decorated=1'{\comma} 'xlabel="x-axis"'} and \texttt{'ylabel="y-axis"'.}}}\end{flushleft}
\begin{center}\includegraphics[keepaspectratio=true,height=6cm,width=\textwidth]{img/gmic_stdlib6.jpg}\\
{\footnotesize \textbf{Example 6~:} \texttt{1024{\comma}1{\comma}1{\comma}2{\comma}'t=x/40;if(c==0{\comma}sin(t){\comma}cos(t))*(exp(cos(t))-2*cos(4*t)-sin(t/12)\textasciicircum 5)' display\_parametric 512{\comma}512}}
\\\includegraphics[keepaspectratio=true,height=6cm,width=\textwidth]{img/gmic_stdlib7.jpg}\\
{\footnotesize \textbf{Example 7~:} \texttt{1000{\comma}1{\comma}1{\comma}2{\comma}u(-100{\comma}100) quantize 4{\comma}1 noise 12 channels 0{\comma}2 --normalize 0{\comma}255 append c display\_parametric 512{\comma}512{\comma}0.1{\comma}8}}
\end{center}

\subsection{\emph{display\_parallel\index{display\_parallel}} }\vspace*{-0.7em}
Display each selected image in a separate interactive display window.
~\\(\emph{eq. to} {\small \texttt{'dp'}}).


\subsection{\emph{display\_parallel0\index{display\_parallel0}} }\vspace*{-0.7em}
Display each selected image in a separate interactive display window{\comma} without value normalization.
~\\(\emph{eq. to} {\small \texttt{'dp0'}}).


\subsection{\emph{display\_polar\index{display\_polar}} }\vspace*{-0.7em}
~\\\textbf{\Cb{Arguments: }}\begin{flushleft}
{\small \Cb{\hspace*{0.5cm}$\bullet$~~\texttt{\_width$>$32{\comma}\_height$>$32{\comma}\_outline\_type{\comma}\_fill\_R{\comma}\_fill\_G{\comma}\_fill\_B{\comma}\_\-theta\_start{\comma}\_theta\_end{\comma}\_xlabel{\comma}\_ylabel}}}\end{flushleft}
Render polar curve from selected image data.
~\\'outline\_type' can be \{ r$<$0=dots with radius -r ~$|$~ 0=no outline ~$|$~ r$>$0=lines+dots with radius r \}.
~\\'fill\_color' can be \{ -1=no fill ~$|$~ R{\comma}G{\comma}B=fill with specified color \}.
\begin{flushleft}\Cc{\textbf{Default values}:\\~\\\hspace*{0.5cm}{\small $\bullet$~~\texttt{'width=500'{\comma} 'height=width'{\comma} 'outline\_type=1'{\comma} 'fill\_R=fill\_G=fill\_B=200'{\comma} 'theta\_start=0'{\comma} 'theta\_end=360'{\comma} 'xlabel="x-axis"'} and \texttt{'ylabel="y-axis"'.}}}\end{flushleft}
\begin{center}\includegraphics[keepaspectratio=true,height=6cm,width=\textwidth]{img/gmic_stdlib8.jpg}\\
{\footnotesize \textbf{Example 8~:} \texttt{300{\comma}1{\comma}1{\comma}1{\comma}'0.3+abs(cos(10*pi*x/w))+u(0.4)' display\_polar 512{\comma}512{\comma}4{\comma}200{\comma}255{\comma}200}}
\\\includegraphics[keepaspectratio=true,height=6cm,width=\textwidth]{img/gmic_stdlib9.jpg}\\
{\footnotesize \textbf{Example 9~:} \texttt{3000{\comma}1{\comma}1{\comma}1{\comma}'x\textasciicircum 3/1e10' display\_polar 400{\comma}400{\comma}1{\comma}-1{\comma}{\comma}{\comma}0{\comma}\{15*360\}}}
\end{center}

\subsection{\emph{display\_quiver\index{display\_quiver}} }\vspace*{-0.7em}
~\\\textbf{\Cb{Arguments: }}\begin{flushleft}
{\small \Cb{\hspace*{0.5cm}$\bullet$~~\texttt{\_size\_factor$>$0{\comma}\_arrow\_size$>$=0{\comma}\_color\_mode=\{ 0=monochrome ~$|$~ 1\-=grayscale ~$|$~ 2=color \}}}}\end{flushleft}
Render selected image of 2d vectors as a field of 2d arrows.
~\\(\emph{eq. to} {\small \texttt{'dq'}}).
\begin{flushleft}\Cc{\textbf{Default values}:\\~\\\hspace*{0.5cm}{\small $\bullet$~~\texttt{'size\_factor=16'{\comma} 'arrow\_size=1.5'} and \texttt{'color\_mode=1'.}}}\end{flushleft}
\begin{center}\includegraphics[keepaspectratio=true,height=6cm,width=\textwidth]{img/gmic_stdlib10.jpg}\\
{\footnotesize \textbf{Example 10~:} \texttt{image.jpg --luminance gradient[-1] xy rv[-2{\comma}-1] *[-2] -1 a[-2{\comma}-1] c crop 60{\comma}10{\comma}90{\comma}30 --display\_quiver[1] {\comma}}}
\end{center}

\subsection{\emph{display\_rgba\index{display\_rgba}} }\vspace*{-0.7em}
~\\\textbf{\Cb{Arguments: }}\begin{flushleft}
{\small \Cb{\hspace*{0.5cm}$\bullet$~~\texttt{\_background\_RGB\_color}}}\end{flushleft}
Render selected RGBA images over a checkerboard background.
~\\(\emph{eq. to} {\small \texttt{'drgba'}}).
\begin{flushleft}\Cc{\textbf{Default values}:\\~\\\hspace*{0.5cm}{\small $\bullet$~~\texttt{'background\_RGB\_color=undefined' (checkerboard).}}}\end{flushleft}
\begin{center}\includegraphics[keepaspectratio=true,height=6cm,width=\textwidth]{img/gmic_stdlib11.jpg}\\
{\footnotesize \textbf{Example 11~:} \texttt{image.jpg --norm threshold[-1] 40\% blur[-1] 3 normalize[-1] 0{\comma}255 append c display\_rgba}}
\end{center}

\subsection{\emph{display\_tensors\index{display\_tensors}} }\vspace*{-0.7em}
~\\\textbf{\Cb{Arguments: }}\begin{flushleft}
{\small \Cb{\hspace*{0.5cm}$\bullet$~~\texttt{\_size\_factor$>$0{\comma}\_ellipse\_size$>$=0{\comma}\_color\_mode=\{ 0=monochrome ~$|$~\- 1=grayscale ~$|$~ 2=color \}{\comma}\_outline$>$=0}}}\end{flushleft}
Render selected image of tensors as a field of 2d ellipses.
~\\(\emph{eq. to} {\small \texttt{'dt'}}).
\begin{flushleft}\Cc{\textbf{Default values}:\\~\\\hspace*{0.5cm}{\small $\bullet$~~\texttt{'size\_factor=16'{\comma} 'ellipse\_size=1.5'{\comma} 'color\_mode=2'} and \texttt{'outline=2'.}}}\end{flushleft}
\begin{center}\includegraphics[keepaspectratio=true,height=6cm,width=\textwidth]{img/gmic_stdlib12.jpg}\\
{\footnotesize \textbf{Example 12~:} \texttt{image.jpg diffusiontensors 0.1{\comma}0.9 resize2dx 32 --display\_tensors 64{\comma}2}}
\end{center}
~\\
~\textbf{Tutorial page: }\\\url{http://gmic.eu/tutorial/\_display\_tensors.shtml}


\subsection{\emph{display\_warp\index{display\_warp}} }\vspace*{-0.7em}
~\\\textbf{\Cb{Arguments: }}\begin{flushleft}
{\small \Cb{\hspace*{0.5cm}$\bullet$~~\texttt{\_cell\_size$>$0}}}\end{flushleft}
Render selected 2d warping fields.
~\\(\emph{eq. to} {\small \texttt{'dw'}}).
\begin{flushleft}\Cc{\textbf{Default value}:\\~\\\hspace*{0.5cm}{\small $\bullet$~~\texttt{'cell\_size=15'.}}}\end{flushleft}
\begin{center}\includegraphics[keepaspectratio=true,height=6cm,width=\textwidth]{img/gmic_stdlib13.jpg}\\
{\footnotesize \textbf{Example 13~:} \texttt{400{\comma}400{\comma}1{\comma}2{\comma}'x=x-w/2;y=y-h/2;r=sqrt(x*x+y*y);a=atan2(y{\comma}x);5*sin(r/10)*[cos(a){\comma}sin(a)]' --display\_warp 10}}
\end{center}

\subsection{\emph{document\_gmic\index{document\_gmic}} }\vspace*{-0.7em}
~\\\textbf{\Cb{Arguments: }}\begin{flushleft}
{\small \Cb{\hspace*{0.5cm}$\bullet$~~\texttt{\_format=\{ ascii ~$|$~ bash ~$|$~ html ~$|$~ images ~$|$~ latex \}{\comma}\_image\_path\-{\comma}\_write\_wrapper=\{ 0 ~$|$~ 1 \}}}}\end{flushleft}
Create documentation of .gmic command files (loaded as raw 'uchar' images){\comma} in specified format.
\begin{flushleft}\Cc{\textbf{Default values}:\\~\\\hspace*{0.5cm}{\small $\bullet$~~\texttt{'format=ascii'{\comma} 'image\_path=""'} and \texttt{'write\_wrapper=1'.\textbackslash n}}}\end{flushleft}
~\\Example(s) : raw:filename.gmic{\comma}char document\_gmic html{\comma}img


\subsection{\emph{echo\index{echo}} (+)}\vspace*{-0.7em}
~\\\textbf{\Cb{Arguments: }}\begin{flushleft}
{\small \Cb{\hspace*{0.5cm}$\bullet$~~\texttt{message}}}\end{flushleft}
Output specified message on the error output.
~\\(\emph{eq. to} {\small \texttt{'e').\textbackslash n}}).
~\\Command selection (if any) stands for displayed call stack subset instead of image indices.


\subsection{\emph{echo\_file\index{echo\_file}} }\vspace*{-0.7em}
~\\\textbf{\Cb{Arguments: }}\begin{flushleft}
{\small \Cb{\hspace*{0.5cm}$\bullet$~~\texttt{filename{\comma}message}}}\end{flushleft}
Output specified message{\comma} appending it to specified output file.
~\\(similar to '-echo' for specified output file stream).


\subsection{\emph{echo\_stdout\index{echo\_stdout}} }\vspace*{-0.7em}
~\\\textbf{\Cb{Arguments: }}\begin{flushleft}
{\small \Cb{\hspace*{0.5cm}$\bullet$~~\texttt{message}}}\end{flushleft}
Output specified message{\comma} on the standard output (stdout).
~\\(similar to '-echo' for output on standard output instead of standard error).


\subsection{\emph{function1d\index{function1d}} }\vspace*{-0.7em}
~\\\textbf{\Cb{Arguments: }}\begin{flushleft}
{\small \Cb{\hspace*{0.5cm}$\bullet$~~\texttt{0$<$=smoothness$<$=1{\comma}x0$>$=0{\comma}y0{\comma}x1$>$=0{\comma}y1{\comma}...{\comma}xn$>$=0{\comma}yn}}}\end{flushleft}
Insert continuous 1d function from specified list of keypoints (xk{\comma}yk)
in range [0{\comma}max(xk)] (xk are positive integers).
\begin{flushleft}\Cc{\textbf{Default values}:\\~\\\hspace*{0.5cm}{\small $\bullet$~~\texttt{'smoothness=1'} and \texttt{'x0=y0=0'.}}}\end{flushleft}
\begin{center}\includegraphics[keepaspectratio=true,height=6cm,width=\textwidth]{img/gmic_stdlib14.jpg}\\
{\footnotesize \textbf{Example 14~:} \texttt{function1d 1{\comma}0{\comma}0{\comma}10{\comma}30{\comma}40{\comma}20{\comma}70{\comma}30{\comma}80{\comma}0 --display\_graph 400{\comma}300}}
\end{center}

\subsection{\emph{gmicky\index{gmicky}} }\vspace*{-0.7em}
Insert new image of the G'MIC mascot 'Gmicky'.


\subsection{\emph{gmicky\_deevad\index{gmicky\_deevad}} }\vspace*{-0.7em}
Insert new image of the G'MIC mascot 'Gmicky'{\comma} by David Revoy.
\begin{center}\includegraphics[keepaspectratio=true,height=6cm,width=\textwidth]{img/gmic_stdlib15.jpg}\\
{\footnotesize \textbf{Example 15~:} \texttt{gmicky\_deevad}}
\end{center}

\subsection{\emph{gmicky\_mahvin\index{gmicky\_mahvin}} }\vspace*{-0.7em}
Insert new image of the G'MIC mascot 'Gmicky'{\comma} by Mahvin.
\begin{center}\includegraphics[keepaspectratio=true,height=6cm,width=\textwidth]{img/gmic_stdlib16.jpg}\\
{\footnotesize \textbf{Example 16~:} \texttt{gmicky\_mahvin}}
\end{center}

\subsection{\emph{gmicky\_wilber\index{gmicky\_wilber}} }\vspace*{-0.7em}
Insert new image of the G'MIC mascot 'Gmicky' together with GIMP mascot 'Wilber'{\comma} by Mahvin.
\begin{center}\includegraphics[keepaspectratio=true,height=6cm,width=\textwidth]{img/gmic_stdlib17.jpg}\\
{\footnotesize \textbf{Example 17~:} \texttt{gmicky\_wilber}}
\end{center}

\subsection{\emph{input\index{input}} (+)}\vspace*{-0.7em}
~\\\textbf{\Cb{Arguments: }}\begin{flushleft}
{\small \Cb{\hspace*{0.5cm}$\bullet$~~\texttt{[type:]filename}}}~~~\\
{\small \Cb{\hspace*{0.5cm}$\bullet$~~\texttt{[type:]http://URL}}}~~~\\
{\small \Cb{\hspace*{0.5cm}$\bullet$~~\texttt{[selection]x\_nb\_copies$>$0}}}~~~\\
{\small \Cb{\hspace*{0.5cm}$\bullet$~~\texttt{\{ width$>$0[\%] ~$|$~ [image\_w] \}{\comma}\{ \_height$>$0[\%] ~$|$~ [image\_h] \}{\comma}\{ \_d\-epth$>$0[\%] ~$|$~ [image\_d] \}{\comma}\{ \_spectrum$>$0[\%] ~$|$~ [image\_s] \}{\comma}\_\{ va\-lue1{\comma}\_value2{\comma}... ~$|$~ 'formula' \}}}}~~~\\
{\small \Cb{\hspace*{0.5cm}$\bullet$~~\texttt{(value1\{{\comma}~$|$~;~$|$~/~$|$~\textasciicircum \}value2\{{\comma}~$|$~;~$|$~/~$|$~\textasciicircum \}...)}}}~~~\\
{\small \Cb{\hspace*{0.5cm}$\bullet$~~\texttt{0}}}\end{flushleft}
Insert a new image taken from a filename or from a copy of an existing image [indice]{\comma}
or insert new image with specified dimensions and values. Single quotes may be omitted in
~\\'formula'. Specifying argument '0' inserts an 'empty' image.
~\\(\emph{eq. to} {\small \texttt{'i' ~$|$~ (no arg}}).
\begin{flushleft}\Cc{\textbf{Default values}:\\~\\\hspace*{0.5cm}{\small $\bullet$~~\texttt{'nb\_copies=1'{\comma} 'height=depth=spectrum=1'} and \texttt{'value1=0'.}}}\end{flushleft}
\begin{center}\includegraphics[keepaspectratio=true,height=6cm,width=\textwidth]{img/gmic_stdlib18.jpg}\\
{\footnotesize \textbf{Example 18~:} \texttt{input image.jpg}}
\\\includegraphics[keepaspectratio=true,height=6cm,width=\textwidth]{img/gmic_stdlib19.jpg}\\
{\footnotesize \textbf{Example 19~:} \texttt{input (1{\comma}2{\comma}3;4{\comma}5{\comma}6;7{\comma}8{\comma}9\textasciicircum 9{\comma}8{\comma}7;6{\comma}5{\comma}4;3{\comma}2{\comma}1)}}
\\\includegraphics[keepaspectratio=true,height=6cm,width=\textwidth]{img/gmic_stdlib20.jpg}\\
{\footnotesize \textbf{Example 20~:} \texttt{image.jpg (1{\comma}2{\comma}3;4{\comma}5{\comma}6;7{\comma}8{\comma}9) (255\textasciicircum 128\textasciicircum 64) 400{\comma}400{\comma}1{\comma}3{\comma}'if(x$>$w/2{\comma}x{\comma}y)*c'}}
\end{center}
~\\
~\textbf{Tutorial page: }\\\url{http://gmic.eu/tutorial/\_input.shtml}


\subsection{\emph{input\_cube\index{input\_cube}} }\vspace*{-0.7em}
~\\\textbf{\Cb{Arguments: }}\begin{flushleft}
{\small \Cb{\hspace*{0.5cm}$\bullet$~~\texttt{"filename"{\comma}\_convert\_1d\_cluts\_to\_3d=\{ 0 ~$|$~ 1 \}.}}}\end{flushleft}
Insert CLUT data from a .cube filename (Adobe CLUT file format).
\begin{flushleft}\Cc{\textbf{Default value}:\\~\\\hspace*{0.5cm}{\small $\bullet$~~\texttt{'convert\_1d\_cluts\_to\_3d=1'.}}}\end{flushleft}


\subsection{\emph{input\_glob\index{input\_glob}} }\vspace*{-0.7em}
~\\\textbf{\Cb{Arguments: }}\begin{flushleft}
{\small \Cb{\hspace*{0.5cm}$\bullet$~~\texttt{pattern}}}\end{flushleft}
Insert new images from several filenames that match the specified glob pattern.


\subsection{\emph{input\_gpl\index{input\_gpl}} }\vspace*{-0.7em}
~\\\textbf{\Cb{Arguments: }}\begin{flushleft}
{\small \Cb{\hspace*{0.5cm}$\bullet$~~\texttt{filename}}}\end{flushleft}
Input specified filename as a .gpl palette data file.


\subsection{\emph{output\index{output}} (+)}\vspace*{-0.7em}
~\\\textbf{\Cb{Arguments: }}\begin{flushleft}
{\small \Cb{\hspace*{0.5cm}$\bullet$~~\texttt{[type:]filename{\comma}\_format\_options}}}\end{flushleft}
Output selected images as one or several numbered file(s).
~\\(\emph{eq. to} {\small \texttt{'o'}}).
\begin{flushleft}\Cc{\textbf{Default value}:\\~\\\hspace*{0.5cm}{\small $\bullet$~~\texttt{'format\_options'=(undefined).}}}\end{flushleft}


\subsection{\emph{output\_cube\index{output\_cube}} }\vspace*{-0.7em}
~\\\textbf{\Cb{Arguments: }}\begin{flushleft}
{\small \Cb{\hspace*{0.5cm}$\bullet$~~\texttt{filename}}}\end{flushleft}
Output selected CLUTs as a .cube file (Adobe CLUT format).


\subsection{\emph{output\_ggr\index{output\_ggr}} }\vspace*{-0.7em}
~\\\textbf{\Cb{Arguments: }}\begin{flushleft}
{\small \Cb{\hspace*{0.5cm}$\bullet$~~\texttt{filename{\comma}\_gradient\_name}}}\end{flushleft}
Output selected images as .ggr gradient files (GIMP).
~\\If no gradient name is specified{\comma} it is deduced from the filename.


\subsection{\emph{outputn\index{outputn}} }\vspace*{-0.7em}
~\\\textbf{\Cb{Arguments: }}\begin{flushleft}
{\small \Cb{\hspace*{0.5cm}$\bullet$~~\texttt{filename}}}\end{flushleft}
Output selected images as automatically numbered filenames in repeat...done loops.
~\\(\emph{eq. to} {\small \texttt{'on'}}).


\subsection{\emph{outputp\index{outputp}} }\vspace*{-0.7em}
~\\\textbf{\Cb{Arguments: }}\begin{flushleft}
{\small \Cb{\hspace*{0.5cm}$\bullet$~~\texttt{prefix}}}\end{flushleft}
Output selected images as prefixed versions of their original filenames.
~\\(\emph{eq. to} {\small \texttt{'op'}}).
\begin{flushleft}\Cc{\textbf{Default value}:\\~\\\hspace*{0.5cm}{\small $\bullet$~~\texttt{'prefix=\_'.}}}\end{flushleft}


\subsection{\emph{outputw\index{outputw}} }\vspace*{-0.7em}
Output selected images by overwritting their original location.
~\\(\emph{eq. to} {\small \texttt{'ow'}}).


\subsection{\emph{outputx\index{outputx}} }\vspace*{-0.7em}
~\\\textbf{\Cb{Arguments: }}\begin{flushleft}
{\small \Cb{\hspace*{0.5cm}$\bullet$~~\texttt{extension1{\comma}\_extension2{\comma}\_...{\comma}\_extensionN{\comma}\_output\_at\_same\_loca\-tion=\{ 0 ~$|$~ 1 \}}}}\end{flushleft}
Output selected images with same base filenames but for N different extensions.
~\\(\emph{eq. to} {\small \texttt{'ox'}}).
\begin{flushleft}\Cc{\textbf{Default value}:\\~\\\hspace*{0.5cm}{\small $\bullet$~~\texttt{'output\_at\_same\_location=0'.}}}\end{flushleft}


\subsection{\emph{pass\index{pass}} (+)}\vspace*{-0.7em}
~\\\textbf{\Cb{Arguments: }}\begin{flushleft}
{\small \Cb{\hspace*{0.5cm}$\bullet$~~\texttt{\_shared\_state=\{ 0=non-shared (copy) ~$|$~ 1=shared ~$|$~ 2=adaptive \-\}}}}\end{flushleft}
Insert images from parent context of a custom command or a local environment.
~\\Command selection (if any) stands for a selection of images in the parent context.
~\\By default (adaptive shared state){\comma} selected images are inserted in a shared state if they do not belong to the context (selection) of the current custom command or local environment as well.
~\\Typical use of command '-pass' concerns the design of custom commands that take images as arguments.
\begin{flushleft}\Cc{\textbf{Default value}:\\~\\\hspace*{0.5cm}{\small $\bullet$~~\texttt{'shared\_state=2'.}}}\end{flushleft}
\begin{center}\includegraphics[keepaspectratio=true,height=6cm,width=\textwidth]{img/gmic_stdlib21.jpg}\\
{\footnotesize \textbf{Example 21~:} \texttt{command "average : pass\$""1 add[\textasciicircum -1] [-1] remove[-1] div 2" sample ? --mirror y --average[0] [1]}}
\end{center}

\subsection{\emph{plot\index{plot}} (+)}\vspace*{-0.7em}
~\\\textbf{\Cb{Arguments: }}\begin{flushleft}
{\small \Cb{\hspace*{0.5cm}$\bullet$~~\texttt{\_plot\_type{\comma}\_vertex\_type{\comma}\_xmin{\comma}\_xmax{\comma}\_ymin{\comma}\_ymax{\comma}\_exit\_on\_any\-key=\{ 0 ~$|$~ 1 \}}}}~~~\\
{\small \Cb{\hspace*{0.5cm}$\bullet$~~\texttt{'formula'{\comma}\_resolution$>$=0{\comma}\_plot\_type{\comma}\_vertex\_type{\comma}\_xmin{\comma}xmax{\comma}\-\_ymin{\comma}\_ymax{\comma}\_exit\_on\_anykey=\{ 0 ~$|$~ 1 \}}}}\end{flushleft}
Display selected image or formula in an interactive viewer (use the instant display window [0] if opened).
~\\'plot\_type' can be \{ 0=none ~$|$~ 1=lines ~$|$~ 2=splines ~$|$~ 3=bar \}.
~\\'vertex\_type' can be \{ 0=none ~$|$~ 1=points ~$|$~ 2{\comma}3=crosses ~$|$~ 4{\comma}5=circles ~$|$~ 6{\comma}7=squares \}.
~\\'xmin'{\comma}'xmax'{\comma}'ymin'{\comma}'ymax' set the coordinates of the displayed xy-axes.
\begin{flushleft}\Cc{\textbf{Default values}:\\~\\\hspace*{0.5cm}{\small $\bullet$~~\texttt{'plot\_type=1'{\comma} 'vertex\_type=1'{\comma} 'xmin=xmax=ymin=ymax=0 (auto)'} and \texttt{'exit\_on\_anykey=0'.}}}\end{flushleft}


\subsection{\emph{print\index{print}} (+)}\vspace*{-0.7em}
Output information on selected images{\comma} on the standard error (stderr).
~\\(\emph{eq. to} {\small \texttt{'p'}}).


\subsection{\emph{rainbow\_lut\index{rainbow\_lut}} }\vspace*{-0.7em}
Input a 256-entries RGB colormap of rainbow colors.
\begin{center}\includegraphics[keepaspectratio=true,height=6cm,width=\textwidth]{img/gmic_stdlib22.jpg}\\
{\footnotesize \textbf{Example 22~:} \texttt{image.jpg rainbow\_lut --luminance[-2] map[-1] [-2]}}
\end{center}

\subsection{\emph{roddy\index{roddy}} }\vspace*{-0.7em}
Input a new image of the G'MIC Rodilius mascot 'Roddy'.
\begin{center}\includegraphics[keepaspectratio=true,height=6cm,width=\textwidth]{img/gmic_stdlib23.jpg}\\
{\footnotesize \textbf{Example 23~:} \texttt{roddy}}
\end{center}

\subsection{\emph{screen\index{screen}} (+)}\vspace*{-0.7em}
~\\\textbf{\Cb{Arguments: }}\begin{flushleft}
{\small \Cb{\hspace*{0.5cm}$\bullet$~~\texttt{\_x0[\%]{\comma}\_y0[\%]{\comma}\_x1[\%]{\comma}\_y1[\%]}}}\end{flushleft}
Take screenshot{\comma} optionally grabbed with specified coordinates{\comma} and insert it
at the end of the image list.


\subsection{\emph{select\index{select}} (+)}\vspace*{-0.7em}
~\\\textbf{\Cb{Arguments: }}\begin{flushleft}
{\small \Cb{\hspace*{0.5cm}$\bullet$~~\texttt{feature\_type{\comma}\_X{\comma}\_Y{\comma}\_Z{\comma}\_exit\_on\_anykey=\{ 0 ~$|$~ 1 \}}}}\end{flushleft}
Interactively select a feature from selected images (use the instant display window [0] if opened).
~\\'feature\_type' can be \{ 0=point ~$|$~ 1=segment ~$|$~ 2=rectangle ~$|$~ 3=ellipse \}.
~\\Arguments 'X'{\comma}'Y'{\comma}'Z' determine the initial selection view{\comma} for 3d volumetric images.
~\\The retrieved feature is returned as a 3d vector (if 'feature\_type==0') or as a 6d vector
~\\(if 'feature\_type!=0') containing the feature coordinates.
~\\The coordinates of the last selected features are also returned as the status value.
\begin{flushleft}\Cc{\textbf{Default values}:\\~\\\hspace*{0.5cm}{\small $\bullet$~~\texttt{'X=Y=Z=(undefined)'} and \texttt{'exit\_on\_anykey=0'.}}}\end{flushleft}


\subsection{\emph{serialize\index{serialize}} (+)}\vspace*{-0.7em}
~\\\textbf{\Cb{Arguments: }}\begin{flushleft}
{\small \Cb{\hspace*{0.5cm}$\bullet$~~\texttt{\_datatype{\comma}\_is\_compressed=\{ 0 ~$|$~ 1 \}{\comma}\_store\_names=\{ 0 ~$|$~ 1 \}}}}\end{flushleft}
Serialize selected list of images into a single image{\comma} optionnally in a compressed form.
~\\'datatype' can be \{ uchar ~$|$~ char ~$|$~ ushort ~$|$~ short ~$|$~ uint ~$|$~ int ~$|$~ uint64 ~$|$~ int64 ~$|$~ float ~$|$~ double \}.
~\\Specify 'datatype' if all selected images have a range of values constrained to a particular datatype{\comma} in order to minimize the memory footprint.
~\\The resulting image has only integers values in [0{\comma}255] and can then be saved as a raw image of
unsigned chars (doing so will output a valid .cimg[z] or .gmz file).
~\\If 'store\_names' is set to '1'{\comma} serialization uses the .gmz format to store data in memory (otherwise the .cimg[z] format).
\begin{flushleft}\Cc{\textbf{Default values}:\\~\\\hspace*{0.5cm}{\small $\bullet$~~\texttt{'datatype=float'{\comma} 'is\_compressed=1'} and \texttt{'store\_names=1'.}}}\end{flushleft}
\begin{center}\includegraphics[keepaspectratio=true,height=6cm,width=\textwidth]{img/gmic_stdlib24.jpg}\\
{\footnotesize \textbf{Example 24~:} \texttt{image.jpg --serialize uchar --unserialize[-1]}}
\end{center}

\subsection{\emph{shape\_heart\index{shape\_heart}} }\vspace*{-0.7em}
~\\\textbf{\Cb{Arguments: }}\begin{flushleft}
{\small \Cb{\hspace*{0.5cm}$\bullet$~~\texttt{\_size$>$=0}}}\end{flushleft}
Input a 2d heart binary shape with specified size.
\begin{flushleft}\Cc{\textbf{Default value}:\\~\\\hspace*{0.5cm}{\small $\bullet$~~\texttt{'size=512'.}}}\end{flushleft}
\begin{center}\includegraphics[keepaspectratio=true,height=6cm,width=\textwidth]{img/gmic_stdlib25.jpg}\\
{\footnotesize \textbf{Example 25~:} \texttt{shape\_heart {\comma}}}
\end{center}

\subsection{\emph{shape\_circle\index{shape\_circle}} }\vspace*{-0.7em}
~\\\textbf{\Cb{Arguments: }}\begin{flushleft}
{\small \Cb{\hspace*{0.5cm}$\bullet$~~\texttt{\_size$>$=0}}}\end{flushleft}
Input a 2d circle binary shape with specified size.
\begin{flushleft}\Cc{\textbf{Default value}:\\~\\\hspace*{0.5cm}{\small $\bullet$~~\texttt{'size=512'.}}}\end{flushleft}
\begin{center}\includegraphics[keepaspectratio=true,height=6cm,width=\textwidth]{img/gmic_stdlib26.jpg}\\
{\footnotesize \textbf{Example 26~:} \texttt{shape\_circle {\comma}}}
\end{center}

\subsection{\emph{shape\_cupid\index{shape\_cupid}} }\vspace*{-0.7em}
~\\\textbf{\Cb{Arguments: }}\begin{flushleft}
{\small \Cb{\hspace*{0.5cm}$\bullet$~~\texttt{\_size$>$=0}}}\end{flushleft}
Input a 2d cupid binary shape with specified size.
\begin{flushleft}\Cc{\textbf{Default value}:\\~\\\hspace*{0.5cm}{\small $\bullet$~~\texttt{'size=512'.}}}\end{flushleft}
\begin{center}\includegraphics[keepaspectratio=true,height=6cm,width=\textwidth]{img/gmic_stdlib27.jpg}\\
{\footnotesize \textbf{Example 27~:} \texttt{shape\_cupid {\comma}}}
\end{center}

\subsection{\emph{shape\_diamond\index{shape\_diamond}} }\vspace*{-0.7em}
~\\\textbf{\Cb{Arguments: }}\begin{flushleft}
{\small \Cb{\hspace*{0.5cm}$\bullet$~~\texttt{\_size$>$=0}}}\end{flushleft}
Input a 2d diamond binary shape with specified size.
\begin{flushleft}\Cc{\textbf{Default value}:\\~\\\hspace*{0.5cm}{\small $\bullet$~~\texttt{'size=512'.}}}\end{flushleft}
\begin{center}\includegraphics[keepaspectratio=true,height=6cm,width=\textwidth]{img/gmic_stdlib28.jpg}\\
{\footnotesize \textbf{Example 28~:} \texttt{shape\_diamond {\comma}}}
\end{center}

\subsection{\emph{shape\_fern\index{shape\_fern}} }\vspace*{-0.7em}
~\\\textbf{\Cb{Arguments: }}\begin{flushleft}
{\small \Cb{\hspace*{0.5cm}$\bullet$~~\texttt{\_size$>$=0{\comma}\_density[\%]$>$=0{\comma}\_angle{\comma}0$<$=\_opacity$<$=1{\comma}\_type=\{ 0=Aspl\-enium adiantum-nigrum ~$|$~ 1=Thelypteridaceae \}}}}\end{flushleft}
Input a 2d Barnsley fern with specified size.
\begin{flushleft}\Cc{\textbf{Default value}:\\~\\\hspace*{0.5cm}{\small $\bullet$~~\texttt{'size=512'{\comma} 'density=50\%'{\comma} 'angle=30'{\comma} 'opacity=0.3'} and \texttt{'type=0'.}}}\end{flushleft}
\begin{center}\includegraphics[keepaspectratio=true,height=6cm,width=\textwidth]{img/gmic_stdlib29.jpg}\\
{\footnotesize \textbf{Example 29~:} \texttt{shape\_fern {\comma}}}
\end{center}

\subsection{\emph{shape\_polygon\index{shape\_polygon}} }\vspace*{-0.7em}
~\\\textbf{\Cb{Arguments: }}\begin{flushleft}
{\small \Cb{\hspace*{0.5cm}$\bullet$~~\texttt{\_size$>$=0{\comma}\_nb\_vertices$>$=3{\comma}\_angle}}}\end{flushleft}
Input a 2d polygonal binary shape with specified geometry.
\begin{flushleft}\Cc{\textbf{Default value}:\\~\\\hspace*{0.5cm}{\small $\bullet$~~\texttt{'size=512'{\comma} 'nb\_vertices=5'} and \texttt{'angle=0'.}}}\end{flushleft}
\begin{center}\includegraphics[keepaspectratio=true,height=6cm,width=\textwidth]{img/gmic_stdlib30.jpg}\\
{\footnotesize \textbf{Example 30~:} \texttt{repeat 6 shape\_polygon 256{\comma}\{3+\$$>$\} done}}
\end{center}

\subsection{\emph{shape\_snowflake\index{shape\_snowflake}} }\vspace*{-0.7em}
~\\\textbf{\Cb{Arguments: }}\begin{flushleft}
{\small \Cb{\hspace*{0.5cm}$\bullet$~~\texttt{size$>$=0{\comma}0$<$=\_nb\_recursions$<$=6}}}\end{flushleft}
Input a 2d snowflake binary shape with specified size.
\begin{flushleft}\Cc{\textbf{Default values}:\\~\\\hspace*{0.5cm}{\small $\bullet$~~\texttt{'size=512'} and \texttt{'nb\_recursions=5'.}}}\end{flushleft}
\begin{center}\includegraphics[keepaspectratio=true,height=6cm,width=\textwidth]{img/gmic_stdlib31.jpg}\\
{\footnotesize \textbf{Example 31~:} \texttt{repeat 6 shape\_snowflake 256{\comma}\$$>$ done}}
\end{center}

\subsection{\emph{shape\_star\index{shape\_star}} }\vspace*{-0.7em}
~\\\textbf{\Cb{Arguments: }}\begin{flushleft}
{\small \Cb{\hspace*{0.5cm}$\bullet$~~\texttt{\_size$>$=0{\comma}\_nb\_branches$>$0{\comma}0$<$=\_thickness$<$=1}}}\end{flushleft}
Input a 2d star binary shape with specified size.
\begin{flushleft}\Cc{\textbf{Default values}:\\~\\\hspace*{0.5cm}{\small $\bullet$~~\texttt{'size=512'{\comma} 'nb\_branches=5'} and \texttt{'thickness=0.38'.}}}\end{flushleft}
\begin{center}\includegraphics[keepaspectratio=true,height=6cm,width=\textwidth]{img/gmic_stdlib32.jpg}\\
{\footnotesize \textbf{Example 32~:} \texttt{repeat 9 shape\_star 256{\comma}\{\$$>$+2\} done}}
\end{center}

\subsection{\emph{shared\index{shared}} (+)}\vspace*{-0.7em}
~\\\textbf{\Cb{Arguments: }}\begin{flushleft}
{\small \Cb{\hspace*{0.5cm}$\bullet$~~\texttt{x0[\%]{\comma}x1[\%]{\comma}y[\%]{\comma}z[\%]{\comma}v[\%]}}}~~~\\
{\small \Cb{\hspace*{0.5cm}$\bullet$~~\texttt{y0[\%]{\comma}y1[\%]{\comma}z[\%]{\comma}v[\%]}}}~~~\\
{\small \Cb{\hspace*{0.5cm}$\bullet$~~\texttt{z0[\%]{\comma}z1[\%]{\comma}v[\%]}}}~~~\\
{\small \Cb{\hspace*{0.5cm}$\bullet$~~\texttt{v0[\%]{\comma}v1[\%]}}}~~~\\
{\small \Cb{\hspace*{0.5cm}$\bullet$~~\texttt{v0[\%]}}}~~~\\
{\small \Cb{\hspace*{0.5cm}$\bullet$~~\texttt{(no arg)}}}\end{flushleft}
Insert shared buffers from (opt. points/rows/planes/channels of) selected images.
~\\Shared buffers cannot be returned by a command{\comma} nor a local environment.
~\\(\emph{eq. to} {\small \texttt{'sh'}}).
\begin{center}\includegraphics[keepaspectratio=true,height=6cm,width=\textwidth]{img/gmic_stdlib33.jpg}\\
{\footnotesize \textbf{Example 33~:} \texttt{image.jpg shared 1 blur[-1] 3 remove[-1]}}
\\\includegraphics[keepaspectratio=true,height=6cm,width=\textwidth]{img/gmic_stdlib34.jpg}\\
{\footnotesize \textbf{Example 34~:} \texttt{image.jpg repeat \{s\} shared 25\%{\comma}75\%{\comma}0{\comma}\$$>$ mirror[-1] x remove[-1] done}}
\end{center}
~\\
~\textbf{Tutorial page: }\\\url{http://gmic.eu/tutorial/\_shared.shtml}


\subsection{\emph{sample\index{sample}} }\vspace*{-0.7em}
~\\\textbf{\Cb{Arguments: }}\begin{flushleft}
{\small \Cb{\hspace*{0.5cm}$\bullet$~~\texttt{\_name1=\{ ? ~$|$~ apples ~$|$~ barbara ~$|$~ boats ~$|$~ bottles ~$|$~ butterfly \-~$|$~ cameraman ~$|$~ car ~$|$~ cat ~$|$~ cliff ~$|$~ david ~$|$~ dog ~$|$~ duck ~$|$~ eagle\- ~$|$~ elephant ~$|$~ earth ~$|$~ flower ~$|$~ fruits ~$|$~ greece ~$|$~ gummy ~$|$~ hou\-se ~$|$~ inside ~$|$~ landscape ~$|$~ leaf ~$|$~ lena ~$|$~ leno ~$|$~ lion ~$|$~ mandri\-ll ~$|$~ monalisa ~$|$~ monkey ~$|$~ parrots ~$|$~ pencils ~$|$~ peppers ~$|$~ roost\-er ~$|$~ rose ~$|$~ square ~$|$~ teddy ~$|$~ tiger ~$|$~ wall ~$|$~ waterfall ~$|$~ zeld\-a \}{\comma}\_name2{\comma}...{\comma}\_nameN{\comma}\_width=\{ $>$=0 ~$|$~ 0 (auto) \}{\comma}\_height = \{ \-$>$=0 ~$|$~ 0 (auto) \}}}}~~~\\
{\small \Cb{\hspace*{0.5cm}$\bullet$~~\texttt{(no arg)}}}\end{flushleft}
Input a new sample RGB image (opt. with specified size).
~\\(\emph{eq. to} {\small \texttt{'sp').\textbackslash n}}).
~\\Argument 'name' can be replaced by an integer which serves as a sample index.
\begin{center}\includegraphics[keepaspectratio=true,height=6cm,width=\textwidth]{img/gmic_stdlib35.jpg}\\
{\footnotesize \textbf{Example 35~:} \texttt{repeat 6 sample done}}
\end{center}

\subsection{\emph{srand\index{srand}} (+)}\vspace*{-0.7em}
~\\\textbf{\Cb{Arguments: }}\begin{flushleft}
{\small \Cb{\hspace*{0.5cm}$\bullet$~~\texttt{value}}}~~~\\
{\small \Cb{\hspace*{0.5cm}$\bullet$~~\texttt{(no arg)}}}\end{flushleft}
Set random generator seed.
~\\If no argument is specified{\comma} a random value is used as the random generator seed.


\subsection{\emph{string\index{string}} }\vspace*{-0.7em}
~\\\textbf{\Cb{Arguments: }}\begin{flushleft}
{\small \Cb{\hspace*{0.5cm}$\bullet$~~\texttt{"string"}}}\end{flushleft}
Insert new image containing the ascii codes of specified string.
\begin{center}\includegraphics[keepaspectratio=true,height=6cm,width=\textwidth]{img/gmic_stdlib36.jpg}\\
{\footnotesize \textbf{Example 36~:} \texttt{string "foo bar"}}
\end{center}

\subsection{\emph{testimage2d\index{testimage2d}} }\vspace*{-0.7em}
~\\\textbf{\Cb{Arguments: }}\begin{flushleft}
{\small \Cb{\hspace*{0.5cm}$\bullet$~~\texttt{\_width$>$0{\comma}\_height$>$0{\comma}\_spectrum$>$0}}}\end{flushleft}
Input a 2d synthetic image.
\begin{flushleft}\Cc{\textbf{Default values}:\\~\\\hspace*{0.5cm}{\small $\bullet$~~\texttt{'width=512'{\comma} 'height=width'} and \texttt{'spectrum=3'.}}}\end{flushleft}
\begin{center}\includegraphics[keepaspectratio=true,height=6cm,width=\textwidth]{img/gmic_stdlib37.jpg}\\
{\footnotesize \textbf{Example 37~:} \texttt{testimage2d 512}}
\end{center}

\subsection{\emph{uncommand\index{uncommand}} (+)}\vspace*{-0.7em}
~\\\textbf{\Cb{Arguments: }}\begin{flushleft}
{\small \Cb{\hspace*{0.5cm}$\bullet$~~\texttt{command\_name[{\comma}\_command\_name2{\comma}...]}}}~~~\\
{\small \Cb{\hspace*{0.5cm}$\bullet$~~\texttt{*}}}\end{flushleft}
Discard definition of specified custom commands.
~\\Set argument to '*' for discarding all existing custom commands.


\subsection{\emph{uniform\_distribution\index{uniform\_distribution}} }\vspace*{-0.7em}
~\\\textbf{\Cb{Arguments: }}\begin{flushleft}
{\small \Cb{\hspace*{0.5cm}$\bullet$~~\texttt{nb\_levels$>$=1{\comma}spectrum$>$=1}}}\end{flushleft}
Input set of uniformly distributed spectrum-d points in [0{\comma}1]\textasciicircum spectrum.
\begin{center}\includegraphics[keepaspectratio=true,height=6cm,width=\textwidth]{img/gmic_stdlib38.jpg}\\
{\footnotesize \textbf{Example 38~:} \texttt{uniform\_distribution 64{\comma}3 * 255 --distribution3d circles3d[-1] 10}}
\end{center}

\subsection{\emph{unserialize\index{unserialize}} (+)}\vspace*{-0.7em}
Recreate lists of images from serialized image buffers{\comma} obtained with command '-serialize'.


\subsection{\emph{update\index{update}} }\vspace*{-0.7em}
Update commands from the latest definition file on the G'MIC server.
~\\(\emph{eq. to} {\small \texttt{'up'}}).


\subsection{\emph{verbose\index{verbose}} (+)}\vspace*{-0.7em}
~\\\textbf{\Cb{Arguments: }}\begin{flushleft}
{\small \Cb{\hspace*{0.5cm}$\bullet$~~\texttt{level}}}~~~\\
{\small \Cb{\hspace*{0.5cm}$\bullet$~~\texttt{\{ + ~$|$~ - \}}}}\end{flushleft}
Set or increment/decrement the verbosity level. Default level is 0.
~\\(\emph{eq. to} {\small \texttt{'v').\textbackslash n}}).
~\\When 'level'$>$=0{\comma} G'MIC log messages are displayed on the standard error (stderr).
\begin{flushleft}\Cc{\textbf{Default value}:\\~\\\hspace*{0.5cm}{\small $\bullet$~~\texttt{'level=0'.}}}\end{flushleft}


\subsection{\emph{wait\index{wait}} (+)}\vspace*{-0.7em}
~\\\textbf{\Cb{Arguments: }}\begin{flushleft}
{\small \Cb{\hspace*{0.5cm}$\bullet$~~\texttt{delay}}}~~~\\
{\small \Cb{\hspace*{0.5cm}$\bullet$~~\texttt{(no arg)}}}\end{flushleft}
Wait for a given delay (in ms){\comma} optionally since the last call to '-wait'.
or wait for a user event occuring on the selected instant display windows.
~\\'delay' can be \{ $<$0=delay+flush events ~$|$~ 0=event ~$|$~ $>$0=delay \}.
~\\Command selection (if any) stands for instant display window indices instead of image indices.
~\\If no window indices are specified and if 'delay' is positive{\comma} the command results
in a 'hard' sleep during specified delay.
\begin{flushleft}\Cc{\textbf{Default value}:\\~\\\hspace*{0.5cm}{\small $\bullet$~~\texttt{'delay=0'.}}}\end{flushleft}


\subsection{\emph{warn\index{warn}} (+)}\vspace*{-0.7em}
~\\\textbf{\Cb{Arguments: }}\begin{flushleft}
{\small \Cb{\hspace*{0.5cm}$\bullet$~~\texttt{\_force\_visible=\{ 0 ~$|$~ 1 \}{\comma}\_message}}}\end{flushleft}
Print specified warning message{\comma} on the standard error (stderr).
~\\Command selection (if any) stands for displayed call stack subset instead of image indices.


\subsection{\emph{window\index{window}} (+)}\vspace*{-0.7em}
~\\\textbf{\Cb{Arguments: }}\begin{flushleft}
{\small \Cb{\hspace*{0.5cm}$\bullet$~~\texttt{\_width[\%]$>$=-1{\comma}\_height[\%]$>$=-1{\comma}\_normalization{\comma}\_fullscreen{\comma}\_pos\-\_x[\%]{\comma}\_pos\_y[\%]{\comma}\_title}}}\end{flushleft}
Display selected images into an instant display window with specified size{\comma} normalization type{\comma}
fullscreen mode and title.
~\\(\emph{eq. to} {\small \texttt{'w').\textbackslash n}}).
~\\If 'width' or 'height' is set to -1{\comma} the corresponding dimension is adjusted to the window
or image size.
~\\When arguments 'pos\_x' and 'pos\_y' are both different than -1{\comma} the window is moved to
the specified coordinates.
~\\'width'=0 or 'height'=0 closes the instant display window.
~\\'normalization' can be \{ -1=keep same ~$|$~ 0=none ~$|$~ 1=always ~$|$~ 2=1st-time ~$|$~ 3=auto \}.
~\\'fullscreen' can be \{ -1=keep same ~$|$~ 0=no ~$|$~ 1=yes \}.
~\\You can manage up to 10 different instant display windows by using the numbered variants
~\\'-w0' (default{\comma} eq. to 'w'){\comma}'-w1'{\comma}...{\comma}'-w9' of the command '-w'.
~\\Invoke '-window' with no selection to make the window visible{\comma} if is has been closed by the user.
\begin{flushleft}\Cc{\textbf{Default values}:\\~\\\hspace*{0.5cm}{\small $\bullet$~~\texttt{'width=height=normalization=fullscreen=-1'} and \texttt{'title=(undefined)'.}}}\end{flushleft}

\section{List manipulation}


\subsection{\emph{keep\index{keep}} (+)}\vspace*{-0.7em}
Keep only selected images.
~\\(\emph{eq. to} {\small \texttt{'k'}}).
\begin{center}\includegraphics[keepaspectratio=true,height=6cm,width=\textwidth]{img/gmic_stdlib39.jpg}\\
{\footnotesize \textbf{Example 39~:} \texttt{image.jpg split x keep[0-50\%:2] append x}}
\\\includegraphics[keepaspectratio=true,height=6cm,width=\textwidth]{img/gmic_stdlib40.jpg}\\
{\footnotesize \textbf{Example 40~:} \texttt{image.jpg split x keep[\textasciicircum 30\%-70\%] append x}}
\end{center}

\subsection{\emph{move\index{move}} (+)}\vspace*{-0.7em}
~\\\textbf{\Cb{Arguments: }}\begin{flushleft}
{\small \Cb{\hspace*{0.5cm}$\bullet$~~\texttt{position[\%]}}}\end{flushleft}
Move selected images at specified position.
~\\(\emph{eq. to} {\small \texttt{'mv'}}).
\begin{center}\includegraphics[keepaspectratio=true,height=6cm,width=\textwidth]{img/gmic_stdlib41.jpg}\\
{\footnotesize \textbf{Example 41~:} \texttt{image.jpg split x{\comma}3 move[1] 0}}
\\\includegraphics[keepaspectratio=true,height=6cm,width=\textwidth]{img/gmic_stdlib42.jpg}\\
{\footnotesize \textbf{Example 42~:} \texttt{image.jpg split x move[50\%--1:2] 0 append x}}
\end{center}

\subsection{\emph{name\index{name}} (+)}\vspace*{-0.7em}
~\\\textbf{\Cb{Arguments: }}\begin{flushleft}
{\small \Cb{\hspace*{0.5cm}$\bullet$~~\texttt{"name"}}}\end{flushleft}
Set name of selected images.
~\\(\emph{eq. to} {\small \texttt{'nm'}}).
\begin{center}\includegraphics[keepaspectratio=true,height=6cm,width=\textwidth]{img/gmic_stdlib43.jpg}\\
{\footnotesize \textbf{Example 43~:} \texttt{image.jpg -name image blur[image] 2}}
\end{center}
~\\
~\textbf{Tutorial page: }\\\url{http://gmic.eu/tutorial/\_name.shtml}


\subsection{\emph{names\index{names}} }\vspace*{-0.7em}
~\\\textbf{\Cb{Arguments: }}\begin{flushleft}
{\small \Cb{\hspace*{0.5cm}$\bullet$~~\texttt{name1{\comma}name2{\comma}...{\comma}nameN}}}\end{flushleft}
Set each name of (multiple) selected images from the sequence of the provided arguments.
~\\(\emph{eq. to} {\small \texttt{'nms'}}).


\subsection{\emph{remove\index{remove}} (+)}\vspace*{-0.7em}
Remove selected images.
~\\(\emph{eq. to} {\small \texttt{'rm'}}).
\begin{center}\includegraphics[keepaspectratio=true,height=6cm,width=\textwidth]{img/gmic_stdlib44.jpg}\\
{\footnotesize \textbf{Example 44~:} \texttt{image.jpg split x remove[30\%-70\%] append x}}
\\\includegraphics[keepaspectratio=true,height=6cm,width=\textwidth]{img/gmic_stdlib45.jpg}\\
{\footnotesize \textbf{Example 45~:} \texttt{image.jpg split x remove[0-50\%:2] append x}}
\end{center}

\subsection{\emph{remove\_duplicates\index{remove\_duplicates}} }\vspace*{-0.7em}
Remove duplicates images in the selected images list.
\begin{center}\includegraphics[keepaspectratio=true,height=6cm,width=\textwidth]{img/gmic_stdlib46.jpg}\\
{\footnotesize \textbf{Example 46~:} \texttt{(1{\comma}2{\comma}3{\comma}4{\comma}2{\comma}4{\comma}3{\comma}1{\comma}3{\comma}4{\comma}2{\comma}1) split x remove\_duplicates append x}}
\end{center}

\subsection{\emph{remove\_empty\index{remove\_empty}} }\vspace*{-0.7em}
Remove empty images in the selected image list.


\subsection{\emph{reverse\index{reverse}} (+)}\vspace*{-0.7em}
Reverse positions of selected images.
~\\(\emph{eq. to} {\small \texttt{'rv'}}).
\begin{center}\includegraphics[keepaspectratio=true,height=6cm,width=\textwidth]{img/gmic_stdlib47.jpg}\\
{\footnotesize \textbf{Example 47~:} \texttt{image.jpg split x{\comma}3 reverse[-2{\comma}-1]}}
\\\includegraphics[keepaspectratio=true,height=6cm,width=\textwidth]{img/gmic_stdlib48.jpg}\\
{\footnotesize \textbf{Example 48~:} \texttt{image.jpg split x{\comma}-16 reverse[50\%-100\%] append x}}
\end{center}

\subsection{\emph{sort\_list\index{sort\_list}} }\vspace*{-0.7em}
~\\\textbf{\Cb{Arguments: }}\begin{flushleft}
{\small \Cb{\hspace*{0.5cm}$\bullet$~~\texttt{\_ordering=\{ + ~$|$~ - \}{\comma}\_criterion}}}\end{flushleft}
Sort list of selected images according to the specified image criterion.
\begin{flushleft}\Cc{\textbf{Default values}:\\~\\\hspace*{0.5cm}{\small $\bullet$~~\texttt{'ordering=+'{\comma} 'criterion=i'.}}}\end{flushleft}
\begin{center}\includegraphics[keepaspectratio=true,height=6cm,width=\textwidth]{img/gmic_stdlib49.jpg}\\
{\footnotesize \textbf{Example 49~:} \texttt{(1;4;7;3;9;2;4;7;6;3;9;1;0;3;3;2) split y sort\_list +{\comma}i append y}}
\end{center}

\subsection{\emph{sort\_str\index{sort\_str}} }\vspace*{-0.7em}
Sort selected images (viewed as a list of strings) in lexicographic order.

\section{Mathematical operators}


\subsection{\emph{abs\index{abs}} (+)}\vspace*{-0.7em}
Compute the pointwise absolute values of selected images.
\begin{center}\includegraphics[keepaspectratio=true,height=6cm,width=\textwidth]{img/gmic_stdlib50.jpg}\\
{\footnotesize \textbf{Example 50~:} \texttt{image.jpg --sub \{ia\} abs[-1]}}
\\\includegraphics[keepaspectratio=true,height=6cm,width=\textwidth]{img/gmic_stdlib51.jpg}\\
{\footnotesize \textbf{Example 51~:} \texttt{300{\comma}1{\comma}1{\comma}1{\comma}'cos(20*x/w)' --abs display\_graph 400{\comma}300}}
\end{center}

\subsection{\emph{acos\index{acos}} (+)}\vspace*{-0.7em}
Compute the pointwise arc-cosine of selected images.
\begin{center}\includegraphics[keepaspectratio=true,height=6cm,width=\textwidth]{img/gmic_stdlib52.jpg}\\
{\footnotesize \textbf{Example 52~:} \texttt{image.jpg --normalize -1{\comma}1 acos[-1]}}
\\\includegraphics[keepaspectratio=true,height=6cm,width=\textwidth]{img/gmic_stdlib53.jpg}\\
{\footnotesize \textbf{Example 53~:} \texttt{300{\comma}1{\comma}1{\comma}1{\comma}'x/w+0.1*u' --acos display\_graph 400{\comma}300}}
\end{center}
~\\
~\textbf{Tutorial page: }\\\url{http://gmic.eu/tutorial/trigometric-and-inverse-trigometric-commands.shtml}


\subsection{\emph{add\index{add}} (+)}\vspace*{-0.7em}
~\\\textbf{\Cb{Arguments: }}\begin{flushleft}
{\small \Cb{\hspace*{0.5cm}$\bullet$~~\texttt{value[\%]}}}~~~\\
{\small \Cb{\hspace*{0.5cm}$\bullet$~~\texttt{[image]}}}~~~\\
{\small \Cb{\hspace*{0.5cm}$\bullet$~~\texttt{'formula'}}}~~~\\
{\small \Cb{\hspace*{0.5cm}$\bullet$~~\texttt{(no arg)}}}\end{flushleft}
Add specified value{\comma} image or mathematical expression to selected images{\comma}
or compute the pointwise sum of selected images.
~\\(\emph{eq. to} {\small \texttt{'+'}}).
\begin{center}\includegraphics[keepaspectratio=true,height=6cm,width=\textwidth]{img/gmic_stdlib54.jpg}\\
{\footnotesize \textbf{Example 54~:} \texttt{image.jpg --add 30\% cut 0{\comma}255}}
\\\includegraphics[keepaspectratio=true,height=6cm,width=\textwidth]{img/gmic_stdlib55.jpg}\\
{\footnotesize \textbf{Example 55~:} \texttt{image.jpg --blur 5 normalize 0{\comma}255 add[1] [0]}}
\\\includegraphics[keepaspectratio=true,height=6cm,width=\textwidth]{img/gmic_stdlib56.jpg}\\
{\footnotesize \textbf{Example 56~:} \texttt{image.jpg add '80*cos(80*(x/w-0.5)*(y/w-0.5)+c)' cut 0{\comma}255}}
\\\includegraphics[keepaspectratio=true,height=6cm,width=\textwidth]{img/gmic_stdlib57.jpg}\\
{\footnotesize \textbf{Example 57~:} \texttt{image.jpg repeat 9 --rotate[0] \{\$$>$*36\}{\comma}1{\comma}0{\comma}50\%{\comma}50\% done add div 10}}
\end{center}

\subsection{\emph{and\index{and}} (+)}\vspace*{-0.7em}
~\\\textbf{\Cb{Arguments: }}\begin{flushleft}
{\small \Cb{\hspace*{0.5cm}$\bullet$~~\texttt{value[\%]}}}~~~\\
{\small \Cb{\hspace*{0.5cm}$\bullet$~~\texttt{[image]}}}~~~\\
{\small \Cb{\hspace*{0.5cm}$\bullet$~~\texttt{'formula'}}}~~~\\
{\small \Cb{\hspace*{0.5cm}$\bullet$~~\texttt{(no arg)}}}\end{flushleft}
Compute the bitwise AND of selected images with specified value{\comma} image or mathematical
expression{\comma} or compute the pointwise sequential bitwise AND of selected images.
~\\(\emph{eq. to} {\small \texttt{'\&'}}).
\begin{center}\includegraphics[keepaspectratio=true,height=6cm,width=\textwidth]{img/gmic_stdlib58.jpg}\\
{\footnotesize \textbf{Example 58~:} \texttt{image.jpg and \{128+64\}}}
\\\includegraphics[keepaspectratio=true,height=6cm,width=\textwidth]{img/gmic_stdlib59.jpg}\\
{\footnotesize \textbf{Example 59~:} \texttt{image.jpg --mirror x and}}
\end{center}

\subsection{\emph{argmax\index{argmax}} }\vspace*{-0.7em}
Compute the argmax of selected images. Returns a single image
with each pixel value being the indice of the input image with maximal value.
\begin{center}\includegraphics[keepaspectratio=true,height=6cm,width=\textwidth]{img/gmic_stdlib60.jpg}\\
{\footnotesize \textbf{Example 60~:} \texttt{image.jpg sample lena{\comma}lion{\comma}square --argmax}}
\end{center}

\subsection{\emph{argmin\index{argmin}} }\vspace*{-0.7em}
Compute the argmin of selected images. Returns a single image
with each pixel value being the indice of the input image with minimal value.
\begin{center}\includegraphics[keepaspectratio=true,height=6cm,width=\textwidth]{img/gmic_stdlib61.jpg}\\
{\footnotesize \textbf{Example 61~:} \texttt{image.jpg sample lena{\comma}lion{\comma}square --argmin}}
\end{center}

\subsection{\emph{asin\index{asin}} (+)}\vspace*{-0.7em}
Compute the pointwise arc-sine of selected images.
\begin{center}\includegraphics[keepaspectratio=true,height=6cm,width=\textwidth]{img/gmic_stdlib62.jpg}\\
{\footnotesize \textbf{Example 62~:} \texttt{image.jpg --normalize -1{\comma}1 -asin[-1]}}
\\\includegraphics[keepaspectratio=true,height=6cm,width=\textwidth]{img/gmic_stdlib63.jpg}\\
{\footnotesize \textbf{Example 63~:} \texttt{300{\comma}1{\comma}1{\comma}1{\comma}'x/w+0.1*u' --asin display\_graph 400{\comma}300}}
\end{center}
~\\
~\textbf{Tutorial page: }\\\url{http://gmic.eu/tutorial/trigometric-and-inverse-trigometric-commands.shtml}


\subsection{\emph{atan\index{atan}} (+)}\vspace*{-0.7em}
Compute the pointwise arc-tangent of selected images.
\begin{center}\includegraphics[keepaspectratio=true,height=6cm,width=\textwidth]{img/gmic_stdlib64.jpg}\\
{\footnotesize \textbf{Example 64~:} \texttt{image.jpg --normalize 0{\comma}8 atan[-1]}}
\\\includegraphics[keepaspectratio=true,height=6cm,width=\textwidth]{img/gmic_stdlib65.jpg}\\
{\footnotesize \textbf{Example 65~:} \texttt{300{\comma}1{\comma}1{\comma}1{\comma}'4*x/w+u' --atan display\_graph 400{\comma}300}}
\end{center}
~\\
~\textbf{Tutorial page: }\\\url{http://gmic.eu/tutorial/trigometric-and-inverse-trigometric-commands.shtml}


\subsection{\emph{atan2\index{atan2}} (+)}\vspace*{-0.7em}
~\\\textbf{\Cb{Arguments: }}\begin{flushleft}
{\small \Cb{\hspace*{0.5cm}$\bullet$~~\texttt{[x\_argument]}}}\end{flushleft}
Compute the pointwise oriented arc-tangent of selected images.
~\\Each selected image is regarded as the y-argument of the arc-tangent function{\comma} while the
specified image gives the corresponding x-argument.
\begin{center}\includegraphics[keepaspectratio=true,height=6cm,width=\textwidth]{img/gmic_stdlib66.jpg}\\
{\footnotesize \textbf{Example 66~:} \texttt{(-1{\comma}1) (-1;1) resize 400{\comma}400{\comma}1{\comma}1{\comma}3 atan2[1] [0] keep[1] mod \{pi/8\}}}
\end{center}
~\\
~\textbf{Tutorial page: }\\\url{http://gmic.eu/tutorial/trigometric-and-inverse-trigometric-commands.shtml}


\subsection{\emph{bsl\index{bsl}} (+)}\vspace*{-0.7em}
~\\\textbf{\Cb{Arguments: }}\begin{flushleft}
{\small \Cb{\hspace*{0.5cm}$\bullet$~~\texttt{value[\%]}}}~~~\\
{\small \Cb{\hspace*{0.5cm}$\bullet$~~\texttt{[image]}}}~~~\\
{\small \Cb{\hspace*{0.5cm}$\bullet$~~\texttt{'formula'}}}~~~\\
{\small \Cb{\hspace*{0.5cm}$\bullet$~~\texttt{(no arg)}}}\end{flushleft}
Compute the bitwise left shift of selected images with specified value{\comma} image or
mathematical expression{\comma} or compute the pointwise sequential bitwise left shift of
selected images.
~\\(\emph{eq. to} {\small \texttt{'$<$$<$'}}).
\begin{center}\includegraphics[keepaspectratio=true,height=6cm,width=\textwidth]{img/gmic_stdlib67.jpg}\\
{\footnotesize \textbf{Example 67~:} \texttt{image.jpg bsl 'round(3*x/w{\comma}0)' cut 0{\comma}255}}
\end{center}

\subsection{\emph{bsr\index{bsr}} (+)}\vspace*{-0.7em}
~\\\textbf{\Cb{Arguments: }}\begin{flushleft}
{\small \Cb{\hspace*{0.5cm}$\bullet$~~\texttt{value[\%]}}}~~~\\
{\small \Cb{\hspace*{0.5cm}$\bullet$~~\texttt{[image]}}}~~~\\
{\small \Cb{\hspace*{0.5cm}$\bullet$~~\texttt{'formula'}}}~~~\\
{\small \Cb{\hspace*{0.5cm}$\bullet$~~\texttt{(no arg)}}}\end{flushleft}
Compute the bitwise right shift of selected images with specified value{\comma} image or"
mathematical expression{\comma} or compute the pointwise sequential bitwise right shift of
selected images.
~\\(\emph{eq. to} {\small \texttt{'$>$$>$'}}).
\begin{center}\includegraphics[keepaspectratio=true,height=6cm,width=\textwidth]{img/gmic_stdlib68.jpg}\\
{\footnotesize \textbf{Example 68~:} \texttt{image.jpg bsr 'round(3*x/w{\comma}0)' cut 0{\comma}255}}
\end{center}

\subsection{\emph{cos\index{cos}} (+)}\vspace*{-0.7em}
Compute the pointwise cosine of selected images.
\begin{center}\includegraphics[keepaspectratio=true,height=6cm,width=\textwidth]{img/gmic_stdlib69.jpg}\\
{\footnotesize \textbf{Example 69~:} \texttt{image.jpg --normalize 0{\comma}\{2*pi\} cos[-1]}}
\\\includegraphics[keepaspectratio=true,height=6cm,width=\textwidth]{img/gmic_stdlib70.jpg}\\
{\footnotesize \textbf{Example 70~:} \texttt{300{\comma}1{\comma}1{\comma}1{\comma}'20*x/w+u' --cos display\_graph 400{\comma}300}}
\end{center}
~\\
~\textbf{Tutorial page: }\\\url{http://gmic.eu/tutorial/trigometric-and-inverse-trigometric-commands.shtml}


\subsection{\emph{cosh\index{cosh}} (+)}\vspace*{-0.7em}
Compute the pointwise hyperbolic cosine of selected images.
\begin{center}\includegraphics[keepaspectratio=true,height=6cm,width=\textwidth]{img/gmic_stdlib71.jpg}\\
{\footnotesize \textbf{Example 71~:} \texttt{image.jpg --normalize -3{\comma}3 cosh[-1]}}
\\\includegraphics[keepaspectratio=true,height=6cm,width=\textwidth]{img/gmic_stdlib72.jpg}\\
{\footnotesize \textbf{Example 72~:} \texttt{300{\comma}1{\comma}1{\comma}1{\comma}'4*x/w+u' --cosh display\_graph 400{\comma}300}}
\end{center}

\subsection{\emph{div\index{div}} (+)}\vspace*{-0.7em}
~\\\textbf{\Cb{Arguments: }}\begin{flushleft}
{\small \Cb{\hspace*{0.5cm}$\bullet$~~\texttt{value[\%]}}}~~~\\
{\small \Cb{\hspace*{0.5cm}$\bullet$~~\texttt{[image]}}}~~~\\
{\small \Cb{\hspace*{0.5cm}$\bullet$~~\texttt{'formula'}}}~~~\\
{\small \Cb{\hspace*{0.5cm}$\bullet$~~\texttt{(no arg)}}}\end{flushleft}
Divide selected image by specified value{\comma} image or mathematical expression{\comma}
or compute the pointwise quotient of selected images.
~\\(\emph{eq. to} {\small \texttt{'/'}}).
\begin{center}\includegraphics[keepaspectratio=true,height=6cm,width=\textwidth]{img/gmic_stdlib73.jpg}\\
{\footnotesize \textbf{Example 73~:} \texttt{image.jpg div '1+abs(cos(x/10)*sin(y/10))'}}
\\\includegraphics[keepaspectratio=true,height=6cm,width=\textwidth]{img/gmic_stdlib74.jpg}\\
{\footnotesize \textbf{Example 74~:} \texttt{image.jpg --norm add[-1] 1 --div}}
\end{center}

\subsection{\emph{div\_complex\index{div\_complex}} }\vspace*{-0.7em}
~\\\textbf{\Cb{Arguments: }}\begin{flushleft}
{\small \Cb{\hspace*{0.5cm}$\bullet$~~\texttt{[divider\_real{\comma}divider\_imag]{\comma}\_epsilon$>$=0}}}\end{flushleft}
Perform division of the selected complex pairs (real1{\comma}imag1{\comma}...{\comma}realN{\comma}imagN) of images by specified complex pair of images (divider\_real{\comma}divider\_imag).
~\\In complex pairs{\comma} the real image must be always located before the imaginary image in the image list.
\begin{flushleft}\Cc{\textbf{Default value}:\\~\\\hspace*{0.5cm}{\small $\bullet$~~\texttt{'epsilon=1e-8'.}}}\end{flushleft}


\subsection{\emph{eq\index{eq}} (+)}\vspace*{-0.7em}
~\\\textbf{\Cb{Arguments: }}\begin{flushleft}
{\small \Cb{\hspace*{0.5cm}$\bullet$~~\texttt{value[\%]}}}~~~\\
{\small \Cb{\hspace*{0.5cm}$\bullet$~~\texttt{[image]}}}~~~\\
{\small \Cb{\hspace*{0.5cm}$\bullet$~~\texttt{'formula'}}}~~~\\
{\small \Cb{\hspace*{0.5cm}$\bullet$~~\texttt{(no arg)}}}\end{flushleft}
Compute the boolean equality of selected images with specified value{\comma} image or
mathematical expression{\comma} or compute the boolean equality of selected images.
~\\(\emph{eq. to} {\small \texttt{'=='}}).
\begin{center}\includegraphics[keepaspectratio=true,height=6cm,width=\textwidth]{img/gmic_stdlib75.jpg}\\
{\footnotesize \textbf{Example 75~:} \texttt{image.jpg round 40 eq \{round(ia{\comma}40)\}}}
\\\includegraphics[keepaspectratio=true,height=6cm,width=\textwidth]{img/gmic_stdlib76.jpg}\\
{\footnotesize \textbf{Example 76~:} \texttt{image.jpg --mirror x eq}}
\end{center}

\subsection{\emph{exp\index{exp}} (+)}\vspace*{-0.7em}
Compute the pointwise exponential of selected images.
\begin{center}\includegraphics[keepaspectratio=true,height=6cm,width=\textwidth]{img/gmic_stdlib77.jpg}\\
{\footnotesize \textbf{Example 77~:} \texttt{image.jpg --normalize 0{\comma}2 exp[-1]}}
\\\includegraphics[keepaspectratio=true,height=6cm,width=\textwidth]{img/gmic_stdlib78.jpg}\\
{\footnotesize \textbf{Example 78~:} \texttt{300{\comma}1{\comma}1{\comma}1{\comma}'7*x/w+u' --exp display\_graph 400{\comma}300}}
\end{center}

\subsection{\emph{ge\index{ge}} (+)}\vspace*{-0.7em}
~\\\textbf{\Cb{Arguments: }}\begin{flushleft}
{\small \Cb{\hspace*{0.5cm}$\bullet$~~\texttt{value[\%]}}}~~~\\
{\small \Cb{\hspace*{0.5cm}$\bullet$~~\texttt{[image]}}}~~~\\
{\small \Cb{\hspace*{0.5cm}$\bullet$~~\texttt{'formula'}}}~~~\\
{\small \Cb{\hspace*{0.5cm}$\bullet$~~\texttt{(no arg)}}}\end{flushleft}
Compute the boolean 'greater or equal than' of selected images with specified value{\comma} image
or mathematical expression{\comma} or compute the boolean 'greater or equal than' of selected images.
~\\(\emph{eq. to} {\small \texttt{'$>$='}}).
\begin{center}\includegraphics[keepaspectratio=true,height=6cm,width=\textwidth]{img/gmic_stdlib79.jpg}\\
{\footnotesize \textbf{Example 79~:} \texttt{image.jpg ge \{ia\}}}
\\\includegraphics[keepaspectratio=true,height=6cm,width=\textwidth]{img/gmic_stdlib80.jpg}\\
{\footnotesize \textbf{Example 80~:} \texttt{image.jpg --mirror x ge}}
\end{center}

\subsection{\emph{gt\index{gt}} (+)}\vspace*{-0.7em}
~\\\textbf{\Cb{Arguments: }}\begin{flushleft}
{\small \Cb{\hspace*{0.5cm}$\bullet$~~\texttt{value[\%]}}}~~~\\
{\small \Cb{\hspace*{0.5cm}$\bullet$~~\texttt{[image]}}}~~~\\
{\small \Cb{\hspace*{0.5cm}$\bullet$~~\texttt{'formula'}}}~~~\\
{\small \Cb{\hspace*{0.5cm}$\bullet$~~\texttt{(no arg)}}}\end{flushleft}
Compute the boolean 'greater than' of selected images with specified value{\comma} image or
mathematical expression{\comma} or compute the boolean 'greater than' of selected images.
~\\(\emph{eq. to} {\small \texttt{'$>$'}}).
\begin{center}\includegraphics[keepaspectratio=true,height=6cm,width=\textwidth]{img/gmic_stdlib81.jpg}\\
{\footnotesize \textbf{Example 81~:} \texttt{image.jpg gt \{ia\}}}
\\\includegraphics[keepaspectratio=true,height=6cm,width=\textwidth]{img/gmic_stdlib82.jpg}\\
{\footnotesize \textbf{Example 82~:} \texttt{image.jpg --mirror x gt}}
\end{center}

\subsection{\emph{le\index{le}} (+)}\vspace*{-0.7em}
~\\\textbf{\Cb{Arguments: }}\begin{flushleft}
{\small \Cb{\hspace*{0.5cm}$\bullet$~~\texttt{value[\%]}}}~~~\\
{\small \Cb{\hspace*{0.5cm}$\bullet$~~\texttt{[image]}}}~~~\\
{\small \Cb{\hspace*{0.5cm}$\bullet$~~\texttt{'formula'}}}~~~\\
{\small \Cb{\hspace*{0.5cm}$\bullet$~~\texttt{(no arg)}}}\end{flushleft}
Compute the boolean 'less or equal than' of selected images with specified value{\comma} image or
mathematical expression{\comma} or compute the boolean 'less or equal than' of selected images.
~\\(\emph{eq. to} {\small \texttt{'$<$='}}).
\begin{center}\includegraphics[keepaspectratio=true,height=6cm,width=\textwidth]{img/gmic_stdlib83.jpg}\\
{\footnotesize \textbf{Example 83~:} \texttt{image.jpg le \{ia\}}}
\\\includegraphics[keepaspectratio=true,height=6cm,width=\textwidth]{img/gmic_stdlib84.jpg}\\
{\footnotesize \textbf{Example 84~:} \texttt{image.jpg --mirror x le}}
\end{center}

\subsection{\emph{lt\index{lt}} (+)}\vspace*{-0.7em}
~\\\textbf{\Cb{Arguments: }}\begin{flushleft}
{\small \Cb{\hspace*{0.5cm}$\bullet$~~\texttt{value[\%]}}}~~~\\
{\small \Cb{\hspace*{0.5cm}$\bullet$~~\texttt{[image]}}}~~~\\
{\small \Cb{\hspace*{0.5cm}$\bullet$~~\texttt{'formula'}}}~~~\\
{\small \Cb{\hspace*{0.5cm}$\bullet$~~\texttt{(no arg)}}}\end{flushleft}
Compute the boolean 'less than' of selected images with specified value{\comma} image or
mathematical expression{\comma} or compute the boolean 'less than' of selected images.
~\\(\emph{eq. to} {\small \texttt{'$<$'}}).
\begin{center}\includegraphics[keepaspectratio=true,height=6cm,width=\textwidth]{img/gmic_stdlib85.jpg}\\
{\footnotesize \textbf{Example 85~:} \texttt{image.jpg lt \{ia\}}}
\\\includegraphics[keepaspectratio=true,height=6cm,width=\textwidth]{img/gmic_stdlib86.jpg}\\
{\footnotesize \textbf{Example 86~:} \texttt{image.jpg --mirror x lt}}
\end{center}

\subsection{\emph{log\index{log}} (+)}\vspace*{-0.7em}
Compute the pointwise base-e logarithm of selected images.
\begin{center}\includegraphics[keepaspectratio=true,height=6cm,width=\textwidth]{img/gmic_stdlib87.jpg}\\
{\footnotesize \textbf{Example 87~:} \texttt{image.jpg --add 1 log[-1]}}
\\\includegraphics[keepaspectratio=true,height=6cm,width=\textwidth]{img/gmic_stdlib88.jpg}\\
{\footnotesize \textbf{Example 88~:} \texttt{300{\comma}1{\comma}1{\comma}1{\comma}'7*x/w+u' --log display\_graph 400{\comma}300}}
\end{center}

\subsection{\emph{log10\index{log10}} (+)}\vspace*{-0.7em}
Compute the pointwise base-10 logarithm of selected images.
\begin{center}\includegraphics[keepaspectratio=true,height=6cm,width=\textwidth]{img/gmic_stdlib89.jpg}\\
{\footnotesize \textbf{Example 89~:} \texttt{image.jpg --add 1 log10[-1]}}
\\\includegraphics[keepaspectratio=true,height=6cm,width=\textwidth]{img/gmic_stdlib90.jpg}\\
{\footnotesize \textbf{Example 90~:} \texttt{300{\comma}1{\comma}1{\comma}1{\comma}'7*x/w+u' --log10 display\_graph 400{\comma}300}}
\end{center}

\subsection{\emph{log2\index{log2}} (+)}\vspace*{-0.7em}
Compute the pointwise base-2 logarithm of selected images
\begin{center}\includegraphics[keepaspectratio=true,height=6cm,width=\textwidth]{img/gmic_stdlib91.jpg}\\
{\footnotesize \textbf{Example 91~:} \texttt{image.jpg --add 1 log2[-1]}}
\\\includegraphics[keepaspectratio=true,height=6cm,width=\textwidth]{img/gmic_stdlib92.jpg}\\
{\footnotesize \textbf{Example 92~:} \texttt{300{\comma}1{\comma}1{\comma}1{\comma}'7*x/w+u' --log2 display\_graph 400{\comma}300}}
\end{center}

\subsection{\emph{max\index{max}} (+)}\vspace*{-0.7em}
~\\\textbf{\Cb{Arguments: }}\begin{flushleft}
{\small \Cb{\hspace*{0.5cm}$\bullet$~~\texttt{value[\%]}}}~~~\\
{\small \Cb{\hspace*{0.5cm}$\bullet$~~\texttt{[image]}}}~~~\\
{\small \Cb{\hspace*{0.5cm}$\bullet$~~\texttt{'formula'}}}~~~\\
{\small \Cb{\hspace*{0.5cm}$\bullet$~~\texttt{(no arg)}}}\end{flushleft}
Compute the maximum between selected images and specified value{\comma} image or
mathematical expression{\comma} or compute the pointwise maxima between selected images.
\begin{center}\includegraphics[keepaspectratio=true,height=6cm,width=\textwidth]{img/gmic_stdlib93.jpg}\\
{\footnotesize \textbf{Example 93~:} \texttt{image.jpg --mirror x max}}
\\\includegraphics[keepaspectratio=true,height=6cm,width=\textwidth]{img/gmic_stdlib94.jpg}\\
{\footnotesize \textbf{Example 94~:} \texttt{image.jpg max 'R=((x/w-0.5)\textasciicircum 2+(y/h-0.5)\textasciicircum 2)\textasciicircum 0.5;255*R'}}
\end{center}

\subsection{\emph{mdiv\index{mdiv}} (+)}\vspace*{-0.7em}
~\\\textbf{\Cb{Arguments: }}\begin{flushleft}
{\small \Cb{\hspace*{0.5cm}$\bullet$~~\texttt{value[\%]}}}~~~\\
{\small \Cb{\hspace*{0.5cm}$\bullet$~~\texttt{[image]}}}~~~\\
{\small \Cb{\hspace*{0.5cm}$\bullet$~~\texttt{'formula'}}}~~~\\
{\small \Cb{\hspace*{0.5cm}$\bullet$~~\texttt{(no arg)}}}\end{flushleft}
Compute the matrix division of selected matrices/vectors by specified value{\comma} image or
mathematical expression{\comma} or compute the matrix division of selected images.
~\\(\emph{eq. to} {\small \texttt{'m/'}}).


\subsection{\emph{min\index{min}} (+)}\vspace*{-0.7em}
~\\\textbf{\Cb{Arguments: }}\begin{flushleft}
{\small \Cb{\hspace*{0.5cm}$\bullet$~~\texttt{value[\%]}}}~~~\\
{\small \Cb{\hspace*{0.5cm}$\bullet$~~\texttt{[image]}}}~~~\\
{\small \Cb{\hspace*{0.5cm}$\bullet$~~\texttt{'formula'}}}~~~\\
{\small \Cb{\hspace*{0.5cm}$\bullet$~~\texttt{(no arg)}}}\end{flushleft}
Compute the minimum between selected images and specified value{\comma} image or
mathematical expression{\comma} or compute the pointwise minima between selected images.
\begin{center}\includegraphics[keepaspectratio=true,height=6cm,width=\textwidth]{img/gmic_stdlib95.jpg}\\
{\footnotesize \textbf{Example 95~:} \texttt{image.jpg --mirror x min}}
\\\includegraphics[keepaspectratio=true,height=6cm,width=\textwidth]{img/gmic_stdlib96.jpg}\\
{\footnotesize \textbf{Example 96~:} \texttt{image.jpg min 'R=((x/w-0.5)\textasciicircum 2+(y/h-0.5)\textasciicircum 2)\textasciicircum 0.5;255*R'}}
\end{center}

\subsection{\emph{mod\index{mod}} (+)}\vspace*{-0.7em}
~\\\textbf{\Cb{Arguments: }}\begin{flushleft}
{\small \Cb{\hspace*{0.5cm}$\bullet$~~\texttt{value[\%]}}}~~~\\
{\small \Cb{\hspace*{0.5cm}$\bullet$~~\texttt{[image]}}}~~~\\
{\small \Cb{\hspace*{0.5cm}$\bullet$~~\texttt{'formula'}}}~~~\\
{\small \Cb{\hspace*{0.5cm}$\bullet$~~\texttt{(no arg)}}}\end{flushleft}
Compute the modulo of selected images with specified value{\comma} image or mathematical
expression{\comma} or compute the pointwise sequential modulo of selected images.
~\\(\emph{eq. to} {\small \texttt{'\%'}}).
\begin{center}\includegraphics[keepaspectratio=true,height=6cm,width=\textwidth]{img/gmic_stdlib97.jpg}\\
{\footnotesize \textbf{Example 97~:} \texttt{image.jpg --mirror x mod}}
\\\includegraphics[keepaspectratio=true,height=6cm,width=\textwidth]{img/gmic_stdlib98.jpg}\\
{\footnotesize \textbf{Example 98~:} \texttt{image.jpg mod 'R=((x/w-0.5)\textasciicircum 2+(y/h-0.5)\textasciicircum 2)\textasciicircum 0.5;255*R'}}
\end{center}

\subsection{\emph{mmul\index{mmul}} (+)}\vspace*{-0.7em}
~\\\textbf{\Cb{Arguments: }}\begin{flushleft}
{\small \Cb{\hspace*{0.5cm}$\bullet$~~\texttt{value[\%]}}}~~~\\
{\small \Cb{\hspace*{0.5cm}$\bullet$~~\texttt{[image]}}}~~~\\
{\small \Cb{\hspace*{0.5cm}$\bullet$~~\texttt{'formula'}}}~~~\\
{\small \Cb{\hspace*{0.5cm}$\bullet$~~\texttt{(no arg)}}}\end{flushleft}
Compute the matrix right multiplication of selected matrices/vectors by specified value{\comma} image or
mathematical expression{\comma} or compute the matrix right multiplication of selected images.
~\\(\emph{eq. to} {\small \texttt{'m*'}}).
\begin{center}\includegraphics[keepaspectratio=true,height=6cm,width=\textwidth]{img/gmic_stdlib99.jpg}\\
{\footnotesize \textbf{Example 99~:} \texttt{(0{\comma}1{\comma}0;0{\comma}0{\comma}1;1{\comma}0{\comma}0) (1;2;3) --mmul}}
\end{center}

\subsection{\emph{mul\index{mul}} (+)}\vspace*{-0.7em}
~\\\textbf{\Cb{Arguments: }}\begin{flushleft}
{\small \Cb{\hspace*{0.5cm}$\bullet$~~\texttt{value[\%]}}}~~~\\
{\small \Cb{\hspace*{0.5cm}$\bullet$~~\texttt{[image]}}}~~~\\
{\small \Cb{\hspace*{0.5cm}$\bullet$~~\texttt{'formula'}}}~~~\\
{\small \Cb{\hspace*{0.5cm}$\bullet$~~\texttt{(no arg)}}}\end{flushleft}
Multiply selected images by specified value{\comma} image or mathematical expression{\comma}
or compute the pointwise product of selected images.
~\\(\emph{eq. to} {\small \texttt{'*'}}).
\begin{center}\includegraphics[keepaspectratio=true,height=6cm,width=\textwidth]{img/gmic_stdlib100.jpg}\\
{\footnotesize \textbf{Example 100~:} \texttt{image.jpg --mul 2 cut 0{\comma}255}}
\\\includegraphics[keepaspectratio=true,height=6cm,width=\textwidth]{img/gmic_stdlib101.jpg}\\
{\footnotesize \textbf{Example 101~:} \texttt{image.jpg (1{\comma}2{\comma}3{\comma}4{\comma}5{\comma}6{\comma}7{\comma}8) resize[-1] [0] mul[0] [-1]}}
\\\includegraphics[keepaspectratio=true,height=6cm,width=\textwidth]{img/gmic_stdlib102.jpg}\\
{\footnotesize \textbf{Example 102~:} \texttt{image.jpg mul '1-3*abs(x/w-0.5)' cut 0{\comma}255}}
\\\includegraphics[keepaspectratio=true,height=6cm,width=\textwidth]{img/gmic_stdlib103.jpg}\\
{\footnotesize \textbf{Example 103~:} \texttt{image.jpg --luminance negate[-1] --mul}}
\end{center}

\subsection{\emph{mul\_channels\index{mul\_channels}} }\vspace*{-0.7em}
~\\\textbf{\Cb{Arguments: }}\begin{flushleft}
{\small \Cb{\hspace*{0.5cm}$\bullet$~~\texttt{value1{\comma}\_value2{\comma}...{\comma}\_valueN}}}\end{flushleft}
Multiply channels of selected images by specified sequence of values.
\begin{center}\includegraphics[keepaspectratio=true,height=6cm,width=\textwidth]{img/gmic_stdlib104.jpg}\\
{\footnotesize \textbf{Example 104~:} \texttt{image.jpg --mul\_channels 1{\comma}0.5{\comma}0.8}}
\end{center}

\subsection{\emph{mul\_complex\index{mul\_complex}} }\vspace*{-0.7em}
~\\\textbf{\Cb{Arguments: }}\begin{flushleft}
{\small \Cb{\hspace*{0.5cm}$\bullet$~~\texttt{[multiplier\_real{\comma}multiplier\_imag]}}}\end{flushleft}
Perform multiplication of the selected complex pairs (real1{\comma}imag1{\comma}...{\comma}realN{\comma}imagN) of images by specified complex pair of images (multiplier\_real{\comma}multiplier\_imag).
~\\In complex pairs{\comma} the real image must be always located before the imaginary image in the image list.


\subsection{\emph{neq\index{neq}} (+)}\vspace*{-0.7em}
~\\\textbf{\Cb{Arguments: }}\begin{flushleft}
{\small \Cb{\hspace*{0.5cm}$\bullet$~~\texttt{value[\%]}}}~~~\\
{\small \Cb{\hspace*{0.5cm}$\bullet$~~\texttt{[image]}}}~~~\\
{\small \Cb{\hspace*{0.5cm}$\bullet$~~\texttt{'formula'}}}~~~\\
{\small \Cb{\hspace*{0.5cm}$\bullet$~~\texttt{(no arg)}}}\end{flushleft}
Compute the boolean inequality of selected images with specified value{\comma} image or
mathematical expression{\comma} or compute the boolean inequality of selected images.
~\\(\emph{eq. to} {\small \texttt{'!='}}).
\begin{center}\includegraphics[keepaspectratio=true,height=6cm,width=\textwidth]{img/gmic_stdlib105.jpg}\\
{\footnotesize \textbf{Example 105~:} \texttt{image.jpg round 40 neq \{round(ia{\comma}40)\}}}
\end{center}

\subsection{\emph{or\index{or}} (+)}\vspace*{-0.7em}
~\\\textbf{\Cb{Arguments: }}\begin{flushleft}
{\small \Cb{\hspace*{0.5cm}$\bullet$~~\texttt{value[\%]}}}~~~\\
{\small \Cb{\hspace*{0.5cm}$\bullet$~~\texttt{[image]}}}~~~\\
{\small \Cb{\hspace*{0.5cm}$\bullet$~~\texttt{'formula'}}}~~~\\
{\small \Cb{\hspace*{0.5cm}$\bullet$~~\texttt{(no arg)}}}\end{flushleft}
Compute the bitwise OR of selected images with specified value{\comma} image or mathematical
expression{\comma} or compute the pointwise sequential bitwise OR of selected images.
~\\(\emph{eq. to} {\small \texttt{'~$|$~'}}).
\begin{center}\includegraphics[keepaspectratio=true,height=6cm,width=\textwidth]{img/gmic_stdlib106.jpg}\\
{\footnotesize \textbf{Example 106~:} \texttt{image.jpg or 128}}
\\\includegraphics[keepaspectratio=true,height=6cm,width=\textwidth]{img/gmic_stdlib107.jpg}\\
{\footnotesize \textbf{Example 107~:} \texttt{image.jpg --mirror x or}}
\end{center}

\subsection{\emph{pow\index{pow}} (+)}\vspace*{-0.7em}
~\\\textbf{\Cb{Arguments: }}\begin{flushleft}
{\small \Cb{\hspace*{0.5cm}$\bullet$~~\texttt{value[\%]}}}~~~\\
{\small \Cb{\hspace*{0.5cm}$\bullet$~~\texttt{[image]}}}~~~\\
{\small \Cb{\hspace*{0.5cm}$\bullet$~~\texttt{'formula'}}}~~~\\
{\small \Cb{\hspace*{0.5cm}$\bullet$~~\texttt{(no arg)}}}\end{flushleft}
Raise selected image to the power of specified value{\comma} image or mathematical
expression{\comma} or compute the pointwise sequential powers of selected images.
~\\(\emph{eq. to} {\small \texttt{'\textasciicircum '}}).
\begin{center}\includegraphics[keepaspectratio=true,height=6cm,width=\textwidth]{img/gmic_stdlib108.jpg}\\
{\footnotesize \textbf{Example 108~:} \texttt{image.jpg div 255 --pow 0.5 mul 255}}
\\\includegraphics[keepaspectratio=true,height=6cm,width=\textwidth]{img/gmic_stdlib109.jpg}\\
{\footnotesize \textbf{Example 109~:} \texttt{image.jpg gradient pow 2 add pow 0.2}}
\end{center}

\subsection{\emph{rol\index{rol}} (+)}\vspace*{-0.7em}
~\\\textbf{\Cb{Arguments: }}\begin{flushleft}
{\small \Cb{\hspace*{0.5cm}$\bullet$~~\texttt{value[\%]}}}~~~\\
{\small \Cb{\hspace*{0.5cm}$\bullet$~~\texttt{[image]}}}~~~\\
{\small \Cb{\hspace*{0.5cm}$\bullet$~~\texttt{'formula'}}}~~~\\
{\small \Cb{\hspace*{0.5cm}$\bullet$~~\texttt{(no arg)}}}\end{flushleft}
Compute the bitwise left rotation of selected images with specified value{\comma} image or
mathematical expression{\comma} or compute the pointwise sequential bitwise left rotation of
selected images.
\begin{center}\includegraphics[keepaspectratio=true,height=6cm,width=\textwidth]{img/gmic_stdlib110.jpg}\\
{\footnotesize \textbf{Example 110~:} \texttt{image.jpg rol 'round(3*x/w{\comma}0)' cut 0{\comma}255}}
\end{center}

\subsection{\emph{ror\index{ror}} (+)}\vspace*{-0.7em}
~\\\textbf{\Cb{Arguments: }}\begin{flushleft}
{\small \Cb{\hspace*{0.5cm}$\bullet$~~\texttt{value[\%]}}}~~~\\
{\small \Cb{\hspace*{0.5cm}$\bullet$~~\texttt{[image]}}}~~~\\
{\small \Cb{\hspace*{0.5cm}$\bullet$~~\texttt{'formula'}}}~~~\\
{\small \Cb{\hspace*{0.5cm}$\bullet$~~\texttt{(no arg)}}}\end{flushleft}
Compute the bitwise right rotation of selected images with specified value{\comma} image or
mathematical expression{\comma} or compute the pointwise sequential bitwise right rotation of
selected images.
\begin{center}\includegraphics[keepaspectratio=true,height=6cm,width=\textwidth]{img/gmic_stdlib111.jpg}\\
{\footnotesize \textbf{Example 111~:} \texttt{image.jpg ror 'round(3*x/w{\comma}0)' cut 0{\comma}255}}
\end{center}

\subsection{\emph{sign\index{sign}} (+)}\vspace*{-0.7em}
Compute the pointwise sign of selected images.
\begin{center}\includegraphics[keepaspectratio=true,height=6cm,width=\textwidth]{img/gmic_stdlib112.jpg}\\
{\footnotesize \textbf{Example 112~:} \texttt{image.jpg --sub \{ia\} sign[-1]}}
\\\includegraphics[keepaspectratio=true,height=6cm,width=\textwidth]{img/gmic_stdlib113.jpg}\\
{\footnotesize \textbf{Example 113~:} \texttt{300{\comma}1{\comma}1{\comma}1{\comma}'cos(20*x/w+u)' --sign display\_graph 400{\comma}300}}
\end{center}

\subsection{\emph{sin\index{sin}} (+)}\vspace*{-0.7em}
Compute the pointwise sine of selected images.
\begin{center}\includegraphics[keepaspectratio=true,height=6cm,width=\textwidth]{img/gmic_stdlib114.jpg}\\
{\footnotesize \textbf{Example 114~:} \texttt{image.jpg --normalize 0{\comma}\{2*pi\} sin[-1]}}
\\\includegraphics[keepaspectratio=true,height=6cm,width=\textwidth]{img/gmic_stdlib115.jpg}\\
{\footnotesize \textbf{Example 115~:} \texttt{300{\comma}1{\comma}1{\comma}1{\comma}'20*x/w+u' --sin display\_graph 400{\comma}300}}
\end{center}
~\\
~\textbf{Tutorial page: }\\\url{http://gmic.eu/tutorial/trigometric-and-inverse-trigometric-commands.shtml}


\subsection{\emph{sinc\index{sinc}} (+)}\vspace*{-0.7em}
Compute the pointwise sinc function of selected images.
\begin{center}\includegraphics[keepaspectratio=true,height=6cm,width=\textwidth]{img/gmic_stdlib116.jpg}\\
{\footnotesize \textbf{Example 116~:} \texttt{image.jpg --normalize \{-2*pi\}{\comma}\{2*pi\} sinc[-1]}}
\\\includegraphics[keepaspectratio=true,height=6cm,width=\textwidth]{img/gmic_stdlib117.jpg}\\
{\footnotesize \textbf{Example 117~:} \texttt{300{\comma}1{\comma}1{\comma}1{\comma}'20*x/w+u' --sinc display\_graph 400{\comma}300}}
\end{center}

\subsection{\emph{sinh\index{sinh}} (+)}\vspace*{-0.7em}
Compute the pointwise hyperbolic sine of selected images.
\begin{center}\includegraphics[keepaspectratio=true,height=6cm,width=\textwidth]{img/gmic_stdlib118.jpg}\\
{\footnotesize \textbf{Example 118~:} \texttt{image.jpg --normalize -3{\comma}3 sinh[-1]}}
\\\includegraphics[keepaspectratio=true,height=6cm,width=\textwidth]{img/gmic_stdlib119.jpg}\\
{\footnotesize \textbf{Example 119~:} \texttt{300{\comma}1{\comma}1{\comma}1{\comma}'4*x/w+u' --sinh display\_graph 400{\comma}300}}
\end{center}

\subsection{\emph{sqr\index{sqr}} (+)}\vspace*{-0.7em}
Compute the pointwise square function of selected images.
\begin{center}\includegraphics[keepaspectratio=true,height=6cm,width=\textwidth]{img/gmic_stdlib120.jpg}\\
{\footnotesize \textbf{Example 120~:} \texttt{image.jpg --sqr}}
\\\includegraphics[keepaspectratio=true,height=6cm,width=\textwidth]{img/gmic_stdlib121.jpg}\\
{\footnotesize \textbf{Example 121~:} \texttt{300{\comma}1{\comma}1{\comma}1{\comma}'40*x/w+u' --sqr display\_graph 400{\comma}300}}
\end{center}

\subsection{\emph{sqrt\index{sqrt}} (+)}\vspace*{-0.7em}
Compute the pointwise square root of selected images.
\begin{center}\includegraphics[keepaspectratio=true,height=6cm,width=\textwidth]{img/gmic_stdlib122.jpg}\\
{\footnotesize \textbf{Example 122~:} \texttt{image.jpg --sqrt}}
\\\includegraphics[keepaspectratio=true,height=6cm,width=\textwidth]{img/gmic_stdlib123.jpg}\\
{\footnotesize \textbf{Example 123~:} \texttt{300{\comma}1{\comma}1{\comma}1{\comma}'40*x/w+u' --sqrt display\_graph 400{\comma}300}}
\end{center}

\subsection{\emph{sub\index{sub}} (+)}\vspace*{-0.7em}
~\\\textbf{\Cb{Arguments: }}\begin{flushleft}
{\small \Cb{\hspace*{0.5cm}$\bullet$~~\texttt{value[\%]}}}~~~\\
{\small \Cb{\hspace*{0.5cm}$\bullet$~~\texttt{[image]}}}~~~\\
{\small \Cb{\hspace*{0.5cm}$\bullet$~~\texttt{'formula'}}}~~~\\
{\small \Cb{\hspace*{0.5cm}$\bullet$~~\texttt{(no arg)}}}\end{flushleft}
Subtract specified value{\comma} image or mathematical expression to selected images{\comma}
or compute the pointwise difference of selected images.
~\\(\emph{eq. to} {\small \texttt{'-'}}).
\begin{center}\includegraphics[keepaspectratio=true,height=6cm,width=\textwidth]{img/gmic_stdlib124.jpg}\\
{\footnotesize \textbf{Example 124~:} \texttt{image.jpg --sub 30\% cut 0{\comma}255}}
\\\includegraphics[keepaspectratio=true,height=6cm,width=\textwidth]{img/gmic_stdlib125.jpg}\\
{\footnotesize \textbf{Example 125~:} \texttt{image.jpg --mirror x sub[-1] [0]}}
\\\includegraphics[keepaspectratio=true,height=6cm,width=\textwidth]{img/gmic_stdlib126.jpg}\\
{\footnotesize \textbf{Example 126~:} \texttt{image.jpg sub 'i(w/2+0.9*(x-w/2){\comma}y)'}}
\\\includegraphics[keepaspectratio=true,height=6cm,width=\textwidth]{img/gmic_stdlib127.jpg}\\
{\footnotesize \textbf{Example 127~:} \texttt{image.jpg --mirror x sub}}
\end{center}

\subsection{\emph{tan\index{tan}} (+)}\vspace*{-0.7em}
Compute the pointwise tangent of selected images.
\begin{center}\includegraphics[keepaspectratio=true,height=6cm,width=\textwidth]{img/gmic_stdlib128.jpg}\\
{\footnotesize \textbf{Example 128~:} \texttt{image.jpg --normalize \{-0.47*pi\}{\comma}\{0.47*pi\} tan[-1]}}
\\\includegraphics[keepaspectratio=true,height=6cm,width=\textwidth]{img/gmic_stdlib129.jpg}\\
{\footnotesize \textbf{Example 129~:} \texttt{300{\comma}1{\comma}1{\comma}1{\comma}'20*x/w+u' --tan display\_graph 400{\comma}300}}
\end{center}
~\\
~\textbf{Tutorial page: }\\\url{http://gmic.eu/tutorial/trigometric-and-inverse-trigometric-commands.shtml}


\subsection{\emph{tanh\index{tanh}} (+)}\vspace*{-0.7em}
Compute the pointwise hyperbolic tangent of selected images.
\begin{center}\includegraphics[keepaspectratio=true,height=6cm,width=\textwidth]{img/gmic_stdlib130.jpg}\\
{\footnotesize \textbf{Example 130~:} \texttt{image.jpg --normalize -3{\comma}3 tanh[-1]}}
\\\includegraphics[keepaspectratio=true,height=6cm,width=\textwidth]{img/gmic_stdlib131.jpg}\\
{\footnotesize \textbf{Example 131~:} \texttt{300{\comma}1{\comma}1{\comma}1{\comma}'4*x/w+u' --tanh display\_graph 400{\comma}300}}
\end{center}

\subsection{\emph{xor\index{xor}} (+)}\vspace*{-0.7em}
~\\\textbf{\Cb{Arguments: }}\begin{flushleft}
{\small \Cb{\hspace*{0.5cm}$\bullet$~~\texttt{value[\%]}}}~~~\\
{\small \Cb{\hspace*{0.5cm}$\bullet$~~\texttt{[image]}}}~~~\\
{\small \Cb{\hspace*{0.5cm}$\bullet$~~\texttt{'formula'}}}~~~\\
{\small \Cb{\hspace*{0.5cm}$\bullet$~~\texttt{(no arg)}}}\end{flushleft}
Compute the bitwise XOR of selected images with specified value{\comma} image or mathematical
expression{\comma} or compute the pointwise sequential bitwise XOR of selected images.
\begin{center}\includegraphics[keepaspectratio=true,height=6cm,width=\textwidth]{img/gmic_stdlib132.jpg}\\
{\footnotesize \textbf{Example 132~:} \texttt{image.jpg xor 128}}
\\\includegraphics[keepaspectratio=true,height=6cm,width=\textwidth]{img/gmic_stdlib133.jpg}\\
{\footnotesize \textbf{Example 133~:} \texttt{image.jpg --mirror x xor}}
\end{center}
\section{Values manipulation}


\subsection{\emph{apply\_curve\index{apply\_curve}} }\vspace*{-0.7em}
~\\\textbf{\Cb{Arguments: }}\begin{flushleft}
{\small \Cb{\hspace*{0.5cm}$\bullet$~~\texttt{0$<$=smoothness$<$=1{\comma}x0{\comma}y0{\comma}x1{\comma}y1{\comma}x2{\comma}y2{\comma}...{\comma}xN{\comma}yN}}}\end{flushleft}
Apply curve transformation to image values.
\begin{flushleft}\Cc{\textbf{Default values}:\\~\\\hspace*{0.5cm}{\small $\bullet$~~\texttt{'smoothness=1'{\comma} 'x0=0'{\comma} 'y0=100'.}}}\end{flushleft}
\begin{center}\includegraphics[keepaspectratio=true,height=6cm,width=\textwidth]{img/gmic_stdlib134.jpg}\\
{\footnotesize \textbf{Example 134~:} \texttt{image.jpg --apply\_curve 1{\comma}0{\comma}0{\comma}128{\comma}255{\comma}255{\comma}0}}
\end{center}

\subsection{\emph{apply\_gamma\index{apply\_gamma}} }\vspace*{-0.7em}
~\\\textbf{\Cb{Arguments: }}\begin{flushleft}
{\small \Cb{\hspace*{0.5cm}$\bullet$~~\texttt{gamma$>$=0}}}\end{flushleft}
Apply gamma correction to selected images.
\begin{center}\includegraphics[keepaspectratio=true,height=6cm,width=\textwidth]{img/gmic_stdlib135.jpg}\\
{\footnotesize \textbf{Example 135~:} \texttt{image.jpg --apply\_gamma 2}}
\end{center}

\subsection{\emph{balance\_gamma\index{balance\_gamma}} }\vspace*{-0.7em}
~\\\textbf{\Cb{Arguments: }}\begin{flushleft}
{\small \Cb{\hspace*{0.5cm}$\bullet$~~\texttt{\_ref\_color1{\comma}...}}}\end{flushleft}
Compute gamma-corrected color balance of selected image{\comma} with respect to specified reference color.
\begin{flushleft}\Cc{\textbf{Default value}:\\~\\\hspace*{0.5cm}{\small $\bullet$~~\texttt{'ref\_color1=128'.}}}\end{flushleft}
\begin{center}\includegraphics[keepaspectratio=true,height=6cm,width=\textwidth]{img/gmic_stdlib136.jpg}\\
{\footnotesize \textbf{Example 136~:} \texttt{image.jpg --balance\_gamma 128{\comma}64{\comma}64}}
\end{center}

\subsection{\emph{cast\index{cast}} }\vspace*{-0.7em}
~\\\textbf{\Cb{Arguments: }}\begin{flushleft}
{\small \Cb{\hspace*{0.5cm}$\bullet$~~\texttt{datatype\_source{\comma}datatype\_target}}}\end{flushleft}
Cast datatype of image buffer from specified source type to specified target type.
~\\'datatype\_source' and 'datatype\_target' can be \{ uchar ~$|$~ char ~$|$~ ushort ~$|$~ short ~$|$~ uint ~$|$~ int ~$|$~ uint64 ~$|$~ int64 ~$|$~ float ~$|$~ double \}.


\subsection{\emph{complex2polar\index{complex2polar}} }\vspace*{-0.7em}
Compute complex to polar transforms of selected images.
\begin{center}\includegraphics[keepaspectratio=true,height=6cm,width=\textwidth]{img/gmic_stdlib137.jpg}\\
{\footnotesize \textbf{Example 137~:} \texttt{image.jpg --fft complex2polar[-2{\comma}-1] log[-2] shift[-2] 50\%{\comma}50\%{\comma}0{\comma}0{\comma}2 remove[-1]}}
\end{center}

\subsection{\emph{compress\_clut\index{compress\_clut}} }\vspace*{-0.7em}
~\\\textbf{\Cb{Arguments: }}\begin{flushleft}
{\small \Cb{\hspace*{0.5cm}$\bullet$~~\texttt{\_max\_nbpoints$>$=1{\comma}\_max\_error$>$=0{\comma}\_avg\_error$>$=0}}}\end{flushleft}
Compress selected color LUTs as sequences of colored keypoints.
\begin{flushleft}\Cc{\textbf{Default values}:\\~\\\hspace*{0.5cm}{\small $\bullet$~~\texttt{'max\_nb\_points=2048'{\comma} 'max\_error=17.5'} and \texttt{'avg\_error=1.75'.}}}\end{flushleft}


\subsection{\emph{compress\_rle\index{compress\_rle}} }\vspace*{-0.7em}
~\\\textbf{\Cb{Arguments: }}\begin{flushleft}
{\small \Cb{\hspace*{0.5cm}$\bullet$~~\texttt{\_is\_binary\_data=\{ 0 ~$|$~ 1 \}{\comma}\_maximum\_sequence\_length$>$=0}}}\end{flushleft}
Compress selected images as 2xN data matrices{\comma} using RLE algorithm.
~\\Set 'maximum\_sequence\_length=0' to disable maximum length constraint.
\begin{flushleft}\Cc{\textbf{Default values}:\\~\\\hspace*{0.5cm}{\small $\bullet$~~\texttt{'is\_binary\_data=0'} and \texttt{'maximum\_sequence\_length=0'.}}}\end{flushleft}
\begin{center}\includegraphics[keepaspectratio=true,height=6cm,width=\textwidth]{img/gmic_stdlib138.jpg}\\
{\footnotesize \textbf{Example 138~:} \texttt{image.jpg resize2dy 100 quantize 4 round --compress\_rle {\comma} --decompress\_rle[-1]}}
\end{center}

\subsection{\emph{cumulate\index{cumulate}} (+)}\vspace*{-0.7em}
~\\\textbf{\Cb{Arguments: }}\begin{flushleft}
{\small \Cb{\hspace*{0.5cm}$\bullet$~~\texttt{\{ x ~$|$~ y ~$|$~ z ~$|$~ c \}...\{ x ~$|$~ y ~$|$~ z ~$|$~ c \}}}}~~~\\
{\small \Cb{\hspace*{0.5cm}$\bullet$~~\texttt{(no arg)}}}\end{flushleft}
Compute the cumulative function of specified image data{\comma} optionally along the specified axes.
\begin{center}\includegraphics[keepaspectratio=true,height=6cm,width=\textwidth]{img/gmic_stdlib139.jpg}\\
{\footnotesize \textbf{Example 139~:} \texttt{image.jpg --histogram --cumulate[-1] display\_graph[-2{\comma}-1] 400{\comma}300{\comma}3}}
\end{center}

\subsection{\emph{cut\index{cut}} (+)}\vspace*{-0.7em}
~\\\textbf{\Cb{Arguments: }}\begin{flushleft}
{\small \Cb{\hspace*{0.5cm}$\bullet$~~\texttt{\{ value0[\%] ~$|$~ [image0] \}{\comma}\{ value1[\%] ~$|$~ [image1] \}}}}~~~\\
{\small \Cb{\hspace*{0.5cm}$\bullet$~~\texttt{[image]}}}~~~\\
{\small \Cb{\hspace*{0.5cm}$\bullet$~~\texttt{(no arg)}}}\end{flushleft}
Cut values of selected images in specified range.
~\\(\emph{eq. to} {\small \texttt{'c').\textbackslash n}}).
~\\(no arg) runs interactive mode (uses the instant display window [0] if opened).
~\\In this case{\comma} the chosen cut values are returned in the status.
\begin{center}\includegraphics[keepaspectratio=true,height=6cm,width=\textwidth]{img/gmic_stdlib140.jpg}\\
{\footnotesize \textbf{Example 140~:} \texttt{image.jpg --add 30\% cut[-1] 0{\comma}255}}
\\\includegraphics[keepaspectratio=true,height=6cm,width=\textwidth]{img/gmic_stdlib141.jpg}\\
{\footnotesize \textbf{Example 141~:} \texttt{image.jpg --cut 25\%{\comma}75\%}}
\end{center}

\subsection{\emph{decompress\_clut\index{decompress\_clut}} }\vspace*{-0.7em}
~\\\textbf{\Cb{Arguments: }}\begin{flushleft}
{\small \Cb{\hspace*{0.5cm}$\bullet$~~\texttt{\_width$>$0{\comma}\_height$>$0{\comma}\_depth$>$0}}}\end{flushleft}
Decompress color LUT expressed as a list of colored keypoints.
\begin{flushleft}\Cc{\textbf{Default value}:\\~\\\hspace*{0.5cm}{\small $\bullet$~~\texttt{'width=height=depth=64'.}}}\end{flushleft}


\subsection{\emph{decompress\_rle\index{decompress\_rle}} }\vspace*{-0.7em}
Decompress selected 2xN data matrices{\comma} using RLE algorithm.


\subsection{\emph{discard\index{discard}} (+)}\vspace*{-0.7em}
~\\\textbf{\Cb{Arguments: }}\begin{flushleft}
{\small \Cb{\hspace*{0.5cm}$\bullet$~~\texttt{\_value1{\comma}\_value2{\comma}...}}}~~~\\
{\small \Cb{\hspace*{0.5cm}$\bullet$~~\texttt{\{ x ~$|$~ y ~$|$~ z ~$|$~ c\}...\{ x ~$|$~ y ~$|$~ z ~$|$~ c\}{\comma}\_value1{\comma}\_value2{\comma}...}}}~~~\\
{\small \Cb{\hspace*{0.5cm}$\bullet$~~\texttt{(no arg)}}}\end{flushleft}
Discard specified values in selected images or discard neighboring duplicate values{\comma}
optionally only for the values along the first of a specified axis.
~\\If no values are specified{\comma} neighboring duplicate values are discarded.
~\\If all pixels of a selected image are discarded{\comma} an empty image is returned.
\begin{center}\includegraphics[keepaspectratio=true,height=6cm,width=\textwidth]{img/gmic_stdlib142.jpg}\\
{\footnotesize \textbf{Example 142~:} \texttt{(1;2;3;4;3;2;1) --discard 2}}
\\\includegraphics[keepaspectratio=true,height=6cm,width=\textwidth]{img/gmic_stdlib143.jpg}\\
{\footnotesize \textbf{Example 143~:} \texttt{(1{\comma}2{\comma}2{\comma}3{\comma}3{\comma}3{\comma}4{\comma}4{\comma}4{\comma}4) --discard x}}
\end{center}

\subsection{\emph{eigen2tensor\index{eigen2tensor}} }\vspace*{-0.7em}
Recompose selected pairs of eigenvalues/eigenvectors as 2x2 or 3x3 tensor fields.

~\\
~\textbf{Tutorial page: }\\\url{http://gmic.eu/tutorial/\_eigen2tensor.shtml}


\subsection{\emph{endian\index{endian}} (+)}\vspace*{-0.7em}
~\\\textbf{\Cb{Arguments: }}\begin{flushleft}
{\small \Cb{\hspace*{0.5cm}$\bullet$~~\texttt{\_datatype}}}\end{flushleft}
Reverse data endianness of selected images{\comma} eventually considering the pixel being of the specified datatype.
~\\'datatype' can be \{ uchar ~$|$~ char ~$|$~ ushort ~$|$~ short ~$|$~ uint ~$|$~ int ~$|$~ uint64 ~$|$~ int64 ~$|$~ float ~$|$~ double \}.


\subsection{\emph{equalize\index{equalize}} (+)}\vspace*{-0.7em}
~\\\textbf{\Cb{Arguments: }}\begin{flushleft}
{\small \Cb{\hspace*{0.5cm}$\bullet$~~\texttt{\_nb\_levels$>$0[\%]{\comma}\_value\_min[\%]{\comma}\_value\_max[\%]}}}\end{flushleft}
Equalize histograms of selected images.
~\\If value range is specified{\comma} the equalization is done only for pixels in the specified
value range.
\begin{flushleft}\Cc{\textbf{Default values}:\\~\\\hspace*{0.5cm}{\small $\bullet$~~\texttt{'nb\_levels=256'{\comma} 'value\_min=0\%'} and \texttt{'value\_max=100\%'.}}}\end{flushleft}
\begin{center}\includegraphics[keepaspectratio=true,height=6cm,width=\textwidth]{img/gmic_stdlib144.jpg}\\
{\footnotesize \textbf{Example 144~:} \texttt{image.jpg --equalize}}
\\\includegraphics[keepaspectratio=true,height=6cm,width=\textwidth]{img/gmic_stdlib145.jpg}\\
{\footnotesize \textbf{Example 145~:} \texttt{image.jpg --equalize 4{\comma}0{\comma}128}}
\end{center}

\subsection{\emph{fill\index{fill}} (+)}\vspace*{-0.7em}
~\\\textbf{\Cb{Arguments: }}\begin{flushleft}
{\small \Cb{\hspace*{0.5cm}$\bullet$~~\texttt{value1{\comma}\_value2{\comma}...}}}~~~\\
{\small \Cb{\hspace*{0.5cm}$\bullet$~~\texttt{[image]}}}~~~\\
{\small \Cb{\hspace*{0.5cm}$\bullet$~~\texttt{'formula'}}}\end{flushleft}
Fill selected images with values read from the specified value list{\comma} existing image
or mathematical expression. Single quotes may be omitted in 'formula'.
~\\(\emph{eq. to} {\small \texttt{'f'}}).
\begin{center}\includegraphics[keepaspectratio=true,height=6cm,width=\textwidth]{img/gmic_stdlib146.jpg}\\
{\footnotesize \textbf{Example 146~:} \texttt{4{\comma}4 fill 1{\comma}2{\comma}3{\comma}4{\comma}5{\comma}6{\comma}7}}
\\\includegraphics[keepaspectratio=true,height=6cm,width=\textwidth]{img/gmic_stdlib147.jpg}\\
{\footnotesize \textbf{Example 147~:} \texttt{4{\comma}4 (1{\comma}2{\comma}3{\comma}4{\comma}5{\comma}6{\comma}7) fill[-2] [-1]}}
\\\includegraphics[keepaspectratio=true,height=6cm,width=\textwidth]{img/gmic_stdlib148.jpg}\\
{\footnotesize \textbf{Example 148~:} \texttt{400{\comma}400{\comma}1{\comma}3 fill "X=x-w/2; Y=y-h/2; R=sqrt(X\textasciicircum 2+Y\textasciicircum 2); a=atan2(Y{\comma}X); if(R$<$=180{\comma}255*abs(cos(c+200*(x/w-0.5)*(y/h-0.5))){\comma}850*(a\%(0.1*(c+1))))"}}
\end{center}
~\\
~\textbf{Tutorial page: }\\\url{http://gmic.eu/tutorial/\_fill.shtml}


\subsection{\emph{float2int8\index{float2int8}} }\vspace*{-0.7em}
Convert selected float-valued images to 8bits integer representations.


\subsection{\emph{int82float\index{int82float}} }\vspace*{-0.7em}
Convert selected 8bits integer representations to float-valued images.


\subsection{\emph{index\index{index}} (+)}\vspace*{-0.7em}
~\\\textbf{\Cb{Arguments: }}\begin{flushleft}
{\small \Cb{\hspace*{0.5cm}$\bullet$~~\texttt{\{ [palette] ~$|$~ predefined\_palette \}{\comma}0$<$=\_dithering$<$=1{\comma}\_map\_pal\-ette=\{ 0 ~$|$~ 1 \}}}}\end{flushleft}
Index selected vector-valued images by specified vector-valued palette.
~\\'predefined\_palette' can be \{ 0=default ~$|$~ 1=HSV ~$|$~ 2=lines ~$|$~ 3=hot ~$|$~ 4=cool ~$|$~ 5=jet ~$|$~ 6=flag ~$|$~ 7=cube \}.
\begin{flushleft}\Cc{\textbf{Default values}:\\~\\\hspace*{0.5cm}{\small $\bullet$~~\texttt{'dithering=0'} and \texttt{'map\_palette=0'.}}}\end{flushleft}
\begin{center}\includegraphics[keepaspectratio=true,height=6cm,width=\textwidth]{img/gmic_stdlib149.jpg}\\
{\footnotesize \textbf{Example 149~:} \texttt{image.jpg --index 1{\comma}1{\comma}1}}
\\\includegraphics[keepaspectratio=true,height=6cm,width=\textwidth]{img/gmic_stdlib150.jpg}\\
{\footnotesize \textbf{Example 150~:} \texttt{image.jpg (0;255;255\textasciicircum 0;128;255\textasciicircum 0;0;255) --index[-2] [-1]{\comma}1{\comma}1}}
\end{center}
~\\
~\textbf{Tutorial page: }\\\url{http://gmic.eu/tutorial/\_index.shtml}


\subsection{\emph{inrange\index{inrange}} }\vspace*{-0.7em}
~\\\textbf{\Cb{Arguments: }}\begin{flushleft}
{\small \Cb{\hspace*{0.5cm}$\bullet$~~\texttt{min[\%]{\comma}max[\%]}}}\end{flushleft}
Detect pixels whose values are in specified range [min{\comma}max]{\comma} in selected images.
~\\(\emph{eq. to} {\small \texttt{'ir'}}).
\begin{center}\includegraphics[keepaspectratio=true,height=6cm,width=\textwidth]{img/gmic_stdlib151.jpg}\\
{\footnotesize \textbf{Example 151~:} \texttt{image.jpg --inrange 25\%{\comma}75\%}}
\end{center}

\subsection{\emph{map\index{map}} (+)}\vspace*{-0.7em}
~\\\textbf{\Cb{Arguments: }}\begin{flushleft}
{\small \Cb{\hspace*{0.5cm}$\bullet$~~\texttt{[palette]{\comma}\_boundary\_conditions}}}~~~\\
{\small \Cb{\hspace*{0.5cm}$\bullet$~~\texttt{predefined\_palette{\comma}\_boundary\_conditions}}}\end{flushleft}
Map specified vector-valued palette to selected indexed scalar images.
~\\'predefined\_palette' can be \{ 0=default ~$|$~ 1=HSV ~$|$~ 2=lines ~$|$~ 3=hot ~$|$~ 4=cool ~$|$~ 5=jet ~$|$~ 6=flag ~$|$~ 7=cube \}.
~\\'boundary\_conditions' can be \{ 0=dirichlet ~$|$~ 1=neumann ~$|$~ 2=periodic ~$|$~ 3=mirror \}.
\begin{flushleft}\Cc{\textbf{Default value}:\\~\\\hspace*{0.5cm}{\small $\bullet$~~\texttt{'boundary\_conditions=0'.}}}\end{flushleft}
\begin{center}\includegraphics[keepaspectratio=true,height=6cm,width=\textwidth]{img/gmic_stdlib152.jpg}\\
{\footnotesize \textbf{Example 152~:} \texttt{image.jpg --luminance map[-1] 3}}
\\\includegraphics[keepaspectratio=true,height=6cm,width=\textwidth]{img/gmic_stdlib153.jpg}\\
{\footnotesize \textbf{Example 153~:} \texttt{image.jpg --rgb2ycbcr split[-1] c (0{\comma}255{\comma}0) resize[-1] 256{\comma}1{\comma}1{\comma}1{\comma}3 map[-4] [-1] remove[-1] append[-3--1] c ycbcr2rgb[-1]}}
\end{center}
~\\
~\textbf{Tutorial page: }\\\url{http://gmic.eu/tutorial/\_map.shtml}


\subsection{\emph{map\_clut\index{map\_clut}} }\vspace*{-0.7em}
~\\\textbf{\Cb{Arguments: }}\begin{flushleft}
{\small \Cb{\hspace*{0.5cm}$\bullet$~~\texttt{[clut] ~$|$~ "clut\_name"}}}\end{flushleft}
Map specified RGB color LUT to selected images.
\begin{center}\includegraphics[keepaspectratio=true,height=6cm,width=\textwidth]{img/gmic_stdlib154.jpg}\\
{\footnotesize \textbf{Example 154~:} \texttt{image.jpg uniform\_distribution \{2\textasciicircum 6\}{\comma}3 mirror[-1] x --map\_clut[0] [1]}}
\end{center}

\subsection{\emph{mix\_channels\index{mix\_channels}} }\vspace*{-0.7em}
~\\\textbf{\Cb{Arguments: }}\begin{flushleft}
{\small \Cb{\hspace*{0.5cm}$\bullet$~~\texttt{(a00{\comma}...{\comma}aMN)}}}\end{flushleft}
Apply specified matrix to channels of selected images.
\begin{center}\includegraphics[keepaspectratio=true,height=6cm,width=\textwidth]{img/gmic_stdlib155.jpg}\\
{\footnotesize \textbf{Example 155~:} \texttt{image.jpg --mix\_channels (0{\comma}1{\comma}0;1{\comma}0{\comma}0;0{\comma}0{\comma}1)}}
\end{center}

\subsection{\emph{negate\index{negate}} }\vspace*{-0.7em}
~\\\textbf{\Cb{Arguments: }}\begin{flushleft}
{\small \Cb{\hspace*{0.5cm}$\bullet$~~\texttt{base\_value}}}~~~\\
{\small \Cb{\hspace*{0.5cm}$\bullet$~~\texttt{(no arg)}}}\end{flushleft}
Negate image values.
\begin{flushleft}\Cc{\textbf{Default value}:\\~\\\hspace*{0.5cm}{\small $\bullet$~~\texttt{'base\_value=(undefined)'.}}}\end{flushleft}
\begin{center}\includegraphics[keepaspectratio=true,height=6cm,width=\textwidth]{img/gmic_stdlib156.jpg}\\
{\footnotesize \textbf{Example 156~:} \texttt{image.jpg --negate}}
\end{center}

\subsection{\emph{noise\index{noise}} (+)}\vspace*{-0.7em}
~\\\textbf{\Cb{Arguments: }}\begin{flushleft}
{\small \Cb{\hspace*{0.5cm}$\bullet$~~\texttt{std\_variation$>$=0[\%]{\comma}\_noise\_type}}}\end{flushleft}
Add random noise to selected images.
~\\'noise\_type' can be \{ 0=gaussian ~$|$~ 1=uniform ~$|$~ 2=salt\&pepper ~$|$~ 3=poisson ~$|$~ 4=rice \}.
\begin{flushleft}\Cc{\textbf{Default value}:\\~\\\hspace*{0.5cm}{\small $\bullet$~~\texttt{'noise\_type=0'.}}}\end{flushleft}
\begin{center}\includegraphics[keepaspectratio=true,height=6cm,width=\textwidth]{img/gmic_stdlib157.jpg}\\
{\footnotesize \textbf{Example 157~:} \texttt{image.jpg --noise[0] 50{\comma}0 --noise[0] 50{\comma}1 --noise[0] 10{\comma}2 cut 0{\comma}255}}
\\\includegraphics[keepaspectratio=true,height=6cm,width=\textwidth]{img/gmic_stdlib158.jpg}\\
{\footnotesize \textbf{Example 158~:} \texttt{300{\comma}300{\comma}1{\comma}3 [0] noise[0] 20{\comma}0 noise[1] 20{\comma}1 --histogram 100 display\_graph[-2{\comma}-1] 400{\comma}300{\comma}3}}
\end{center}

\subsection{\emph{normp\index{normp}} }\vspace*{-0.7em}
~\\\textbf{\Cb{Arguments: }}\begin{flushleft}
{\small \Cb{\hspace*{0.5cm}$\bullet$~~\texttt{p$>$=0}}}\end{flushleft}
Compute the pointwise Lp-norm norm of vector-valued pixels in selected images.
\begin{flushleft}\Cc{\textbf{Default value}:\\~\\\hspace*{0.5cm}{\small $\bullet$~~\texttt{'p=2'.}}}\end{flushleft}
\begin{center}\includegraphics[keepaspectratio=true,height=6cm,width=\textwidth]{img/gmic_stdlib159.jpg}\\
{\footnotesize \textbf{Example 159~:} \texttt{image.jpg --normp[0] 0 --normp[0] 1 --normp[0] 2 --normp[0] inf}}
\end{center}

\subsection{\emph{norm\index{norm}} }\vspace*{-0.7em}
Compute the pointwise euclidean norm of vector-valued pixels in selected images.
\begin{center}\includegraphics[keepaspectratio=true,height=6cm,width=\textwidth]{img/gmic_stdlib160.jpg}\\
{\footnotesize \textbf{Example 160~:} \texttt{image.jpg --norm}}
\end{center}
~\\
~\textbf{Tutorial page: }\\\url{http://gmic.eu/tutorial/\_norm.shtml}


\subsection{\emph{normalize\index{normalize}} (+)}\vspace*{-0.7em}
~\\\textbf{\Cb{Arguments: }}\begin{flushleft}
{\small \Cb{\hspace*{0.5cm}$\bullet$~~\texttt{\{ value0[\%] ~$|$~ [image0] \}{\comma}\{ value1[\%] ~$|$~ [image1] \}}}}~~~\\
{\small \Cb{\hspace*{0.5cm}$\bullet$~~\texttt{[image]}}}\end{flushleft}
Linearly normalize values of selected images in specified range.
~\\(\emph{eq. to} {\small \texttt{'n'}}).
\begin{center}\includegraphics[keepaspectratio=true,height=6cm,width=\textwidth]{img/gmic_stdlib161.jpg}\\
{\footnotesize \textbf{Example 161~:} \texttt{image.jpg split x{\comma}2 normalize[-1] 64{\comma}196 append x}}
\end{center}
~\\
~\textbf{Tutorial page: }\\\url{http://gmic.eu/tutorial/\_normalize.shtml}


\subsection{\emph{normalize\_sum\index{normalize\_sum}} }\vspace*{-0.7em}
Normalize selected images with a unitary sum.
\begin{center}\includegraphics[keepaspectratio=true,height=6cm,width=\textwidth]{img/gmic_stdlib162.jpg}\\
{\footnotesize \textbf{Example 162~:} \texttt{image.jpg --histogram normalize\_sum[-1] display\_graph[-1] 400{\comma}300}}
\end{center}

\subsection{\emph{not\index{not}} }\vspace*{-0.7em}
Apply boolean not operation on selected images.
\begin{center}\includegraphics[keepaspectratio=true,height=6cm,width=\textwidth]{img/gmic_stdlib163.jpg}\\
{\footnotesize \textbf{Example 163~:} \texttt{image.jpg --ge 50\% --not[-1]}}
\end{center}

\subsection{\emph{orientation\index{orientation}} }\vspace*{-0.7em}
Compute the pointwise orientation of vector-valued pixels in selected images.
\begin{center}\includegraphics[keepaspectratio=true,height=6cm,width=\textwidth]{img/gmic_stdlib164.jpg}\\
{\footnotesize \textbf{Example 164~:} \texttt{image.jpg --orientation --norm[-2] negate[-1] mul[-2] [-1] reverse[-2{\comma}-1]}}
\end{center}
~\\
~\textbf{Tutorial page: }\\\url{http://gmic.eu/tutorial/\_orientation.shtml}


\subsection{\emph{oneminus\index{oneminus}} }\vspace*{-0.7em}
For each selected image{\comma} compute one minus image.
\begin{center}\includegraphics[keepaspectratio=true,height=6cm,width=\textwidth]{img/gmic_stdlib165.jpg}\\
{\footnotesize \textbf{Example 165~:} \texttt{image.jpg normalize 0{\comma}1 --oneminus}}
\end{center}

\subsection{\emph{otsu\index{otsu}} }\vspace*{-0.7em}
~\\\textbf{\Cb{Arguments: }}\begin{flushleft}
{\small \Cb{\hspace*{0.5cm}$\bullet$~~\texttt{\_nb\_levels$>$0}}}\end{flushleft}
Hard-threshold selected images using Otsu's method.
~\\The computed thresholds are returned as a list of values in the status.
\begin{flushleft}\Cc{\textbf{Default value}:\\~\\\hspace*{0.5cm}{\small $\bullet$~~\texttt{'nb\_levels=256'.}}}\end{flushleft}
\begin{center}\includegraphics[keepaspectratio=true,height=6cm,width=\textwidth]{img/gmic_stdlib166.jpg}\\
{\footnotesize \textbf{Example 166~:} \texttt{image.jpg luminance --otsu {\comma}}}
\end{center}

\subsection{\emph{polar2complex\index{polar2complex}} }\vspace*{-0.7em}
Compute polar to complex transforms of selected images.


\subsection{\emph{quantize\index{quantize}} }\vspace*{-0.7em}
~\\\textbf{\Cb{Arguments: }}\begin{flushleft}
{\small \Cb{\hspace*{0.5cm}$\bullet$~~\texttt{nb\_levels$>$=1{\comma}\_keep\_values=\{ 0 ~$|$~ 1 \}{\comma}\_is\_uniform=\{ 0 ~$|$~ 1 \}}}}\end{flushleft}
Quantize selected images.
\begin{flushleft}\Cc{\textbf{Default value}:\\~\\\hspace*{0.5cm}{\small $\bullet$~~\texttt{'keep\_values=1'} and \texttt{'is\_uniform=0'.}}}\end{flushleft}
\begin{center}\includegraphics[keepaspectratio=true,height=6cm,width=\textwidth]{img/gmic_stdlib167.jpg}\\
{\footnotesize \textbf{Example 167~:} \texttt{image.jpg luminance --quantize 3}}
\\\includegraphics[keepaspectratio=true,height=6cm,width=\textwidth]{img/gmic_stdlib168.jpg}\\
{\footnotesize \textbf{Example 168~:} \texttt{200{\comma}200{\comma}1{\comma}1{\comma}'cos(x/10)*sin(y/10)' --quantize[0] 6 --quantize[0] 4 --quantize[0] 3 --quantize[0] 2}}
\end{center}

\subsection{\emph{rand\index{rand}} (+)}\vspace*{-0.7em}
~\\\textbf{\Cb{Arguments: }}\begin{flushleft}
{\small \Cb{\hspace*{0.5cm}$\bullet$~~\texttt{\{ value0[\%] ~$|$~ [image0] \}{\comma}\_\{ value1[\%] ~$|$~ [image1] \}}}}~~~\\
{\small \Cb{\hspace*{0.5cm}$\bullet$~~\texttt{[image]}}}\end{flushleft}
Fill selected images with random values uniformly distributed in the specified range.
\begin{center}\includegraphics[keepaspectratio=true,height=6cm,width=\textwidth]{img/gmic_stdlib169.jpg}\\
{\footnotesize \textbf{Example 169~:} \texttt{400{\comma}400{\comma}1{\comma}3 rand -10{\comma}10 --blur 10 sign[-1]}}
\end{center}

\subsection{\emph{replace\index{replace}} }\vspace*{-0.7em}
~\\\textbf{\Cb{Arguments: }}\begin{flushleft}
{\small \Cb{\hspace*{0.5cm}$\bullet$~~\texttt{source{\comma}target}}}\end{flushleft}
Replace pixel values in selected images.
\begin{center}\includegraphics[keepaspectratio=true,height=6cm,width=\textwidth]{img/gmic_stdlib170.jpg}\\
{\footnotesize \textbf{Example 170~:} \texttt{(1;2;3;4) --replace 2{\comma}3}}
\end{center}

\subsection{\emph{replace\_inf\index{replace\_inf}} }\vspace*{-0.7em}
~\\\textbf{\Cb{Arguments: }}\begin{flushleft}
{\small \Cb{\hspace*{0.5cm}$\bullet$~~\texttt{\_expression}}}\end{flushleft}
Replace all infinite values in selected images by specified expression.
\begin{center}\includegraphics[keepaspectratio=true,height=6cm,width=\textwidth]{img/gmic_stdlib171.jpg}\\
{\footnotesize \textbf{Example 171~:} \texttt{(0;1;2) log --replace\_inf 2}}
\end{center}

\subsection{\emph{replace\_nan\index{replace\_nan}} }\vspace*{-0.7em}
~\\\textbf{\Cb{Arguments: }}\begin{flushleft}
{\small \Cb{\hspace*{0.5cm}$\bullet$~~\texttt{\_expression}}}\end{flushleft}
Replace all NaN values in selected images by specified expression.
\begin{center}\includegraphics[keepaspectratio=true,height=6cm,width=\textwidth]{img/gmic_stdlib172.jpg}\\
{\footnotesize \textbf{Example 172~:} \texttt{(-1;0;2) sqrt --replace\_nan 2}}
\end{center}

\subsection{\emph{replace\_seq\index{replace\_seq}} }\vspace*{-0.7em}
~\\\textbf{\Cb{Arguments: }}\begin{flushleft}
{\small \Cb{\hspace*{0.5cm}$\bullet$~~\texttt{"search\_seq"{\comma}"replace\_seq"}}}\end{flushleft}
Search and replace a sequence of values in selected images.
\begin{center}\includegraphics[keepaspectratio=true,height=6cm,width=\textwidth]{img/gmic_stdlib173.jpg}\\
{\footnotesize \textbf{Example 173~:} \texttt{(1;2;3;4;5) --replace\_seq "2{\comma}3{\comma}4"{\comma}"7{\comma}8"}}
\end{center}

\subsection{\emph{replace\_str\index{replace\_str}} }\vspace*{-0.7em}
~\\\textbf{\Cb{Arguments: }}\begin{flushleft}
{\small \Cb{\hspace*{0.5cm}$\bullet$~~\texttt{"search\_str"{\comma}"replace\_str"}}}\end{flushleft}
Search and replace a string in selected images (viewed as strings{\comma} i.e. sequences of ascii codes).
\begin{center}\includegraphics[keepaspectratio=true,height=6cm,width=\textwidth]{img/gmic_stdlib174.jpg}\\
{\footnotesize \textbf{Example 174~:} \texttt{(\{'"Hello there{\comma} how are you ?"'\}) --replace\_str "Hello there"{\comma}"Hi David"}}
\end{center}

\subsection{\emph{round\index{round}} (+)}\vspace*{-0.7em}
~\\\textbf{\Cb{Arguments: }}\begin{flushleft}
{\small \Cb{\hspace*{0.5cm}$\bullet$~~\texttt{rounding\_value$>$=0{\comma}\_rounding\_type}}}~~~\\
{\small \Cb{\hspace*{0.5cm}$\bullet$~~\texttt{(no arg)}}}\end{flushleft}
Round values of selected images.
~\\'rounding\_type' can be \{ -1=backward ~$|$~ 0=nearest ~$|$~ 1=forward \}.
\begin{flushleft}\Cc{\textbf{Default value}:\\~\\\hspace*{0.5cm}{\small $\bullet$~~\texttt{'rounding\_type=0'.}}}\end{flushleft}
\begin{center}\includegraphics[keepaspectratio=true,height=6cm,width=\textwidth]{img/gmic_stdlib175.jpg}\\
{\footnotesize \textbf{Example 175~:} \texttt{image.jpg --round 100}}
\\\includegraphics[keepaspectratio=true,height=6cm,width=\textwidth]{img/gmic_stdlib176.jpg}\\
{\footnotesize \textbf{Example 176~:} \texttt{image.jpg mul \{pi/180\} sin --round}}
\end{center}

\subsection{\emph{roundify\index{roundify}} }\vspace*{-0.7em}
~\\\textbf{\Cb{Arguments: }}\begin{flushleft}
{\small \Cb{\hspace*{0.5cm}$\bullet$~~\texttt{gamma$>$=0}}}\end{flushleft}
Apply roundify transformation on float-valued data{\comma} with specified gamma.
\begin{flushleft}\Cc{\textbf{Default value}:\\~\\\hspace*{0.5cm}{\small $\bullet$~~\texttt{'gamma=0'.}}}\end{flushleft}
\begin{center}\includegraphics[keepaspectratio=true,height=6cm,width=\textwidth]{img/gmic_stdlib177.jpg}\\
{\footnotesize \textbf{Example 177~:} \texttt{1000 fill '4*x/w' repeat 5 --roundify[0] \{\$$>$*0.2\} done append c display\_graph 400{\comma}300}}
\end{center}

\subsection{\emph{set\index{set}} (+)}\vspace*{-0.7em}
~\\\textbf{\Cb{Arguments: }}\begin{flushleft}
{\small \Cb{\hspace*{0.5cm}$\bullet$~~\texttt{value{\comma}\_x[\%]{\comma}\_y[\%]{\comma}\_z[\%]{\comma}\_c[\%]}}}\end{flushleft}
Set pixel value in selected images{\comma} at specified coordinates.
~\\(\emph{eq. to} {\small \texttt{'=').\textbackslash n}}).
~\\If specified coordinates are outside the image bounds{\comma} no action is performed.
\begin{flushleft}\Cc{\textbf{Default values}:\\~\\\hspace*{0.5cm}{\small $\bullet$~~\texttt{'x=y=z=c=0'.}}}\end{flushleft}
\begin{center}\includegraphics[keepaspectratio=true,height=6cm,width=\textwidth]{img/gmic_stdlib178.jpg}\\
{\footnotesize \textbf{Example 178~:} \texttt{2{\comma}2 set 1{\comma}0{\comma}0 set 2{\comma}1{\comma}0 set 3{\comma}0{\comma}1 set 4{\comma}1{\comma}1}}
\\\includegraphics[keepaspectratio=true,height=6cm,width=\textwidth]{img/gmic_stdlib179.jpg}\\
{\footnotesize \textbf{Example 179~:} \texttt{image.jpg repeat 10000 set 255{\comma}\{u(100)\}\%{\comma}\{u(100)\}\%{\comma}0{\comma}\{u(100)\}\% done}}
\end{center}

\subsection{\emph{threshold\index{threshold}} (+)}\vspace*{-0.7em}
~\\\textbf{\Cb{Arguments: }}\begin{flushleft}
{\small \Cb{\hspace*{0.5cm}$\bullet$~~\texttt{value[\%]{\comma}\_is\_soft=\{ 0 ~$|$~ 1 \}}}}~~~\\
{\small \Cb{\hspace*{0.5cm}$\bullet$~~\texttt{(no arg)}}}\end{flushleft}
Threshold values of selected images.
~\\'soft' can be \{ 0=hard-thresholding ~$|$~ 1=soft-thresholding \}.
~\\(no arg) runs interactive mode (uses the instant display window [0] if opened).
~\\In this case{\comma} the chosen threshold value is returned in the status.
\begin{flushleft}\Cc{\textbf{Default value}:\\~\\\hspace*{0.5cm}{\small $\bullet$~~\texttt{'is\_soft=0'.}}}\end{flushleft}
\begin{center}\includegraphics[keepaspectratio=true,height=6cm,width=\textwidth]{img/gmic_stdlib180.jpg}\\
{\footnotesize \textbf{Example 180~:} \texttt{image.jpg --threshold[0] 50\% --threshold[0] 50\%{\comma}1}}
\end{center}
~\\
~\textbf{Tutorial page: }\\\url{http://gmic.eu/tutorial/\_threshold.shtml}


\subsection{\emph{unrepeat\index{unrepeat}} }\vspace*{-0.7em}
Remove repetition of adjacent values in selected images.
\begin{center}\includegraphics[keepaspectratio=true,height=6cm,width=\textwidth]{img/gmic_stdlib181.jpg}\\
{\footnotesize \textbf{Example 181~:} \texttt{(1;1;1;1;1;2;2;2;3;4;4;4;5;5;5) --unrepeat}}
\end{center}

\subsection{\emph{vector2tensor\index{vector2tensor}} }\vspace*{-0.7em}
Convert selected vector fields to corresponding tensor fields.

\section{Colors manipulation}


\subsection{\emph{adjust\_colors\index{adjust\_colors}} }\vspace*{-0.7em}
~\\\textbf{\Cb{Arguments: }}\begin{flushleft}
{\small \Cb{\hspace*{0.5cm}$\bullet$~~\texttt{-100$<$=\_brightness$<$=100{\comma}-100$<$=\_contrast$<$=100{\comma}-100$<$=\_gamma$<$=10\-0{\comma}-100$<$=\_hue\_shift$<$=100{\comma}-100$<$=\_saturation$<$=100{\comma}\_value\_min{\comma}\_v\-alue\_max}}}\end{flushleft}
Perform a global adjustment of colors on selected images.
~\\Range of correct image values are considered to be in [value\_min{\comma}value\_max] (e.g. [0{\comma}255]).
~\\If 'value\_min==value\_max==0'{\comma} value range is estimated from min/max values of selected images.
~\\Processed images have pixel values constrained in [value\_min{\comma}value\_max].
\begin{flushleft}\Cc{\textbf{Default values}:\\~\\\hspace*{0.5cm}{\small $\bullet$~~\texttt{'brightness=0'{\comma} 'contrast=0'{\comma} 'gamma=0'{\comma} 'hue\_shift=0'{\comma} 'saturation=0'{\comma} 'value\_min=value\_max=0'.}}}\end{flushleft}
\begin{center}\includegraphics[keepaspectratio=true,height=6cm,width=\textwidth]{img/gmic_stdlib182.jpg}\\
{\footnotesize \textbf{Example 182~:} \texttt{image.jpg --adjust\_colors 0{\comma}30{\comma}0{\comma}0{\comma}30}}
\end{center}

\subsection{\emph{apply\_channels\index{apply\_channels}} }\vspace*{-0.7em}
~\\\textbf{\Cb{Arguments: }}\begin{flushleft}
{\small \Cb{\hspace*{0.5cm}$\bullet$~~\texttt{"command"{\comma}channels{\comma}\_value\_action=\{ 0=none ~$|$~ 1=cut ~$|$~ 2=normal\-ize \}}}}\end{flushleft}
Apply specified command on the chosen color channel(s) of each selected images.
~\\(\emph{eq. to} {\small \texttt{'ac').\textbackslash n}}).
~\\Argument 'channels' refers to a colorspace{\comma} and can be basically one of \{ all ~$|$~ rgba ~$|$~ rgb ~$|$~ lrgb ~$|$~ ycbcr ~$|$~ lab ~$|$~ lch ~$|$~ hsv ~$|$~ hsi ~$|$~ hsl ~$|$~ cmy ~$|$~ cmyk ~$|$~ yiq \}.
~\\You can also make the processing focus on one particular channel of this colorspace{\comma} by setting 'channels' as 'colorspace\_channel' (e.g. 'hsv\_h' for the hue).
~\\All channel values are considered to be in the [0{\comma}255] range.
\begin{flushleft}\Cc{\textbf{Default value}:\\~\\\hspace*{0.5cm}{\small $\bullet$~~\texttt{'value\_action=0'.}}}\end{flushleft}
\begin{center}\includegraphics[keepaspectratio=true,height=6cm,width=\textwidth]{img/gmic_stdlib183.jpg}\\
{\footnotesize \textbf{Example 183~:} \texttt{image.jpg --apply\_channels "equalize blur 2"{\comma}ycbcr\_cbcr}}
\end{center}

\subsection{\emph{autoindex\index{autoindex}} }\vspace*{-0.7em}
~\\\textbf{\Cb{Arguments: }}\begin{flushleft}
{\small \Cb{\hspace*{0.5cm}$\bullet$~~\texttt{nb\_colors$>$0{\comma}0$<$=\_dithering$<$=1{\comma}\_method=\{ 0=median-cut ~$|$~ 1=k-me\-ans \}}}}\end{flushleft}
Index selected vector-valued images by adapted colormaps.
\begin{flushleft}\Cc{\textbf{Default values}:\\~\\\hspace*{0.5cm}{\small $\bullet$~~\texttt{'dithering=0'} and \texttt{'method=1'.}}}\end{flushleft}
\begin{center}\includegraphics[keepaspectratio=true,height=6cm,width=\textwidth]{img/gmic_stdlib184.jpg}\\
{\footnotesize \textbf{Example 184~:} \texttt{image.jpg --autoindex[0] 4 --autoindex[0] 8 --autoindex[0] 16}}
\end{center}

\subsection{\emph{bayer2rgb\index{bayer2rgb}} }\vspace*{-0.7em}
~\\\textbf{\Cb{Arguments: }}\begin{flushleft}
{\small \Cb{\hspace*{0.5cm}$\bullet$~~\texttt{\_GM\_smoothness{\comma}\_RB\_smoothness1{\comma}\_RB\_smoothness2}}}\end{flushleft}
Transform selected RGB-Bayer sampled images to color images.
\begin{flushleft}\Cc{\textbf{Default values}:\\~\\\hspace*{0.5cm}{\small $\bullet$~~\texttt{'GM\_smoothness=RB\_smoothness=1'} and \texttt{'RB\_smoothness2=0.5'.}}}\end{flushleft}
\begin{center}\includegraphics[keepaspectratio=true,height=6cm,width=\textwidth]{img/gmic_stdlib185.jpg}\\
{\footnotesize \textbf{Example 185~:} \texttt{image.jpg rgb2bayer 0 --bayer2rgb 1{\comma}1{\comma}0.5}}
\end{center}

\subsection{\emph{cmy2rgb\index{cmy2rgb}} }\vspace*{-0.7em}
Convert selected images from CMY to RGB colorbases.


\subsection{\emph{cmyk2rgb\index{cmyk2rgb}} }\vspace*{-0.7em}
Convert selected images from CMYK to RGB colorbases.


\subsection{\emph{colorblind\index{colorblind}} }\vspace*{-0.7em}
~\\\textbf{\Cb{Arguments: }}\begin{flushleft}
{\small \Cb{\hspace*{0.5cm}$\bullet$~~\texttt{type=\{ 0=protanopia ~$|$~ 1=protanomaly ~$|$~ 2=deuteranopia ~$|$~ 3=deu\-teranomaly ~$|$~ 4=tritanopia ~$|$~ 5=tritanomaly ~$|$~ 6=achromatopsia \-~$|$~ 7=achromatomaly \}}}}\end{flushleft}
Simulate color blindness vision.
\begin{center}\includegraphics[keepaspectratio=true,height=6cm,width=\textwidth]{img/gmic_stdlib186.jpg}\\
{\footnotesize \textbf{Example 186~:} \texttt{image.jpg --colorblind 0}}
\end{center}

\subsection{\emph{colormap\index{colormap}} }\vspace*{-0.7em}
~\\\textbf{\Cb{Arguments: }}\begin{flushleft}
{\small \Cb{\hspace*{0.5cm}$\bullet$~~\texttt{nb\_levels$>$=0{\comma}\_method=\{ 0=median-cut ~$|$~ 1=k-means \}{\comma}\_sort\_vect\-ors=\{ 0 ~$|$~ 1 \}}}}\end{flushleft}
Estimate best-fitting colormap with 'nb\_colors' entries{\comma} to index selected images.
~\\Set 'nb\_levels==0' to extract all existing colors of an image.
\begin{flushleft}\Cc{\textbf{Default value}:\\~\\\hspace*{0.5cm}{\small $\bullet$~~\texttt{'method=1'} and \texttt{'sort\_vectors=1'.}}}\end{flushleft}
\begin{center}\includegraphics[keepaspectratio=true,height=6cm,width=\textwidth]{img/gmic_stdlib187.jpg}\\
{\footnotesize \textbf{Example 187~:} \texttt{image.jpg --colormap[0] 4 --colormap[0] 8 --colormap[0] 16}}
\end{center}
~\\
~\textbf{Tutorial page: }\\\url{http://gmic.eu/tutorial/\_colormap.shtml}


\subsection{\emph{compose\_channels\index{compose\_channels}} }\vspace*{-0.7em}
Compose all channels of each selected image{\comma} using specified arithmetic operator (+{\comma}-{\comma}or{\comma}min{\comma}...).
\begin{flushleft}\Cc{\textbf{Default value}:\\~\\\hspace*{0.5cm}{\small $\bullet$~~\texttt{'1=+'.}}}\end{flushleft}
\begin{center}\includegraphics[keepaspectratio=true,height=6cm,width=\textwidth]{img/gmic_stdlib188.jpg}\\
{\footnotesize \textbf{Example 188~:} \texttt{image.jpg --compose\_channels and}}
\end{center}
~\\
~\textbf{Tutorial page: }\\\url{http://gmic.eu/tutorial/\_compose\_channels.shtml}


\subsection{\emph{direction2rgb\index{direction2rgb}} }\vspace*{-0.7em}
Compute RGB representation of selected 2d direction fields.
\begin{center}\includegraphics[keepaspectratio=true,height=6cm,width=\textwidth]{img/gmic_stdlib189.jpg}\\
{\footnotesize \textbf{Example 189~:} \texttt{image.jpg luminance gradient append c blur 2 orientation --direction2rgb}}
\end{center}

\subsection{\emph{ditheredbw\index{ditheredbw}} }\vspace*{-0.7em}
Create dithered B\&W version of selected images.
\begin{center}\includegraphics[keepaspectratio=true,height=6cm,width=\textwidth]{img/gmic_stdlib190.jpg}\\
{\footnotesize \textbf{Example 190~:} \texttt{image.jpg --equalize ditheredbw[-1]}}
\end{center}

\subsection{\emph{fill\_color\index{fill\_color}} }\vspace*{-0.7em}
~\\\textbf{\Cb{Arguments: }}\begin{flushleft}
{\small \Cb{\hspace*{0.5cm}$\bullet$~~\texttt{col1{\comma}...{\comma}colN}}}\end{flushleft}
Fill selected images with specified color.
~\\(\emph{eq. to} {\small \texttt{'fc'}}).
\begin{center}\includegraphics[keepaspectratio=true,height=6cm,width=\textwidth]{img/gmic_stdlib191.jpg}\\
{\footnotesize \textbf{Example 191~:} \texttt{image.jpg --fill\_color 255{\comma}0{\comma}255}}
\end{center}
~\\
~\textbf{Tutorial page: }\\\url{http://gmic.eu/tutorial/\_fill\_color.shtml}


\subsection{\emph{gradient2rgb\index{gradient2rgb}} }\vspace*{-0.7em}
~\\\textbf{\Cb{Arguments: }}\begin{flushleft}
{\small \Cb{\hspace*{0.5cm}$\bullet$~~\texttt{\_is\_orientation=\{ 0 ~$|$~ 1 \}}}}\end{flushleft}
Compute RGB representation of 2d gradient of selected images.
\begin{flushleft}\Cc{\textbf{Default value}:\\~\\\hspace*{0.5cm}{\small $\bullet$~~\texttt{'is\_orientation=0'.}}}\end{flushleft}
\begin{center}\includegraphics[keepaspectratio=true,height=6cm,width=\textwidth]{img/gmic_stdlib192.jpg}\\
{\footnotesize \textbf{Example 192~:} \texttt{image.jpg --gradient2rgb 0 equalize[-1]}}
\end{center}

\subsection{\emph{hsi2rgb\index{hsi2rgb}} (+)}\vspace*{-0.7em}
Convert selected images from HSI to RGB colorbases.


\subsection{\emph{hsi82rgb\index{hsi82rgb}} }\vspace*{-0.7em}
Convert selected images from HSI8 to RGB color bases.


\subsection{\emph{hsl2rgb\index{hsl2rgb}} (+)}\vspace*{-0.7em}
Convert selected images from HSL to RGB colorbases.


\subsection{\emph{hsl82rgb\index{hsl82rgb}} }\vspace*{-0.7em}
Convert selected images from HSL8 to RGB color bases.


\subsection{\emph{hsv2rgb\index{hsv2rgb}} (+)}\vspace*{-0.7em}
Convert selected images from HSV to RGB colorbases.
\begin{center}\includegraphics[keepaspectratio=true,height=6cm,width=\textwidth]{img/gmic_stdlib193.jpg}\\
{\footnotesize \textbf{Example 193~:} \texttt{(0{\comma}360;0{\comma}360\textasciicircum 0{\comma}0;1{\comma}1\textasciicircum 1{\comma}1;1{\comma}1) resize 400{\comma}400{\comma}1{\comma}3{\comma}3 hsv2rgb}}
\end{center}

\subsection{\emph{hsv82rgb\index{hsv82rgb}} }\vspace*{-0.7em}
Convert selected images from HSV8 to RGB color bases.


\subsection{\emph{int2rgb\index{int2rgb}} }\vspace*{-0.7em}
Convert selected images from INT24 scalars to RGB.


\subsection{\emph{lab2lch\index{lab2lch}} }\vspace*{-0.7em}
Convert selected images from Lab to Lch color bases.


\subsection{\emph{lab2rgb\index{lab2rgb}} (+)}\vspace*{-0.7em}
~\\\textbf{\Cb{Arguments: }}\begin{flushleft}
{\small \Cb{\hspace*{0.5cm}$\bullet$~~\texttt{illuminant=\{ 0=D50 ~$|$~ 1=D65 \}}}}~~~\\
{\small \Cb{\hspace*{0.5cm}$\bullet$~~\texttt{(no arg)}}}\end{flushleft}
Convert selected images from Lab to RGB colorbases.
\begin{center}\includegraphics[keepaspectratio=true,height=6cm,width=\textwidth]{img/gmic_stdlib194.jpg}\\
{\footnotesize \textbf{Example 194~:} \texttt{(50{\comma}50;50{\comma}50\textasciicircum -3{\comma}3;-3{\comma}3\textasciicircum -3{\comma}-3;3{\comma}3) resize 400{\comma}400{\comma}1{\comma}3{\comma}3 lab2rgb}}
\end{center}

\subsection{\emph{lab82rgb\index{lab82rgb}} }\vspace*{-0.7em}
~\\\textbf{\Cb{Arguments: }}\begin{flushleft}
{\small \Cb{\hspace*{0.5cm}$\bullet$~~\texttt{illuminant=\{ 0=D50 ~$|$~ 1=D65 \}}}}~~~\\
{\small \Cb{\hspace*{0.5cm}$\bullet$~~\texttt{(no arg)}}}\end{flushleft}
Convert selected images from Lab8 to RGB color bases.


\subsection{\emph{lch2lab\index{lch2lab}} }\vspace*{-0.7em}
Convert selected images from Lch to Lab color bases.


\subsection{\emph{lch2rgb\index{lch2rgb}} }\vspace*{-0.7em}
Convert selected images from Lch to RGB color bases.


\subsection{\emph{lch82rgb\index{lch82rgb}} }\vspace*{-0.7em}
Convert selected images from Lch8 to RGB color bases.


\subsection{\emph{luminance\index{luminance}} }\vspace*{-0.7em}
Compute luminance of selected sRGB images.
\begin{center}\includegraphics[keepaspectratio=true,height=6cm,width=\textwidth]{img/gmic_stdlib195.jpg}\\
{\footnotesize \textbf{Example 195~:} \texttt{image.jpg --luminance}}
\end{center}

\subsection{\emph{mix\_rgb\index{mix\_rgb}} }\vspace*{-0.7em}
~\\\textbf{\Cb{Arguments: }}\begin{flushleft}
{\small \Cb{\hspace*{0.5cm}$\bullet$~~\texttt{a11{\comma}a12{\comma}a13{\comma}a21{\comma}a22{\comma}a23{\comma}a31{\comma}a32{\comma}a33}}}\end{flushleft}
Apply 3x3 specified matrix to RGB colors of selected images.
\begin{flushleft}\Cc{\textbf{Default values}:\\~\\\hspace*{0.5cm}{\small $\bullet$~~\texttt{'a11=1'{\comma} 'a12=a13=a21=0'{\comma} 'a22=1'{\comma} 'a23=a31=a32=0'} and \texttt{'a33=1'.}}}\end{flushleft}
\begin{center}\includegraphics[keepaspectratio=true,height=6cm,width=\textwidth]{img/gmic_stdlib196.jpg}\\
{\footnotesize \textbf{Example 196~:} \texttt{image.jpg --mix\_rgb 0{\comma}1{\comma}0{\comma}1{\comma}0{\comma}0{\comma}0{\comma}0{\comma}1}}
\end{center}
~\\
~\textbf{Tutorial page: }\\\url{http://gmic.eu/tutorial/\_mix\_rgb.shtml}


\subsection{\emph{pseudogray\index{pseudogray}} }\vspace*{-0.7em}
~\\\textbf{\Cb{Arguments: }}\begin{flushleft}
{\small \Cb{\hspace*{0.5cm}$\bullet$~~\texttt{\_max\_increment$>$=0{\comma}\_JND\_threshold$>$=0{\comma}\_bits\_depth$>$0}}}\end{flushleft}
Generate pseudogray colormap with specified increment and perceptual threshold.
~\\If 'JND\_threshold' is 0{\comma} no perceptual constraints are applied.
\begin{flushleft}\Cc{\textbf{Default values}:\\~\\\hspace*{0.5cm}{\small $\bullet$~~\texttt{'max\_increment=5'{\comma} 'JND\_threshold=2.3'} and \texttt{'bits\_depth=8'.}}}\end{flushleft}
\begin{center}\includegraphics[keepaspectratio=true,height=6cm,width=\textwidth]{img/gmic_stdlib197.jpg}\\
{\footnotesize \textbf{Example 197~:} \texttt{pseudogray 5}}
\end{center}

\subsection{\emph{replace\_color\index{replace\_color}} }\vspace*{-0.7em}
~\\\textbf{\Cb{Arguments: }}\begin{flushleft}
{\small \Cb{\hspace*{0.5cm}$\bullet$~~\texttt{tolerance[\%]$>$=0{\comma}smoothness[\%]$>$=0{\comma}src1{\comma}src2{\comma}...{\comma}dest1{\comma}dest2{\comma}.\-..}}}\end{flushleft}
Replace pixels from/to specified colors in selected images.
\begin{center}\includegraphics[keepaspectratio=true,height=6cm,width=\textwidth]{img/gmic_stdlib198.jpg}\\
{\footnotesize \textbf{Example 198~:} \texttt{image.jpg --replace\_color 40{\comma}3{\comma}204{\comma}153{\comma}110{\comma}255{\comma}0{\comma}0}}
\end{center}

\subsection{\emph{retinex\index{retinex}} }\vspace*{-0.7em}
~\\\textbf{\Cb{Arguments: }}\begin{flushleft}
{\small \Cb{\hspace*{0.5cm}$\bullet$~~\texttt{\_value\_offset$>$0{\comma}\_colorspace=\{ hsi ~$|$~ hsv ~$|$~ lab ~$|$~ lrgb ~$|$~ rgb ~$|$~\- ycbcr \}{\comma}0$<$=\_min\_cut$<$=100{\comma}0$<$=\_max\_cut$<$=100{\comma}\_sigma\_low$>$0{\comma}\_sig\-ma\_mid$>$0{\comma}\_sigma\_high$>$0}}}\end{flushleft}
Apply multi-scale retinex algorithm on selected image to improve color consistency.
~\\(as described in the page http://www.ipol.im/pub/art/2014/107/).
\begin{flushleft}\Cc{\textbf{Default values}:\\~\\\hspace*{0.5cm}{\small $\bullet$~~\texttt{'offset=1'{\comma} 'colorspace=hsv'{\comma} 'min\_cut=1'{\comma} 'max\_cut=1'{\comma} 'sigma\_low=15'{\comma}'sigma\_mid=80'} and \texttt{'sigma\_high=250'.}}}\end{flushleft}


\subsection{\emph{rgb2bayer\index{rgb2bayer}} }\vspace*{-0.7em}
~\\\textbf{\Cb{Arguments: }}\begin{flushleft}
{\small \Cb{\hspace*{0.5cm}$\bullet$~~\texttt{\_start\_pattern=0{\comma}\_color\_grid=0}}}\end{flushleft}
Transform selected color images to RGB-Bayer sampled images.
\begin{flushleft}\Cc{\textbf{Default values}:\\~\\\hspace*{0.5cm}{\small $\bullet$~~\texttt{'start\_pattern=0'} and \texttt{'color\_grid=0'.}}}\end{flushleft}
\begin{center}\includegraphics[keepaspectratio=true,height=6cm,width=\textwidth]{img/gmic_stdlib199.jpg}\\
{\footnotesize \textbf{Example 199~:} \texttt{image.jpg --rgb2bayer 0}}
\end{center}

\subsection{\emph{rgb2cmy\index{rgb2cmy}} }\vspace*{-0.7em}
Convert selected images from RGB to CMY colorbases.
\begin{center}\includegraphics[keepaspectratio=true,height=6cm,width=\textwidth]{img/gmic_stdlib200.jpg}\\
{\footnotesize \textbf{Example 200~:} \texttt{image.jpg rgb2cmy split c}}
\end{center}

\subsection{\emph{rgb2cmyk\index{rgb2cmyk}} }\vspace*{-0.7em}
Convert selected images from RGB to CMYK colorbases.
\begin{center}\includegraphics[keepaspectratio=true,height=6cm,width=\textwidth]{img/gmic_stdlib201.jpg}\\
{\footnotesize \textbf{Example 201~:} \texttt{image.jpg rgb2cmyk split c}}
\\\includegraphics[keepaspectratio=true,height=6cm,width=\textwidth]{img/gmic_stdlib202.jpg}\\
{\footnotesize \textbf{Example 202~:} \texttt{image.jpg rgb2cmyk split c fill[3] 0 append c cmyk2rgb}}
\end{center}

\subsection{\emph{rgb2hcy\index{rgb2hcy}} }\vspace*{-0.7em}
Convert selected images from RGB to HCY colorbases.
\begin{center}\includegraphics[keepaspectratio=true,height=6cm,width=\textwidth]{img/gmic_stdlib203.jpg}\\
{\footnotesize \textbf{Example 203~:} \texttt{image.jpg rgb2hcy split c}}
\end{center}

\subsection{\emph{hcy2rgb :\index{hcy2rgb :}} }\vspace*{-0.7em}
Convert selected images from HCY to RGB colorbases.


\subsection{\emph{rgb2hsi\index{rgb2hsi}} (+)}\vspace*{-0.7em}
Convert selected images from RGB to HSI colorbases.
\begin{center}\includegraphics[keepaspectratio=true,height=6cm,width=\textwidth]{img/gmic_stdlib204.jpg}\\
{\footnotesize \textbf{Example 204~:} \texttt{image.jpg rgb2hsi split c}}
\end{center}

\subsection{\emph{rgb2hsi8\index{rgb2hsi8}} }\vspace*{-0.7em}
Convert selected images from RGB to HSI8 color bases.
\begin{center}\includegraphics[keepaspectratio=true,height=6cm,width=\textwidth]{img/gmic_stdlib205.jpg}\\
{\footnotesize \textbf{Example 205~:} \texttt{image.jpg rgb2hsi8 split c}}
\end{center}

\subsection{\emph{rgb2hsl\index{rgb2hsl}} (+)}\vspace*{-0.7em}
Convert selected images from RGB to HSL colorbases.
\begin{center}\includegraphics[keepaspectratio=true,height=6cm,width=\textwidth]{img/gmic_stdlib206.jpg}\\
{\footnotesize \textbf{Example 206~:} \texttt{image.jpg rgb2hsl split c}}
\\\includegraphics[keepaspectratio=true,height=6cm,width=\textwidth]{img/gmic_stdlib207.jpg}\\
{\footnotesize \textbf{Example 207~:} \texttt{image.jpg rgb2hsl --split c add[-3] 100 mod[-3] 360 append[-3--1] c hsl2rgb}}
\end{center}

\subsection{\emph{rgb2hsl8\index{rgb2hsl8}} }\vspace*{-0.7em}
Convert selected images from RGB to HSL8 color bases.
\begin{center}\includegraphics[keepaspectratio=true,height=6cm,width=\textwidth]{img/gmic_stdlib208.jpg}\\
{\footnotesize \textbf{Example 208~:} \texttt{image.jpg rgb2hsl8 split c}}
\end{center}

\subsection{\emph{rgb2hsv\index{rgb2hsv}} (+)}\vspace*{-0.7em}
Convert selected images from RGB to HSV colorbases.
\begin{center}\includegraphics[keepaspectratio=true,height=6cm,width=\textwidth]{img/gmic_stdlib209.jpg}\\
{\footnotesize \textbf{Example 209~:} \texttt{image.jpg rgb2hsv split c}}
\\\includegraphics[keepaspectratio=true,height=6cm,width=\textwidth]{img/gmic_stdlib210.jpg}\\
{\footnotesize \textbf{Example 210~:} \texttt{image.jpg rgb2hsv --split c add[-2] 0.3 cut[-2] 0{\comma}1 append[-3--1] c hsv2rgb}}
\end{center}

\subsection{\emph{rgb2hsv8\index{rgb2hsv8}} }\vspace*{-0.7em}
Convert selected images from RGB to HSV8 color bases.
\begin{center}\includegraphics[keepaspectratio=true,height=6cm,width=\textwidth]{img/gmic_stdlib211.jpg}\\
{\footnotesize \textbf{Example 211~:} \texttt{image.jpg rgb2hsv8 split c}}
\end{center}

\subsection{\emph{rgb2lab\index{rgb2lab}} (+)}\vspace*{-0.7em}
~\\\textbf{\Cb{Arguments: }}\begin{flushleft}
{\small \Cb{\hspace*{0.5cm}$\bullet$~~\texttt{illuminant=\{ 0=D50 ~$|$~ 1=D65 \}}}}~~~\\
{\small \Cb{\hspace*{0.5cm}$\bullet$~~\texttt{(no arg)}}}\end{flushleft}
Convert selected images from RGB to Lab colorbases.
\begin{center}\includegraphics[keepaspectratio=true,height=6cm,width=\textwidth]{img/gmic_stdlib212.jpg}\\
{\footnotesize \textbf{Example 212~:} \texttt{image.jpg rgb2lab split c}}
\\\includegraphics[keepaspectratio=true,height=6cm,width=\textwidth]{img/gmic_stdlib213.jpg}\\
{\footnotesize \textbf{Example 213~:} \texttt{image.jpg rgb2lab --split c mul[-2{\comma}-1] 2.5 append[-3--1] c lab2rgb}}
\end{center}

\subsection{\emph{rgb2lab8\index{rgb2lab8}} }\vspace*{-0.7em}
~\\\textbf{\Cb{Arguments: }}\begin{flushleft}
{\small \Cb{\hspace*{0.5cm}$\bullet$~~\texttt{illuminant=\{ 0=D50 ~$|$~ 1=D65 \}}}}~~~\\
{\small \Cb{\hspace*{0.5cm}$\bullet$~~\texttt{(no arg)}}}\end{flushleft}
Convert selected images from RGB to Lab8 color bases.
\begin{center}\includegraphics[keepaspectratio=true,height=6cm,width=\textwidth]{img/gmic_stdlib214.jpg}\\
{\footnotesize \textbf{Example 214~:} \texttt{image.jpg rgb2lab8 split c}}
\end{center}

\subsection{\emph{rgb2lch\index{rgb2lch}} }\vspace*{-0.7em}
Convert selected images from RGB to Lch color bases.
\begin{center}\includegraphics[keepaspectratio=true,height=6cm,width=\textwidth]{img/gmic_stdlib215.jpg}\\
{\footnotesize \textbf{Example 215~:} \texttt{image.jpg rgb2lch split c}}
\end{center}

\subsection{\emph{rgb2lch8\index{rgb2lch8}} }\vspace*{-0.7em}
Convert selected images from RGB to Lch8 color bases.
\begin{center}\includegraphics[keepaspectratio=true,height=6cm,width=\textwidth]{img/gmic_stdlib216.jpg}\\
{\footnotesize \textbf{Example 216~:} \texttt{image.jpg rgb2lch8 split c}}
\end{center}

\subsection{\emph{rgb2luv\index{rgb2luv}} }\vspace*{-0.7em}
Convert selected images from RGB to LUV color bases.
\begin{center}\includegraphics[keepaspectratio=true,height=6cm,width=\textwidth]{img/gmic_stdlib217.jpg}\\
{\footnotesize \textbf{Example 217~:} \texttt{image.jpg rgb2luv split c}}
\end{center}

\subsection{\emph{rgb2int\index{rgb2int}} }\vspace*{-0.7em}
Convert selected images from RGB to INT24 scalars.
\begin{center}\includegraphics[keepaspectratio=true,height=6cm,width=\textwidth]{img/gmic_stdlib218.jpg}\\
{\footnotesize \textbf{Example 218~:} \texttt{image.jpg rgb2int}}
\end{center}

\subsection{\emph{rgb2srgb\index{rgb2srgb}} (+)}\vspace*{-0.7em}
Convert selected images from RGB to sRGB colorbases.


\subsection{\emph{rgb2xyz\index{rgb2xyz}} }\vspace*{-0.7em}
~\\\textbf{\Cb{Arguments: }}\begin{flushleft}
{\small \Cb{\hspace*{0.5cm}$\bullet$~~\texttt{illuminant=\{ 0=D50 ~$|$~ 1=D65 \}}}}~~~\\
{\small \Cb{\hspace*{0.5cm}$\bullet$~~\texttt{(no arg)}}}\end{flushleft}
Convert selected images from RGB to XYZ colorbases.
\begin{flushleft}\Cc{\textbf{Default value}:\\~\\\hspace*{0.5cm}{\small $\bullet$~~\texttt{'illuminant=0'.}}}\end{flushleft}
\begin{center}\includegraphics[keepaspectratio=true,height=6cm,width=\textwidth]{img/gmic_stdlib219.jpg}\\
{\footnotesize \textbf{Example 219~:} \texttt{image.jpg rgb2xyz split c}}
\end{center}

\subsection{\emph{rgb2xyz8\index{rgb2xyz8}} }\vspace*{-0.7em}
~\\\textbf{\Cb{Arguments: }}\begin{flushleft}
{\small \Cb{\hspace*{0.5cm}$\bullet$~~\texttt{illuminant=\{ 0=D50 ~$|$~ 1=D65 \}}}}~~~\\
{\small \Cb{\hspace*{0.5cm}$\bullet$~~\texttt{(no arg)}}}\end{flushleft}
Convert selected images from RGB to XYZ8 color bases.
\begin{center}\includegraphics[keepaspectratio=true,height=6cm,width=\textwidth]{img/gmic_stdlib220.jpg}\\
{\footnotesize \textbf{Example 220~:} \texttt{image.jpg rgb2xyz8 split c}}
\end{center}

\subsection{\emph{rgb2yiq\index{rgb2yiq}} }\vspace*{-0.7em}
Convert selected images from RGB to YIQ colorbases.
\begin{center}\includegraphics[keepaspectratio=true,height=6cm,width=\textwidth]{img/gmic_stdlib221.jpg}\\
{\footnotesize \textbf{Example 221~:} \texttt{image.jpg rgb2yiq split c}}
\end{center}

\subsection{\emph{rgb2yiq8\index{rgb2yiq8}} }\vspace*{-0.7em}
Convert selected images from RGB to YIQ8 colorbases.
\begin{center}\includegraphics[keepaspectratio=true,height=6cm,width=\textwidth]{img/gmic_stdlib222.jpg}\\
{\footnotesize \textbf{Example 222~:} \texttt{image.jpg rgb2yiq8 split c}}
\end{center}

\subsection{\emph{rgb2ycbcr\index{rgb2ycbcr}} }\vspace*{-0.7em}
Convert selected images from RGB to YCbCr colorbases.
\begin{center}\includegraphics[keepaspectratio=true,height=6cm,width=\textwidth]{img/gmic_stdlib223.jpg}\\
{\footnotesize \textbf{Example 223~:} \texttt{image.jpg rgb2ycbcr split c}}
\end{center}

\subsection{\emph{rgb2yuv\index{rgb2yuv}} }\vspace*{-0.7em}
Convert selected images from RGB to YUV colorbases.
\begin{center}\includegraphics[keepaspectratio=true,height=6cm,width=\textwidth]{img/gmic_stdlib224.jpg}\\
{\footnotesize \textbf{Example 224~:} \texttt{image.jpg rgb2yuv split c}}
\end{center}

\subsection{\emph{rgb2yuv8\index{rgb2yuv8}} }\vspace*{-0.7em}
Convert selected images from RGB to YUV8 color bases.
\begin{center}\includegraphics[keepaspectratio=true,height=6cm,width=\textwidth]{img/gmic_stdlib225.jpg}\\
{\footnotesize \textbf{Example 225~:} \texttt{image.jpg rgb2yuv8 split c}}
\end{center}

\subsection{\emph{remove\_opacity\index{remove\_opacity}} }\vspace*{-0.7em}
Remove opacity channel of selected images.


\subsection{\emph{select\_color\index{select\_color}} }\vspace*{-0.7em}
~\\\textbf{\Cb{Arguments: }}\begin{flushleft}
{\small \Cb{\hspace*{0.5cm}$\bullet$~~\texttt{tolerance[\%]$>$=0{\comma}col1{\comma}...{\comma}colN}}}\end{flushleft}
Select pixels with specified color in selected images.
\begin{center}\includegraphics[keepaspectratio=true,height=6cm,width=\textwidth]{img/gmic_stdlib226.jpg}\\
{\footnotesize \textbf{Example 226~:} \texttt{image.jpg --select\_color 40{\comma}204{\comma}153{\comma}110}}
\end{center}
~\\
~\textbf{Tutorial page: }\\\url{http://gmic.eu/tutorial/\_select\_color.shtml}


\subsection{\emph{sepia\index{sepia}} }\vspace*{-0.7em}
Apply sepia tones effect on selected images.
\begin{center}\includegraphics[keepaspectratio=true,height=6cm,width=\textwidth]{img/gmic_stdlib227.jpg}\\
{\footnotesize \textbf{Example 227~:} \texttt{image.jpg --sepia}}
\end{center}

\subsection{\emph{solarize\index{solarize}} }\vspace*{-0.7em}
Solarize selected images.
\begin{center}\includegraphics[keepaspectratio=true,height=6cm,width=\textwidth]{img/gmic_stdlib228.jpg}\\
{\footnotesize \textbf{Example 228~:} \texttt{image.jpg --solarize}}
\end{center}

\subsection{\emph{split\_colors\index{split\_colors}} }\vspace*{-0.7em}
~\\\textbf{\Cb{Arguments: }}\begin{flushleft}
{\small \Cb{\hspace*{0.5cm}$\bullet$~~\texttt{\_tolerance$>$=0{\comma}\_max\_nb\_outputs$>$0{\comma}\_min\_area$>$0}}}\end{flushleft}
Split selected images as several image containing a single color.
~\\One selected image can be split as at most 'max\_nb\_outputs' images.
~\\Output images are sorted by decreasing area of extracted color regions and have an additional alpha-channel.
\begin{flushleft}\Cc{\textbf{Default values}:\\~\\\hspace*{0.5cm}{\small $\bullet$~~\texttt{'tolerance=0'{\comma} 'max\_nb\_outputs=256'} and \texttt{'min\_area=8'.}}}\end{flushleft}
\begin{center}\includegraphics[keepaspectratio=true,height=6cm,width=\textwidth]{img/gmic_stdlib229.jpg}\\
{\footnotesize \textbf{Example 229~:} \texttt{image.jpg quantize 5 --split\_colors {\comma} display\_rgba}}
\end{center}

\subsection{\emph{split\_opacity\index{split\_opacity}} }\vspace*{-0.7em}
Split color and opacity parts of selected images.


\subsection{\emph{srgb2rgb\index{srgb2rgb}} (+)}\vspace*{-0.7em}
Convert selected images from sRGB to RGB colorbases.


\subsection{\emph{to\_a\index{to\_a}} }\vspace*{-0.7em}
Force selected images to have an alpha channel.


\subsection{\emph{to\_color\index{to\_color}} }\vspace*{-0.7em}
Force selected images to be in color mode (RGB or RGBA).


\subsection{\emph{to\_colormode\index{to\_colormode}} }\vspace*{-0.7em}
~\\\textbf{\Cb{Arguments: }}\begin{flushleft}
{\small \Cb{\hspace*{0.5cm}$\bullet$~~\texttt{mode=\{ 0=adaptive ~$|$~ 1=G ~$|$~ 2=GA ~$|$~ 3=RGB ~$|$~ 4=RGBA \}}}}\end{flushleft}
Force selected images to be in a given color mode.
\begin{flushleft}\Cc{\textbf{Default value}:\\~\\\hspace*{0.5cm}{\small $\bullet$~~\texttt{'mode=0'.}}}\end{flushleft}


\subsection{\emph{to\_gray\index{to\_gray}} }\vspace*{-0.7em}
Force selected images to be in GRAY mode.
\begin{center}\includegraphics[keepaspectratio=true,height=6cm,width=\textwidth]{img/gmic_stdlib230.jpg}\\
{\footnotesize \textbf{Example 230~:} \texttt{image.jpg --to\_gray}}
\end{center}

\subsection{\emph{to\_graya\index{to\_graya}} }\vspace*{-0.7em}
Force selected images to be in GRAYA mode.


\subsection{\emph{to\_pseudogray\index{to\_pseudogray}} }\vspace*{-0.7em}
~\\\textbf{\Cb{Arguments: }}\begin{flushleft}
{\small \Cb{\hspace*{0.5cm}$\bullet$~~\texttt{\_max\_step$>$=0{\comma}\_is\_perceptual\_constraint=\{ 0 ~$|$~ 1 \}{\comma}\_bits\_depth\-$>$0}}}\end{flushleft}
Convert selected scalar images ([0-255]-valued) to pseudo-gray color images.
~\\Default parameters : 'max\_step=5'{\comma} 'is\_perceptual\_constraint=1' and 'bits\_depth=8'.
~\\The original pseudo-gray technique has been introduced by Rich Franzen [http://r0k.us/graphics/pseudoGrey.html].
~\\Extension of this technique to arbitrary increments for more tones{\comma} has been done by David Tschumperle.


\subsection{\emph{to\_rgb\index{to\_rgb}} }\vspace*{-0.7em}
Force selected images to be in RGB mode.


\subsection{\emph{to\_rgba\index{to\_rgba}} }\vspace*{-0.7em}
Force selected images to be in RGBA mode.


\subsection{\emph{transfer\_colors\index{transfer\_colors}} }\vspace*{-0.7em}
~\\\textbf{\Cb{Arguments: }}\begin{flushleft}
{\small \Cb{\hspace*{0.5cm}$\bullet$~~\texttt{[reference\_image]{\comma}\_transfer\_brightness=\{ 0 ~$|$~ 1 \}}}}\end{flushleft}
Transfer colors of the specified reference image to selected images.
\begin{flushleft}\Cc{\textbf{Default value}:\\~\\\hspace*{0.5cm}{\small $\bullet$~~\texttt{'transfer\_brightness=0'.}}}\end{flushleft}
\begin{center}\includegraphics[keepaspectratio=true,height=6cm,width=\textwidth]{img/gmic_stdlib231.jpg}\\
{\footnotesize \textbf{Example 231~:} \texttt{image.jpg --rand 0{\comma}255 --transfer\_colors[0] [1]{\comma}1}}
\end{center}

\subsection{\emph{transfer\_rgb\index{transfer\_rgb}} }\vspace*{-0.7em}
~\\\textbf{\Cb{Arguments: }}\begin{flushleft}
{\small \Cb{\hspace*{0.5cm}$\bullet$~~\texttt{[target]{\comma}\_gamma$>$=0{\comma}\_regularization$>$=0{\comma}\_luminosity\_constraint\-s$>$=0{\comma}\_rgb\_resolution$>$=0{\comma}\_is\_constraints=\{ 0 ~$|$~ 1 \}}}}\end{flushleft}
Transfer colors from selected source images to selected reference image (given as argument).
~\\'gamma' determines the importance of color occurences in the matching process (0=none to 1=huge).
~\\'regularization' determines the number of guided filter iterations to remove quantization effects.
~\\'luminosity\_constraints' tells if luminosity constraints must be applied on non-confident matched colors.
~\\'is\_constraints' tells if additional hard color constraints must be set (opens an interactive window).
\begin{flushleft}\Cc{\textbf{Default values}:\\~\\\hspace*{0.5cm}{\small $\bullet$~~\texttt{'gamma=0.3'{\comma}'regularization=8'{\comma} 'luminosity\_constraints=0.1'{\comma} 'rgb\_resolution=64'} and \texttt{'is\_constraints=0'.}}}\end{flushleft}


\subsection{\emph{xyz2rgb\index{xyz2rgb}} }\vspace*{-0.7em}
~\\\textbf{\Cb{Arguments: }}\begin{flushleft}
{\small \Cb{\hspace*{0.5cm}$\bullet$~~\texttt{illuminant=\{ 0=D50 ~$|$~ 1=D65 \}}}}~~~\\
{\small \Cb{\hspace*{0.5cm}$\bullet$~~\texttt{(no arg)}}}\end{flushleft}
Convert selected images from XYZ to RGB colorbases.
\begin{flushleft}\Cc{\textbf{Default value}:\\~\\\hspace*{0.5cm}{\small $\bullet$~~\texttt{'illuminant=0'.}}}\end{flushleft}


\subsection{\emph{xyz82rgb\index{xyz82rgb}} }\vspace*{-0.7em}
~\\\textbf{\Cb{Arguments: }}\begin{flushleft}
{\small \Cb{\hspace*{0.5cm}$\bullet$~~\texttt{illuminant=\{ 0=D50 ~$|$~ 1=D65 \}}}}~~~\\
{\small \Cb{\hspace*{0.5cm}$\bullet$~~\texttt{(no arg)}}}\end{flushleft}
Convert selected images from XYZ8 to RGB color bases.


\subsection{\emph{ycbcr2rgb\index{ycbcr2rgb}} }\vspace*{-0.7em}
Convert selected images from YCbCr to RGB colorbases.


\subsection{\emph{yiq2rgb\index{yiq2rgb}} }\vspace*{-0.7em}
Convert selected images from YIQ to RGB colorbases.


\subsection{\emph{yiq82rgb\index{yiq82rgb}} }\vspace*{-0.7em}
Convert selected images from YIQ8 to RGB colorbases.


\subsection{\emph{yuv2rgb\index{yuv2rgb}} }\vspace*{-0.7em}
Convert selected images from YUV to RGB colorbases.


\subsection{\emph{yuv82rgb\index{yuv82rgb}} }\vspace*{-0.7em}
Convert selected images from YUV8 to RGB color bases.

\section{Geometry manipulation}


\subsection{\emph{append\index{append}} (+)}\vspace*{-0.7em}
~\\\textbf{\Cb{Arguments: }}\begin{flushleft}
{\small \Cb{\hspace*{0.5cm}$\bullet$~~\texttt{[image]{\comma}axis{\comma}\_centering}}}~~~\\
{\small \Cb{\hspace*{0.5cm}$\bullet$~~\texttt{axis{\comma}\_centering}}}\end{flushleft}
Append specified image to selected images{\comma} or all selected images together{\comma} along specified axis.
~\\(\emph{eq. to} {\small \texttt{'a').\textbackslash n}}).
~\\'axis' can be \{ x ~$|$~ y ~$|$~ z ~$|$~ c \}.
~\\Usual 'centering' values are \{ 0=left-justified ~$|$~ 0.5=centered ~$|$~ 1=right-justified \}.
\begin{flushleft}\Cc{\textbf{Default value}:\\~\\\hspace*{0.5cm}{\small $\bullet$~~\texttt{'centering=0'.}}}\end{flushleft}
\begin{center}\includegraphics[keepaspectratio=true,height=6cm,width=\textwidth]{img/gmic_stdlib232.jpg}\\
{\footnotesize \textbf{Example 232~:} \texttt{image.jpg split y{\comma}10 reverse append y}}
\\\includegraphics[keepaspectratio=true,height=6cm,width=\textwidth]{img/gmic_stdlib233.jpg}\\
{\footnotesize \textbf{Example 233~:} \texttt{image.jpg repeat 5 --rows[0] 0{\comma}\{10+18*\$$>$\}\% done remove[0] append x{\comma}0.5}}
\\\includegraphics[keepaspectratio=true,height=6cm,width=\textwidth]{img/gmic_stdlib234.jpg}\\
{\footnotesize \textbf{Example 234~:} \texttt{image.jpg append[0] [0]{\comma}y}}
\end{center}

\subsection{\emph{append\_tiles\index{append\_tiles}} }\vspace*{-0.7em}
~\\\textbf{\Cb{Arguments: }}\begin{flushleft}
{\small \Cb{\hspace*{0.5cm}$\bullet$~~\texttt{\_M$>$=0{\comma}\_N$>$=0{\comma}0$<$=\_centering\_x$<$=1{\comma}0$<$=\_centering\_y$<$=1}}}\end{flushleft}
Append MxN selected tiles as new images.
~\\If 'N' is set to 0{\comma} number of rows is estimated automatically.
~\\If 'M' is set to 0{\comma} number of columns is estimated automatically.
~\\If 'M' and 'N' are both set to '0'{\comma} auto-mode is used.
~\\If 'M' or 'N' is set to 0{\comma} only a single image is produced.
~\\'centering\_x' and 'centering\_y' tells about the centering of tiles when they have different sizes.
\begin{flushleft}\Cc{\textbf{Default values}:\\~\\\hspace*{0.5cm}{\small $\bullet$~~\texttt{'M=0'{\comma} 'N=0'{\comma} 'centering\_x=centering\_y=0.5'.}}}\end{flushleft}
\begin{center}\includegraphics[keepaspectratio=true,height=6cm,width=\textwidth]{img/gmic_stdlib235.jpg}\\
{\footnotesize \textbf{Example 235~:} \texttt{image.jpg split xy{\comma}4 append\_tiles {\comma}}}
\end{center}

\subsection{\emph{apply\_scales\index{apply\_scales}} }\vspace*{-0.7em}
~\\\textbf{\Cb{Arguments: }}\begin{flushleft}
{\small \Cb{\hspace*{0.5cm}$\bullet$~~\texttt{"command"{\comma}number\_of\_scales$>$0{\comma}\_min\_scale[\%]$>$=0{\comma}\_max\_scale[\%]$>$\-=0{\comma}\_scale\_gamma$>$0{\comma}\_interpolation}}}\end{flushleft}
Apply specified command on different scales of selected images.
~\\'interpolation' can be \{ 0=none ~$|$~ 1=nearest ~$|$~ 2=average ~$|$~ 3=linear ~$|$~ 4=grid ~$|$~ 5=bicubic ~$|$~ 6=lanczos \}.
\begin{flushleft}\Cc{\textbf{Default value}:\\~\\\hspace*{0.5cm}{\small $\bullet$~~\texttt{'min\_scale=25\%'{\comma} 'max\_scale=100\%'} and \texttt{'interpolation=3'.}}}\end{flushleft}
\begin{center}\includegraphics[keepaspectratio=true,height=6cm,width=\textwidth]{img/gmic_stdlib236.jpg}\\
{\footnotesize \textbf{Example 236~:} \texttt{image.jpg apply\_scales "blur 5 sharpen 1000"{\comma}4}}
\end{center}

\subsection{\emph{autocrop\index{autocrop}} (+)}\vspace*{-0.7em}
~\\\textbf{\Cb{Arguments: }}\begin{flushleft}
{\small \Cb{\hspace*{0.5cm}$\bullet$~~\texttt{value1{\comma}value2{\comma}...}}}~~~\\
{\small \Cb{\hspace*{0.5cm}$\bullet$~~\texttt{(no arg)}}}\end{flushleft}
Autocrop selected images by specified vector-valued intensity.
~\\If no arguments are provided{\comma} cropping value is guessed.
\begin{center}\includegraphics[keepaspectratio=true,height=6cm,width=\textwidth]{img/gmic_stdlib237.jpg}\\
{\footnotesize \textbf{Example 237~:} \texttt{400{\comma}400{\comma}1{\comma}3 fill\_color 64{\comma}128{\comma}255 ellipse 50\%{\comma}50\%{\comma}120{\comma}120{\comma}0{\comma}1{\comma}255 --autocrop}}
\end{center}

\subsection{\emph{autocrop\_components\index{autocrop\_components}} }\vspace*{-0.7em}
~\\\textbf{\Cb{Arguments: }}\begin{flushleft}
{\small \Cb{\hspace*{0.5cm}$\bullet$~~\texttt{\_threshold[\%]{\comma}\_min\_area[\%]$>$=0{\comma}\_is\_high\_connectivity=\{ 0 ~$|$~ 1 \-\}{\comma}\_output\_type=\{ 0=crop ~$|$~ 1=segmentation ~$|$~ 2=coordinates \}}}}\end{flushleft}
Autocrop and extract connected components in selected images{\comma} according to a mask given as the last channel of
each of the selected image (e.g. alpha-channel).
\begin{flushleft}\Cc{\textbf{Default values}:\\~\\\hspace*{0.5cm}{\small $\bullet$~~\texttt{'threshold=0\%'{\comma} 'min\_area=0.1\%'{\comma} 'is\_high\_connectivity=0'} and \texttt{'output\_type=1'.}}}\end{flushleft}
\begin{center}\includegraphics[keepaspectratio=true,height=6cm,width=\textwidth]{img/gmic_stdlib238.jpg}\\
{\footnotesize \textbf{Example 238~:} \texttt{256{\comma}256 noise 0.1{\comma}2 dilate\_circ 20 label\_fg 0{\comma}1 normalize 0{\comma}255 --neq 0 *[-1] 255 append c --autocrop\_components {\comma}}}
\end{center}

\subsection{\emph{autocrop\_seq\index{autocrop\_seq}} }\vspace*{-0.7em}
~\\\textbf{\Cb{Arguments: }}\begin{flushleft}
{\small \Cb{\hspace*{0.5cm}$\bullet$~~\texttt{value1{\comma}value2{\comma}... ~$|$~ auto}}}\end{flushleft}
Autocrop selected images using the crop geometry of the last one by specified vector-valued intensity{\comma}
or by automatic guessing the cropping value.
\begin{flushleft}\Cc{\textbf{Default value}:\\~\\\hspace*{0.5cm}{\small $\bullet$~~\texttt{auto mode.}}}\end{flushleft}
\begin{center}\includegraphics[keepaspectratio=true,height=6cm,width=\textwidth]{img/gmic_stdlib239.jpg}\\
{\footnotesize \textbf{Example 239~:} \texttt{image.jpg --fill[-1] 0 ellipse[-1] 50\%{\comma}50\%{\comma}30\%{\comma}20\%{\comma}0{\comma}1{\comma}1 autocrop\_seq 0}}
\end{center}

\subsection{\emph{channels\index{channels}} (+)}\vspace*{-0.7em}
~\\\textbf{\Cb{Arguments: }}\begin{flushleft}
{\small \Cb{\hspace*{0.5cm}$\bullet$~~\texttt{\{ [image0] ~$|$~ c0[\%] \}{\comma}\_\{ [image1] ~$|$~ c1[\%] \}}}}\end{flushleft}
Keep only specified channels of selected images.
~\\Dirichlet boundary is used when specified channels are out of range.
\begin{center}\includegraphics[keepaspectratio=true,height=6cm,width=\textwidth]{img/gmic_stdlib240.jpg}\\
{\footnotesize \textbf{Example 240~:} \texttt{image.jpg channels 0{\comma}1}}
\\\includegraphics[keepaspectratio=true,height=6cm,width=\textwidth]{img/gmic_stdlib241.jpg}\\
{\footnotesize \textbf{Example 241~:} \texttt{image.jpg luminance channels 0{\comma}2}}
\end{center}

\subsection{\emph{columns\index{columns}} (+)}\vspace*{-0.7em}
~\\\textbf{\Cb{Arguments: }}\begin{flushleft}
{\small \Cb{\hspace*{0.5cm}$\bullet$~~\texttt{\{ [image0] ~$|$~ x0[\%] \}{\comma}\_\{ [image1] ~$|$~ x1[\%] \}}}}\end{flushleft}
Keep only specified columns of selected images.
~\\Dirichlet boundary is used when specified columns are out of range.
\begin{center}\includegraphics[keepaspectratio=true,height=6cm,width=\textwidth]{img/gmic_stdlib242.jpg}\\
{\footnotesize \textbf{Example 242~:} \texttt{image.jpg columns -25\%{\comma}50\%}}
\end{center}

\subsection{\emph{crop\index{crop}} (+)}\vspace*{-0.7em}
~\\\textbf{\Cb{Arguments: }}\begin{flushleft}
{\small \Cb{\hspace*{0.5cm}$\bullet$~~\texttt{x0[\%]{\comma}x1[\%]{\comma}\_boundary\_conditions}}}~~~\\
{\small \Cb{\hspace*{0.5cm}$\bullet$~~\texttt{x0[\%]{\comma}y0[\%]{\comma}x1[\%]{\comma}y1[\%]{\comma}\_boundary\_conditions}}}~~~\\
{\small \Cb{\hspace*{0.5cm}$\bullet$~~\texttt{x0[\%]{\comma}y0[\%]{\comma}z0[\%]{\comma}x1[\%]{\comma}y1[\%]{\comma}z1[\%]{\comma}\_boundary\_conditions}}}~~~\\
{\small \Cb{\hspace*{0.5cm}$\bullet$~~\texttt{x0[\%]{\comma}y0[\%]{\comma}z0[\%]{\comma}c0[\%]{\comma}x1[\%]{\comma}y1[\%]{\comma}z1[\%]{\comma}c1[\%]{\comma}\_boundary\_co\-nditions}}}~~~\\
{\small \Cb{\hspace*{0.5cm}$\bullet$~~\texttt{(no arg)}}}\end{flushleft}
Crop selected images with specified region coordinates.
~\\(\emph{eq. to} {\small \texttt{'z').\textbackslash n}}).
~\\'boundary\_conditions' can be \{ 0=dirichlet ~$|$~ 1=neumann ~$|$~ 2=periodic ~$|$~ 3=mirror \}.
~\\(no arg) runs interactive mode (uses the instant display window [0] if opened).
\begin{flushleft}\Cc{\textbf{Default value}:\\~\\\hspace*{0.5cm}{\small $\bullet$~~\texttt{'boundary\_conditions=0'.}}}\end{flushleft}
\begin{center}\includegraphics[keepaspectratio=true,height=6cm,width=\textwidth]{img/gmic_stdlib243.jpg}\\
{\footnotesize \textbf{Example 243~:} \texttt{image.jpg --crop -230{\comma}-230{\comma}280{\comma}280{\comma}1 crop[0] -230{\comma}-230{\comma}280{\comma}280{\comma}0}}
\\\includegraphics[keepaspectratio=true,height=6cm,width=\textwidth]{img/gmic_stdlib244.jpg}\\
{\footnotesize \textbf{Example 244~:} \texttt{image.jpg crop 25\%{\comma}25\%{\comma}75\%{\comma}75\%}}
\end{center}

\subsection{\emph{diagonal\index{diagonal}} }\vspace*{-0.7em}
Transform selected vectors as diagonal matrices.
\begin{center}\includegraphics[keepaspectratio=true,height=6cm,width=\textwidth]{img/gmic_stdlib245.jpg}\\
{\footnotesize \textbf{Example 245~:} \texttt{1{\comma}10{\comma}1{\comma}1{\comma}'y' --diagonal}}
\end{center}

\subsection{\emph{elevate\index{elevate}} }\vspace*{-0.7em}
~\\\textbf{\Cb{Arguments: }}\begin{flushleft}
{\small \Cb{\hspace*{0.5cm}$\bullet$~~\texttt{\_depth{\comma}\_is\_plain=\{ 0 ~$|$~ 1 \}{\comma}\_is\_colored=\{ 0 ~$|$~ 1 \}}}}\end{flushleft}
Elevate selected 2d images into 3d volumes.
\begin{flushleft}\Cc{\textbf{Default values}:\\~\\\hspace*{0.5cm}{\small $\bullet$~~\texttt{'depth=64'{\comma} 'is\_plain=1'} and \texttt{'is\_colored=1'.}}}\end{flushleft}


\subsection{\emph{expand\_x\index{expand\_x}} }\vspace*{-0.7em}
~\\\textbf{\Cb{Arguments: }}\begin{flushleft}
{\small \Cb{\hspace*{0.5cm}$\bullet$~~\texttt{size\_x$>$=0{\comma}\_boundary\_conditions=\{ 0=dirichlet ~$|$~ 1=neumann ~$|$~ 2\-=periodic ~$|$~ 3=mirror \}}}}\end{flushleft}
Expand selected images along the x-axis.
\begin{flushleft}\Cc{\textbf{Default value}:\\~\\\hspace*{0.5cm}{\small $\bullet$~~\texttt{'boundary\_conditions=1'.}}}\end{flushleft}
\begin{center}\includegraphics[keepaspectratio=true,height=6cm,width=\textwidth]{img/gmic_stdlib246.jpg}\\
{\footnotesize \textbf{Example 246~:} \texttt{image.jpg expand\_x 30{\comma}0}}
\end{center}

\subsection{\emph{expand\_xy\index{expand\_xy}} }\vspace*{-0.7em}
~\\\textbf{\Cb{Arguments: }}\begin{flushleft}
{\small \Cb{\hspace*{0.5cm}$\bullet$~~\texttt{size$>$=0{\comma}\_boundary\_conditions=\{ 0=dirichlet ~$|$~ 1=neumann ~$|$~ 2=p\-eriodic ~$|$~ 3=mirror \}}}}\end{flushleft}
Expand selected images along the xy-axes.
\begin{flushleft}\Cc{\textbf{Default value}:\\~\\\hspace*{0.5cm}{\small $\bullet$~~\texttt{'boundary\_conditions=1'.}}}\end{flushleft}
\begin{center}\includegraphics[keepaspectratio=true,height=6cm,width=\textwidth]{img/gmic_stdlib247.jpg}\\
{\footnotesize \textbf{Example 247~:} \texttt{image.jpg expand\_xy 30{\comma}0}}
\end{center}

\subsection{\emph{expand\_xyz\index{expand\_xyz}} }\vspace*{-0.7em}
~\\\textbf{\Cb{Arguments: }}\begin{flushleft}
{\small \Cb{\hspace*{0.5cm}$\bullet$~~\texttt{size$>$=0{\comma}\_boundary\_conditions=\{ 0=dirichlet ~$|$~ 1=neumann ~$|$~ 2=p\-eriodic ~$|$~ 3=mirror \}}}}\end{flushleft}
Expand selected images along the xyz-axes.
\begin{flushleft}\Cc{\textbf{Default value}:\\~\\\hspace*{0.5cm}{\small $\bullet$~~\texttt{'boundary\_conditions=1'.}}}\end{flushleft}


\subsection{\emph{expand\_y\index{expand\_y}} }\vspace*{-0.7em}
~\\\textbf{\Cb{Arguments: }}\begin{flushleft}
{\small \Cb{\hspace*{0.5cm}$\bullet$~~\texttt{size\_y$>$=0{\comma}\_boundary\_conditions=\{ 0=dirichlet ~$|$~ 1=neumann ~$|$~ 2\-=periodic ~$|$~ 3=mirror \}}}}\end{flushleft}
Expand selected images along the y-axis.
\begin{flushleft}\Cc{\textbf{Default value}:\\~\\\hspace*{0.5cm}{\small $\bullet$~~\texttt{'boundary\_conditions=1'.}}}\end{flushleft}
\begin{center}\includegraphics[keepaspectratio=true,height=6cm,width=\textwidth]{img/gmic_stdlib248.jpg}\\
{\footnotesize \textbf{Example 248~:} \texttt{image.jpg expand\_y 30{\comma}0}}
\end{center}

\subsection{\emph{expand\_z\index{expand\_z}} }\vspace*{-0.7em}
~\\\textbf{\Cb{Arguments: }}\begin{flushleft}
{\small \Cb{\hspace*{0.5cm}$\bullet$~~\texttt{size\_z$>$=0{\comma}\_boundary\_conditions=\{ 0=dirichlet ~$|$~ 1=neumann ~$|$~ 2\-=periodic ~$|$~ 3=mirror \}}}}\end{flushleft}
Expand selected images along the z-axis.
\begin{flushleft}\Cc{\textbf{Default value}:\\~\\\hspace*{0.5cm}{\small $\bullet$~~\texttt{'boundary\_conditions=1'.}}}\end{flushleft}


\subsection{\emph{extract\_region\index{extract\_region}} }\vspace*{-0.7em}
~\\\textbf{\Cb{Arguments: }}\begin{flushleft}
{\small \Cb{\hspace*{0.5cm}$\bullet$~~\texttt{[label\_image]{\comma}\_extract\_xyz\_coordinates=\{ 0 ~$|$~ 1 \}{\comma}\_label\_1{\comma}..\-.{\comma}\_label\_M}}}\end{flushleft}
Extract all pixels of selected images whose corresponding label in '[label\_image]' is equal to 'label\_m'{\comma}
and output them as M column images.
\begin{flushleft}\Cc{\textbf{Default value}:\\~\\\hspace*{0.5cm}{\small $\bullet$~~\texttt{'extract\_xyz\_coordinates=0'.}}}\end{flushleft}
\begin{center}\includegraphics[keepaspectratio=true,height=6cm,width=\textwidth]{img/gmic_stdlib249.jpg}\\
{\footnotesize \textbf{Example 249~:} \texttt{image.jpg --blur 3 quantize. 4{\comma}0 --extract\_region[0] [1]{\comma}0{\comma}1{\comma}3}}
\end{center}

\subsection{\emph{montage\index{montage}} }\vspace*{-0.7em}
~\\\textbf{\Cb{Arguments: }}\begin{flushleft}
{\small \Cb{\hspace*{0.5cm}$\bullet$~~\texttt{"\_layout\_code"{\comma}\_montage\_mode=\{ 0$<$=centering$<$=1 ~$|$~ 2$<$=scale+2$<$\-=3 \}{\comma}\_output\_mode=\{ 0=single layer ~$|$~ 1=multiple layers \}{\comma}"\_p\-rocessing\_command"}}}\end{flushleft}
Create a single image montage from selected images{\comma} according to specified layout code :
- 'X' to assemble all images using an automatically estimated layout.
- 'H' to assemble all images horizontally.
- 'V' to assemble all images vertically.
- 'A' to assemble all images as an horizontal array.
- 'B' to assemble all images as a vertical array.
- 'Ha:b' to assemble two blocks 'a' and 'b' horizontally.
- 'Va:b' to assemble two blocks 'a' and 'b' vertically.
- 'Ra' to rotate a block 'a' by 90 deg. ('RRa' for 180 deg. and 'RRRa' for 270 deg.).
- 'Ma' to mirror a block 'a' along the X-axis ('MRRa' for the Y-axis).
~\\A block 'a' can be an image indice (treated periodically) or a nested layout expression 'Hb:c'{\comma}'Vb:c'{\comma}'Rb' or 'Mb' itself.
~\\For example{\comma} layout code 'H0:V1:2' creates an image where image [0] is on the left{\comma} and images [1] and [2] vertically packed on the right.
\begin{flushleft}\Cc{\textbf{Default values}:\\~\\\hspace*{0.5cm}{\small $\bullet$~~\texttt{'layout\_code=X'{\comma} 'montage\_mode=2'{\comma} output\_mode='0'} and \texttt{'processing\_command=""'.}}}\end{flushleft}
\begin{center}\includegraphics[keepaspectratio=true,height=6cm,width=\textwidth]{img/gmic_stdlib250.jpg}\\
{\footnotesize \textbf{Example 250~:} \texttt{image.jpg sample ? --plasma[0] shape\_cupid 256 normalize 0{\comma}255 frame 3{\comma}3{\comma}0 frame 10{\comma}10{\comma}255 to\_rgb --montage A --montage[\textasciicircum -1] H1:V0:VH2:1H0:3}}
\end{center}

\subsection{\emph{mirror\index{mirror}} (+)}\vspace*{-0.7em}
~\\\textbf{\Cb{Arguments: }}\begin{flushleft}
{\small \Cb{\hspace*{0.5cm}$\bullet$~~\texttt{\{ x ~$|$~ y ~$|$~ z \}...\{ x ~$|$~ y ~$|$~ z \}}}}\end{flushleft}
Mirror selected images along specified axes.
\begin{center}\includegraphics[keepaspectratio=true,height=6cm,width=\textwidth]{img/gmic_stdlib251.jpg}\\
{\footnotesize \textbf{Example 251~:} \texttt{image.jpg --mirror y --mirror[0] c}}
\\\includegraphics[keepaspectratio=true,height=6cm,width=\textwidth]{img/gmic_stdlib252.jpg}\\
{\footnotesize \textbf{Example 252~:} \texttt{image.jpg --mirror x --mirror y append\_tiles 2{\comma}2}}
\end{center}

\subsection{\emph{permute\index{permute}} (+)}\vspace*{-0.7em}
~\\\textbf{\Cb{Arguments: }}\begin{flushleft}
{\small \Cb{\hspace*{0.5cm}$\bullet$~~\texttt{permutation\_string}}}\end{flushleft}
Permute selected image axes by specified permutation.
~\\'permutation' is a combination of the character set \{x~$|$~y~$|$~z~$|$~c\}{\comma}
e.g. 'xycz'{\comma} 'cxyz'{\comma} ...
\begin{center}\includegraphics[keepaspectratio=true,height=6cm,width=\textwidth]{img/gmic_stdlib253.jpg}\\
{\footnotesize \textbf{Example 253~:} \texttt{image.jpg permute yxzc}}
\end{center}

\subsection{\emph{resize\index{resize}} (+)}\vspace*{-0.7em}
~\\\textbf{\Cb{Arguments: }}\begin{flushleft}
{\small \Cb{\hspace*{0.5cm}$\bullet$~~\texttt{[image]{\comma}\_interpolation{\comma}\_boundary\_conditions{\comma}\_ax{\comma}\_ay{\comma}\_az{\comma}\_ac}}}~~~\\
{\small \Cb{\hspace*{0.5cm}$\bullet$~~\texttt{\{[image\_w] ~$|$~ width$>$0[\%]\}{\comma}\_\{[image\_h] ~$|$~ height$>$0[\%]\}{\comma}\_\{[image\-\_d] ~$|$~ depth$>$0[\%]\}{\comma}\_\{[image\_s] ~$|$~ spectrum$>$0[\%]\}{\comma}\_interpolatio\-n{\comma}\_boundary\_conditions{\comma}\_ax{\comma}\_ay{\comma}\_az{\comma}\_ac}}}~~~\\
{\small \Cb{\hspace*{0.5cm}$\bullet$~~\texttt{(no arg)}}}\end{flushleft}
Resize selected images with specified geometry.
~\\(\emph{eq. to} {\small \texttt{'r').\textbackslash n}}).
~\\'interpolation' can be \{ -1=none (memory content) ~$|$~ 0=none ~$|$~ 1=nearest ~$|$~ 2=average ~$|$~ 3=linear ~$|$~ 4=grid ~$|$~ 5=bicubic ~$|$~ 6=lanczos \}.
~\\'boundary\_conditions' has different meanings{\comma} according to the chosen 'interpolation' mode :
. When 'interpolation==\{ -1 ~$|$~ 1 ~$|$~ 2 ~$|$~ 4 \}'{\comma} 'boundary\_conditions' is meaningless.
. When 'interpolation==0'{\comma} 'boundary\_conditions' can be \{ 0=dirichlet ~$|$~ 1=neumann ~$|$~ 2=periodic ~$|$~ 3=mirror \}.
. When 'interpolation==\{ 3 ~$|$~ 5 ~$|$~ 6 \}'{\comma} 'boundary\_conditions' can be \{ 0=none ~$|$~ 1=neumann \}.
~\\'ax{\comma}ay{\comma}az{\comma}ac' set the centering along each axis when 'interpolation=0 or 4'
~\\(set to '0' by default{\comma} must be defined in range [0{\comma}1]).
~\\(no arg) runs interactive mode (uses the instant display window [0] if opened).
\begin{flushleft}\Cc{\textbf{Default values}:\\~\\\hspace*{0.5cm}{\small $\bullet$~~\texttt{'interpolation=1'{\comma} 'boundary\_conditions=0'} and \texttt{'ax=ay=az=ac=0'.}}}\end{flushleft}
\begin{center}\includegraphics[keepaspectratio=true,height=6cm,width=\textwidth]{img/gmic_stdlib254.jpg}\\
{\footnotesize \textbf{Example 254~:} \texttt{image.jpg (0{\comma}1;0{\comma}1\textasciicircum 0{\comma}0;1{\comma}1\textasciicircum 1{\comma}1;1{\comma}1) resize[-1] [-2]{\comma}3 mul[-2] [-1]}}
\\\includegraphics[keepaspectratio=true,height=6cm,width=\textwidth]{img/gmic_stdlib255.jpg}\\
{\footnotesize \textbf{Example 255~:} \texttt{image.jpg --resize[-1] 256{\comma}128{\comma}1{\comma}3{\comma}2 --resize[-1] 120\%{\comma}120\%{\comma}1{\comma}3{\comma}0{\comma}1{\comma}0.5{\comma}0.5 --resize[-1] 120\%{\comma}120\%{\comma}1{\comma}3{\comma}0{\comma}0{\comma}0.2{\comma}0.2 --resize[-1] [0]{\comma}[0]{\comma}1{\comma}3{\comma}4}}
\end{center}

\subsection{\emph{resize\_mn\index{resize\_mn}} }\vspace*{-0.7em}
~\\\textbf{\Cb{Arguments: }}\begin{flushleft}
{\small \Cb{\hspace*{0.5cm}$\bullet$~~\texttt{width[\%]$>$=0{\comma}\_height[\%]$>$=0{\comma}\_depth[\%]$>$=0{\comma}\_B\_value{\comma}\_C\_value}}}\end{flushleft}
Resize selected images with Mitchell-Netravali filter (cubic).
~\\For details about the method{\comma} see: https://de.wikipedia.org/wiki/Mitchell-Netravali-Filter
\begin{flushleft}\Cc{\textbf{Default values}:\\~\\\hspace*{0.5cm}{\small $\bullet$~~\texttt{'height=100\%'{\comma} 'depth=100\%'{\comma} 'B=0.3333'} and \texttt{'C=0.3333'.}}}\end{flushleft}
\begin{center}\includegraphics[keepaspectratio=true,height=6cm,width=\textwidth]{img/gmic_stdlib256.jpg}\\
{\footnotesize \textbf{Example 256~:} \texttt{image.jpg resize2dx 32 resize\_mn 800\%{\comma}800\%}}
\end{center}

\subsection{\emph{resize\_pow2\index{resize\_pow2}} }\vspace*{-0.7em}
~\\\textbf{\Cb{Arguments: }}\begin{flushleft}
{\small \Cb{\hspace*{0.5cm}$\bullet$~~\texttt{\_interpolation{\comma}\_boundary\_conditions{\comma}\_ax{\comma}\_ay{\comma}\_az{\comma}\_ac}}}\end{flushleft}
Resize selected images so that each dimension is a power of 2.
~\\'interpolation' can be \{ -1=none (memory content) ~$|$~ 0=none ~$|$~ 1=nearest ~$|$~ 2=average ~$|$~ 3=linear ~$|$~ 4=grid ~$|$~ 5=bicubic ~$|$~ 6=lanczos \}.
~\\'boundary\_conditions' has different meanings{\comma} according to the chosen 'interpolation' mode :
. When 'interpolation==\{ -1 ~$|$~ 1 ~$|$~ 2 ~$|$~ 4 \}'{\comma} 'boundary\_conditions' is meaningless.
. When 'interpolation==0'{\comma} 'boundary\_conditions' can be \{ 0=dirichlet ~$|$~ 1=neumann ~$|$~ 2=periodic ~$|$~ 3=mirror \}.
. When 'interpolation==\{ 3 ~$|$~ 5 ~$|$~ 6 \}'{\comma} 'boundary\_conditions' can be \{ 0=none ~$|$~ 1=neumann \}.
~\\'ax{\comma}ay{\comma}az{\comma}ac' set the centering along each axis when 'interpolation=0'
~\\(set to '0' by default{\comma} must be defined in range [0{\comma}1]).
\begin{flushleft}\Cc{\textbf{Default values}:\\~\\\hspace*{0.5cm}{\small $\bullet$~~\texttt{'interpolation=0'{\comma} 'boundary\_conditions=0'} and \texttt{'ax=ay=az=ac=0'.}}}\end{flushleft}
\begin{center}\includegraphics[keepaspectratio=true,height=6cm,width=\textwidth]{img/gmic_stdlib257.jpg}\\
{\footnotesize \textbf{Example 257~:} \texttt{image.jpg --resize\_pow2[-1] 0}}
\end{center}

\subsection{\emph{resize\_ratio2d\index{resize\_ratio2d}} }\vspace*{-0.7em}
~\\\textbf{\Cb{Arguments: }}\begin{flushleft}
{\small \Cb{\hspace*{0.5cm}$\bullet$~~\texttt{width$>$0{\comma}height$>$0{\comma}\_mode=\{ 0=inside ~$|$~ 1=outside ~$|$~ 2=padded \}{\comma}0\-=$<$\_interpolation$<$=6}}}\end{flushleft}
Resize selected images while preserving their aspect ratio.
~\\(\emph{eq. to} {\small \texttt{'rr2d'}}).
\begin{flushleft}\Cc{\textbf{Default values}:\\~\\\hspace*{0.5cm}{\small $\bullet$~~\texttt{'mode=0'} and \texttt{'interpolation=6'.}}}\end{flushleft}


\subsection{\emph{resize2dx\index{resize2dx}} }\vspace*{-0.7em}
~\\\textbf{\Cb{Arguments: }}\begin{flushleft}
{\small \Cb{\hspace*{0.5cm}$\bullet$~~\texttt{width[\%]$>$0{\comma}\_interpolation{\comma}\_boundary\_conditions{\comma}\_ax{\comma}\_ay{\comma}\_az{\comma}\_\-ac}}}\end{flushleft}
Resize selected images along the x-axis{\comma} preserving 2d ratio.
~\\(\emph{eq. to} {\small \texttt{'r2dx').\textbackslash n}}).
~\\'interpolation' can be \{ -1=none (memory content) ~$|$~ 0=none ~$|$~ 1=nearest ~$|$~ 2=average ~$|$~ 3=linear ~$|$~ 4=grid ~$|$~ 5=bicubic ~$|$~ 6=lanczos \}.
~\\'boundary\_conditions' has different meanings{\comma} according to the chosen 'interpolation' mode :
. When 'interpolation==\{ -1 ~$|$~ 1 ~$|$~ 2 ~$|$~ 4 \}'{\comma} 'boundary\_conditions' is meaningless.
. When 'interpolation==0'{\comma} 'boundary\_conditions' can be \{ 0=dirichlet ~$|$~ 1=neumann ~$|$~ 2=periodic ~$|$~ 3=mirror \}.
. When 'interpolation==\{ 3 ~$|$~ 5 ~$|$~ 6 \}'{\comma} 'boundary\_conditions' can be \{ 0=none ~$|$~ 1=neumann \}.
~\\'ax{\comma}ay{\comma}az{\comma}ac' set the centering along each axis when 'interpolation=0'
~\\(set to '0' by default{\comma} must be defined in range [0{\comma}1]).
\begin{flushleft}\Cc{\textbf{Default values}:\\~\\\hspace*{0.5cm}{\small $\bullet$~~\texttt{'interpolation=3'{\comma} 'boundary\_conditions=0'} and \texttt{'ax=ay=az=ac=0'.}}}\end{flushleft}
\begin{center}\includegraphics[keepaspectratio=true,height=6cm,width=\textwidth]{img/gmic_stdlib258.jpg}\\
{\footnotesize \textbf{Example 258~:} \texttt{image.jpg --resize2dx 100{\comma}2 append x}}
\end{center}

\subsection{\emph{resize2dy\index{resize2dy}} }\vspace*{-0.7em}
~\\\textbf{\Cb{Arguments: }}\begin{flushleft}
{\small \Cb{\hspace*{0.5cm}$\bullet$~~\texttt{height[\%]$>$=0{\comma}\_interpolation{\comma}\_boundary\_conditions{\comma}\_ax{\comma}\_ay{\comma}\_az\-{\comma}\_ac}}}\end{flushleft}
Resize selected images along the y-axis{\comma} preserving 2d ratio.
~\\(\emph{eq. to} {\small \texttt{'r2dy').\textbackslash n}}).
~\\'interpolation' can be \{ -1=none (memory content) ~$|$~ 0=none ~$|$~ 1=nearest ~$|$~ 2=average ~$|$~ 3=linear ~$|$~ 4=grid ~$|$~ 5=bicubic ~$|$~ 6=lanczos \}.
~\\'boundary\_conditions' has different meanings{\comma} according to the chosen 'interpolation' mode :
. When 'interpolation==\{ -1 ~$|$~ 1 ~$|$~ 2 ~$|$~ 4 \}'{\comma} 'boundary\_conditions' is meaningless.
. When 'interpolation==0'{\comma} 'boundary\_conditions' can be \{ 0=dirichlet ~$|$~ 1=neumann ~$|$~ 2=periodic ~$|$~ 3=mirror \}.
. When 'interpolation==\{ 3 ~$|$~ 5 ~$|$~ 6 \}'{\comma} 'boundary\_conditions' can be \{ 0=none ~$|$~ 1=neumann \}.
~\\'ax{\comma}ay{\comma}az{\comma}ac' set the centering along each axis when 'interpolation=0'
~\\(set to '0' by default{\comma} must be defined in range [0{\comma}1]).
\begin{flushleft}\Cc{\textbf{Default values}:\\~\\\hspace*{0.5cm}{\small $\bullet$~~\texttt{'interpolation=3'{\comma} 'boundary\_conditions=0'} and \texttt{'ax=ay=az=ac=0'.}}}\end{flushleft}
\begin{center}\includegraphics[keepaspectratio=true,height=6cm,width=\textwidth]{img/gmic_stdlib259.jpg}\\
{\footnotesize \textbf{Example 259~:} \texttt{image.jpg --resize2dy 100{\comma}2 append x}}
\end{center}

\subsection{\emph{resize3dx\index{resize3dx}} }\vspace*{-0.7em}
~\\\textbf{\Cb{Arguments: }}\begin{flushleft}
{\small \Cb{\hspace*{0.5cm}$\bullet$~~\texttt{width[\%]$>$0{\comma}\_interpolation{\comma}\_boundary\_conditions{\comma}\_ax{\comma}\_ay{\comma}\_az{\comma}\_\-ac}}}\end{flushleft}
Resize selected images along the x-axis{\comma} preserving 3d ratio.
~\\(\emph{eq. to} {\small \texttt{'r3dx').\textbackslash n}}).
~\\'interpolation' can be \{ -1=none (memory content) ~$|$~ 0=none ~$|$~ 1=nearest ~$|$~ 2=average ~$|$~ 3=linear ~$|$~ 4=grid ~$|$~ 5=bicubic ~$|$~ 6=lanczos \}.
~\\'boundary\_conditions' has different meanings{\comma} according to the chosen 'interpolation' mode :
. When 'interpolation==\{ -1 ~$|$~ 1 ~$|$~ 2 ~$|$~ 4 \}'{\comma} 'boundary\_conditions' is meaningless.
. When 'interpolation==0'{\comma} 'boundary\_conditions' can be \{ 0=dirichlet ~$|$~ 1=neumann ~$|$~ 2=periodic ~$|$~ 3=mirror \}.
. When 'interpolation==\{ 3 ~$|$~ 5 ~$|$~ 6 \}'{\comma} 'boundary\_conditions' can be \{ 0=none ~$|$~ 1=neumann \}.
~\\'ax{\comma}ay{\comma}az{\comma}ac' set the centering along each axis when 'interpolation=0'
~\\(set to '0' by default{\comma} must be defined in range [0{\comma}1]).
\begin{flushleft}\Cc{\textbf{Default values}:\\~\\\hspace*{0.5cm}{\small $\bullet$~~\texttt{'interpolation=3'{\comma} 'boundary\_conditions=0'} and \texttt{'ax=ay=az=ac=0'.}}}\end{flushleft}


\subsection{\emph{resize3dy\index{resize3dy}} }\vspace*{-0.7em}
~\\\textbf{\Cb{Arguments: }}\begin{flushleft}
{\small \Cb{\hspace*{0.5cm}$\bullet$~~\texttt{height[\%]$>$0{\comma}\_interpolation{\comma}\_boundary\_conditions{\comma}\_ax{\comma}\_ay{\comma}\_az{\comma}\-\_ac}}}\end{flushleft}
Resize selected images along the y-axis{\comma} preserving 3d ratio.
~\\(\emph{eq. to} {\small \texttt{'r3dy').\textbackslash n}}).
~\\'interpolation' can be \{ -1=none (memory content) ~$|$~ 0=none ~$|$~ 1=nearest ~$|$~ 2=average ~$|$~ 3=linear ~$|$~ 4=grid ~$|$~ 5=bicubic ~$|$~ 6=lanczos \}.
~\\'boundary\_conditions' has different meanings{\comma} according to the chosen 'interpolation' mode :
. When 'interpolation==\{ -1 ~$|$~ 1 ~$|$~ 2 ~$|$~ 4 \}'{\comma} 'boundary\_conditions' is meaningless.
. When 'interpolation==0'{\comma} 'boundary\_conditions' can be \{ 0=dirichlet ~$|$~ 1=neumann ~$|$~ 2=periodic ~$|$~ 3=mirror \}.
. When 'interpolation==\{ 3 ~$|$~ 5 ~$|$~ 6 \}'{\comma} 'boundary\_conditions' can be \{ 0=none ~$|$~ 1=neumann \}.
~\\'ax{\comma}ay{\comma}az{\comma}ac' set the centering along each axis when 'interpolation=0'
~\\(set to '0' by default{\comma} must be defined in range [0{\comma}1]).
\begin{flushleft}\Cc{\textbf{Default values}:\\~\\\hspace*{0.5cm}{\small $\bullet$~~\texttt{'interpolation=3'{\comma} 'boundary\_conditions=0'} and \texttt{'ax=ay=az=ac=0'.}}}\end{flushleft}


\subsection{\emph{resize3dz\index{resize3dz}} }\vspace*{-0.7em}
~\\\textbf{\Cb{Arguments: }}\begin{flushleft}
{\small \Cb{\hspace*{0.5cm}$\bullet$~~\texttt{depth[\%]$>$0{\comma}\_interpolation{\comma}\_boundary\_conditions{\comma}\_ax{\comma}\_ay{\comma}\_az{\comma}\_\-ac}}}\end{flushleft}
Resize selected images along the z-axis{\comma} preserving 3d ratio.
~\\(\emph{eq. to} {\small \texttt{'r3dz').\textbackslash n}}).
~\\'interpolation' can be \{ -1=none (memory content) ~$|$~ 0=none ~$|$~ 1=nearest ~$|$~ 2=average ~$|$~ 3=linear ~$|$~ 4=grid ~$|$~ 5=bicubic ~$|$~ 6=lanczos \}.
~\\'boundary\_conditions' has different meanings{\comma} according to the chosen 'interpolation' mode :
. When 'interpolation==\{ -1 ~$|$~ 1 ~$|$~ 2 ~$|$~ 4 \}'{\comma} 'boundary\_conditions' is meaningless.
. When 'interpolation==0'{\comma} 'boundary\_conditions' can be \{ 0=dirichlet ~$|$~ 1=neumann ~$|$~ 2=periodic ~$|$~ 3=mirror \}.
. When 'interpolation==\{ 3 ~$|$~ 5 ~$|$~ 6 \}'{\comma} 'boundary\_conditions' can be \{ 0=none ~$|$~ 1=neumann \}.
~\\'ax{\comma}ay{\comma}az{\comma}ac' set the centering along each axis when 'interpolation=0'
~\\(set to '0' by default{\comma} must be defined in range [0{\comma}1]).
\begin{flushleft}\Cc{\textbf{Default values}:\\~\\\hspace*{0.5cm}{\small $\bullet$~~\texttt{'interpolation=3'{\comma} 'boundary\_conditions=0'} and \texttt{'ax=ay=az=ac=0'.}}}\end{flushleft}


\subsection{\emph{rotate\index{rotate}} (+)}\vspace*{-0.7em}
~\\\textbf{\Cb{Arguments: }}\begin{flushleft}
{\small \Cb{\hspace*{0.5cm}$\bullet$~~\texttt{angle{\comma}\_interpolation{\comma}\_boundary\_conditions{\comma}\_center\_x[\%]{\comma}\_cent\-er\_y[\%]}}}~~~\\
{\small \Cb{\hspace*{0.5cm}$\bullet$~~\texttt{u{\comma}v{\comma}w{\comma}angle{\comma}interpolation{\comma}boundary\_conditions{\comma}\_center\_x[\%]{\comma}\_\-center\_y[\%]{\comma}\_center\_z[\%]}}}\end{flushleft}
Rotate selected images with specified angle (in deg.){\comma} and optionally 3d axis (u{\comma}v{\comma}w).
~\\'interpolation' can be \{ 0=none ~$|$~ 1=linear ~$|$~ 2=bicubic \}.
~\\'boundary\_conditions' can be \{ 0=dirichlet ~$|$~ 1=neumann ~$|$~ 2=periodic ~$|$~ 3=mirror \}.
~\\When a rotation center (cx{\comma}cy{\comma}\_cz) is specified{\comma} the size of the image is preserved.
\begin{flushleft}\Cc{\textbf{Default values}:\\~\\\hspace*{0.5cm}{\small $\bullet$~~\texttt{'interpolation=1'{\comma} 'boundary\_conditions=0'} and \texttt{'center\_x=center\_y=(undefined)'.}}}\end{flushleft}
\begin{center}\includegraphics[keepaspectratio=true,height=6cm,width=\textwidth]{img/gmic_stdlib260.jpg}\\
{\footnotesize \textbf{Example 260~:} \texttt{image.jpg --rotate -25{\comma}1{\comma}2{\comma}50\%{\comma}50\% rotate[0] 25}}
\end{center}

\subsection{\emph{rotate\_tileable\index{rotate\_tileable}} }\vspace*{-0.7em}
~\\\textbf{\Cb{Arguments: }}\begin{flushleft}
{\small \Cb{\hspace*{0.5cm}$\bullet$~~\texttt{angle{\comma}\_max\_size\_factor$>$=0}}}\end{flushleft}
Rotate selected images by specified angle and make them tileable.
~\\If resulting size of an image is too big{\comma} the image is replaced by a 1x1 image.
\begin{flushleft}\Cc{\textbf{Default values}:\\~\\\hspace*{0.5cm}{\small $\bullet$~~\texttt{'max\_size\_factor=8'.}}}\end{flushleft}


\subsection{\emph{rows\index{rows}} (+)}\vspace*{-0.7em}
~\\\textbf{\Cb{Arguments: }}\begin{flushleft}
{\small \Cb{\hspace*{0.5cm}$\bullet$~~\texttt{\{ [image0] ~$|$~ y0[\%] \}{\comma}\_\{ [image1] ~$|$~ y1[\%] \}}}}\end{flushleft}
Keep only specified rows of selected images.
~\\Dirichlet boundary conditions are used when specified rows are out of range.
\begin{center}\includegraphics[keepaspectratio=true,height=6cm,width=\textwidth]{img/gmic_stdlib261.jpg}\\
{\footnotesize \textbf{Example 261~:} \texttt{image.jpg rows -25\%{\comma}50\%}}
\end{center}

\subsection{\emph{scale2x\index{scale2x}} }\vspace*{-0.7em}
Resize selected images using the Scale2x algorithm.
\begin{center}\includegraphics[keepaspectratio=true,height=6cm,width=\textwidth]{img/gmic_stdlib262.jpg}\\
{\footnotesize \textbf{Example 262~:} \texttt{image.jpg threshold 50\% resize 50\%{\comma}50\% --scale2x}}
\end{center}

\subsection{\emph{scale3x\index{scale3x}} }\vspace*{-0.7em}
Resize selected images using the Scale3x algorithm.
\begin{center}\includegraphics[keepaspectratio=true,height=6cm,width=\textwidth]{img/gmic_stdlib263.jpg}\\
{\footnotesize \textbf{Example 263~:} \texttt{image.jpg threshold 50\% resize 33\%{\comma}33\% --scale3x}}
\end{center}

\subsection{\emph{scale\_dcci2x\index{scale\_dcci2x}} }\vspace*{-0.7em}
~\\\textbf{\Cb{Arguments: }}\begin{flushleft}
{\small \Cb{\hspace*{0.5cm}$\bullet$~~\texttt{\_edge\_threshold$>$=0{\comma}\_exponent$>$0{\comma}\_extend\_1px=\{ 0=false ~$|$~ 1=tru\-e \}}}}\end{flushleft}
Double image size using directional cubic convolution interpolation{\comma}
as described in https://en.wikipedia.org/wiki/Directional\_Cubic\_Convolution\_Interpolation.
\begin{flushleft}\Cc{\textbf{Default values}:\\~\\\hspace*{0.5cm}{\small $\bullet$~~\texttt{'edge\_threshold=1.15'{\comma} 'exponent=5'} and \texttt{'extend\_1px=0'.}}}\end{flushleft}
\begin{center}\includegraphics[keepaspectratio=true,height=6cm,width=\textwidth]{img/gmic_stdlib264.jpg}\\
{\footnotesize \textbf{Example 264~:} \texttt{image.jpg --scale\_dcci2x {\comma}}}
\end{center}

\subsection{\emph{seamcarve\index{seamcarve}} }\vspace*{-0.7em}
~\\\textbf{\Cb{Arguments: }}\begin{flushleft}
{\small \Cb{\hspace*{0.5cm}$\bullet$~~\texttt{\_width[\%]$>$=0{\comma}\_height[\%]$>$=0{\comma}\_is\_priority\_channel=\{ 0 ~$|$~ 1 \}{\comma}\_i\-s\_antialiasing=\{ 0 ~$|$~ 1 \}{\comma}\_maximum\_seams[\%]$>$=0}}}\end{flushleft}
Resize selected images with specified 2d geometry{\comma} using the seam-carving algorithm.
\begin{flushleft}\Cc{\textbf{Default values}:\\~\\\hspace*{0.5cm}{\small $\bullet$~~\texttt{'height=100\%'{\comma} 'is\_priority\_channel=0'{\comma} 'is\_antialiasing=1'} and \texttt{'maximum\_seams=25\%'.}}}\end{flushleft}
\begin{center}\includegraphics[keepaspectratio=true,height=6cm,width=\textwidth]{img/gmic_stdlib265.jpg}\\
{\footnotesize \textbf{Example 265~:} \texttt{image.jpg --seamcarve 60\%}}
\end{center}

\subsection{\emph{shift\index{shift}} (+)}\vspace*{-0.7em}
~\\\textbf{\Cb{Arguments: }}\begin{flushleft}
{\small \Cb{\hspace*{0.5cm}$\bullet$~~\texttt{vx[\%]{\comma}\_vy[\%]{\comma}\_vz[\%]{\comma}\_vc[\%]{\comma}\_boundary\_conditions{\comma}\_interpolati\-on=\{ 0=nearest\_neighbor ~$|$~ 1=linear \}}}}\end{flushleft}
Shift selected images by specified displacement vector.
~\\Displacement vector can be non-integer in which case linear interpolation of the shift is computed.
~\\'boundary\_conditions' can be \{ 0=dirichlet ~$|$~ 1=neumann ~$|$~ 2=periodic ~$|$~ 3=mirror \}.
\begin{flushleft}\Cc{\textbf{Default value}:\\~\\\hspace*{0.5cm}{\small $\bullet$~~\texttt{'boundary\_conditions=0'} and \texttt{'interpolation=0'.}}}\end{flushleft}
\begin{center}\includegraphics[keepaspectratio=true,height=6cm,width=\textwidth]{img/gmic_stdlib266.jpg}\\
{\footnotesize \textbf{Example 266~:} \texttt{image.jpg --shift[0] 50\%{\comma}50\%{\comma}0{\comma}0{\comma}0 --shift[0] 50\%{\comma}50\%{\comma}0{\comma}0{\comma}1 --shift[0] 50\%{\comma}50\%{\comma}0{\comma}0{\comma}2}}
\end{center}

\subsection{\emph{shrink\_x\index{shrink\_x}} }\vspace*{-0.7em}
~\\\textbf{\Cb{Arguments: }}\begin{flushleft}
{\small \Cb{\hspace*{0.5cm}$\bullet$~~\texttt{size\_x$>$=0}}}\end{flushleft}
Shrink selected images along the x-axis.
\begin{center}\includegraphics[keepaspectratio=true,height=6cm,width=\textwidth]{img/gmic_stdlib267.jpg}\\
{\footnotesize \textbf{Example 267~:} \texttt{image.jpg shrink\_x 30}}
\end{center}

\subsection{\emph{shrink\_xy\index{shrink\_xy}} }\vspace*{-0.7em}
~\\\textbf{\Cb{Arguments: }}\begin{flushleft}
{\small \Cb{\hspace*{0.5cm}$\bullet$~~\texttt{size$>$=0}}}\end{flushleft}
Shrink selected images along the xy-axes.
\begin{center}\includegraphics[keepaspectratio=true,height=6cm,width=\textwidth]{img/gmic_stdlib268.jpg}\\
{\footnotesize \textbf{Example 268~:} \texttt{image.jpg shrink\_xy 30}}
\end{center}

\subsection{\emph{shrink\_xyz\index{shrink\_xyz}} }\vspace*{-0.7em}
~\\\textbf{\Cb{Arguments: }}\begin{flushleft}
{\small \Cb{\hspace*{0.5cm}$\bullet$~~\texttt{size$>$=0}}}\end{flushleft}
Shrink selected images along the xyz-axes.


\subsection{\emph{shrink\_y\index{shrink\_y}} }\vspace*{-0.7em}
~\\\textbf{\Cb{Arguments: }}\begin{flushleft}
{\small \Cb{\hspace*{0.5cm}$\bullet$~~\texttt{size\_y$>$=0}}}\end{flushleft}
Shrink selected images along the y-axis.
\begin{center}\includegraphics[keepaspectratio=true,height=6cm,width=\textwidth]{img/gmic_stdlib269.jpg}\\
{\footnotesize \textbf{Example 269~:} \texttt{image.jpg shrink\_y 30}}
\end{center}

\subsection{\emph{shrink\_z\index{shrink\_z}} }\vspace*{-0.7em}
~\\\textbf{\Cb{Arguments: }}\begin{flushleft}
{\small \Cb{\hspace*{0.5cm}$\bullet$~~\texttt{size\_z$>$=0}}}\end{flushleft}
Shrink selected images along the z-axis.


\subsection{\emph{slices\index{slices}} (+)}\vspace*{-0.7em}
~\\\textbf{\Cb{Arguments: }}\begin{flushleft}
{\small \Cb{\hspace*{0.5cm}$\bullet$~~\texttt{\{ [image0] ~$|$~ z0[\%] \}{\comma}\_\{ [image1] ~$|$~ z1[\%] \}}}}\end{flushleft}
Keep only specified slices of selected images.
~\\Dirichlet boundary conditions are used when specified slices are out of range.


\subsection{\emph{sort\index{sort}} (+)}\vspace*{-0.7em}
~\\\textbf{\Cb{Arguments: }}\begin{flushleft}
{\small \Cb{\hspace*{0.5cm}$\bullet$~~\texttt{\_ordering=\{ + ~$|$~ - \}{\comma}\_axis=\{ x ~$|$~ y ~$|$~ z ~$|$~ c \}}}}\end{flushleft}
Sort pixel values of selected images.
~\\If 'axis' is specified{\comma} the sorting is done according to the data of the first column/row/slice/channel
of selected images.
\begin{flushleft}\Cc{\textbf{Default values}:\\~\\\hspace*{0.5cm}{\small $\bullet$~~\texttt{'ordering=+'} and \texttt{'axis=(undefined)'.}}}\end{flushleft}
\begin{center}\includegraphics[keepaspectratio=true,height=6cm,width=\textwidth]{img/gmic_stdlib270.jpg}\\
{\footnotesize \textbf{Example 270~:} \texttt{64 rand 0{\comma}100 --sort display\_graph 400{\comma}300{\comma}3}}
\end{center}

\subsection{\emph{split\index{split}} (+)}\vspace*{-0.7em}
~\\\textbf{\Cb{Arguments: }}\begin{flushleft}
{\small \Cb{\hspace*{0.5cm}$\bullet$~~\texttt{\{ x ~$|$~ y ~$|$~ z ~$|$~ c \}...\{ x ~$|$~ y ~$|$~ z ~$|$~ c \}{\comma}\_split\_mode}}}~~~\\
{\small \Cb{\hspace*{0.5cm}$\bullet$~~\texttt{keep\_splitting\_values=\{ + ~$|$~ - \}{\comma}\_\{ x ~$|$~ y ~$|$~ z ~$|$~ c \}...\{ x ~$|$~ y\- ~$|$~ z ~$|$~ c \}{\comma}value1{\comma}\_value2{\comma}...}}}~~~\\
{\small \Cb{\hspace*{0.5cm}$\bullet$~~\texttt{(no arg)}}}\end{flushleft}
Split selected images along specified axes{\comma} or regarding to a sequence of scalar values (optionally along specified axes too).
~\\(\emph{eq. to} {\small \texttt{'s').\textbackslash n}}).
~\\'split\_mode' can be \{ 0=split according to constant values ~$|$~ $>$0=split in N parts ~$|$~ $<$0=split in parts of size -N \}.
\begin{flushleft}\Cc{\textbf{Default value}:\\~\\\hspace*{0.5cm}{\small $\bullet$~~\texttt{'split\_mode=-1'.}}}\end{flushleft}
\begin{center}\includegraphics[keepaspectratio=true,height=6cm,width=\textwidth]{img/gmic_stdlib271.jpg}\\
{\footnotesize \textbf{Example 271~:} \texttt{image.jpg split c}}
\\\includegraphics[keepaspectratio=true,height=6cm,width=\textwidth]{img/gmic_stdlib272.jpg}\\
{\footnotesize \textbf{Example 272~:} \texttt{image.jpg split y{\comma}3}}
\\\includegraphics[keepaspectratio=true,height=6cm,width=\textwidth]{img/gmic_stdlib273.jpg}\\
{\footnotesize \textbf{Example 273~:} \texttt{image.jpg split x{\comma}-128}}
\\\includegraphics[keepaspectratio=true,height=6cm,width=\textwidth]{img/gmic_stdlib274.jpg}\\
{\footnotesize \textbf{Example 274~:} \texttt{1{\comma}20{\comma}1{\comma}1{\comma}"1{\comma}2{\comma}3{\comma}4" --split -{\comma}2{\comma}3 append[1--1] y}}
\\\includegraphics[keepaspectratio=true,height=6cm,width=\textwidth]{img/gmic_stdlib275.jpg}\\
{\footnotesize \textbf{Example 275~:} \texttt{(1{\comma}2{\comma}2{\comma}3{\comma}3{\comma}3{\comma}4{\comma}4{\comma}4{\comma}4) --split x{\comma}0 append[1--1] y}}
\end{center}

\subsection{\emph{split\_tiles\index{split\_tiles}} }\vspace*{-0.7em}
~\\\textbf{\Cb{Arguments: }}\begin{flushleft}
{\small \Cb{\hspace*{0.5cm}$\bullet$~~\texttt{M!=0{\comma}\_N!=0{\comma}\_is\_homogeneous=\{ 0 ~$|$~ 1 \}}}}\end{flushleft}
Split selected images as a MxN array of tiles.
~\\If M or N is negative{\comma} it stands for the tile size instead.
\begin{flushleft}\Cc{\textbf{Default values}:\\~\\\hspace*{0.5cm}{\small $\bullet$~~\texttt{'N=M'} and \texttt{'is\_homogeneous=0'.}}}\end{flushleft}
\begin{center}\includegraphics[keepaspectratio=true,height=6cm,width=\textwidth]{img/gmic_stdlib276.jpg}\\
{\footnotesize \textbf{Example 276~:} \texttt{image.jpg --local split\_tiles 5{\comma}4 blur 3{\comma}0 sharpen 700 append\_tiles 4{\comma}5 endlocal}}
\end{center}

\subsection{\emph{undistort\index{undistort}} }\vspace*{-0.7em}
~\\\textbf{\Cb{Arguments: }}\begin{flushleft}
{\small \Cb{\hspace*{0.5cm}$\bullet$~~\texttt{-1$<$=\_amplitude$<$=1{\comma}\_aspect\_ratio{\comma}\_zoom{\comma}\_center\_x[\%]{\comma}\_center\_y\-[\%]{\comma}\_boundary\_conditions}}}\end{flushleft}
Correct barrel/pincushion distortions occuring with wide-angle lens.
~\\References:
[1] Zhang Z. (1999). Flexible camera calibration by viewing a plane from unknown orientation.
[2] Andrew W. Fitzgibbon (2001). Simultaneous linear estimation of multiple view geometry and lens distortion.
~\\'boundary\_conditions' can be \{ 0=dirichlet ~$|$~ 1=neumann ~$|$~ 2=periodic ~$|$~ 3=mirror \}.
\begin{flushleft}\Cc{\textbf{Default values}:\\~\\\hspace*{0.5cm}{\small $\bullet$~~\texttt{'amplitude=0.25'{\comma} 'aspect\_ratio=0'{\comma} 'zoom=0'{\comma} 'center\_x=center\_y=50\%' } and \texttt{'boundary\_conditions=0'.}}}\end{flushleft}


\subsection{\emph{unroll\index{unroll}} (+)}\vspace*{-0.7em}
~\\\textbf{\Cb{Arguments: }}\begin{flushleft}
{\small \Cb{\hspace*{0.5cm}$\bullet$~~\texttt{\_axis=\{ x ~$|$~ y ~$|$~ z ~$|$~ c \}}}}\end{flushleft}
Unroll selected images along specified axis.
~\\(\emph{eq. to} {\small \texttt{'y'}}).
\begin{flushleft}\Cc{\textbf{Default value}:\\~\\\hspace*{0.5cm}{\small $\bullet$~~\texttt{'axis=y'.}}}\end{flushleft}
\begin{center}\includegraphics[keepaspectratio=true,height=6cm,width=\textwidth]{img/gmic_stdlib277.jpg}\\
{\footnotesize \textbf{Example 277~:} \texttt{(1{\comma}2{\comma}3;4{\comma}5{\comma}6;7{\comma}8{\comma}9) --unroll y}}
\end{center}

\subsection{\emph{upscale\_smart\index{upscale\_smart}} }\vspace*{-0.7em}
~\\\textbf{\Cb{Arguments: }}\begin{flushleft}
{\small \Cb{\hspace*{0.5cm}$\bullet$~~\texttt{width[\%]{\comma}\_height[\%]{\comma}\_depth{\comma}\_smoothness$>$=0{\comma}\_anisotropy=[0{\comma}1]{\comma}\-sharpening$>$=0}}}\end{flushleft}
Upscale selected images with an edge-preserving algorithm.
\begin{flushleft}\Cc{\textbf{Default values}:\\~\\\hspace*{0.5cm}{\small $\bullet$~~\texttt{'height=100\%'{\comma} 'depth=100\%'{\comma} 'smoothness=2'{\comma} 'anisotropy=0.4'} and \texttt{'sharpening=10'.}}}\end{flushleft}
\begin{center}\includegraphics[keepaspectratio=true,height=6cm,width=\textwidth]{img/gmic_stdlib278.jpg}\\
{\footnotesize \textbf{Example 278~:} \texttt{image.jpg resize2dy 100 --upscale\_smart 500\%{\comma}500\% append x}}
\end{center}

\subsection{\emph{warp\index{warp}} (+)}\vspace*{-0.7em}
~\\\textbf{\Cb{Arguments: }}\begin{flushleft}
{\small \Cb{\hspace*{0.5cm}$\bullet$~~\texttt{[warping\_field]{\comma}\_mode{\comma}\_interpolation{\comma}\_boundary\_conditions{\comma}\_n\-b\_frames$>$0}}}\end{flushleft}
Warp selected image with specified displacement field.
~\\'mode' can be \{ 0=backward-absolute ~$|$~ 1=backward-relative ~$|$~ 2=forward-absolute ~$|$~ 3=forward-relative \}.
~\\'interpolation' can be \{ 0=nearest-neighbor ~$|$~ 1=linear ~$|$~ 2=cubic \}.
~\\'boundary\_conditions' can be \{ 0=dirichlet ~$|$~ 1=neumann ~$|$~ 2=periodic ~$|$~ 3=mirror \}.
\begin{flushleft}\Cc{\textbf{Default values}:\\~\\\hspace*{0.5cm}{\small $\bullet$~~\texttt{'mode=0'{\comma} 'interpolation=1'{\comma} 'boundary\_conditions=1'} and \texttt{'nb\_frames=1'.}}}\end{flushleft}
\begin{center}\includegraphics[keepaspectratio=true,height=6cm,width=\textwidth]{img/gmic_stdlib279.jpg}\\
{\footnotesize \textbf{Example 279~:} \texttt{image.jpg 100\%{\comma}100\%{\comma}1{\comma}2{\comma}'X=x/w-0.5;Y=y/h-0.5;R=(X*X+Y*Y)\textasciicircum 0.5;A=atan2(Y{\comma}X);130*R*if(c==0{\comma}cos(4*A){\comma}sin(8*A))' warp[-2] [-1]{\comma}1{\comma}1{\comma}0 quiver[-1] [-1]{\comma}10{\comma}1{\comma}1{\comma}1{\comma}100}}
\end{center}
~\\
~\textbf{Tutorial page: }\\\url{http://gmic.eu/tutorial/\_warp.shtml}

\section{Filtering}


\subsection{\emph{bandpass\index{bandpass}} }\vspace*{-0.7em}
~\\\textbf{\Cb{Arguments: }}\begin{flushleft}
{\small \Cb{\hspace*{0.5cm}$\bullet$~~\texttt{\_min\_freq[\%]{\comma}\_max\_freq[\%]}}}\end{flushleft}
Apply bandpass filter to selected images.
\begin{flushleft}\Cc{\textbf{Default values}:\\~\\\hspace*{0.5cm}{\small $\bullet$~~\texttt{'min\_freq=0'} and \texttt{'max\_freq=20\%'.}}}\end{flushleft}
\begin{center}\includegraphics[keepaspectratio=true,height=6cm,width=\textwidth]{img/gmic_stdlib280.jpg}\\
{\footnotesize \textbf{Example 280~:} \texttt{image.jpg bandpass 1\%{\comma}3\%}}
\end{center}
~\\
~\textbf{Tutorial page: }\\\url{http://gmic.eu/tutorial/\_bandpass.shtml}


\subsection{\emph{bilateral\index{bilateral}} (+)}\vspace*{-0.7em}
~\\\textbf{\Cb{Arguments: }}\begin{flushleft}
{\small \Cb{\hspace*{0.5cm}$\bullet$~~\texttt{[guide]{\comma}std\_variation\_s[\%]$>$=0{\comma}std\_variation\_r[\%]$>$=0{\comma}\_samplin\-g\_s$>$=0{\comma}\_sampling\_r$>$=0}}}~~~\\
{\small \Cb{\hspace*{0.5cm}$\bullet$~~\texttt{std\_variation\_s[\%]$>$=0{\comma}std\_variation\_r[\%]$>$=0{\comma}\_sampling\_s$>$=0{\comma}\_\-sampling\_r$>$=0}}}\end{flushleft}
Blur selected images by anisotropic (eventually joint/cross) bilateral filtering.
~\\If a guide image is provided{\comma} it is used for drive the smoothing filter.
~\\A guide image must be of the same xyz-size as the selected images.
~\\Set 'sampling' arguments to '0' for automatic adjustment.
\begin{center}\includegraphics[keepaspectratio=true,height=6cm,width=\textwidth]{img/gmic_stdlib281.jpg}\\
{\footnotesize \textbf{Example 281~:} \texttt{image.jpg [0] repeat 5 bilateral[-1] 10{\comma}10 done}}
\end{center}

\subsection{\emph{blur\index{blur}} (+)}\vspace*{-0.7em}
~\\\textbf{\Cb{Arguments: }}\begin{flushleft}
{\small \Cb{\hspace*{0.5cm}$\bullet$~~\texttt{std\_variation$>$=0[\%]{\comma}\_boundary\_conditions{\comma}\_kernel}}}~~~\\
{\small \Cb{\hspace*{0.5cm}$\bullet$~~\texttt{axes{\comma}std\_variation$>$=0[\%]{\comma}\_boundary\_conditions{\comma}\_kernel}}}\end{flushleft}
Blur selected images by a quasi-gaussian or gaussian filter (recursive implementation).
~\\(\emph{eq. to} {\small \texttt{'b').\textbackslash n}}).
~\\'boundary\_conditions' can be \{ 0=dirichlet ~$|$~ 1=neumann \} and 'kernel' can be \{ 0=quasi-gaussian (faster) ~$|$~ 1=gaussian \}.
~\\When specified{\comma} argument 'axes' is a sequence of \{ x ~$|$~ y ~$|$~ z ~$|$~ c \}.
~\\Specifying one axis multiple times apply also the blur multiple times.
\begin{flushleft}\Cc{\textbf{Default values}:\\~\\\hspace*{0.5cm}{\small $\bullet$~~\texttt{'boundary\_conditions=1'} and \texttt{'kernel=0'.}}}\end{flushleft}
\begin{center}\includegraphics[keepaspectratio=true,height=6cm,width=\textwidth]{img/gmic_stdlib282.jpg}\\
{\footnotesize \textbf{Example 282~:} \texttt{image.jpg --blur 5{\comma}0 --blur[0] 5{\comma}1}}
\\\includegraphics[keepaspectratio=true,height=6cm,width=\textwidth]{img/gmic_stdlib283.jpg}\\
{\footnotesize \textbf{Example 283~:} \texttt{image.jpg --blur y{\comma}10\%}}
\end{center}
~\\
~\textbf{Tutorial page: }\\\url{http://gmic.eu/tutorial/\_blur.shtml}


\subsection{\emph{blur\_angular\index{blur\_angular}} }\vspace*{-0.7em}
~\\\textbf{\Cb{Arguments: }}\begin{flushleft}
{\small \Cb{\hspace*{0.5cm}$\bullet$~~\texttt{amplitude[\%]{\comma}\_center\_x[\%]{\comma}\_center\_y[\%]}}}\end{flushleft}
Apply angular blur on selected images.
\begin{flushleft}\Cc{\textbf{Default values}:\\~\\\hspace*{0.5cm}{\small $\bullet$~~\texttt{'center\_x=center\_y=50\%'.}}}\end{flushleft}
\begin{center}\includegraphics[keepaspectratio=true,height=6cm,width=\textwidth]{img/gmic_stdlib284.jpg}\\
{\footnotesize \textbf{Example 284~:} \texttt{image.jpg --blur\_angular 2\%}}
\end{center}
~\\
~\textbf{Tutorial page: }\\\url{http://gmic.eu/tutorial/\_blur\_angular.shtml}


\subsection{\emph{blur\_bloom\index{blur\_bloom}} }\vspace*{-0.7em}
~\\\textbf{\Cb{Arguments: }}\begin{flushleft}
{\small \Cb{\hspace*{0.5cm}$\bullet$~~\texttt{\_amplitude$>$=0{\comma}\_ratio$>$=0{\comma}\_nb\_iter$>$=0{\comma}\_blend\_operator=\{ + ~$|$~ ma\-x ~$|$~ min \}{\comma}\_kernel=\{ 0=quasi-gaussian (faster) ~$|$~ 1=gaussian ~$|$~\- 2=box ~$|$~ 3=triangle ~$|$~ 4=quadratic \}{\comma}\_normalize\_scales=\{ 0 ~$|$~ \-1 \}{\comma}\_axes}}}\end{flushleft}
Apply a bloom filter that blend multiple blur filters of different radii{\comma}
resulting in a larger but sharper glare than a simple blur.
~\\When specified{\comma} argument 'axes' is a sequence of \{ x ~$|$~ y ~$|$~ z ~$|$~ c \}.
~\\Specifying one axis multiple times apply also the blur multiple times.
~\\Reference: Masaki Kawase{\comma} "Practical Implementation of High Dynamic Range Rendering"{\comma} GDC 2004.
\begin{flushleft}\Cc{\textbf{Default values}:\\~\\\hspace*{0.5cm}{\small $\bullet$~~\texttt{'amplitude=1'{\comma} 'ratio=2'{\comma} 'nb\_iter=5'{\comma} 'blend\_operator=+'{\comma} 'kernel=0'{\comma}'normalize\_scales=0'} and \texttt{'axes=(all)'}}}\end{flushleft}
\begin{center}\includegraphics[keepaspectratio=true,height=6cm,width=\textwidth]{img/gmic_stdlib285.jpg}\\
{\footnotesize \textbf{Example 285~:} \texttt{image.jpg --blur\_bloom {\comma}}}
\end{center}

\subsection{\emph{blur\_linear\index{blur\_linear}} }\vspace*{-0.7em}
~\\\textbf{\Cb{Arguments: }}\begin{flushleft}
{\small \Cb{\hspace*{0.5cm}$\bullet$~~\texttt{amplitude1[\%]{\comma}\_amplitude2[\%]{\comma}\_angle{\comma}\_boundary\_conditions=\{ 0\-=dirichlet ~$|$~ 1=neumann \}}}}\end{flushleft}
Apply linear blur on selected images{\comma} with specified angle and amplitudes.
\begin{flushleft}\Cc{\textbf{Default values}:\\~\\\hspace*{0.5cm}{\small $\bullet$~~\texttt{'amplitude2=0'{\comma} 'angle=0'} and \texttt{'boundary\_conditions=1'.}}}\end{flushleft}
\begin{center}\includegraphics[keepaspectratio=true,height=6cm,width=\textwidth]{img/gmic_stdlib286.jpg}\\
{\footnotesize \textbf{Example 286~:} \texttt{image.jpg --blur\_linear 10{\comma}0{\comma}45}}
\end{center}
~\\
~\textbf{Tutorial page: }\\\url{http://gmic.eu/tutorial/\_blur\_linear.shtml}


\subsection{\emph{blur\_radial\index{blur\_radial}} }\vspace*{-0.7em}
~\\\textbf{\Cb{Arguments: }}\begin{flushleft}
{\small \Cb{\hspace*{0.5cm}$\bullet$~~\texttt{amplitude[\%]{\comma}\_center\_x[\%]{\comma}\_center\_y[\%]}}}\end{flushleft}
Apply radial blur on selected images.
\begin{flushleft}\Cc{\textbf{Default values}:\\~\\\hspace*{0.5cm}{\small $\bullet$~~\texttt{'center\_x=center\_y=50\%'.}}}\end{flushleft}
\begin{center}\includegraphics[keepaspectratio=true,height=6cm,width=\textwidth]{img/gmic_stdlib287.jpg}\\
{\footnotesize \textbf{Example 287~:} \texttt{image.jpg --blur\_radial 2\%}}
\end{center}
~\\
~\textbf{Tutorial page: }\\\url{http://gmic.eu/tutorial/\_blur\_radial.shtml}


\subsection{\emph{blur\_selective\index{blur\_selective}} }\vspace*{-0.7em}
~\\\textbf{\Cb{Arguments: }}\begin{flushleft}
{\small \Cb{\hspace*{0.5cm}$\bullet$~~\texttt{sigma$>$=0{\comma}\_edges$>$0{\comma}\_nb\_scales$>$0}}}\end{flushleft}
Blur selected images using selective gaussian scales.
\begin{flushleft}\Cc{\textbf{Default values}:\\~\\\hspace*{0.5cm}{\small $\bullet$~~\texttt{'sigma=5'{\comma} 'edges=0.5'} and \texttt{'nb\_scales=5'.}}}\end{flushleft}
\begin{center}\includegraphics[keepaspectratio=true,height=6cm,width=\textwidth]{img/gmic_stdlib288.jpg}\\
{\footnotesize \textbf{Example 288~:} \texttt{image.jpg noise 20 cut 0{\comma}255 --local[-1] repeat 4 blur\_selective {\comma} done endlocal}}
\end{center}
~\\
~\textbf{Tutorial page: }\\\url{http://gmic.eu/tutorial/\_blur\_selective.shtml}


\subsection{\emph{blur\_x\index{blur\_x}} }\vspace*{-0.7em}
~\\\textbf{\Cb{Arguments: }}\begin{flushleft}
{\small \Cb{\hspace*{0.5cm}$\bullet$~~\texttt{amplitude[\%]$>$=0{\comma}\_boundary\_conditions=\{ 0=dirichlet ~$|$~ 1=neuma\-nn \}}}}\end{flushleft}
Blur selected images along the x-axis.
\begin{flushleft}\Cc{\textbf{Default value}:\\~\\\hspace*{0.5cm}{\small $\bullet$~~\texttt{'boundary\_conditions=1'.}}}\end{flushleft}
\begin{center}\includegraphics[keepaspectratio=true,height=6cm,width=\textwidth]{img/gmic_stdlib289.jpg}\\
{\footnotesize \textbf{Example 289~:} \texttt{image.jpg --blur\_x 6}}
\end{center}
~\\
~\textbf{Tutorial page: }\\\url{http://gmic.eu/tutorial/\_blur\_x.shtml}


\subsection{\emph{blur\_xy\index{blur\_xy}} }\vspace*{-0.7em}
~\\\textbf{\Cb{Arguments: }}\begin{flushleft}
{\small \Cb{\hspace*{0.5cm}$\bullet$~~\texttt{amplitude\_x[\%]{\comma}amplitude\_y[\%]{\comma}\_boundary\_conditions=\{ 0=diric\-hlet ~$|$~ 1=neumann \}}}}\end{flushleft}
Blur selected images along the X and Y axes.
\begin{flushleft}\Cc{\textbf{Default value}:\\~\\\hspace*{0.5cm}{\small $\bullet$~~\texttt{'boundary\_conditions=1'.}}}\end{flushleft}
\begin{center}\includegraphics[keepaspectratio=true,height=6cm,width=\textwidth]{img/gmic_stdlib290.jpg}\\
{\footnotesize \textbf{Example 290~:} \texttt{image.jpg --blur\_xy 6}}
\end{center}
~\\
~\textbf{Tutorial page: }\\\url{http://gmic.eu/tutorial/\_blur\_xy.shtml}


\subsection{\emph{blur\_xyz\index{blur\_xyz}} }\vspace*{-0.7em}
~\\\textbf{\Cb{Arguments: }}\begin{flushleft}
{\small \Cb{\hspace*{0.5cm}$\bullet$~~\texttt{amplitude\_x[\%]{\comma}amplitude\_y[\%]{\comma}amplitude\_z{\comma}\_boundary\_conditio\-ns=\{ 0=dirichlet ~$|$~ 1=neumann \}}}}\end{flushleft}
Blur selected images along the X{\comma} Y and Z axes.
\begin{flushleft}\Cc{\textbf{Default value}:\\~\\\hspace*{0.5cm}{\small $\bullet$~~\texttt{'boundary\_conditions=1'.}}}\end{flushleft}

~\\
~\textbf{Tutorial page: }\\\url{http://gmic.eu/tutorial/\_blur\_xyz.shtml}


\subsection{\emph{blur\_y\index{blur\_y}} }\vspace*{-0.7em}
~\\\textbf{\Cb{Arguments: }}\begin{flushleft}
{\small \Cb{\hspace*{0.5cm}$\bullet$~~\texttt{amplitude[\%]$>$=0{\comma}\_boundary\_conditions=\{ 0=dirichlet ~$|$~ 1=neuma\-nn \}}}}\end{flushleft}
Blur selected images along the y-axis.
\begin{flushleft}\Cc{\textbf{Default value}:\\~\\\hspace*{0.5cm}{\small $\bullet$~~\texttt{'boundary\_conditions=1'.}}}\end{flushleft}
\begin{center}\includegraphics[keepaspectratio=true,height=6cm,width=\textwidth]{img/gmic_stdlib291.jpg}\\
{\footnotesize \textbf{Example 291~:} \texttt{image.jpg --blur\_y 6}}
\end{center}
~\\
~\textbf{Tutorial page: }\\\url{http://gmic.eu/tutorial/\_blur\_y.shtml}


\subsection{\emph{blur\_z\index{blur\_z}} }\vspace*{-0.7em}
~\\\textbf{\Cb{Arguments: }}\begin{flushleft}
{\small \Cb{\hspace*{0.5cm}$\bullet$~~\texttt{amplitude[\%]$>$=0{\comma}\_boundary\_conditions=\{ 0=dirichlet ~$|$~ 1=neuma\-nn \}}}}\end{flushleft}
Blur selected images along the z-axis.
\begin{flushleft}\Cc{\textbf{Default value}:\\~\\\hspace*{0.5cm}{\small $\bullet$~~\texttt{'boundary\_conditions=1'.}}}\end{flushleft}

~\\
~\textbf{Tutorial page: }\\\url{http://gmic.eu/tutorial/\_blur\_z.shtml}


\subsection{\emph{boxfilter\index{boxfilter}} (+)}\vspace*{-0.7em}
~\\\textbf{\Cb{Arguments: }}\begin{flushleft}
{\small \Cb{\hspace*{0.5cm}$\bullet$~~\texttt{size$>$=0[\%]{\comma}\_order{\comma}\_boundary\_conditions{\comma}\_nb\_iter$>$=0}}}~~~\\
{\small \Cb{\hspace*{0.5cm}$\bullet$~~\texttt{axes{\comma}size$>$=0[\%]{\comma}\_order{\comma}\_boundary\_conditions{\comma}\_nb\_iter$>$=0}}}\end{flushleft}
Blur selected images by a box filter of specified size (fast recursive implementation).
~\\'order' can be \{ 0=smooth ~$|$~ 1=1st-derivative ~$|$~ 2=2nd-derivative \}.
~\\'boundary\_conditions' can be \{ 0=dirichlet ~$|$~ 1=neumann \}.
~\\When specified{\comma} argument 'axes' is a sequence of \{ x ~$|$~ y ~$|$~ z ~$|$~ c \}.
~\\Specifying one axis multiple times apply also the blur multiple times.
\begin{flushleft}\Cc{\textbf{Default values}:\\~\\\hspace*{0.5cm}{\small $\bullet$~~\texttt{'order=0'{\comma} 'boundary\_conditions=1'} and \texttt{'nb\_iter=1'.}}}\end{flushleft}
\begin{center}\includegraphics[keepaspectratio=true,height=6cm,width=\textwidth]{img/gmic_stdlib292.jpg}\\
{\footnotesize \textbf{Example 292~:} \texttt{image.jpg --boxfilter 5\%}}
\\\includegraphics[keepaspectratio=true,height=6cm,width=\textwidth]{img/gmic_stdlib293.jpg}\\
{\footnotesize \textbf{Example 293~:} \texttt{image.jpg --boxfilter y{\comma}3{\comma}1}}
\end{center}

\subsection{\emph{compose\_freq\index{compose\_freq}} }\vspace*{-0.7em}
Compose selected low and high frequency parts into new images.
\begin{center}\includegraphics[keepaspectratio=true,height=6cm,width=\textwidth]{img/gmic_stdlib294.jpg}\\
{\footnotesize \textbf{Example 294~:} \texttt{image.jpg split\_freq 2\% mirror[-1] x compose\_freq}}
\end{center}

\subsection{\emph{convolve\index{convolve}} (+)}\vspace*{-0.7em}
~\\\textbf{\Cb{Arguments: }}\begin{flushleft}
{\small \Cb{\hspace*{0.5cm}$\bullet$~~\texttt{[mask]{\comma}\_boundary\_conditions{\comma}\_is\_normalized=\{ 0 ~$|$~ 1 \}}}}\end{flushleft}
Convolve selected images by specified mask.
~\\'boundary\_conditions' can be \{ 0=dirichlet ~$|$~ 1=neumann \}.
\begin{flushleft}\Cc{\textbf{Default values}:\\~\\\hspace*{0.5cm}{\small $\bullet$~~\texttt{'boundary\_conditions=1'} and \texttt{'is\_normalized=0'.}}}\end{flushleft}
\begin{center}\includegraphics[keepaspectratio=true,height=6cm,width=\textwidth]{img/gmic_stdlib295.jpg}\\
{\footnotesize \textbf{Example 295~:} \texttt{image.jpg (0{\comma}1{\comma}0;1{\comma}-4{\comma}1;0{\comma}1{\comma}0) convolve[-2] [-1] keep[-2]}}
\\\includegraphics[keepaspectratio=true,height=6cm,width=\textwidth]{img/gmic_stdlib296.jpg}\\
{\footnotesize \textbf{Example 296~:} \texttt{image.jpg (0{\comma}1{\comma}0) resize[-1] 130{\comma}1{\comma}1{\comma}1{\comma}3 --convolve[0] [1]}}
\end{center}
~\\
~\textbf{Tutorial page: }\\\url{http://gmic.eu/tutorial/\_convolve.shtml}


\subsection{\emph{convolve\_fft\index{convolve\_fft}} }\vspace*{-0.7em}
~\\\textbf{\Cb{Arguments: }}\begin{flushleft}
{\small \Cb{\hspace*{0.5cm}$\bullet$~~\texttt{[mask]}}}\end{flushleft}
Convolve selected images with specified mask{\comma} in the fourier domain.
\begin{center}\includegraphics[keepaspectratio=true,height=6cm,width=\textwidth]{img/gmic_stdlib297.jpg}\\
{\footnotesize \textbf{Example 297~:} \texttt{image.jpg 100\%{\comma}100\% gaussian[-1] 20{\comma}1{\comma}45 --convolve\_fft[0] [1]}}
\end{center}

\subsection{\emph{correlate\index{correlate}} (+)}\vspace*{-0.7em}
~\\\textbf{\Cb{Arguments: }}\begin{flushleft}
{\small \Cb{\hspace*{0.5cm}$\bullet$~~\texttt{[mask]{\comma}\_boundary\_conditions{\comma}\_is\_normalized=\{ 0 ~$|$~ 1 \}}}}\end{flushleft}
Correlate selected images by specified mask.
~\\'boundary\_conditions' can be \{ 0=dirichlet ~$|$~ 1=neumann \}.
\begin{flushleft}\Cc{\textbf{Default values}:\\~\\\hspace*{0.5cm}{\small $\bullet$~~\texttt{'boundary\_conditions=1'} and \texttt{'is\_normalized=0'.}}}\end{flushleft}
\begin{center}\includegraphics[keepaspectratio=true,height=6cm,width=\textwidth]{img/gmic_stdlib298.jpg}\\
{\footnotesize \textbf{Example 298~:} \texttt{image.jpg (0{\comma}1{\comma}0;1{\comma}-4{\comma}1;0{\comma}1{\comma}0) correlate[-2] [-1] keep[-2]}}
\\\includegraphics[keepaspectratio=true,height=6cm,width=\textwidth]{img/gmic_stdlib299.jpg}\\
{\footnotesize \textbf{Example 299~:} \texttt{image.jpg --crop 40\%{\comma}40\%{\comma}60\%{\comma}60\% --correlate[0] [-1]{\comma}0{\comma}1}}
\end{center}

\subsection{\emph{cross\_correlation\index{cross\_correlation}} }\vspace*{-0.7em}
~\\\textbf{\Cb{Arguments: }}\begin{flushleft}
{\small \Cb{\hspace*{0.5cm}$\bullet$~~\texttt{[mask]}}}\end{flushleft}
Compute cross-correlation of selected images with specified mask.
\begin{center}\includegraphics[keepaspectratio=true,height=6cm,width=\textwidth]{img/gmic_stdlib300.jpg}\\
{\footnotesize \textbf{Example 300~:} \texttt{image.jpg --shift -30{\comma}-20 --cross\_correlation[0] [1]}}
\end{center}

\subsection{\emph{curvature\index{curvature}} }\vspace*{-0.7em}
Compute isophote curvatures on selected images.
\begin{center}\includegraphics[keepaspectratio=true,height=6cm,width=\textwidth]{img/gmic_stdlib301.jpg}\\
{\footnotesize \textbf{Example 301~:} \texttt{image.jpg blur 10 curvature}}
\end{center}

\subsection{\emph{dct\index{dct}} }\vspace*{-0.7em}
~\\\textbf{\Cb{Arguments: }}\begin{flushleft}
{\small \Cb{\hspace*{0.5cm}$\bullet$~~\texttt{\_\{ x ~$|$~ y ~$|$~ z \}...\{ x ~$|$~ y ~$|$~ z \}}}}~~~\\
{\small \Cb{\hspace*{0.5cm}$\bullet$~~\texttt{(no arg)}}}\end{flushleft}
Compute the discrete cosine transform of selected images{\comma}
optionally along the specified axes only.
\begin{flushleft}\Cc{\textbf{Default values}:\\~\\\hspace*{0.5cm}{\small $\bullet$~~\texttt{(no arg)}}}\end{flushleft}
\begin{center}\includegraphics[keepaspectratio=true,height=6cm,width=\textwidth]{img/gmic_stdlib302.jpg}\\
{\footnotesize \textbf{Example 302~:} \texttt{image.jpg --dct --idct[-1] abs[-2] +[-2] 1 log[-2]}}
\end{center}
~\\
~\textbf{Tutorial page: }\\\url{http://gmic.eu/tutorial/\_dct-and-idct.shtml}


\subsection{\emph{deblur\index{deblur}} }\vspace*{-0.7em}
~\\\textbf{\Cb{Arguments: }}\begin{flushleft}
{\small \Cb{\hspace*{0.5cm}$\bullet$~~\texttt{amplitude[\%]$>$=0{\comma}\_nb\_iter$>$=0{\comma}\_dt$>$=0{\comma}\_regul$>$=0{\comma}\_regul\_type=\{ 0\-=Tikhonov ~$|$~ 1=meancurv. ~$|$~ 2=TV \}}}}\end{flushleft}
Deblur image using a regularized Jansson-Van Cittert algorithm.
\begin{flushleft}\Cc{\textbf{Default values}:\\~\\\hspace*{0.5cm}{\small $\bullet$~~\texttt{'nb\_iter=10'{\comma} 'dt=20'{\comma} 'regul=0.7'} and \texttt{'regul\_type=1'.}}}\end{flushleft}
\begin{center}\includegraphics[keepaspectratio=true,height=6cm,width=\textwidth]{img/gmic_stdlib303.jpg}\\
{\footnotesize \textbf{Example 303~:} \texttt{image.jpg blur 3 --deblur 3{\comma}40{\comma}20{\comma}0.01}}
\end{center}

\subsection{\emph{deblur\_goldmeinel\index{deblur\_goldmeinel}} }\vspace*{-0.7em}
~\\\textbf{\Cb{Arguments: }}\begin{flushleft}
{\small \Cb{\hspace*{0.5cm}$\bullet$~~\texttt{sigma$>$=0{\comma} \_nb\_iter$>$=0{\comma} \_acceleration$>$=0{\comma} \_kernel\_type=\{ 0=qu\-asi-gaussian (faster) ~$|$~ 1=gaussian \}.}}}\end{flushleft}
Deblur selected images using Gold-Meinel algorithm
\begin{flushleft}\Cc{\textbf{Default values}:\\~\\\hspace*{0.5cm}{\small $\bullet$~~\texttt{'nb\_iter=8'{\comma} 'acceleration=1'} and \texttt{'kernel\_type=1'.}}}\end{flushleft}
\begin{center}\includegraphics[keepaspectratio=true,height=6cm,width=\textwidth]{img/gmic_stdlib304.jpg}\\
{\footnotesize \textbf{Example 304~:} \texttt{image.jpg --blur 1 --deblur\_goldmeinel[-1] 1}}
\end{center}

\subsection{\emph{deblur\_richardsonlucy\index{deblur\_richardsonlucy}} }\vspace*{-0.7em}
~\\\textbf{\Cb{Arguments: }}\begin{flushleft}
{\small \Cb{\hspace*{0.5cm}$\bullet$~~\texttt{sigma$>$=0{\comma} nb\_iter$>$=0{\comma} \_kernel\_type=\{ 0=quasi-gaussian (faste\-r) ~$|$~ 1=gaussian \}.}}}\end{flushleft}
Deblur selected images using Richardson-Lucy algorithm.
\begin{flushleft}\Cc{\textbf{Default values}:\\~\\\hspace*{0.5cm}{\small $\bullet$~~\texttt{'nb\_iter=50'} and \texttt{'kernel\_type=1'.}}}\end{flushleft}
\begin{center}\includegraphics[keepaspectratio=true,height=6cm,width=\textwidth]{img/gmic_stdlib305.jpg}\\
{\footnotesize \textbf{Example 305~:} \texttt{image.jpg --blur 1 --deblur\_richardsonlucy[-1] 1}}
\end{center}

\subsection{\emph{deconvolve\_fft\index{deconvolve\_fft}} }\vspace*{-0.7em}
~\\\textbf{\Cb{Arguments: }}\begin{flushleft}
{\small \Cb{\hspace*{0.5cm}$\bullet$~~\texttt{[kernel]{\comma}\_regularization$>$=0}}}\end{flushleft}
Deconvolve selected images by specified mask in the fourier space.
\begin{flushleft}\Cc{\textbf{Default value}:\\~\\\hspace*{0.5cm}{\small $\bullet$~~\texttt{'regularization$>$=0'.}}}\end{flushleft}
\begin{center}\includegraphics[keepaspectratio=true,height=6cm,width=\textwidth]{img/gmic_stdlib306.jpg}\\
{\footnotesize \textbf{Example 306~:} \texttt{image.jpg --gaussian 5 --convolve\_fft[0] [1] --deconvolve\_fft[-1] [1]}}
\end{center}

\subsection{\emph{deinterlace\index{deinterlace}} }\vspace*{-0.7em}
~\\\textbf{\Cb{Arguments: }}\begin{flushleft}
{\small \Cb{\hspace*{0.5cm}$\bullet$~~\texttt{\_method=\{ 0 ~$|$~ 1 \}}}}\end{flushleft}
Deinterlace selected images ('method' can be \{ 0=standard or 1=motion-compensated \}).
\begin{flushleft}\Cc{\textbf{Default value}:\\~\\\hspace*{0.5cm}{\small $\bullet$~~\texttt{'method=0'.}}}\end{flushleft}
\begin{center}\includegraphics[keepaspectratio=true,height=6cm,width=\textwidth]{img/gmic_stdlib307.jpg}\\
{\footnotesize \textbf{Example 307~:} \texttt{image.jpg --rotate 3{\comma}1{\comma}1{\comma}50\%{\comma}50\% resize 100\%{\comma}50\% resize 100\%{\comma}200\%{\comma}1{\comma}3{\comma}4 shift[-1] 0{\comma}1 add --deinterlace 1}}
\end{center}

\subsection{\emph{denoise\index{denoise}} (+)}\vspace*{-0.7em}
~\\\textbf{\Cb{Arguments: }}\begin{flushleft}
{\small \Cb{\hspace*{0.5cm}$\bullet$~~\texttt{std\_variation\_s$>$=0{\comma}\_std\_variation\_p$>$=0{\comma}\_patch\_size$>$0{\comma}\_lookup\-\_size$>$0{\comma}\_smoothness{\comma}\_fast\_approx=\{ 0 ~$|$~ 1 \}}}}\end{flushleft}
Denoise selected images by non-local patch averaging.
\begin{flushleft}\Cc{\textbf{Default values}:\\~\\\hspace*{0.5cm}{\small $\bullet$~~\texttt{'std\_variation\_p=10'{\comma} 'patch\_size=5'{\comma} 'lookup\_size=6'} and \texttt{'smoothness=1'.}}}\end{flushleft}
\begin{center}\includegraphics[keepaspectratio=true,height=6cm,width=\textwidth]{img/gmic_stdlib308.jpg}\\
{\footnotesize \textbf{Example 308~:} \texttt{image.jpg --denoise 5{\comma}5{\comma}8}}
\end{center}

\subsection{\emph{denoise\_haar\index{denoise\_haar}} }\vspace*{-0.7em}
~\\\textbf{\Cb{Arguments: }}\begin{flushleft}
{\small \Cb{\hspace*{0.5cm}$\bullet$~~\texttt{\_threshold$>$=0{\comma}\_nb\_scales$>$=0{\comma}\_cycle\_spinning$>$0}}}\end{flushleft}
Denoise selected image using haar-wavelet thresholding with cycle spinning.
~\\Set 'nb\_scales==0' to automatically determine the optimal number of scales.
\begin{flushleft}\Cc{\textbf{Default values}:\\~\\\hspace*{0.5cm}{\small $\bullet$~~\texttt{'threshold=1.4'{\comma} 'nb\_scale=0'} and \texttt{'cycle\_spinning=10'.}}}\end{flushleft}
\begin{center}\includegraphics[keepaspectratio=true,height=6cm,width=\textwidth]{img/gmic_stdlib309.jpg}\\
{\footnotesize \textbf{Example 309~:} \texttt{image.jpg noise 20 cut 0{\comma}255 --denoise\_haar[-1] 0.8}}
\end{center}

\subsection{\emph{denoise\_patchpca\index{denoise\_patchpca}} }\vspace*{-0.7em}
~\\\textbf{\Cb{Arguments: }}\begin{flushleft}
{\small \Cb{\hspace*{0.5cm}$\bullet$~~\texttt{\_strength$>$=0{\comma}\_patch\_size$>$0{\comma}\_lookup\_size$>$0{\comma}\_spatial\_sampling$>$\-0}}}\end{flushleft}
Denoise selected images using the patch-pca algorithm.
\begin{flushleft}\Cc{\textbf{Default values}:\\~\\\hspace*{0.5cm}{\small $\bullet$~~\texttt{'patch\_size=7'{\comma} 'lookup\_size=11'{\comma} 'details=1.8'} and \texttt{'spatial\_sampling=5'.}}}\end{flushleft}
\begin{center}\includegraphics[keepaspectratio=true,height=6cm,width=\textwidth]{img/gmic_stdlib310.jpg}\\
{\footnotesize \textbf{Example 310~:} \texttt{image.jpg --noise 20 c[-1] 0{\comma}255 --denoise\_patchpca[-1] {\comma}}}
\end{center}

\subsection{\emph{deriche\index{deriche}} (+)}\vspace*{-0.7em}
~\\\textbf{\Cb{Arguments: }}\begin{flushleft}
{\small \Cb{\hspace*{0.5cm}$\bullet$~~\texttt{std\_variation$>$=0[\%]{\comma}order=\{ 0 ~$|$~ 1 ~$|$~ 2 \}{\comma}axis=\{ x ~$|$~ y ~$|$~ z ~$|$~ c\- \}{\comma}\_boundary\_conditions}}}\end{flushleft}
Apply Deriche recursive filter on selected images{\comma} along specified axis and with
specified standard deviation{\comma} order and boundary conditions.
~\\'boundary\_conditions' can be \{ 0=dirichlet ~$|$~ 1=neumann \}.
\begin{flushleft}\Cc{\textbf{Default value}:\\~\\\hspace*{0.5cm}{\small $\bullet$~~\texttt{'boundary\_conditions=1'.}}}\end{flushleft}
\begin{center}\includegraphics[keepaspectratio=true,height=6cm,width=\textwidth]{img/gmic_stdlib311.jpg}\\
{\footnotesize \textbf{Example 311~:} \texttt{image.jpg --deriche 3{\comma}1{\comma}x}}
\\\includegraphics[keepaspectratio=true,height=6cm,width=\textwidth]{img/gmic_stdlib312.jpg}\\
{\footnotesize \textbf{Example 312~:} \texttt{image.jpg --deriche 30{\comma}0{\comma}x deriche[-2] 30{\comma}0{\comma}y add}}
\end{center}
~\\
~\textbf{Tutorial page: }\\\url{http://gmic.eu/tutorial/\_deriche.shtml}


\subsection{\emph{dilate\index{dilate}} (+)}\vspace*{-0.7em}
~\\\textbf{\Cb{Arguments: }}\begin{flushleft}
{\small \Cb{\hspace*{0.5cm}$\bullet$~~\texttt{size$>$=0}}}~~~\\
{\small \Cb{\hspace*{0.5cm}$\bullet$~~\texttt{size\_x$>$=0{\comma}size\_y$>$=0{\comma}size\_z$>$=0}}}~~~\\
{\small \Cb{\hspace*{0.5cm}$\bullet$~~\texttt{[kernel]{\comma}\_boundary\_conditions{\comma}\_is\_real=\{ 0=binary-mode ~$|$~ 1=r\-eal-mode \}}}}\end{flushleft}
Dilate selected images by a rectangular or the specified structuring element.
~\\'boundary\_conditions' can be \{ 0=dirichlet ~$|$~ 1=neumann \}.
\begin{flushleft}\Cc{\textbf{Default values}:\\~\\\hspace*{0.5cm}{\small $\bullet$~~\texttt{'size\_z=1'{\comma} 'boundary\_conditions=1'} and \texttt{'is\_real=0'.}}}\end{flushleft}
\begin{center}\includegraphics[keepaspectratio=true,height=6cm,width=\textwidth]{img/gmic_stdlib313.jpg}\\
{\footnotesize \textbf{Example 313~:} \texttt{image.jpg --dilate 10}}
\end{center}

\subsection{\emph{dilate\_circ\index{dilate\_circ}} }\vspace*{-0.7em}
~\\\textbf{\Cb{Arguments: }}\begin{flushleft}
{\small \Cb{\hspace*{0.5cm}$\bullet$~~\texttt{\_size$>$=0{\comma}\_boundary\_conditions{\comma}\_is\_normalized=\{ 0 ~$|$~ 1 \}}}}\end{flushleft}
Apply circular dilation of selected image by specified size.
\begin{flushleft}\Cc{\textbf{Default values}:\\~\\\hspace*{0.5cm}{\small $\bullet$~~\texttt{'boundary\_conditions=1'} and \texttt{'is\_normalized=0'.}}}\end{flushleft}
\begin{center}\includegraphics[keepaspectratio=true,height=6cm,width=\textwidth]{img/gmic_stdlib314.jpg}\\
{\footnotesize \textbf{Example 314~:} \texttt{image.jpg --dilate\_circ 7}}
\end{center}

\subsection{\emph{dilate\_oct\index{dilate\_oct}} }\vspace*{-0.7em}
~\\\textbf{\Cb{Arguments: }}\begin{flushleft}
{\small \Cb{\hspace*{0.5cm}$\bullet$~~\texttt{\_size$>$=0{\comma}\_boundary\_conditions{\comma}\_is\_normalized=\{ 0 ~$|$~ 1 \}}}}\end{flushleft}
Apply octagonal dilation of selected image by specified size.
\begin{flushleft}\Cc{\textbf{Default values}:\\~\\\hspace*{0.5cm}{\small $\bullet$~~\texttt{'boundary\_conditions=1'} and \texttt{'is\_normalized=0'.}}}\end{flushleft}
\begin{center}\includegraphics[keepaspectratio=true,height=6cm,width=\textwidth]{img/gmic_stdlib315.jpg}\\
{\footnotesize \textbf{Example 315~:} \texttt{image.jpg --dilate\_oct 7}}
\end{center}

\subsection{\emph{dilate\_threshold\index{dilate\_threshold}} }\vspace*{-0.7em}
~\\\textbf{\Cb{Arguments: }}\begin{flushleft}
{\small \Cb{\hspace*{0.5cm}$\bullet$~~\texttt{size\_x$>$=1{\comma}size\_y$>$=1{\comma}size\_z$>$=1{\comma}\_threshold$>$=0{\comma}\_boundary\_condit\-ions}}}\end{flushleft}
Dilate selected images in the (X{\comma}Y{\comma}Z{\comma}I) space.
~\\'boundary\_conditions' can be \{ 0=dirichlet ~$|$~ 1=neumann \}.
\begin{flushleft}\Cc{\textbf{Default values}:\\~\\\hspace*{0.5cm}{\small $\bullet$~~\texttt{'size\_y=size\_x'{\comma} 'size\_z=1'{\comma} 'threshold=255'} and \texttt{'boundary\_conditions=1'.}}}\end{flushleft}


\subsection{\emph{divergence\index{divergence}} }\vspace*{-0.7em}
Compute divergence of selected vector fields.
\begin{center}\includegraphics[keepaspectratio=true,height=6cm,width=\textwidth]{img/gmic_stdlib316.jpg}\\
{\footnotesize \textbf{Example 316~:} \texttt{image.jpg luminance --gradient append[-2{\comma}-1] c divergence[-1]}}
\end{center}

\subsection{\emph{dog\index{dog}} }\vspace*{-0.7em}
~\\\textbf{\Cb{Arguments: }}\begin{flushleft}
{\small \Cb{\hspace*{0.5cm}$\bullet$~~\texttt{\_sigma1$>$=0[\%]{\comma}\_sigma2$>$=0[\%]}}}\end{flushleft}
Compute difference of gaussian on selected images.
\begin{flushleft}\Cc{\textbf{Default values}:\\~\\\hspace*{0.5cm}{\small $\bullet$~~\texttt{'sigma1=2\%'} and \texttt{'sigma2=3\%'.}}}\end{flushleft}
\begin{center}\includegraphics[keepaspectratio=true,height=6cm,width=\textwidth]{img/gmic_stdlib317.jpg}\\
{\footnotesize \textbf{Example 317~:} \texttt{image.jpg --dog 2{\comma}3}}
\end{center}

\subsection{\emph{diffusiontensors\index{diffusiontensors}} }\vspace*{-0.7em}
~\\\textbf{\Cb{Arguments: }}\begin{flushleft}
{\small \Cb{\hspace*{0.5cm}$\bullet$~~\texttt{\_sharpness$>$=0{\comma}0$<$=\_anisotropy$<$=1{\comma}\_alpha[\%]{\comma}\_sigma[\%]{\comma}is\_sqrt=\-\{ 0 ~$|$~ 1 \}}}}\end{flushleft}
Compute the diffusion tensors of selected images for edge-preserving smoothing algorithms.
\begin{flushleft}\Cc{\textbf{Default values}:\\~\\\hspace*{0.5cm}{\small $\bullet$~~\texttt{'sharpness=0.7'{\comma} 'anisotropy=0.3'{\comma} 'alpha=0.6'{\comma} 'sigma=1.1'} and \texttt{'is\_sqrt=0'.}}}\end{flushleft}
\begin{center}\includegraphics[keepaspectratio=true,height=6cm,width=\textwidth]{img/gmic_stdlib318.jpg}\\
{\footnotesize \textbf{Example 318~:} \texttt{image.jpg diffusiontensors 0.8 abs pow 0.2}}
\end{center}
~\\
~\textbf{Tutorial page: }\\\url{http://gmic.eu/tutorial/\_diffusiontensors.shtml}


\subsection{\emph{edges\index{edges}} }\vspace*{-0.7em}
~\\\textbf{\Cb{Arguments: }}\begin{flushleft}
{\small \Cb{\hspace*{0.5cm}$\bullet$~~\texttt{\_threshold[\%]$>$=0}}}\end{flushleft}
Estimate contours of selected images.
\begin{flushleft}\Cc{\textbf{Default value}:\\~\\\hspace*{0.5cm}{\small $\bullet$~~\texttt{'edges=15\%'}}}\end{flushleft}
\begin{center}\includegraphics[keepaspectratio=true,height=6cm,width=\textwidth]{img/gmic_stdlib319.jpg}\\
{\footnotesize \textbf{Example 319~:} \texttt{image.jpg --edges 15\%}}
\end{center}

\subsection{\emph{erode\index{erode}} (+)}\vspace*{-0.7em}
~\\\textbf{\Cb{Arguments: }}\begin{flushleft}
{\small \Cb{\hspace*{0.5cm}$\bullet$~~\texttt{size$>$=0}}}~~~\\
{\small \Cb{\hspace*{0.5cm}$\bullet$~~\texttt{size\_x$>$=0{\comma}size\_y$>$=0{\comma}\_size\_z$>$=0}}}~~~\\
{\small \Cb{\hspace*{0.5cm}$\bullet$~~\texttt{[kernel]{\comma}\_boundary\_conditions{\comma}\_is\_real=\{ 0=binary-mode ~$|$~ 1=r\-eal-mode \}}}}\end{flushleft}
Erode selected images by a rectangular or the specified structuring element.
~\\'boundary\_conditions' can be \{ 0=dirichlet ~$|$~ 1=neumann \}.
\begin{flushleft}\Cc{\textbf{Default values}:\\~\\\hspace*{0.5cm}{\small $\bullet$~~\texttt{'size\_z=1'{\comma} 'boundary\_conditions=1'} and \texttt{'is\_real=0'.}}}\end{flushleft}
\begin{center}\includegraphics[keepaspectratio=true,height=6cm,width=\textwidth]{img/gmic_stdlib320.jpg}\\
{\footnotesize \textbf{Example 320~:} \texttt{image.jpg --erode 10}}
\end{center}

\subsection{\emph{erode\_circ\index{erode\_circ}} }\vspace*{-0.7em}
~\\\textbf{\Cb{Arguments: }}\begin{flushleft}
{\small \Cb{\hspace*{0.5cm}$\bullet$~~\texttt{\_size$>$=0{\comma}\_boundary\_conditions{\comma}\_is\_normalized=\{ 0 ~$|$~ 1 \}}}}\end{flushleft}
Apply circular erosion of selected images by specified size.
\begin{flushleft}\Cc{\textbf{Default values}:\\~\\\hspace*{0.5cm}{\small $\bullet$~~\texttt{'boundary\_conditions=1'} and \texttt{'is\_normalized=0'.}}}\end{flushleft}
\begin{center}\includegraphics[keepaspectratio=true,height=6cm,width=\textwidth]{img/gmic_stdlib321.jpg}\\
{\footnotesize \textbf{Example 321~:} \texttt{image.jpg --erode\_circ 7}}
\end{center}

\subsection{\emph{erode\_oct\index{erode\_oct}} }\vspace*{-0.7em}
~\\\textbf{\Cb{Arguments: }}\begin{flushleft}
{\small \Cb{\hspace*{0.5cm}$\bullet$~~\texttt{\_size$>$=0{\comma}\_boundary\_conditions{\comma}\_is\_normalized=\{ 0 ~$|$~ 1 \}}}}\end{flushleft}
Apply octagonal erosion of selected images by specified size.
\begin{flushleft}\Cc{\textbf{Default values}:\\~\\\hspace*{0.5cm}{\small $\bullet$~~\texttt{'boundary\_conditions=1'} and \texttt{'is\_normalized=0'.}}}\end{flushleft}
\begin{center}\includegraphics[keepaspectratio=true,height=6cm,width=\textwidth]{img/gmic_stdlib322.jpg}\\
{\footnotesize \textbf{Example 322~:} \texttt{image.jpg --erode\_oct 7}}
\end{center}

\subsection{\emph{erode\_threshold\index{erode\_threshold}} }\vspace*{-0.7em}
~\\\textbf{\Cb{Arguments: }}\begin{flushleft}
{\small \Cb{\hspace*{0.5cm}$\bullet$~~\texttt{size\_x$>$=1{\comma}size\_y$>$=1{\comma}size\_z$>$=1{\comma}\_threshold$>$=0{\comma}\_boundary\_condit\-ions}}}\end{flushleft}
Erode selected images in the (X{\comma}Y{\comma}Z{\comma}I) space.
~\\'boundary\_conditions' can be \{ 0=dirichlet ~$|$~ 1=neumann \}.
\begin{flushleft}\Cc{\textbf{Default values}:\\~\\\hspace*{0.5cm}{\small $\bullet$~~\texttt{'size\_y=size\_x'{\comma} 'size\_z=1'{\comma} 'threshold=255'} and \texttt{'boundary\_conditions=1'.}}}\end{flushleft}


\subsection{\emph{fft\index{fft}} (+)}\vspace*{-0.7em}
~\\\textbf{\Cb{Arguments: }}\begin{flushleft}
{\small \Cb{\hspace*{0.5cm}$\bullet$~~\texttt{\_\{ x ~$|$~ y ~$|$~ z \}...\{ x ~$|$~ y ~$|$~ z \}}}}\end{flushleft}
Compute the direct fourier transform (real and imaginary parts) of selected images{\comma}
optionally along the specified axes only.
\begin{center}\includegraphics[keepaspectratio=true,height=6cm,width=\textwidth]{img/gmic_stdlib323.jpg}\\
{\footnotesize \textbf{Example 323~:} \texttt{image.jpg luminance --fft append[-2{\comma}-1] c norm[-1] log[-1] shift[-1] 50\%{\comma}50\%{\comma}0{\comma}0{\comma}2}}
\\\includegraphics[keepaspectratio=true,height=6cm,width=\textwidth]{img/gmic_stdlib324.jpg}\\
{\footnotesize \textbf{Example 324~:} \texttt{image.jpg w2=\{int(w/2)\} h2=\{int(h/2)\} fft shift \$w2{\comma}\$h2{\comma}0{\comma}0{\comma}2 ellipse \$w2{\comma}\$h2{\comma}30{\comma}30{\comma}0{\comma}1{\comma}0 shift -\$w2{\comma}-\$h2{\comma}0{\comma}0{\comma}2 ifft remove[-1]}}
\end{center}
~\\
~\textbf{Tutorial page: }\\\url{http://gmic.eu/tutorial/\_fft.shtml}


\subsection{\emph{gradient\index{gradient}} (+)}\vspace*{-0.7em}
~\\\textbf{\Cb{Arguments: }}\begin{flushleft}
{\small \Cb{\hspace*{0.5cm}$\bullet$~~\texttt{\{ x ~$|$~ y ~$|$~ z \}...\{ x ~$|$~ y ~$|$~ z \}{\comma}\_scheme}}}~~~\\
{\small \Cb{\hspace*{0.5cm}$\bullet$~~\texttt{(no arg)}}}\end{flushleft}
Compute the gradient components (first derivatives) of selected images.
~\\(\emph{eq. to} {\small \texttt{'g').\textbackslash n}}).
~\\'scheme' can be \{ -1=backward ~$|$~ 0=centered ~$|$~ 1=forward ~$|$~ 2=sobel ~$|$~ 3=rotation-invariant (default) ~$|$~ 4=deriche ~$|$~ 5=vanvliet \}.
~\\(no arg) compute all significant 2d/3d components.
\begin{flushleft}\Cc{\textbf{Default value}:\\~\\\hspace*{0.5cm}{\small $\bullet$~~\texttt{'scheme=3'.}}}\end{flushleft}
\begin{center}\includegraphics[keepaspectratio=true,height=6cm,width=\textwidth]{img/gmic_stdlib325.jpg}\\
{\footnotesize \textbf{Example 325~:} \texttt{image.jpg gradient}}
\end{center}
~\\
~\textbf{Tutorial page: }\\\url{http://gmic.eu/tutorial/\_gradient.shtml}


\subsection{\emph{gradient\_norm\index{gradient\_norm}} }\vspace*{-0.7em}
Compute gradient norm of selected images.
\begin{center}\includegraphics[keepaspectratio=true,height=6cm,width=\textwidth]{img/gmic_stdlib326.jpg}\\
{\footnotesize \textbf{Example 326~:} \texttt{image.jpg --gradient\_norm equalize[-1]}}
\end{center}
~\\
~\textbf{Tutorial page: }\\\url{http://gmic.eu/tutorial/\_gradient\_norm.shtml}


\subsection{\emph{gradient\_orientation\index{gradient\_orientation}} }\vspace*{-0.7em}
~\\\textbf{\Cb{Arguments: }}\begin{flushleft}
{\small \Cb{\hspace*{0.5cm}$\bullet$~~\texttt{\_dimension=\{1{\comma}2{\comma}3\}}}}\end{flushleft}
Compute N-d gradient orientation of selected images.
\begin{flushleft}\Cc{\textbf{Default value}:\\~\\\hspace*{0.5cm}{\small $\bullet$~~\texttt{'dimension=3'.}}}\end{flushleft}
\begin{center}\includegraphics[keepaspectratio=true,height=6cm,width=\textwidth]{img/gmic_stdlib327.jpg}\\
{\footnotesize \textbf{Example 327~:} \texttt{image.jpg --gradient\_orientation 2}}
\end{center}

\subsection{\emph{guided\index{guided}} (+)}\vspace*{-0.7em}
~\\\textbf{\Cb{Arguments: }}\begin{flushleft}
{\small \Cb{\hspace*{0.5cm}$\bullet$~~\texttt{[guide]{\comma}radius[\%]$>$=0{\comma}regularization[\%]$>$=0}}}~~~\\
{\small \Cb{\hspace*{0.5cm}$\bullet$~~\texttt{radius[\%]$>$=0{\comma}regularization[\%]$>$=0}}}\end{flushleft}
Blur selected images by guided image filtering.
~\\If a guide image is provided{\comma} it is used to drive the smoothing process.
~\\A guide image must be of the same xyz-size as the selected images.
~\\This command implements the filtering algorithm described in:
~\\He{\comma} Kaiming; Sun{\comma} Jian; Tang{\comma} Xiaoou{\comma} "Guided Image Filtering{\comma}" Pattern Analysis and Machine Intelligence{\comma}
~\\IEEE Transactions on {\comma} vol.35{\comma} no.6{\comma} pp.1397{\comma}1409{\comma} June 2013
\begin{center}\includegraphics[keepaspectratio=true,height=6cm,width=\textwidth]{img/gmic_stdlib328.jpg}\\
{\footnotesize \textbf{Example 328~:} \texttt{image.jpg [0] --guided 5{\comma}400}}
\end{center}

\subsection{\emph{haar\index{haar}} }\vspace*{-0.7em}
~\\\textbf{\Cb{Arguments: }}\begin{flushleft}
{\small \Cb{\hspace*{0.5cm}$\bullet$~~\texttt{scale$>$0}}}\end{flushleft}
Compute the direct haar multiscale wavelet transform of selected images.

~\\
~\textbf{Tutorial page: }\\\url{http://gmic.eu/tutorial/\_haar.shtml}


\subsection{\emph{heat\_flow\index{heat\_flow}} }\vspace*{-0.7em}
~\\\textbf{\Cb{Arguments: }}\begin{flushleft}
{\small \Cb{\hspace*{0.5cm}$\bullet$~~\texttt{\_nb\_iter$>$=0{\comma}\_dt{\comma}\_keep\_sequence=\{ 0 ~$|$~ 1 \}}}}\end{flushleft}
Apply iterations of the heat flow on selected images.
\begin{flushleft}\Cc{\textbf{Default values}:\\~\\\hspace*{0.5cm}{\small $\bullet$~~\texttt{'nb\_iter=10'{\comma} 'dt=30'} and \texttt{'keep\_sequence=0'.}}}\end{flushleft}
\begin{center}\includegraphics[keepaspectratio=true,height=6cm,width=\textwidth]{img/gmic_stdlib329.jpg}\\
{\footnotesize \textbf{Example 329~:} \texttt{image.jpg --heat\_flow 20}}
\end{center}

\subsection{\emph{hessian\index{hessian}} (+)}\vspace*{-0.7em}
~\\\textbf{\Cb{Arguments: }}\begin{flushleft}
{\small \Cb{\hspace*{0.5cm}$\bullet$~~\texttt{\{ xx ~$|$~ xy ~$|$~ xz ~$|$~ yy ~$|$~ yz ~$|$~ zz \}...\{ xx ~$|$~ xy ~$|$~ xz ~$|$~ yy ~$|$~ yz ~$|$~\- zz \}}}}~~~\\
{\small \Cb{\hspace*{0.5cm}$\bullet$~~\texttt{(no arg)}}}\end{flushleft}
Compute the hessian components (second derivatives) of selected images.
~\\(no arg) compute all significant components.
\begin{center}\includegraphics[keepaspectratio=true,height=6cm,width=\textwidth]{img/gmic_stdlib330.jpg}\\
{\footnotesize \textbf{Example 330~:} \texttt{image.jpg hessian}}
\end{center}

\subsection{\emph{idct\index{idct}} }\vspace*{-0.7em}
~\\\textbf{\Cb{Arguments: }}\begin{flushleft}
{\small \Cb{\hspace*{0.5cm}$\bullet$~~\texttt{\_\{ x ~$|$~ y ~$|$~ z \}...\{ x ~$|$~ y ~$|$~ z \}}}}~~~\\
{\small \Cb{\hspace*{0.5cm}$\bullet$~~\texttt{(no arg)}}}\end{flushleft}
Compute the inverse discrete cosine transform of selected images{\comma}
optionally along the specified axes only.
\begin{flushleft}\Cc{\textbf{Default values}:\\~\\\hspace*{0.5cm}{\small $\bullet$~~\texttt{(no arg)}}}\end{flushleft}

~\\
~\textbf{Tutorial page: }\\\url{http://gmic.eu/tutorial/\_dct-and-idct.shtml}


\subsection{\emph{iee\index{iee}} }\vspace*{-0.7em}
Compute gradient-orthogonal-directed 2nd derivative of image(s).
\begin{center}\includegraphics[keepaspectratio=true,height=6cm,width=\textwidth]{img/gmic_stdlib331.jpg}\\
{\footnotesize \textbf{Example 331~:} \texttt{image.jpg iee}}
\end{center}

\subsection{\emph{ifft\index{ifft}} (+)}\vspace*{-0.7em}
~\\\textbf{\Cb{Arguments: }}\begin{flushleft}
{\small \Cb{\hspace*{0.5cm}$\bullet$~~\texttt{\_\{ x ~$|$~ y ~$|$~ z \}...\{ x ~$|$~ y ~$|$~ z \}}}}\end{flushleft}
Compute the inverse fourier transform (real and imaginary parts) of selected images.
optionally along the specified axes only.

~\\
~\textbf{Tutorial page: }\\\url{http://gmic.eu/tutorial/\_fft.shtml}


\subsection{\emph{ihaar\index{ihaar}} }\vspace*{-0.7em}
~\\\textbf{\Cb{Arguments: }}\begin{flushleft}
{\small \Cb{\hspace*{0.5cm}$\bullet$~~\texttt{scale$>$0}}}\end{flushleft}
Compute the inverse haar multiscale wavelet transform of selected images.


\subsection{\emph{inn\index{inn}} }\vspace*{-0.7em}
Compute gradient-directed 2nd derivative of image(s).
\begin{center}\includegraphics[keepaspectratio=true,height=6cm,width=\textwidth]{img/gmic_stdlib332.jpg}\\
{\footnotesize \textbf{Example 332~:} \texttt{image.jpg inn}}
\end{center}

\subsection{\emph{inpaint\index{inpaint}} (+)}\vspace*{-0.7em}
~\\\textbf{\Cb{Arguments: }}\begin{flushleft}
{\small \Cb{\hspace*{0.5cm}$\bullet$~~\texttt{[mask]}}}~~~\\
{\small \Cb{\hspace*{0.5cm}$\bullet$~~\texttt{[mask]{\comma}0{\comma}\_fast\_method}}}~~~\\
{\small \Cb{\hspace*{0.5cm}$\bullet$~~\texttt{[mask]{\comma}\_patch\_size$>$=1{\comma}\_lookup\_size$>$=1{\comma}\_lookup\_factor$>$=0{\comma}\_loo\-kup\_increment!=0{\comma}\_blend\_size$>$=0{\comma}0$<$=\_blend\_threshold$<$=1{\comma}\_blen\-d\_decay$>$=0{\comma}\_blend\_scales$>$=1{\comma}\_is\_blend\_outer=\{ 0 ~$|$~ 1 \}}}}\end{flushleft}
Inpaint selected images by specified mask.
~\\If no patch size (or 0) is specified{\comma} inpainting is done using a fast average or median algorithm.
~\\Otherwise{\comma} it used a patch-based reconstruction method{\comma} that can be very time consuming.
~\\'fast\_method' can be \{ 0=low-connectivity average ~$|$~ 1=high-connectivity average ~$|$~ 2=low-connectivity median ~$|$~ 3=high-connectivity median \}.
\begin{flushleft}\Cc{\textbf{Default values}:\\~\\\hspace*{0.5cm}{\small $\bullet$~~\texttt{'patch\_size=0'{\comma} 'fast\_method=1'{\comma} 'lookup\_size=22'{\comma} 'lookup\_factor=0.5'{\comma} 'lookup\_increment=1'{\comma} 'blend\_size=0'{\comma} 'blend\_threshold=0'{\comma} 'blend\_decay=0.05'{\comma} 'blend\_scales=10'} and \texttt{'is\_blend\_outer=1'.}}}\end{flushleft}
\begin{center}\includegraphics[keepaspectratio=true,height=6cm,width=\textwidth]{img/gmic_stdlib333.jpg}\\
{\footnotesize \textbf{Example 333~:} \texttt{image.jpg 100\%{\comma}100\% ellipse 50\%{\comma}50\%{\comma}30{\comma}30{\comma}0{\comma}1{\comma}255 ellipse 20\%{\comma}20\%{\comma}30{\comma}10{\comma}0{\comma}1{\comma}255 --inpaint[-2] [-1] remove[-2]}}
\\\includegraphics[keepaspectratio=true,height=6cm,width=\textwidth]{img/gmic_stdlib334.jpg}\\
{\footnotesize \textbf{Example 334~:} \texttt{image.jpg 100\%{\comma}100\% circle 30\%{\comma}30\%{\comma}30{\comma}1{\comma}255{\comma}0{\comma}255 circle 70\%{\comma}70\%{\comma}50{\comma}1{\comma}255{\comma}0{\comma}255 --inpaint[0] [1]{\comma}5{\comma}15{\comma}0.5{\comma}1{\comma}9{\comma}0 remove[1]}}
\end{center}

\subsection{\emph{inpaint\_flow\index{inpaint\_flow}} }\vspace*{-0.7em}
~\\\textbf{\Cb{Arguments: }}\begin{flushleft}
{\small \Cb{\hspace*{0.5cm}$\bullet$~~\texttt{[mask]{\comma}\_nb\_global\_iter$>$=0{\comma}\_nb\_local\_iter$>$=0{\comma}\_dt$>$0{\comma}\_alpha$>$=0{\comma}\-\_sigma$>$=0}}}\end{flushleft}
Apply iteration of the inpainting flow on selected images.
\begin{flushleft}\Cc{\textbf{Default values}:\\~\\\hspace*{0.5cm}{\small $\bullet$~~\texttt{'nb\_global\_iter=4'{\comma} 'nb\_global\_iter=15'{\comma} 'dt=10'{\comma} 'alpha=1'} and \texttt{'sigma=3'.}}}\end{flushleft}
\begin{center}\includegraphics[keepaspectratio=true,height=6cm,width=\textwidth]{img/gmic_stdlib335.jpg}\\
{\footnotesize \textbf{Example 335~:} \texttt{image.jpg 100\%{\comma}100\% ellipse[-1] 30\%{\comma}30\%{\comma}40{\comma}30{\comma}0{\comma}1{\comma}255 inpaint\_flow[0] [1]}}
\end{center}

\subsection{\emph{inpaint\_holes\index{inpaint\_holes}} }\vspace*{-0.7em}
~\\\textbf{\Cb{Arguments: }}\begin{flushleft}
{\small \Cb{\hspace*{0.5cm}$\bullet$~~\texttt{maximal\_area[\%]$>$=0{\comma}\_tolerance$>$=0{\comma}\_is\_high\_connectivity=\{ 0 ~$|$~\- 1 \}}}}\end{flushleft}
Inpaint all connected regions having an area less than specified value.
\begin{flushleft}\Cc{\textbf{Default values}:\\~\\\hspace*{0.5cm}{\small $\bullet$~~\texttt{'maximal\_area=4'{\comma} 'tolerance=0'} and \texttt{'is\_high\_connectivity=0'.}}}\end{flushleft}
\begin{center}\includegraphics[keepaspectratio=true,height=6cm,width=\textwidth]{img/gmic_stdlib336.jpg}\\
{\footnotesize \textbf{Example 336~:} \texttt{image.jpg noise 5\%{\comma}2 --inpaint\_holes 8{\comma}40}}
\end{center}

\subsection{\emph{inpaint\_morpho\index{inpaint\_morpho}} }\vspace*{-0.7em}
~\\\textbf{\Cb{Arguments: }}\begin{flushleft}
{\small \Cb{\hspace*{0.5cm}$\bullet$~~\texttt{[mask]}}}\end{flushleft}
Inpaint selected images by specified mask using morphological operators.
\begin{center}\includegraphics[keepaspectratio=true,height=6cm,width=\textwidth]{img/gmic_stdlib337.jpg}\\
{\footnotesize \textbf{Example 337~:} \texttt{image.jpg 100\%{\comma}100\% ellipse[-1] 30\%{\comma}30\%{\comma}40{\comma}30{\comma}0{\comma}1{\comma}255 --inpaint\_morpho[0] [1]}}
\end{center}

\subsection{\emph{inpaint\_patchmatch\index{inpaint\_patchmatch}} }\vspace*{-0.7em}
~\\\textbf{\Cb{Arguments: }}\begin{flushleft}
{\small \Cb{\hspace*{0.5cm}$\bullet$~~\texttt{[mask]{\comma}\_nb\_scales=\{ 0=auto ~$|$~ $>$0 \}{\comma}\_patch\_size$>$0{\comma}\_nb\_iteratio\-ns\_per\_scale$>$0{\comma}\_blend\_size$>$=0{\comma}\_allow\_outer\_blending=\{ 0 ~$|$~ 1 \-\}{\comma}\_is\_already\_initialized=\{ 0 ~$|$~ 1 \}}}}\end{flushleft}
Inpaint selected images by specified binary mask{\comma} using a multi-scale patchmatch algorithm.
\begin{flushleft}\Cc{\textbf{Default values}:\\~\\\hspace*{0.5cm}{\small $\bullet$~~\texttt{'nb\_scales=0'{\comma} 'patch\_size=9'{\comma} 'nb\_iterations\_per\_scale=10'{\comma} 'blend\_size=5'{\comma} 'allow\_outer\_blending=1'} and \texttt{'is\_already\_initialized=0'.}}}\end{flushleft}
\begin{center}\includegraphics[keepaspectratio=true,height=6cm,width=\textwidth]{img/gmic_stdlib338.jpg}\\
{\footnotesize \textbf{Example 338~:} \texttt{image.jpg 100\%{\comma}100\% ellipse[-1] 30\%{\comma}30\%{\comma}40{\comma}30{\comma}0{\comma}1{\comma}255 --inpaint\_patchmatch[0] [1]}}
\end{center}

\subsection{\emph{inpaint\_diffusion\index{inpaint\_diffusion}} }\vspace*{-0.7em}
~\\\textbf{\Cb{Arguments: }}\begin{flushleft}
{\small \Cb{\hspace*{0.5cm}$\bullet$~~\texttt{[mask]{\comma}\_nb\_scales[\%]$>$=0{\comma}\_diffusion\_type=\{ 0=isotropic ~$|$~ 1=de\-launay-guided ~$|$~ 2=edge-guided ~$|$~ 3=mask-guided \}{\comma}\_diffusion\_i\-ter$>$=0}}}\end{flushleft}
Inpaint selected images by specified mask using a multiscale transport-diffusion algorithm.
~\\If 'diffusion type==3'{\comma} non-zero values of the mask (e.g. a distance function) are used to guide the diffusion process.
\begin{flushleft}\Cc{\textbf{Default values}:\\~\\\hspace*{0.5cm}{\small $\bullet$~~\texttt{'nb\_scales=75\%'{\comma} 'diffusion\_type=1'} and \texttt{'diffusion\_iter=20'.}}}\end{flushleft}
\begin{center}\includegraphics[keepaspectratio=true,height=6cm,width=\textwidth]{img/gmic_stdlib339.jpg}\\
{\footnotesize \textbf{Example 339~:} \texttt{image.jpg 100\%{\comma}100\% ellipse[-1] 30\%{\comma}30\%{\comma}40{\comma}30{\comma}0{\comma}1{\comma}255 --inpaint\_diffusion[0] [1]}}
\end{center}

\subsection{\emph{kuwahara\index{kuwahara}} }\vspace*{-0.7em}
~\\\textbf{\Cb{Arguments: }}\begin{flushleft}
{\small \Cb{\hspace*{0.5cm}$\bullet$~~\texttt{size$>$0}}}\end{flushleft}
Apply Kuwahara filter of specified size on selected images.
\begin{center}\includegraphics[keepaspectratio=true,height=6cm,width=\textwidth]{img/gmic_stdlib340.jpg}\\
{\footnotesize \textbf{Example 340~:} \texttt{image.jpg --kuwahara 5}}
\end{center}

\subsection{\emph{laplacian\index{laplacian}} }\vspace*{-0.7em}
Compute Laplacian of selected images.
\begin{center}\includegraphics[keepaspectratio=true,height=6cm,width=\textwidth]{img/gmic_stdlib341.jpg}\\
{\footnotesize \textbf{Example 341~:} \texttt{image.jpg laplacian}}
\end{center}

\subsection{\emph{lic\index{lic}} }\vspace*{-0.7em}
~\\\textbf{\Cb{Arguments: }}\begin{flushleft}
{\small \Cb{\hspace*{0.5cm}$\bullet$~~\texttt{\_amplitude$>$0{\comma}\_channels$>$0}}}\end{flushleft}
Render LIC representation of selected vector fields.
\begin{flushleft}\Cc{\textbf{Default values}:\\~\\\hspace*{0.5cm}{\small $\bullet$~~\texttt{'amplitude=30'} and \texttt{'channels=1'.}}}\end{flushleft}
\begin{center}\includegraphics[keepaspectratio=true,height=6cm,width=\textwidth]{img/gmic_stdlib342.jpg}\\
{\footnotesize \textbf{Example 342~:} \texttt{400{\comma}400{\comma}1{\comma}2{\comma}'if(c==0{\comma}x-w/2{\comma}y-h/2)' --lic 200{\comma}3 quiver[-2] [-2]{\comma}10{\comma}1{\comma}1{\comma}1{\comma}255}}
\end{center}

\subsection{\emph{map\_tones\index{map\_tones}} }\vspace*{-0.7em}
~\\\textbf{\Cb{Arguments: }}\begin{flushleft}
{\small \Cb{\hspace*{0.5cm}$\bullet$~~\texttt{\_threshold$>$=0{\comma}\_gamma$>$=0{\comma}\_smoothness$>$=0{\comma}nb\_iter$>$=0}}}\end{flushleft}
Apply tone mapping operator on selected images{\comma} based on Poisson equation.
\begin{flushleft}\Cc{\textbf{Default values}:\\~\\\hspace*{0.5cm}{\small $\bullet$~~\texttt{'threshold=0.1'{\comma} 'gamma=0.8'{\comma} 'smoothness=0.5'} and \texttt{'nb\_iter=30'.}}}\end{flushleft}
\begin{center}\includegraphics[keepaspectratio=true,height=6cm,width=\textwidth]{img/gmic_stdlib343.jpg}\\
{\footnotesize \textbf{Example 343~:} \texttt{image.jpg --map\_tones {\comma}}}
\end{center}

\subsection{\emph{map\_tones\_fast\index{map\_tones\_fast}} }\vspace*{-0.7em}
~\\\textbf{\Cb{Arguments: }}\begin{flushleft}
{\small \Cb{\hspace*{0.5cm}$\bullet$~~\texttt{\_radius[\%]$>$=0{\comma}\_power$>$=0}}}\end{flushleft}
Apply fast tone mapping operator on selected images.
\begin{flushleft}\Cc{\textbf{Default values}:\\~\\\hspace*{0.5cm}{\small $\bullet$~~\texttt{'radius=3\%'} and \texttt{'power=0.3'.}}}\end{flushleft}
\begin{center}\includegraphics[keepaspectratio=true,height=6cm,width=\textwidth]{img/gmic_stdlib344.jpg}\\
{\footnotesize \textbf{Example 344~:} \texttt{image.jpg --map\_tones\_fast {\comma}}}
\end{center}

\subsection{\emph{meancurvature\_flow\index{meancurvature\_flow}} }\vspace*{-0.7em}
~\\\textbf{\Cb{Arguments: }}\begin{flushleft}
{\small \Cb{\hspace*{0.5cm}$\bullet$~~\texttt{\_nb\_iter$>$=0{\comma}\_dt{\comma}\_keep\_sequence=\{ 0 ~$|$~ 1 \}}}}\end{flushleft}
Apply iterations of the mean curvature flow on selected images.
\begin{flushleft}\Cc{\textbf{Default values}:\\~\\\hspace*{0.5cm}{\small $\bullet$~~\texttt{'nb\_iter=10'{\comma} 'dt=30'} and \texttt{'keep\_sequence=0'.}}}\end{flushleft}
\begin{center}\includegraphics[keepaspectratio=true,height=6cm,width=\textwidth]{img/gmic_stdlib345.jpg}\\
{\footnotesize \textbf{Example 345~:} \texttt{image.jpg --meancurvature\_flow 20}}
\end{center}

\subsection{\emph{median\index{median}} (+)}\vspace*{-0.7em}
~\\\textbf{\Cb{Arguments: }}\begin{flushleft}
{\small \Cb{\hspace*{0.5cm}$\bullet$~~\texttt{size$>$=0{\comma}\_threshold$>$0}}}\end{flushleft}
Apply (opt. thresholded) median filter on selected images with structuring element size x size.
\begin{center}\includegraphics[keepaspectratio=true,height=6cm,width=\textwidth]{img/gmic_stdlib346.jpg}\\
{\footnotesize \textbf{Example 346~:} \texttt{image.jpg --median 5}}
\end{center}

\subsection{\emph{nlmeans\index{nlmeans}} }\vspace*{-0.7em}
~\\\textbf{\Cb{Arguments: }}\begin{flushleft}
{\small \Cb{\hspace*{0.5cm}$\bullet$~~\texttt{[guide]{\comma}\_patch\_radius$>$0{\comma}\_spatial\_bandwidth$>$0{\comma}\_tonal\_bandwidt\-h$>$0{\comma}\_patch\_measure\_command}}}~~~\\
{\small \Cb{\hspace*{0.5cm}$\bullet$~~\texttt{\_patch\_radius$>$0{\comma}\_spatial\_bandwidth$>$0{\comma}\_tonal\_bandwidth$>$0{\comma}\_pat\-ch\_measure\_command}}}\end{flushleft}
Apply non local means denoising of Buades et al{\comma} 2005. on selected images.
~\\The patch is a gaussian function of 'std \_patch\_radius'.
~\\The spatial kernel is a rectangle of radius 'spatial\_bandwidth'.
~\\The tonal kernel is exponential (exp(-d\textasciicircum 2/\_tonal\_bandwidth\textasciicircum 2))
with d the euclidiean distance between image patches.
\begin{flushleft}\Cc{\textbf{Default values}:\\~\\\hspace*{0.5cm}{\small $\bullet$~~\texttt{'patch\_radius=4'{\comma} 'spatial\_bandwidth=4'{\comma} 'tonal\_bandwidth=10'} and \texttt{'patch\_measure\_command=-norm'.}}}\end{flushleft}
\begin{center}\includegraphics[keepaspectratio=true,height=6cm,width=\textwidth]{img/gmic_stdlib347.jpg}\\
{\footnotesize \textbf{Example 347~:} \texttt{image.jpg --noise 10 nlmeans[-1] 4{\comma}4{\comma}\{0.6*\$\{-std\_noise\}\}}}
\end{center}

\subsection{\emph{nlmeans\_core\index{nlmeans\_core}} }\vspace*{-0.7em}
~\\\textbf{\Cb{Arguments: }}\begin{flushleft}
{\small \Cb{\hspace*{0.5cm}$\bullet$~~\texttt{\_reference\_image{\comma}\_scaling\_map{\comma}\_patch\_radius$>$0{\comma}\_spatial\_bandw\-idth$>$0}}}\end{flushleft}
Apply non local means denoising using a image for weigth and a map for scaling


\subsection{\emph{normalize\_local\index{normalize\_local}} }\vspace*{-0.7em}
~\\\textbf{\Cb{Arguments: }}\begin{flushleft}
{\small \Cb{\hspace*{0.5cm}$\bullet$~~\texttt{\_amplitude$>$=0{\comma}\_radius$>$0{\comma}\_n\_smooth$>$=0[\%]{\comma}\_a\_smooth$>$=0[\%]{\comma}\_is\_\-cut=\{ 0 ~$|$~ 1 \}{\comma}\_min=0{\comma}\_max=255}}}\end{flushleft}
Normalize selected images locally.
\begin{flushleft}\Cc{\textbf{Default values}:\\~\\\hspace*{0.5cm}{\small $\bullet$~~\texttt{'amplitude=3'{\comma} 'radius=16'{\comma} 'n\_smooth=4\%'{\comma} 'a\_smooth=2\%'{\comma} 'is\_cut=1'{\comma} 'min=0'} and \texttt{'max=255'.}}}\end{flushleft}
\begin{center}\includegraphics[keepaspectratio=true,height=6cm,width=\textwidth]{img/gmic_stdlib348.jpg}\\
{\footnotesize \textbf{Example 348~:} \texttt{image.jpg --normalize\_local 8{\comma}10}}
\end{center}

\subsection{\emph{normalized\_cross\_correlation\index{normalized\_cross\_correlation}} }\vspace*{-0.7em}
~\\\textbf{\Cb{Arguments: }}\begin{flushleft}
{\small \Cb{\hspace*{0.5cm}$\bullet$~~\texttt{[mask]}}}\end{flushleft}
Compute normalized cross-correlation of selected images with specified mask.
\begin{center}\includegraphics[keepaspectratio=true,height=6cm,width=\textwidth]{img/gmic_stdlib349.jpg}\\
{\footnotesize \textbf{Example 349~:} \texttt{image.jpg --shift -30{\comma}-20 --normalized\_cross\_correlation[0] [1]}}
\end{center}

\subsection{\emph{peronamalik\_flow\index{peronamalik\_flow}} }\vspace*{-0.7em}
~\\\textbf{\Cb{Arguments: }}\begin{flushleft}
{\small \Cb{\hspace*{0.5cm}$\bullet$~~\texttt{K\_factor$>$0{\comma}\_nb\_iter$>$=0{\comma}\_dt{\comma}\_keep\_sequence=\{ 0 ~$|$~ 1 \}}}}\end{flushleft}
Apply iterations of the Perona-Malik flow on selected images.
\begin{flushleft}\Cc{\textbf{Default values}:\\~\\\hspace*{0.5cm}{\small $\bullet$~~\texttt{'K\_factor=20'{\comma} 'nb\_iter=5'{\comma} 'dt=5'} and \texttt{'keep\_sequence=0'.}}}\end{flushleft}
\begin{center}\includegraphics[keepaspectratio=true,height=6cm,width=\textwidth]{img/gmic_stdlib350.jpg}\\
{\footnotesize \textbf{Example 350~:} \texttt{image.jpg --heat\_flow 20}}
\end{center}

\subsection{\emph{phase\_correlation\index{phase\_correlation}} }\vspace*{-0.7em}
~\\\textbf{\Cb{Arguments: }}\begin{flushleft}
{\small \Cb{\hspace*{0.5cm}$\bullet$~~\texttt{[destination]}}}\end{flushleft}
Estimate translation vector between selected source images and specified destination.
\begin{center}\includegraphics[keepaspectratio=true,height=6cm,width=\textwidth]{img/gmic_stdlib351.jpg}\\
{\footnotesize \textbf{Example 351~:} \texttt{image.jpg --shift -30{\comma}-20 --phase\_correlation[0] [1] unroll[-1] y}}
\end{center}

\subsection{\emph{pde\_flow\index{pde\_flow}} }\vspace*{-0.7em}
~\\\textbf{\Cb{Arguments: }}\begin{flushleft}
{\small \Cb{\hspace*{0.5cm}$\bullet$~~\texttt{\_nb\_iter$>$=0{\comma}\_dt{\comma}\_velocity\_command{\comma}\_keep\_sequence=\{ 0 ~$|$~ 1 \}}}}\end{flushleft}
Apply iterations of a generic PDE flow on selected images.
\begin{flushleft}\Cc{\textbf{Default values}:\\~\\\hspace*{0.5cm}{\small $\bullet$~~\texttt{'nb\_iter=10'{\comma} 'dt=30'{\comma} 'velocity\_command=laplacian'} and \texttt{'keep\_sequence=0'.}}}\end{flushleft}
\begin{center}\includegraphics[keepaspectratio=true,height=6cm,width=\textwidth]{img/gmic_stdlib352.jpg}\\
{\footnotesize \textbf{Example 352~:} \texttt{image.jpg --pde\_flow 20}}
\end{center}

\subsection{\emph{periodize\_poisson\index{periodize\_poisson}} }\vspace*{-0.7em}
Periodize selected images using a Poisson solver in Fourier space.
\begin{center}\includegraphics[keepaspectratio=true,height=6cm,width=\textwidth]{img/gmic_stdlib353.jpg}\\
{\footnotesize \textbf{Example 353~:} \texttt{image.jpg --periodize\_poisson array 2{\comma}2{\comma}2}}
\end{center}

\subsection{\emph{red\_eye\index{red\_eye}} }\vspace*{-0.7em}
~\\\textbf{\Cb{Arguments: }}\begin{flushleft}
{\small \Cb{\hspace*{0.5cm}$\bullet$~~\texttt{0$<$=\_threshold$<$=100{\comma}\_smoothness$>$=0{\comma}0$<$=attenuation$<$=1}}}\end{flushleft}
Attenuate red-eye effect in selected images.
\begin{flushleft}\Cc{\textbf{Default values}:\\~\\\hspace*{0.5cm}{\small $\bullet$~~\texttt{'threshold=75'{\comma} 'smoothness=3.5'} and \texttt{'attenuation=0.1'.}}}\end{flushleft}
\begin{center}\includegraphics[keepaspectratio=true,height=6cm,width=\textwidth]{img/gmic_stdlib354.jpg}\\
{\footnotesize \textbf{Example 354~:} \texttt{image.jpg --red\_eye {\comma}}}
\end{center}

\subsection{\emph{remove\_hotpixels\index{remove\_hotpixels}} }\vspace*{-0.7em}
~\\\textbf{\Cb{Arguments: }}\begin{flushleft}
{\small \Cb{\hspace*{0.5cm}$\bullet$~~\texttt{\_mask\_size$>$0{\comma} \_threshold[\%]$>$0}}}\end{flushleft}
Remove hot pixels in selected images.
\begin{flushleft}\Cc{\textbf{Default values}:\\~\\\hspace*{0.5cm}{\small $\bullet$~~\texttt{'mask\_size=3'} and \texttt{'threshold=10\%'.}}}\end{flushleft}
\begin{center}\includegraphics[keepaspectratio=true,height=6cm,width=\textwidth]{img/gmic_stdlib355.jpg}\\
{\footnotesize \textbf{Example 355~:} \texttt{image.jpg noise 10{\comma}2 --remove\_hotpixels {\comma}}}
\end{center}

\subsection{\emph{remove\_pixels\index{remove\_pixels}} }\vspace*{-0.7em}
~\\\textbf{\Cb{Arguments: }}\begin{flushleft}
{\small \Cb{\hspace*{0.5cm}$\bullet$~~\texttt{number\_of\_pixels[\%]$>$=0}}}\end{flushleft}
Remove specified number of pixels (i.e. set them to 0) from the set of non-zero pixels in selected images.
\begin{center}\includegraphics[keepaspectratio=true,height=6cm,width=\textwidth]{img/gmic_stdlib356.jpg}\\
{\footnotesize \textbf{Example 356~:} \texttt{image.jpg --remove\_pixels 50\%}}
\end{center}

\subsection{\emph{rolling\_guidance\index{rolling\_guidance}} }\vspace*{-0.7em}
~\\\textbf{\Cb{Arguments: }}\begin{flushleft}
{\small \Cb{\hspace*{0.5cm}$\bullet$~~\texttt{std\_variation\_s[\%]$>$=0{\comma}std\_variation\_r[\%]$>$=0{\comma}\_precision$>$=0}}}\end{flushleft}
Apply the rolling guidance filter on selected image.
~\\Rolling guidance filter is a fast image abstraction filter{\comma} described in:
"Rolling Guidance Filter"{\comma} Qi Zhang Xiaoyong{\comma} Shen Li{\comma} Xu Jiaya Jia{\comma} ECCV'2014.
\begin{flushleft}\Cc{\textbf{Default values}:\\~\\\hspace*{0.5cm}{\small $\bullet$~~\texttt{'std\_variation\_s=4'{\comma} 'std\_variation\_r=10'} and \texttt{'precision=0.5'.}}}\end{flushleft}
\begin{center}\includegraphics[keepaspectratio=true,height=6cm,width=\textwidth]{img/gmic_stdlib357.jpg}\\
{\footnotesize \textbf{Example 357~:} \texttt{image.jpg --rolling\_guidance {\comma} ---}}
\end{center}

\subsection{\emph{sharpen\index{sharpen}} (+)}\vspace*{-0.7em}
~\\\textbf{\Cb{Arguments: }}\begin{flushleft}
{\small \Cb{\hspace*{0.5cm}$\bullet$~~\texttt{amplitude$>$=0}}}~~~\\
{\small \Cb{\hspace*{0.5cm}$\bullet$~~\texttt{amplitude$>$=0{\comma}edge$>$=0{\comma}\_alpha{\comma}\_sigma}}}\end{flushleft}
Sharpen selected images by inverse diffusion or shock filters methods.
~\\'edge' must be specified to enable shock-filter method.
\begin{flushleft}\Cc{\textbf{Default values}:\\~\\\hspace*{0.5cm}{\small $\bullet$~~\texttt{'alpha=0'} and \texttt{'sigma=0'.}}}\end{flushleft}
\begin{center}\includegraphics[keepaspectratio=true,height=6cm,width=\textwidth]{img/gmic_stdlib358.jpg}\\
{\footnotesize \textbf{Example 358~:} \texttt{image.jpg --sharpen 300}}
\\\includegraphics[keepaspectratio=true,height=6cm,width=\textwidth]{img/gmic_stdlib359.jpg}\\
{\footnotesize \textbf{Example 359~:} \texttt{image.jpg blur 5 --sharpen[-1] 300{\comma}1}}
\end{center}

\subsection{\emph{smooth\index{smooth}} (+)}\vspace*{-0.7em}
~\\\textbf{\Cb{Arguments: }}\begin{flushleft}
{\small \Cb{\hspace*{0.5cm}$\bullet$~~\texttt{amplitude$>$=0{\comma}\_sharpness$>$=0{\comma}\_anisotropy{\comma}\_alpha{\comma}\_sigma{\comma}\_dl$>$0{\comma}\_\-da$>$0{\comma}\_precision$>$0{\comma}interpolation{\comma}\_fast\_approx=\{ 0 ~$|$~ 1 \}}}}~~~\\
{\small \Cb{\hspace*{0.5cm}$\bullet$~~\texttt{nb\_iterations$>$=0{\comma}\_sharpness$>$=0{\comma}\_anisotropy{\comma}\_alpha{\comma}\_sigma{\comma}\_dt\-$>$0{\comma}0}}}~~~\\
{\small \Cb{\hspace*{0.5cm}$\bullet$~~\texttt{[tensor\_field]{\comma}\_amplitude$>$=0{\comma}\_dl$>$0{\comma}\_da$>$0{\comma}\_precision$>$0{\comma}\_inter\-polation{\comma}\_fast\_approx=\{ 0 ~$|$~ 1 \}}}}~~~\\
{\small \Cb{\hspace*{0.5cm}$\bullet$~~\texttt{[tensor\_field]{\comma}\_nb\_iters$>$=0{\comma}\_dt$>$0{\comma}0}}}\end{flushleft}
Smooth selected images anisotropically using diffusion PDE's{\comma} with specified field of
diffusion tensors.
~\\'anisotropy' must be in [0{\comma}1].
~\\'interpolation' can be \{ 0=nearest ~$|$~ 1=linear ~$|$~ 2=runge-kutta \}.
\begin{flushleft}\Cc{\textbf{Default values}:\\~\\\hspace*{0.5cm}{\small $\bullet$~~\texttt{'sharpness=0.7'{\comma} 'anisotropy=0.3'{\comma} 'alpha=0.6'{\comma} 'sigma=1.1'{\comma} 'dl=0.8'{\comma} 'da=30'{\comma} 'precision=2'{\comma} 'interpolation=0'} and \texttt{'fast\_approx=1'.}}}\end{flushleft}
\begin{center}\includegraphics[keepaspectratio=true,height=6cm,width=\textwidth]{img/gmic_stdlib360.jpg}\\
{\footnotesize \textbf{Example 360~:} \texttt{image.jpg [0] repeat 3 smooth 20 done}}
\\\includegraphics[keepaspectratio=true,height=6cm,width=\textwidth]{img/gmic_stdlib361.jpg}\\
{\footnotesize \textbf{Example 361~:} \texttt{image.jpg 100\%{\comma}100\%{\comma}1{\comma}2 rand[-1] -100{\comma}100 repeat 2 smooth[-1] 100{\comma}0.2{\comma}1{\comma}4{\comma}4 done --warp[0] [-1]{\comma}1{\comma}1}}
\end{center}
~\\
~\textbf{Tutorial page: }\\\url{http://gmic.eu/tutorial/\_smooth.shtml}


\subsection{\emph{split\_freq\index{split\_freq}} }\vspace*{-0.7em}
~\\\textbf{\Cb{Arguments: }}\begin{flushleft}
{\small \Cb{\hspace*{0.5cm}$\bullet$~~\texttt{smoothness$>$0[\%]}}}\end{flushleft}
Split selected images into low and high frequency parts.
\begin{center}\includegraphics[keepaspectratio=true,height=6cm,width=\textwidth]{img/gmic_stdlib362.jpg}\\
{\footnotesize \textbf{Example 362~:} \texttt{image.jpg split\_freq 2\%}}
\end{center}

\subsection{\emph{solve\_poisson\index{solve\_poisson}} }\vspace*{-0.7em}
~\\\textbf{\Cb{Arguments: }}\begin{flushleft}
{\small \Cb{\hspace*{0.5cm}$\bullet$~~\texttt{"laplacian\_command"{\comma}\_nb\_iterations$>$=0{\comma}\_time\_step$>$0{\comma}\_nb\_scale\-s$>$=0}}}\end{flushleft}
Solve Poisson equation so that applying '-laplacian[n]' is close to the result of '-laplacian\_command[n]'.
~\\Solving is performed using a multi-scale gradient descent algorithm.
~\\If 'nb\_scales=0'{\comma} the number of scales is automatically determined.
\begin{flushleft}\Cc{\textbf{Default values}:\\~\\\hspace*{0.5cm}{\small $\bullet$~~\texttt{'nb\_iterations=60'{\comma} 'dt=5'} and \texttt{'nb\_scales=0'.}}}\end{flushleft}
\begin{center}\includegraphics[keepaspectratio=true,height=6cm,width=\textwidth]{img/gmic_stdlib363.jpg}\\
{\footnotesize \textbf{Example 363~:} \texttt{image.jpg command "foo : gradient x" --solve\_poisson foo --foo[0] --laplacian[1]}}
\end{center}

\subsection{\emph{split\_details\index{split\_details}} }\vspace*{-0.7em}
~\\\textbf{\Cb{Arguments: }}\begin{flushleft}
{\small \Cb{\hspace*{0.5cm}$\bullet$~~\texttt{\_nb\_scales$>$0{\comma}\_base\_scale[\%]$>$=0{\comma}\_detail\_scale[\%]$>$=0}}}\end{flushleft}
Split selected images into 'nb\_scales' detail scales.
~\\If 'base\_scale'=='detail\_scale'==0{\comma} the image decomposition is done with 'a trous' wavelets.
~\\Otherwise{\comma} it uses laplacian pyramids with linear standard deviations.
\begin{flushleft}\Cc{\textbf{Default values}:\\~\\\hspace*{0.5cm}{\small $\bullet$~~\texttt{'nb\_scales=4'{\comma} 'base\_scale=0'} and \texttt{'detail\_scale=0'.}}}\end{flushleft}
\begin{center}\includegraphics[keepaspectratio=true,height=6cm,width=\textwidth]{img/gmic_stdlib364.jpg}\\
{\footnotesize \textbf{Example 364~:} \texttt{image.jpg split\_details {\comma}}}
\end{center}

\subsection{\emph{structuretensors\index{structuretensors}} (+)}\vspace*{-0.7em}
~\\\textbf{\Cb{Arguments: }}\begin{flushleft}
{\small \Cb{\hspace*{0.5cm}$\bullet$~~\texttt{\_scheme=\{ 0=centered ~$|$~ 1=forward/backward \}}}}\end{flushleft}
Compute the structure tensor field of selected images.
\begin{flushleft}\Cc{\textbf{Default value}:\\~\\\hspace*{0.5cm}{\small $\bullet$~~\texttt{'scheme=1'.}}}\end{flushleft}
\begin{center}\includegraphics[keepaspectratio=true,height=6cm,width=\textwidth]{img/gmic_stdlib365.jpg}\\
{\footnotesize \textbf{Example 365~:} \texttt{image.jpg structuretensors abs pow 0.2}}
\end{center}
~\\
~\textbf{Tutorial page: }\\\url{http://gmic.eu/tutorial/\_structuretensors.shtml}


\subsection{\emph{solidify\index{solidify}} }\vspace*{-0.7em}
~\\\textbf{\Cb{Arguments: }}\begin{flushleft}
{\small \Cb{\hspace*{0.5cm}$\bullet$~~\texttt{\_smoothness[\%]$>$=0{\comma}\_diffusion\_type=\{ 0=isotropic ~$|$~ 1=delaunay\--oriented ~$|$~ 2=edge-oriented \}{\comma}\_diffusion\_iter$>$=0}}}\end{flushleft}
Solidify selected transparent images.
\begin{flushleft}\Cc{\textbf{Default values}:\\~\\\hspace*{0.5cm}{\small $\bullet$~~\texttt{'smoothness=75\%'{\comma} 'diffusion\_type=1'} and \texttt{'diffusion\_iter=20'.}}}\end{flushleft}
\begin{center}\includegraphics[keepaspectratio=true,height=6cm,width=\textwidth]{img/gmic_stdlib366.jpg}\\
{\footnotesize \textbf{Example 366~:} \texttt{image.jpg 100\%{\comma}100\% circle[-1] 50\%{\comma}50\%{\comma}25\%{\comma}1{\comma}255 append c --solidify {\comma} display\_rgba}}
\end{center}

\subsection{\emph{syntexturize\index{syntexturize}} }\vspace*{-0.7em}
~\\\textbf{\Cb{Arguments: }}\begin{flushleft}
{\small \Cb{\hspace*{0.5cm}$\bullet$~~\texttt{\_width[\%]$>$0{\comma}\_height[\%]$>$0}}}\end{flushleft}
Resynthetize 'width'x'height' versions of selected micro-textures by phase randomization.
~\\The texture synthesis algorithm is a straightforward implementation of the method described in :
http://www.ipol.im/pub/art/2011/ggm\_rpn/
\begin{flushleft}\Cc{\textbf{Default values}:\\~\\\hspace*{0.5cm}{\small $\bullet$~~\texttt{'width=height=100\%'.}}}\end{flushleft}
\begin{center}\includegraphics[keepaspectratio=true,height=6cm,width=\textwidth]{img/gmic_stdlib367.jpg}\\
{\footnotesize \textbf{Example 367~:} \texttt{image.jpg crop 2{\comma}282{\comma}50{\comma}328 --syntexturize 320{\comma}320}}
\end{center}

\subsection{\emph{syntexturize\_patchmatch\index{syntexturize\_patchmatch}} }\vspace*{-0.7em}
~\\\textbf{\Cb{Arguments: }}\begin{flushleft}
{\small \Cb{\hspace*{0.5cm}$\bullet$~~\texttt{\_width[\%]$>$0{\comma}\_height[\%]$>$0{\comma}\_nb\_scales$>$=0{\comma}\_patch\_size$>$0{\comma}\_blendi\-ng\_size$>$=0{\comma}\_precision$>$=0}}}\end{flushleft}
Resynthetize 'width'x'height' versions of selected micro-textures using a patch-matching algorithm.
~\\If 'nbscales==0'{\comma} the number of scales used is estimated from the image size.
\begin{flushleft}\Cc{\textbf{Default values}:\\~\\\hspace*{0.5cm}{\small $\bullet$~~\texttt{'width=height=100\%'{\comma} 'nb\_scales=0'{\comma} 'patch\_size=7'{\comma} 'blending\_size=5'} and \texttt{'precision=1'.}}}\end{flushleft}
\begin{center}\includegraphics[keepaspectratio=true,height=6cm,width=\textwidth]{img/gmic_stdlib368.jpg}\\
{\footnotesize \textbf{Example 368~:} \texttt{image.jpg crop 2{\comma}282{\comma}50{\comma}328 --syntexturize\_patchmatch 320{\comma}320}}
\end{center}

\subsection{\emph{tv\_flow\index{tv\_flow}} }\vspace*{-0.7em}
~\\\textbf{\Cb{Arguments: }}\begin{flushleft}
{\small \Cb{\hspace*{0.5cm}$\bullet$~~\texttt{\_nb\_iter$>$=0{\comma}\_dt{\comma}\_keep\_sequence=\{ 0 ~$|$~ 1 \}}}}\end{flushleft}
Apply iterations of the total variation flow on selected images.
\begin{flushleft}\Cc{\textbf{Default values}:\\~\\\hspace*{0.5cm}{\small $\bullet$~~\texttt{'nb\_iter=10'{\comma} 'dt=30'} and \texttt{'keep\_sequence=0'.}}}\end{flushleft}
\begin{center}\includegraphics[keepaspectratio=true,height=6cm,width=\textwidth]{img/gmic_stdlib369.jpg}\\
{\footnotesize \textbf{Example 369~:} \texttt{image.jpg --tv\_flow 40}}
\end{center}

\subsection{\emph{unsharp\index{unsharp}} }\vspace*{-0.7em}
~\\\textbf{\Cb{Arguments: }}\begin{flushleft}
{\small \Cb{\hspace*{0.5cm}$\bullet$~~\texttt{radius[\%]$>$=0{\comma}\_amount$>$=0{\comma}\_threshold[\%]$>$=0}}}\end{flushleft}
Apply unsharp mask on selected images.
\begin{flushleft}\Cc{\textbf{Default values}:\\~\\\hspace*{0.5cm}{\small $\bullet$~~\texttt{'amount=2'} and \texttt{'threshold=0'.}}}\end{flushleft}
\begin{center}\includegraphics[keepaspectratio=true,height=6cm,width=\textwidth]{img/gmic_stdlib370.jpg}\\
{\footnotesize \textbf{Example 370~:} \texttt{image.jpg blur 3 --unsharp 1.5{\comma}15 cut 0{\comma}255}}
\end{center}

\subsection{\emph{unsharp\_octave\index{unsharp\_octave}} }\vspace*{-0.7em}
~\\\textbf{\Cb{Arguments: }}\begin{flushleft}
{\small \Cb{\hspace*{0.5cm}$\bullet$~~\texttt{\_nb\_scales$>$0{\comma}\_radius[\%]$>$=0{\comma}\_amount$>$=0{\comma}threshold[\%]$>$=0}}}\end{flushleft}
Apply octave sharpening on selected images.
\begin{flushleft}\Cc{\textbf{Default values}:\\~\\\hspace*{0.5cm}{\small $\bullet$~~\texttt{'nb\_scales=4'{\comma} 'radius=1'{\comma} 'amount=2'} and \texttt{'threshold=0'.}}}\end{flushleft}
\begin{center}\includegraphics[keepaspectratio=true,height=6cm,width=\textwidth]{img/gmic_stdlib371.jpg}\\
{\footnotesize \textbf{Example 371~:} \texttt{image.jpg blur 3 --unsharp\_octave 4{\comma}5{\comma}15 cut 0{\comma}255}}
\end{center}

\subsection{\emph{vanvliet\index{vanvliet}} (+)}\vspace*{-0.7em}
~\\\textbf{\Cb{Arguments: }}\begin{flushleft}
{\small \Cb{\hspace*{0.5cm}$\bullet$~~\texttt{std\_variation$>$=0[\%]{\comma}order=\{ 0 ~$|$~ 1 ~$|$~ 2 ~$|$~ 3 \}{\comma}axis=\{ x ~$|$~ y ~$|$~ z\- ~$|$~ c \}{\comma}\_boundary\_conditions}}}\end{flushleft}
Apply Vanvliet recursive filter on selected images{\comma} along specified axis and with
specified standard deviation{\comma} order and boundary conditions.
~\\'boundary\_conditions' can be \{ 0=dirichlet ~$|$~ 1=neumann \}.
\begin{flushleft}\Cc{\textbf{Default value}:\\~\\\hspace*{0.5cm}{\small $\bullet$~~\texttt{'boundary\_conditions=1'.}}}\end{flushleft}
\begin{center}\includegraphics[keepaspectratio=true,height=6cm,width=\textwidth]{img/gmic_stdlib372.jpg}\\
{\footnotesize \textbf{Example 372~:} \texttt{image.jpg --vanvliet 3{\comma}1{\comma}x}}
\\\includegraphics[keepaspectratio=true,height=6cm,width=\textwidth]{img/gmic_stdlib373.jpg}\\
{\footnotesize \textbf{Example 373~:} \texttt{image.jpg --vanvliet 30{\comma}0{\comma}x vanvliet[-2] 30{\comma}0{\comma}y add}}
\end{center}

\subsection{\emph{watermark\_fourier\index{watermark\_fourier}} }\vspace*{-0.7em}
~\\\textbf{\Cb{Arguments: }}\begin{flushleft}
{\small \Cb{\hspace*{0.5cm}$\bullet$~~\texttt{text{\comma}\_size$>$0}}}\end{flushleft}
Add a textual watermark in the frequency domain of selected images.
\begin{flushleft}\Cc{\textbf{Default value}:\\~\\\hspace*{0.5cm}{\small $\bullet$~~\texttt{'size=33'.}}}\end{flushleft}
\begin{center}\includegraphics[keepaspectratio=true,height=6cm,width=\textwidth]{img/gmic_stdlib374.jpg}\\
{\footnotesize \textbf{Example 374~:} \texttt{image.jpg --watermark\_fourier "Watermarked!" --display\_fft remove[-3{\comma}-1] normalize 0{\comma}255 append[-4{\comma}-2] y append[-2{\comma}-1] y}}
\end{center}

\subsection{\emph{watershed\index{watershed}} (+)}\vspace*{-0.7em}
~\\\textbf{\Cb{Arguments: }}\begin{flushleft}
{\small \Cb{\hspace*{0.5cm}$\bullet$~~\texttt{[priority\_image]{\comma}\_is\_high\_connectivity=\{ 0 ~$|$~ 1 \}}}}\end{flushleft}
Compute the watershed transform of selected images.
\begin{flushleft}\Cc{\textbf{Default value}:\\~\\\hspace*{0.5cm}{\small $\bullet$~~\texttt{'is\_high\_connectivity=1'.}}}\end{flushleft}
\begin{center}\includegraphics[keepaspectratio=true,height=6cm,width=\textwidth]{img/gmic_stdlib375.jpg}\\
{\footnotesize \textbf{Example 375~:} \texttt{400{\comma}400 noise 0.2{\comma}2 --distance 1 mul[-1] -1 label[-2] watershed[-2] [-1] mod[-2] 256 map[-2] 0 reverse}}
\end{center}
\section{Features extraction}


\subsection{\emph{area\index{area}} }\vspace*{-0.7em}
~\\\textbf{\Cb{Arguments: }}\begin{flushleft}
{\small \Cb{\hspace*{0.5cm}$\bullet$~~\texttt{tolerance$>$=0{\comma}is\_high\_connectivity=\{ 0 ~$|$~ 1 \}}}}\end{flushleft}
Compute area of connected components in selected images.
\begin{flushleft}\Cc{\textbf{Default values}:\\~\\\hspace*{0.5cm}{\small $\bullet$~~\texttt{'is\_high\_connectivity=0'.}}}\end{flushleft}
\begin{center}\includegraphics[keepaspectratio=true,height=6cm,width=\textwidth]{img/gmic_stdlib376.jpg}\\
{\footnotesize \textbf{Example 376~:} \texttt{image.jpg luminance stencil[-1] 1 --area 0}}
\end{center}
~\\
~\textbf{Tutorial page: }\\\url{http://gmic.eu/tutorial/\_area.shtml}


\subsection{\emph{area\_fg\index{area\_fg}} }\vspace*{-0.7em}
~\\\textbf{\Cb{Arguments: }}\begin{flushleft}
{\small \Cb{\hspace*{0.5cm}$\bullet$~~\texttt{tolerance$>$=0{\comma}is\_high\_connectivity=\{ 0 ~$|$~ 1 \}}}}\end{flushleft}
Compute area of connected components for non-zero values in selected images.
~\\Similar to '-area' except that 0-valued pixels are not considered.
\begin{flushleft}\Cc{\textbf{Default values}:\\~\\\hspace*{0.5cm}{\small $\bullet$~~\texttt{'is\_high\_connectivity=0'.}}}\end{flushleft}
\begin{center}\includegraphics[keepaspectratio=true,height=6cm,width=\textwidth]{img/gmic_stdlib377.jpg}\\
{\footnotesize \textbf{Example 377~:} \texttt{image.jpg luminance stencil[-1] 1 --area\_fg 0}}
\end{center}

\subsection{\emph{at\_line\index{at\_line}} }\vspace*{-0.7em}
~\\\textbf{\Cb{Arguments: }}\begin{flushleft}
{\small \Cb{\hspace*{0.5cm}$\bullet$~~\texttt{x0[\%]{\comma}y0[\%]{\comma}z0[\%]{\comma}x1[\%]{\comma}y1[\%]{\comma}z1[\%]}}}\end{flushleft}
Retrieve pixels of the selected images belonging to the specified line (x0{\comma}y0{\comma}z0)-(x1{\comma}y1{\comma}z1).
\begin{center}\includegraphics[keepaspectratio=true,height=6cm,width=\textwidth]{img/gmic_stdlib378.jpg}\\
{\footnotesize \textbf{Example 378~:} \texttt{image.jpg --at\_line 0{\comma}0{\comma}0{\comma}100\%{\comma}100\%{\comma}0}}
\end{center}

\subsection{\emph{at\_quadrangle\index{at\_quadrangle}} }\vspace*{-0.7em}
~\\\textbf{\Cb{Arguments: }}\begin{flushleft}
{\small \Cb{\hspace*{0.5cm}$\bullet$~~\texttt{x0[\%]{\comma}y0[\%]{\comma}x1[\%]{\comma}y1[\%]{\comma}x2[\%]{\comma}y2[\%]{\comma}x3[\%]{\comma}y3[\%]{\comma}\_interpolati\-on{\comma}\_boundary\_conditions}}}~~~\\
{\small \Cb{\hspace*{0.5cm}$\bullet$~~\texttt{x0[\%]{\comma}y0[\%]{\comma}z0[\%]{\comma}x1[\%]{\comma}y1[\%]{\comma}z1[\%]{\comma}x2[\%]{\comma}y2[\%]{\comma}z2[\%]{\comma}x3[\%]{\comma}\-y3[\%]{\comma}z3[\%]{\comma}\_interpolation{\comma}\_boundary\_conditions}}}\end{flushleft}
Retrieve pixels of the selected images belonging to the specified 2d or 3d quadrangle.
~\\'interpolation' can be \{ 0=nearest-neighbor ~$|$~ 1=linear ~$|$~ 2=cubic \}.
~\\'boundary\_conditions' can be \{ 0=dirichlet ~$|$~ 1=neumann ~$|$~ 2=periodic ~$|$~ 3=mirror \}.
\begin{center}\includegraphics[keepaspectratio=true,height=6cm,width=\textwidth]{img/gmic_stdlib379.jpg}\\
{\footnotesize \textbf{Example 379~:} \texttt{image.jpg params=5\%{\comma}5\%{\comma}95\%{\comma}5\%{\comma}60\%{\comma}95\%{\comma}40\%{\comma}95\% --at\_quadrangle \$params polygon.. 4{\comma}\$params{\comma}0.5{\comma}255}}
\end{center}

\subsection{\emph{barycenter\index{barycenter}} }\vspace*{-0.7em}
Compute the barycenter vector of pixel values.
\begin{center}\includegraphics[keepaspectratio=true,height=6cm,width=\textwidth]{img/gmic_stdlib380.jpg}\\
{\footnotesize \textbf{Example 380~:} \texttt{256{\comma}256 ellipse 50\%{\comma}50\%{\comma}20\%{\comma}20\%{\comma}0{\comma}1{\comma}1 deform 20 --barycenter --ellipse[-2] \{@0{\comma}1\}{\comma}5{\comma}5{\comma}0{\comma}10}}
\end{center}

\subsection{\emph{detect\_skin\index{detect\_skin}} }\vspace*{-0.7em}
~\\\textbf{\Cb{Arguments: }}\begin{flushleft}
{\small \Cb{\hspace*{0.5cm}$\bullet$~~\texttt{0$<$=tolerance$<$=1{\comma}\_skin\_x{\comma}\_skin\_y{\comma}\_skin\_radius$>$=0}}}\end{flushleft}
Detect skin in selected color images and output an appartenance probability map.
~\\Detection is performed using CbCr chromaticity data of skin pixels.
~\\If arguments 'skin\_x'{\comma} 'skin\_y' and 'skin\_radius' are provided{\comma} skin pixels are learnt
from the sample pixels inside the circle located at ('skin\_x'{\comma}'skin\_y') with radius 'skin\_radius'.
\begin{flushleft}\Cc{\textbf{Default value}:\\~\\\hspace*{0.5cm}{\small $\bullet$~~\texttt{'tolerance=0.5'} and \texttt{'skin\_x=skiny=radius=-1'.}}}\end{flushleft}


\subsection{\emph{displacement\index{displacement}} (+)}\vspace*{-0.7em}
~\\\textbf{\Cb{Arguments: }}\begin{flushleft}
{\small \Cb{\hspace*{0.5cm}$\bullet$~~\texttt{[source\_image]{\comma}\_smoothness{\comma}\_precision$>$=0{\comma}\_nb\_scales$>$=0{\comma}\_iter\-ation\_max$>$=0{\comma}is\_backward=\{ 0 ~$|$~ 1 \}{\comma}\_[guide]}}}\end{flushleft}
Estimate displacement field between specified source and selected target images.
~\\If 'smoothness$>$=0'{\comma} regularization type is set to isotropic{\comma} else to anisotropic.
~\\If 'nbscales==0'{\comma} the number of scales used is estimated from the image size.
\begin{flushleft}\Cc{\textbf{Default values}:\\~\\\hspace*{0.5cm}{\small $\bullet$~~\texttt{'smoothness=0.1'{\comma} 'precision=5'{\comma} 'nb\_scales=0'{\comma} 'iteration\_max=10000'{\comma} 'is\_backward=1'} and \texttt{'[guide]=(unused)'.}}}\end{flushleft}
\begin{center}\includegraphics[keepaspectratio=true,height=6cm,width=\textwidth]{img/gmic_stdlib381.jpg}\\
{\footnotesize \textbf{Example 381~:} \texttt{image.jpg --rotate 3{\comma}1{\comma}0{\comma}50\%{\comma}50\% --displacement[-1] [-2] quiver[-1] [-1]{\comma}15{\comma}1{\comma}1{\comma}1{\comma}\{1.5*iM\}}}
\end{center}

\subsection{\emph{distance\index{distance}} (+)}\vspace*{-0.7em}
~\\\textbf{\Cb{Arguments: }}\begin{flushleft}
{\small \Cb{\hspace*{0.5cm}$\bullet$~~\texttt{isovalue[\%]{\comma}\_metric}}}~~~\\
{\small \Cb{\hspace*{0.5cm}$\bullet$~~\texttt{isovalue[\%]{\comma}[metric]{\comma}\_method}}}\end{flushleft}
Compute the unsigned distance function to specified isovalue{\comma} opt. according to a custom metric.
~\\'metric' can be \{ 0=chebyshev ~$|$~ 1=manhattan ~$|$~ 2=euclidean ~$|$~ 3=squared-euclidean \}.
~\\'method' can be \{ 0=fast-marching ~$|$~ 1=low-connectivity dijkstra ~$|$~ 2=high-connectivity dijkstra ~$|$~ 3=1+return path ~$|$~ 4=2+return path \}.
\begin{flushleft}\Cc{\textbf{Default value}:\\~\\\hspace*{0.5cm}{\small $\bullet$~~\texttt{'metric=2'} and \texttt{'method=0'.}}}\end{flushleft}
\begin{center}\includegraphics[keepaspectratio=true,height=6cm,width=\textwidth]{img/gmic_stdlib382.jpg}\\
{\footnotesize \textbf{Example 382~:} \texttt{image.jpg threshold 20\% distance 0 pow 0.3}}
\\\includegraphics[keepaspectratio=true,height=6cm,width=\textwidth]{img/gmic_stdlib383.jpg}\\
{\footnotesize \textbf{Example 383~:} \texttt{400{\comma}400 set 1{\comma}50\%{\comma}50\% --distance[0] 1{\comma}2 --distance[0] 1{\comma}1 distance[0] 1{\comma}0 mod 32 threshold 16 append c}}
\end{center}
~\\
~\textbf{Tutorial page: }\\\url{http://gmic.eu/tutorial/\_distance.shtml}


\subsection{\emph{float2fft8\index{float2fft8}} }\vspace*{-0.7em}
Convert selected float-valued images to 8bits fourier representations.


\subsection{\emph{fft82float\index{fft82float}} }\vspace*{-0.7em}
Convert selected 8bits fourier representations to float-valued images.


\subsection{\emph{fftpolar\index{fftpolar}} }\vspace*{-0.7em}
Compute fourier transform of selected images{\comma} as centered magnitude/phase images.
\begin{center}\includegraphics[keepaspectratio=true,height=6cm,width=\textwidth]{img/gmic_stdlib384.jpg}\\
{\footnotesize \textbf{Example 384~:} \texttt{image.jpg fftpolar ellipse 50\%{\comma}50\%{\comma}10{\comma}10{\comma}0{\comma}1{\comma}0 ifftpolar}}
\end{center}

\subsection{\emph{histogram\index{histogram}} (+)}\vspace*{-0.7em}
~\\\textbf{\Cb{Arguments: }}\begin{flushleft}
{\small \Cb{\hspace*{0.5cm}$\bullet$~~\texttt{\_nb\_levels$>$0[\%]{\comma}\_value0[\%]{\comma}\_value1[\%]}}}\end{flushleft}
Compute the histogram of selected images.
~\\If value range is set{\comma} the histogram is estimated only for pixels in the specified
value range. Argument 'value1' must be specified if 'value0' is set.
\begin{flushleft}\Cc{\textbf{Default values}:\\~\\\hspace*{0.5cm}{\small $\bullet$~~\texttt{'nb\_levels=256'{\comma} 'value0=0\%'} and \texttt{'value1=100\%'.}}}\end{flushleft}
\begin{center}\includegraphics[keepaspectratio=true,height=6cm,width=\textwidth]{img/gmic_stdlib385.jpg}\\
{\footnotesize \textbf{Example 385~:} \texttt{image.jpg --histogram 64 display\_graph[-1] 400{\comma}300{\comma}3}}
\end{center}

\subsection{\emph{histogram\_nd\index{histogram\_nd}} }\vspace*{-0.7em}
~\\\textbf{\Cb{Arguments: }}\begin{flushleft}
{\small \Cb{\hspace*{0.5cm}$\bullet$~~\texttt{nb\_levels$>$0[\%]{\comma}\_value0[\%]{\comma}\_value1[\%]}}}\end{flushleft}
Compute the 1d{\comma}2d or 3d histogram of selected multi-channels images (having 1{\comma}2 or 3 channels).
~\\If value range is set{\comma} the histogram is estimated only for pixels in the specified
value range.
\begin{flushleft}\Cc{\textbf{Default values}:\\~\\\hspace*{0.5cm}{\small $\bullet$~~\texttt{'value0=0\%'} and \texttt{'value1=100\%'.}}}\end{flushleft}
\begin{center}\includegraphics[keepaspectratio=true,height=6cm,width=\textwidth]{img/gmic_stdlib386.jpg}\\
{\footnotesize \textbf{Example 386~:} \texttt{image.jpg channels 0{\comma}1 --histogram\_nd 256}}
\end{center}

\subsection{\emph{histogram\_cumul\index{histogram\_cumul}} }\vspace*{-0.7em}
~\\\textbf{\Cb{Arguments: }}\begin{flushleft}
{\small \Cb{\hspace*{0.5cm}$\bullet$~~\texttt{\_nb\_levels$>$0{\comma}\_is\_normalized=\{ 0 ~$|$~ 1 \}{\comma}\_val0[\%]{\comma}\_val1[\%]}}}\end{flushleft}
Compute cumulative histogram of selected images.
\begin{flushleft}\Cc{\textbf{Default values}:\\~\\\hspace*{0.5cm}{\small $\bullet$~~\texttt{'nb\_levels=256'{\comma} 'is\_normalized=0'} and \texttt{'val0=val1=0'.}}}\end{flushleft}
\begin{center}\includegraphics[keepaspectratio=true,height=6cm,width=\textwidth]{img/gmic_stdlib387.jpg}\\
{\footnotesize \textbf{Example 387~:} \texttt{image.jpg --histogram\_cumul 256 histogram[0] 256 display\_graph 400{\comma}300{\comma}3}}
\end{center}

\subsection{\emph{histogram\_pointwise\index{histogram\_pointwise}} }\vspace*{-0.7em}
~\\\textbf{\Cb{Arguments: }}\begin{flushleft}
{\small \Cb{\hspace*{0.5cm}$\bullet$~~\texttt{nb\_levels$>$0[\%]{\comma}\_value0[\%]{\comma}\_value1[\%]}}}\end{flushleft}
Compute the histogram of each vector-valued point of selected images.
~\\If value range is set{\comma} the histogram is estimated only for values in the specified
value range.
\begin{flushleft}\Cc{\textbf{Default values}:\\~\\\hspace*{0.5cm}{\small $\bullet$~~\texttt{'value0=0\%'} and \texttt{'value1=100\%'.}}}\end{flushleft}


\subsection{\emph{hough\index{hough}} }\vspace*{-0.7em}
~\\\textbf{\Cb{Arguments: }}\begin{flushleft}
{\small \Cb{\hspace*{0.5cm}$\bullet$~~\texttt{\_width$>$0{\comma}\_height$>$0{\comma}gradient\_norm\_voting=\{ 0 ~$|$~ 1 \}}}}\end{flushleft}
Compute hough transform (theta{\comma}rho) of selected images.
\begin{flushleft}\Cc{\textbf{Default values}:\\~\\\hspace*{0.5cm}{\small $\bullet$~~\texttt{'width=512'{\comma} 'height=width'} and \texttt{'gradient\_norm\_voting=1'.}}}\end{flushleft}
\begin{center}\includegraphics[keepaspectratio=true,height=6cm,width=\textwidth]{img/gmic_stdlib388.jpg}\\
{\footnotesize \textbf{Example 388~:} \texttt{image.jpg --blur 1.5 hough[-1] 400{\comma}400 blur[-1] 0.5 add[-1] 1 log[-1]}}
\end{center}

\subsection{\emph{ifftpolar\index{ifftpolar}} }\vspace*{-0.7em}
Compute inverse fourier transform of selected images{\comma} from centered magnitude/phase images.


\subsection{\emph{isophotes\index{isophotes}} }\vspace*{-0.7em}
~\\\textbf{\Cb{Arguments: }}\begin{flushleft}
{\small \Cb{\hspace*{0.5cm}$\bullet$~~\texttt{\_nb\_levels$>$0}}}\end{flushleft}
Render isophotes of selected images on a transparent background.
\begin{flushleft}\Cc{\textbf{Default value}:\\~\\\hspace*{0.5cm}{\small $\bullet$~~\texttt{'nb\_levels=64'}}}\end{flushleft}
\begin{center}\includegraphics[keepaspectratio=true,height=6cm,width=\textwidth]{img/gmic_stdlib389.jpg}\\
{\footnotesize \textbf{Example 389~:} \texttt{image.jpg blur 2 isophotes 6 dilate\_circ 5 display\_rgba}}
\end{center}

\subsection{\emph{label\index{label}} (+)}\vspace*{-0.7em}
~\\\textbf{\Cb{Arguments: }}\begin{flushleft}
{\small \Cb{\hspace*{0.5cm}$\bullet$~~\texttt{\_tolerance$>$=0{\comma}is\_high\_connectivity=\{ 0 ~$|$~ 1 \}}}}\end{flushleft}
Label connected components in selected images.
\begin{flushleft}\Cc{\textbf{Default values}:\\~\\\hspace*{0.5cm}{\small $\bullet$~~\texttt{'tolerance=0'} and \texttt{'is\_high\_connectivity=0'.}}}\end{flushleft}
\begin{center}\includegraphics[keepaspectratio=true,height=6cm,width=\textwidth]{img/gmic_stdlib390.jpg}\\
{\footnotesize \textbf{Example 390~:} \texttt{image.jpg luminance threshold 60\% label normalize 0{\comma}255 map 0}}
\\\includegraphics[keepaspectratio=true,height=6cm,width=\textwidth]{img/gmic_stdlib391.jpg}\\
{\footnotesize \textbf{Example 391~:} \texttt{400{\comma}400 set 1{\comma}50\%{\comma}50\% distance 1 mod 16 threshold 8 label mod 255 map 2}}
\end{center}
~\\
~\textbf{Tutorial page: }\\\url{http://gmic.eu/tutorial/\_label.shtml}


\subsection{\emph{label\_fg\index{label\_fg}} }\vspace*{-0.7em}
~\\\textbf{\Cb{Arguments: }}\begin{flushleft}
{\small \Cb{\hspace*{0.5cm}$\bullet$~~\texttt{tolerance$>$=0{\comma}is\_high\_connectivity=\{ 0 ~$|$~ 1 \}}}}\end{flushleft}
Label connected components for non-zero values (foreground) in selected images.
~\\Similar to '-label' except that 0-valued pixels are not labeled.
\begin{flushleft}\Cc{\textbf{Default value}:\\~\\\hspace*{0.5cm}{\small $\bullet$~~\texttt{'is\_high\_connectivity=0'.}}}\end{flushleft}


\subsection{\emph{max\_patch\index{max\_patch}} }\vspace*{-0.7em}
~\\\textbf{\Cb{Arguments: }}\begin{flushleft}
{\small \Cb{\hspace*{0.5cm}$\bullet$~~\texttt{\_patch\_size$>$=1}}}\end{flushleft}
Return locations of maximal values in local patch-based neighborhood of given size for selected images.
\begin{flushleft}\Cc{\textbf{Default value}:\\~\\\hspace*{0.5cm}{\small $\bullet$~~\texttt{'patch\_size=16'.}}}\end{flushleft}
\begin{center}\includegraphics[keepaspectratio=true,height=6cm,width=\textwidth]{img/gmic_stdlib392.jpg}\\
{\footnotesize \textbf{Example 392~:} \texttt{image.jpg norm --max\_patch 16}}
\end{center}

\subsection{\emph{min\_patch\index{min\_patch}} }\vspace*{-0.7em}
~\\\textbf{\Cb{Arguments: }}\begin{flushleft}
{\small \Cb{\hspace*{0.5cm}$\bullet$~~\texttt{\_patch\_size$>$=1}}}\end{flushleft}
Return locations of minimal values in local patch-based neighborhood of given size for selected images.
\begin{flushleft}\Cc{\textbf{Default value}:\\~\\\hspace*{0.5cm}{\small $\bullet$~~\texttt{'patch\_size=16'.}}}\end{flushleft}
\begin{center}\includegraphics[keepaspectratio=true,height=6cm,width=\textwidth]{img/gmic_stdlib393.jpg}\\
{\footnotesize \textbf{Example 393~:} \texttt{image.jpg norm --min\_patch 16}}
\end{center}

\subsection{\emph{minimal\_path\index{minimal\_path}} }\vspace*{-0.7em}
~\\\textbf{\Cb{Arguments: }}\begin{flushleft}
{\small \Cb{\hspace*{0.5cm}$\bullet$~~\texttt{x0[\%]$>$=0{\comma}y0[\%]$>$=0{\comma}z0[\%]$>$=0{\comma}x1[\%]$>$=0{\comma}y1[\%]$>$=0{\comma}z1[\%]$>$=0{\comma}\_is\_hi\-gh\_connectivity=\{ 0 ~$|$~ 1 \}}}}\end{flushleft}
Compute minimal path between two points on selected potential maps.
\begin{flushleft}\Cc{\textbf{Default value}:\\~\\\hspace*{0.5cm}{\small $\bullet$~~\texttt{'is\_high\_connectivity=0'.}}}\end{flushleft}
\begin{center}\includegraphics[keepaspectratio=true,height=6cm,width=\textwidth]{img/gmic_stdlib394.jpg}\\
{\footnotesize \textbf{Example 394~:} \texttt{image.jpg --gradient\_norm fill[-1] 1/(1+i) minimal\_path[-1] 0{\comma}0{\comma}0{\comma}100\%{\comma}100\%{\comma}0 pointcloud[-1] 0 *[-1] 280 to\_rgb[-1] resize[-1] [-2]{\comma}0 or}}
\end{center}

\subsection{\emph{mse\index{mse}} (+)}\vspace*{-0.7em}
Compute MSE (Mean-Squared Error) matrix between selected images.
\begin{center}\includegraphics[keepaspectratio=true,height=6cm,width=\textwidth]{img/gmic_stdlib395.jpg}\\
{\footnotesize \textbf{Example 395~:} \texttt{image.jpg --noise 30 --noise[0] 35 --noise[0] 38 cut. 0{\comma}255 -mse}}
\end{center}

\subsection{\emph{patches\index{patches}} }\vspace*{-0.7em}
~\\\textbf{\Cb{Arguments: }}\begin{flushleft}
{\small \Cb{\hspace*{0.5cm}$\bullet$~~\texttt{patch\_width$>$0{\comma}patch\_height$>$0{\comma}patch\_depth$>$0{\comma}x0{\comma}y0{\comma}z0{\comma}\_x1{\comma}\_y1{\comma}\-\_z1{\comma}...{\comma}\_xN{\comma}\_yN{\comma}\_zN}}}\end{flushleft}
Extract N+1 patches from selected images{\comma} centered at specified locations.
\begin{center}\includegraphics[keepaspectratio=true,height=6cm,width=\textwidth]{img/gmic_stdlib396.jpg}\\
{\footnotesize \textbf{Example 396~:} \texttt{image.jpg --patches 64{\comma}64{\comma}1{\comma}153{\comma}124{\comma}0{\comma}184{\comma}240{\comma}0{\comma}217{\comma}126{\comma}0{\comma}275{\comma}38{\comma}0}}
\end{center}

\subsection{\emph{patchmatch\index{patchmatch}} (+)}\vspace*{-0.7em}
~\\\textbf{\Cb{Arguments: }}\begin{flushleft}
{\small \Cb{\hspace*{0.5cm}$\bullet$~~\texttt{[patch\_image]{\comma}patch\_width$>$=1{\comma}\_patch\_height$>$=1{\comma}\_patch\_depth$>$=\-1{\comma}\_nb\_iterations$>$=0{\comma}\_nb\_randoms$>$=0{\comma}\_output\_score=\{ 0 ~$|$~ 1 \}{\comma}\_\-[guide]}}}\end{flushleft}
Estimate correspondence map between selected images and specified patch image{\comma} using
the patchmatch algorithm{\comma} as described in the paper :
"PatchMatch: A Randomized Correspondence Algorithm for Structural Image Editing"{\comma} by
~\\Connelly Barnes{\comma} Eli Shechtman{\comma} Adam Finkelstein{\comma} Dan B Goldman(2009).
~\\Each pixel of the returned correspondence map gives the location (p{\comma}q) of the closest patch in
the specified patch image. If 'output\_score=1'{\comma} the third channel gives the corresponding
matching score for each patch as well.
\begin{flushleft}\Cc{\textbf{Default values}:\\~\\\hspace*{0.5cm}{\small $\bullet$~~\texttt{'patch\_height=patch\_width'{\comma} 'patch\_depth=1'{\comma} 'nb\_iterations=5'{\comma} 'nb\_randoms=5'{\comma} 'output\_score=0'} and \texttt{'guide=(undefined)'.}}}\end{flushleft}
\begin{center}\includegraphics[keepaspectratio=true,height=6cm,width=\textwidth]{img/gmic_stdlib397.jpg}\\
{\footnotesize \textbf{Example 397~:} \texttt{image.jpg sample ? to\_rgb --patchmatch[0] [1]{\comma}3 --warp[-2] [-1]{\comma}0}}
\end{center}

\subsection{\emph{plot2value\index{plot2value}} }\vspace*{-0.7em}
Retrieve values from selected 2d graph plots.
\begin{center}\includegraphics[keepaspectratio=true,height=6cm,width=\textwidth]{img/gmic_stdlib398.jpg}\\
{\footnotesize \textbf{Example 398~:} \texttt{400{\comma}300{\comma}1{\comma}1{\comma}'if(y$>$300*abs(cos(x/10+2*u)){\comma}1{\comma}0)' --plot2value --display\_graph[-1] 400{\comma}300}}
\end{center}

\subsection{\emph{pointcloud\index{pointcloud}} }\vspace*{-0.7em}
~\\\textbf{\Cb{Arguments: }}\begin{flushleft}
{\small \Cb{\hspace*{0.5cm}$\bullet$~~\texttt{\_type = \{ -X=-X-opacity ~$|$~ 0=binary ~$|$~ 1=cumulative ~$|$~ 2=label \-\}{\comma}\_width{\comma}\_height$>$0{\comma}\_depth$>$0}}}\end{flushleft}
Convert a Nx1{\comma} Nx2{\comma} Nx3 or NxM image as a point cloud in a 1d/2d or 3d binary image.
~\\If 'M'$>$3{\comma} the 3-to-M lines sets the (M-3)-dimensional color at each point.
~\\Parameters 'width'{\comma}'height' and 'depth' are related to the size of the final image :
- If set to 0{\comma} the size is automatically set along the specified axis.
- If set to N$>$0{\comma} the size along the specified axis is N.
- If set to N$<$0{\comma} the size along the specified axis is at most N.
~\\Points with coordinates that are negative or higher than specified ('width'{\comma}'height'{\comma}'depth')
are not plotted.
\begin{flushleft}\Cc{\textbf{Default values}:\\~\\\hspace*{0.5cm}{\small $\bullet$~~\texttt{'type=0'} and \texttt{'max\_width=max\_height=max\_depth=0'.}}}\end{flushleft}
\begin{center}\includegraphics[keepaspectratio=true,height=6cm,width=\textwidth]{img/gmic_stdlib399.jpg}\\
{\footnotesize \textbf{Example 399~:} \texttt{3000{\comma}2 rand 0{\comma}400 --pointcloud 0 dilate[-1] 3}}
\\\includegraphics[keepaspectratio=true,height=6cm,width=\textwidth]{img/gmic_stdlib400.jpg}\\
{\footnotesize \textbf{Example 400~:} \texttt{3000{\comma}2 rand 0{\comma}400 \{w\} \{w\}{\comma}3 rand[-1] 0{\comma}255 append y --pointcloud 0 dilate[-1] 3}}
\end{center}

\subsection{\emph{psnr\index{psnr}} }\vspace*{-0.7em}
~\\\textbf{\Cb{Arguments: }}\begin{flushleft}
{\small \Cb{\hspace*{0.5cm}$\bullet$~~\texttt{\_max\_value}}}\end{flushleft}
Compute PSNR (Peak Signal-to-Noise Ratio) matrix between selected images.
\begin{flushleft}\Cc{\textbf{Default value}:\\~\\\hspace*{0.5cm}{\small $\bullet$~~\texttt{'max\_value=255'.}}}\end{flushleft}
\begin{center}\includegraphics[keepaspectratio=true,height=6cm,width=\textwidth]{img/gmic_stdlib401.jpg}\\
{\footnotesize \textbf{Example 401~:} \texttt{image.jpg --noise 30 --noise[0] 35 --noise[0] 38 cut[-1] 0{\comma}255 psnr 255 replace\_inf 0}}
\end{center}

\subsection{\emph{segment\_watershed\index{segment\_watershed}} }\vspace*{-0.7em}
~\\\textbf{\Cb{Arguments: }}\begin{flushleft}
{\small \Cb{\hspace*{0.5cm}$\bullet$~~\texttt{\_threshold$>$=0}}}\end{flushleft}
Apply watershed segmentation on selected images.
\begin{flushleft}\Cc{\textbf{Default values}:\\~\\\hspace*{0.5cm}{\small $\bullet$~~\texttt{'threshold=2'.}}}\end{flushleft}
\begin{center}\includegraphics[keepaspectratio=true,height=6cm,width=\textwidth]{img/gmic_stdlib402.jpg}\\
{\footnotesize \textbf{Example 402~:} \texttt{image.jpg --segment\_watershed 2}}
\end{center}

\subsection{\emph{skeleton\index{skeleton}} }\vspace*{-0.7em}
~\\\textbf{\Cb{Arguments: }}\begin{flushleft}
{\small \Cb{\hspace*{0.5cm}$\bullet$~~\texttt{\_smoothness[\%]$>$=0}}}\end{flushleft}
Compute skeleton of binary shapes using distance transform.
\begin{flushleft}\Cc{\textbf{Default value}:\\~\\\hspace*{0.5cm}{\small $\bullet$~~\texttt{'smoothness=0'.}}}\end{flushleft}
\begin{center}\includegraphics[keepaspectratio=true,height=6cm,width=\textwidth]{img/gmic_stdlib403.jpg}\\
{\footnotesize \textbf{Example 403~:} \texttt{image.jpg threshold 50\% --skeleton 0}}
\end{center}

\subsection{\emph{ssd\_patch\index{ssd\_patch}} }\vspace*{-0.7em}
~\\\textbf{\Cb{Arguments: }}\begin{flushleft}
{\small \Cb{\hspace*{0.5cm}$\bullet$~~\texttt{[patch]{\comma}\_use\_fourier=\{ 0 ~$|$~ 1 \}{\comma}\_boundary\_conditions=\{ 0=diri\-chlet ~$|$~ 1=neumann \}}}}\end{flushleft}
Compute fields of SSD between selected images and specified patch.
~\\Argument 'boundary\_conditions' is valid only when 'use\_fourier=0'.
\begin{flushleft}\Cc{\textbf{Default value}:\\~\\\hspace*{0.5cm}{\small $\bullet$~~\texttt{'use\_fourier=0'} and \texttt{'boundary\_conditions=0'.}}}\end{flushleft}
\begin{center}\includegraphics[keepaspectratio=true,height=6cm,width=\textwidth]{img/gmic_stdlib404.jpg}\\
{\footnotesize \textbf{Example 404~:} \texttt{image.jpg --crop 20\%{\comma}20\%{\comma}35\%{\comma}35\% --ssd\_patch[0] [1]{\comma}0{\comma}0}}
\end{center}

\subsection{\emph{thinning\index{thinning}} }\vspace*{-0.7em}
Compute skeleton of binary shapes using morphological thinning
~\\(This is a quite slow iterative proces)
\begin{center}\includegraphics[keepaspectratio=true,height=6cm,width=\textwidth]{img/gmic_stdlib405.jpg}\\
{\footnotesize \textbf{Example 405~:} \texttt{image.jpg threshold 50\% --thinning}}
\end{center}

\subsection{\emph{tones\index{tones}} }\vspace*{-0.7em}
~\\\textbf{\Cb{Arguments: }}\begin{flushleft}
{\small \Cb{\hspace*{0.5cm}$\bullet$~~\texttt{N$>$0}}}\end{flushleft}
Get N tones masks from selected images.
\begin{center}\includegraphics[keepaspectratio=true,height=6cm,width=\textwidth]{img/gmic_stdlib406.jpg}\\
{\footnotesize \textbf{Example 406~:} \texttt{image.jpg --tones 3}}
\end{center}

\subsection{\emph{topographic\_map\index{topographic\_map}} }\vspace*{-0.7em}
~\\\textbf{\Cb{Arguments: }}\begin{flushleft}
{\small \Cb{\hspace*{0.5cm}$\bullet$~~\texttt{\_nb\_levels$>$0{\comma}\_smoothness}}}\end{flushleft}
Render selected images as topographic maps.
\begin{flushleft}\Cc{\textbf{Default values}:\\~\\\hspace*{0.5cm}{\small $\bullet$~~\texttt{'nb\_levels=16'} and \texttt{'smoothness=2'.}}}\end{flushleft}
\begin{center}\includegraphics[keepaspectratio=true,height=6cm,width=\textwidth]{img/gmic_stdlib407.jpg}\\
{\footnotesize \textbf{Example 407~:} \texttt{image.jpg --topographic\_map 10}}
\end{center}

\subsection{\emph{variance\_patch\index{variance\_patch}} }\vspace*{-0.7em}
~\\\textbf{\Cb{Arguments: }}\begin{flushleft}
{\small \Cb{\hspace*{0.5cm}$\bullet$~~\texttt{\_patch\_size$>$=1}}}\end{flushleft}
Compute variance of each images patch centered at (x{\comma}y){\comma} in selected images.
\begin{flushleft}\Cc{\textbf{Default value}:\\~\\\hspace*{0.5cm}{\small $\bullet$~~\texttt{'patch\_size=16'}}}\end{flushleft}
\begin{center}\includegraphics[keepaspectratio=true,height=6cm,width=\textwidth]{img/gmic_stdlib408.jpg}\\
{\footnotesize \textbf{Example 408~:} \texttt{image.jpg --variance\_patch}}
\end{center}
\section{Image drawing}


\subsection{\emph{arrow\index{arrow}} }\vspace*{-0.7em}
~\\\textbf{\Cb{Arguments: }}\begin{flushleft}
{\small \Cb{\hspace*{0.5cm}$\bullet$~~\texttt{x0[\%]{\comma}y0[\%]{\comma}x1[\%]{\comma}y1[\%]{\comma}\_thickness[\%]$>$=0{\comma}\_head\_length[\%]$>$=0{\comma}\-\_head\_thickness[\%]$>$=0{\comma}\_opacity{\comma}\_pattern{\comma}\_color1{\comma}...}}}\end{flushleft}
Draw specified arrow on selected images.
~\\'pattern' is an hexadecimal number starting with '0x' which can be omitted
even if a color is specified. If a pattern is specified{\comma} the arrow is
drawn outlined instead of filled.
\begin{flushleft}\Cc{\textbf{Default values}:\\~\\\hspace*{0.5cm}{\small $\bullet$~~\texttt{'thickness=1\%'{\comma} 'head\_length=10\%'{\comma} 'head\_thickness=3\%'{\comma} 'opacity=1'{\comma} 'pattern=(undefined)'} and \texttt{'color1=0'.}}}\end{flushleft}
\begin{center}\includegraphics[keepaspectratio=true,height=6cm,width=\textwidth]{img/gmic_stdlib409.jpg}\\
{\footnotesize \textbf{Example 409~:} \texttt{400{\comma}400{\comma}1{\comma}3 repeat 100 arrow 50\%{\comma}50\%{\comma}\{u(100)\}\%{\comma}\{u(100)\}\%{\comma}3{\comma}20{\comma}10{\comma}0.3{\comma}\$\{-RGB\} done}}
\end{center}

\subsection{\emph{axes\index{axes}} }\vspace*{-0.7em}
~\\\textbf{\Cb{Arguments: }}\begin{flushleft}
{\small \Cb{\hspace*{0.5cm}$\bullet$~~\texttt{x0{\comma}x1{\comma}y0{\comma}y1{\comma}\_font\_height$>$=0{\comma}\_opacity{\comma}\_pattern{\comma}\_color1{\comma}...}}}\end{flushleft}
Draw xy-axes on selected images.
~\\'pattern' is an hexadecimal number starting with '0x' which can be omitted
even if a color is specified.
~\\To draw only one x-axis at row Y{\comma} set both 'y0' and 'y1' to Y.
~\\To draw only one y-axis at column X{\comma} set both 'x0' and 'x1' to X.
\begin{flushleft}\Cc{\textbf{Default values}:\\~\\\hspace*{0.5cm}{\small $\bullet$~~\texttt{'font\_height=14'{\comma} 'opacity=1'{\comma} 'pattern=(undefined)'} and \texttt{'color1=0'.}}}\end{flushleft}
\begin{center}\includegraphics[keepaspectratio=true,height=6cm,width=\textwidth]{img/gmic_stdlib410.jpg}\\
{\footnotesize \textbf{Example 410~:} \texttt{400{\comma}400{\comma}1{\comma}3{\comma}255 axes -1{\comma}1{\comma}1{\comma}-1}}
\end{center}

\subsection{\emph{ball\index{ball}} }\vspace*{-0.7em}
~\\\textbf{\Cb{Arguments: }}\begin{flushleft}
{\small \Cb{\hspace*{0.5cm}$\bullet$~~\texttt{\_size$>$0{\comma} \_R{\comma}\_G{\comma}\_B{\comma}0$<$=\_specular\_light$<$=8{\comma}0$<$=\_specular\_size$<$=8\-{\comma}\_shadow$>$=0}}}\end{flushleft}
Input a 2d RGBA colored ball sprite.
\begin{flushleft}\Cc{\textbf{Default values}:\\~\\\hspace*{0.5cm}{\small $\bullet$~~\texttt{'size=64'{\comma} 'R=255'{\comma} 'G=R'{\comma} 'B=R'{\comma} 'specular\_light=0.8'{\comma} 'specular\_size=1'} and \texttt{'shading=1.5'.}}}\end{flushleft}
\begin{center}\includegraphics[keepaspectratio=true,height=6cm,width=\textwidth]{img/gmic_stdlib411.jpg}\\
{\footnotesize \textbf{Example 411~:} \texttt{repeat 9 ball \{1.5\textasciicircum (\$$>$+2)\}{\comma}\$\{-RGB\} done append x}}
\end{center}

\subsection{\emph{chessboard\index{chessboard}} }\vspace*{-0.7em}
~\\\textbf{\Cb{Arguments: }}\begin{flushleft}
{\small \Cb{\hspace*{0.5cm}$\bullet$~~\texttt{size1$>$0{\comma}\_size2$>$0{\comma}\_offset1{\comma}\_offset2{\comma}\_angle{\comma}\_opacity{\comma}\_color1{\comma}.\-..{\comma}\_color2{\comma}...}}}\end{flushleft}
Draw chessboard on selected images.
\begin{flushleft}\Cc{\textbf{Default values}:\\~\\\hspace*{0.5cm}{\small $\bullet$~~\texttt{'size2=size1'{\comma} 'offset1=offset2=0'{\comma} 'angle=0'{\comma} 'opacity=1'{\comma} 'color1=0'} and \texttt{'color2=255'.}}}\end{flushleft}
\begin{center}\includegraphics[keepaspectratio=true,height=6cm,width=\textwidth]{img/gmic_stdlib412.jpg}\\
{\footnotesize \textbf{Example 412~:} \texttt{image.jpg chessboard 32{\comma}32{\comma}0{\comma}0{\comma}25{\comma}0.3{\comma}255{\comma}128{\comma}0{\comma}0{\comma}128{\comma}255}}
\end{center}

\subsection{\emph{cie1931\index{cie1931}} }\vspace*{-0.7em}
Draw CIE-1931 chromaticity diagram on selected images.
\begin{center}\includegraphics[keepaspectratio=true,height=6cm,width=\textwidth]{img/gmic_stdlib413.jpg}\\
{\footnotesize \textbf{Example 413~:} \texttt{500{\comma}400{\comma}1{\comma}3 cie1931}}
\end{center}

\subsection{\emph{circle\index{circle}} }\vspace*{-0.7em}
~\\\textbf{\Cb{Arguments: }}\begin{flushleft}
{\small \Cb{\hspace*{0.5cm}$\bullet$~~\texttt{x[\%]{\comma}y[\%]{\comma}R[\%]{\comma}\_opacity{\comma}\_pattern{\comma}\_color1{\comma}...}}}\end{flushleft}
Draw specified colored circle on selected images.
~\\A radius of '100\%' stands for 'sqrt(width\textasciicircum 2+height\textasciicircum 2)'.
~\\'pattern' is an hexadecimal number starting with '0x' which can be omitted
even if a color is specified. If a pattern is specified{\comma} the circle is
drawn outlined instead of filled.
\begin{flushleft}\Cc{\textbf{Default values}:\\~\\\hspace*{0.5cm}{\small $\bullet$~~\texttt{'opacity=1'{\comma} 'pattern=(undefined)'} and \texttt{'color1=0'.}}}\end{flushleft}
\begin{center}\includegraphics[keepaspectratio=true,height=6cm,width=\textwidth]{img/gmic_stdlib414.jpg}\\
{\footnotesize \textbf{Example 414~:} \texttt{image.jpg repeat 300 circle \{u(100)\}\%{\comma}\{u(100)\}\%{\comma}\{u(30)\}{\comma}0.3{\comma}\$\{-RGB\} done circle 50\%{\comma}50\%{\comma}100{\comma}0.7{\comma}255}}
\end{center}

\subsection{\emph{close\_binary\index{close\_binary}} }\vspace*{-0.7em}
~\\\textbf{\Cb{Arguments: }}\begin{flushleft}
{\small \Cb{\hspace*{0.5cm}$\bullet$~~\texttt{0$<$=\_endpoint\_rate$<$=100{\comma}\_endpoint\_connectivity$>$=0{\comma}\_spline\_dis\-tmax$>$=0{\comma}\_segment\_distmax$>$=0{\comma}0$<$=\_spline\_anglemax$<$=180{\comma}\_spline\-\_roundness$>$=0{\comma}\_area\_min$>$=0{\comma}\_allow\_self\_intersection=\{ 0 ~$|$~ 1 \-\}{\comma}0$<$=\_hd\_detection\_rate$<$=1{\comma}0$<$=\_hd\_angle$<$=180{\comma}\_hd\_grouping\_di\-st$>$=0{\comma}0$<$=\_hd\_grouping\_angle$<$=180{\comma}0$<$=\_hd\_grouping\_ratio$<$=1}}}\end{flushleft}
Automatically close open shapes in binary images (defining white strokes on black background).
~\\Parameter names starting with 'hd\_' are related to the hatch detection algorithm. Set 'hd\_detection\_rate$>$0' to disable this module.
\begin{flushleft}\Cc{\textbf{Default values}:\\~\\\hspace*{0.5cm}{\small $\bullet$~~\texttt{'endpoint\_rate=85'{\comma} 'endpoint\_connectivity=2'{\comma} 'spline\_distmax=80'{\comma} 'segment\_distmax=20'{\comma} 'spline\_anglemax=90'{\comma} 'spline\_roundness=1'{\comma}'area\_min=100'{\comma} 'allow\_self\_intersection=1'{\comma} 'hd\_detection\_rate=0'{\comma} 'hd\_angle=25'{\comma} 'hd\_grouping\_dist=64'{\comma} 'hd\_grouping\_angle=5'} and \texttt{'hd\_grouping\_ratio=0.2'.}}}\end{flushleft}


\subsection{\emph{ellipse\index{ellipse}} (+)}\vspace*{-0.7em}
~\\\textbf{\Cb{Arguments: }}\begin{flushleft}
{\small \Cb{\hspace*{0.5cm}$\bullet$~~\texttt{x[\%]{\comma}y[\%]{\comma}R[\%]{\comma}r[\%]{\comma}\_angle{\comma}\_opacity{\comma}\_pattern{\comma}\_color1{\comma}...}}}\end{flushleft}
Draw specified colored ellipse on selected images.
~\\A radius of '100\%' stands for 'sqrt(width\textasciicircum 2+height\textasciicircum 2)'.
~\\'pattern' is an hexadecimal number starting with '0x' which can be omitted
even if a color is specified. If a pattern is specified{\comma} the ellipse is
drawn outlined instead of filled.
\begin{flushleft}\Cc{\textbf{Default values}:\\~\\\hspace*{0.5cm}{\small $\bullet$~~\texttt{'opacity=1'{\comma} 'pattern=(undefined)'} and \texttt{'color1=0'.}}}\end{flushleft}
\begin{center}\includegraphics[keepaspectratio=true,height=6cm,width=\textwidth]{img/gmic_stdlib415.jpg}\\
{\footnotesize \textbf{Example 415~:} \texttt{image.jpg repeat 300 ellipse \{u(100)\}\%{\comma}\{u(100)\}\%{\comma}\{u(30)\}{\comma}\{u(30)\}{\comma}\{u(180)\}{\comma}0.3{\comma}\$\{-RGB\} done ellipse 50\%{\comma}50\%{\comma}100{\comma}100{\comma}0{\comma}0.7{\comma}255}}
\end{center}

\subsection{\emph{flood\index{flood}} (+)}\vspace*{-0.7em}
~\\\textbf{\Cb{Arguments: }}\begin{flushleft}
{\small \Cb{\hspace*{0.5cm}$\bullet$~~\texttt{x[\%]{\comma}\_y[\%]{\comma}\_z[\%]{\comma}\_tolerance$>$=0{\comma}\_is\_high\_connectivity=\{ 0 ~$|$~ 1\- \}{\comma}\_opacity{\comma}\_color1{\comma}...}}}\end{flushleft}
Flood-fill selected images using specified value and tolerance.
\begin{flushleft}\Cc{\textbf{Default values}:\\~\\\hspace*{0.5cm}{\small $\bullet$~~\texttt{'y=z=0'{\comma} 'tolerance=0'{\comma} 'is\_high\_connectivity=0'{\comma} 'opacity=1'} and \texttt{'color1=0'.}}}\end{flushleft}
\begin{center}\includegraphics[keepaspectratio=true,height=6cm,width=\textwidth]{img/gmic_stdlib416.jpg}\\
{\footnotesize \textbf{Example 416~:} \texttt{image.jpg repeat 1000 flood \{u(100)\}\%{\comma}\{u(100)\}\%{\comma}0{\comma}20{\comma}0{\comma}1{\comma}\$\{-RGB\} done}}
\end{center}

\subsection{\emph{gaussian\index{gaussian}} }\vspace*{-0.7em}
~\\\textbf{\Cb{Arguments: }}\begin{flushleft}
{\small \Cb{\hspace*{0.5cm}$\bullet$~~\texttt{\_sigma1[\%]{\comma}\_sigma2[\%]{\comma}\_angle}}}\end{flushleft}
Draw a centered gaussian on selected images{\comma} with specified standard deviations and orientation.
\begin{flushleft}\Cc{\textbf{Default values}:\\~\\\hspace*{0.5cm}{\small $\bullet$~~\texttt{'sigma1=3'{\comma} 'sigma2=sigma1'} and \texttt{'angle=0'.}}}\end{flushleft}
\begin{center}\includegraphics[keepaspectratio=true,height=6cm,width=\textwidth]{img/gmic_stdlib417.jpg}\\
{\footnotesize \textbf{Example 417~:} \texttt{400{\comma}400 gaussian 100{\comma}30{\comma}45}}
\end{center}
~\\
~\textbf{Tutorial page: }\\\url{http://gmic.eu/tutorial/\_gaussian.shtml}


\subsection{\emph{graph\index{graph}} (+)}\vspace*{-0.7em}
~\\\textbf{\Cb{Arguments: }}\begin{flushleft}
{\small \Cb{\hspace*{0.5cm}$\bullet$~~\texttt{[function\_image]{\comma}\_plot\_type{\comma}\_vertex\_type{\comma}\_ymin{\comma}\_ymax{\comma}\_opacit\-y{\comma}\_pattern{\comma}\_color1{\comma}...}}}~~~\\
{\small \Cb{\hspace*{0.5cm}$\bullet$~~\texttt{'formula'{\comma}\_resolution$>$=0{\comma}\_plot\_type{\comma}\_vertex\_type{\comma}\_xmin{\comma}xmax{\comma}\-\_ymin{\comma}\_ymax{\comma}\_opacity{\comma}\_pattern{\comma}\_color1{\comma}...}}}\end{flushleft}
Draw specified function graph on selected images.
~\\'plot\_type' can be \{ 0=none ~$|$~ 1=lines ~$|$~ 2=splines ~$|$~ 3=bar \}.
~\\'vertex\_type' can be \{ 0=none ~$|$~ 1=points ~$|$~ 2{\comma}3=crosses ~$|$~ 4{\comma}5=circles ~$|$~ 6{\comma}7=squares \}.
~\\'pattern' is an hexadecimal number starting with '0x' which can be omitted
even if a color is specified.
\begin{flushleft}\Cc{\textbf{Default values}:\\~\\\hspace*{0.5cm}{\small $\bullet$~~\texttt{'plot\_type=1'{\comma} 'vertex\_type=1'{\comma} 'ymin=ymax=0 (auto)'{\comma} 'opacity=1'{\comma} 'pattern=(undefined)'}}}\end{flushleft}
and 'color1=0'.
\begin{center}\includegraphics[keepaspectratio=true,height=6cm,width=\textwidth]{img/gmic_stdlib418.jpg}\\
{\footnotesize \textbf{Example 418~:} \texttt{image.jpg --rows 50\% blur[-1] 3 split[-1] c div[0] 1.5 graph[0] [1]{\comma}2{\comma}0{\comma}0{\comma}0{\comma}1{\comma}255{\comma}0{\comma}0 graph[0] [2]{\comma}2{\comma}0{\comma}0{\comma}0{\comma}1{\comma}0{\comma}255{\comma}0 graph[0] [3]{\comma}2{\comma}0{\comma}0{\comma}0{\comma}1{\comma}0{\comma}0{\comma}255 keep[0]}}
\end{center}

\subsection{\emph{grid\index{grid}} }\vspace*{-0.7em}
~\\\textbf{\Cb{Arguments: }}\begin{flushleft}
{\small \Cb{\hspace*{0.5cm}$\bullet$~~\texttt{size\_x[\%]$>$=0{\comma}size\_y[\%]$>$=0{\comma}\_offset\_x[\%]{\comma}\_offset\_y[\%]{\comma}\_opacity\-{\comma}\_pattern{\comma}\_color1{\comma}...}}}\end{flushleft}
Draw xy-grid on selected images.
~\\'pattern' is an hexadecimal number starting with '0x' which can be omitted
even if a color is specified.
\begin{flushleft}\Cc{\textbf{Default values}:\\~\\\hspace*{0.5cm}{\small $\bullet$~~\texttt{'offset\_x=offset\_y=0'{\comma} 'opacity=1'{\comma} 'pattern=(undefined)'} and \texttt{'color1=0'.}}}\end{flushleft}
\begin{center}\includegraphics[keepaspectratio=true,height=6cm,width=\textwidth]{img/gmic_stdlib419.jpg}\\
{\footnotesize \textbf{Example 419~:} \texttt{image.jpg grid 10\%{\comma}10\%{\comma}0{\comma}0{\comma}0.5{\comma}255}}
\\\includegraphics[keepaspectratio=true,height=6cm,width=\textwidth]{img/gmic_stdlib420.jpg}\\
{\footnotesize \textbf{Example 420~:} \texttt{400{\comma}400{\comma}1{\comma}3{\comma}255 grid 10\%{\comma}10\%{\comma}0{\comma}0{\comma}0.3{\comma}0xCCCCCCCC{\comma}128{\comma}32{\comma}16}}
\end{center}

\subsection{\emph{image\index{image}} (+)}\vspace*{-0.7em}
~\\\textbf{\Cb{Arguments: }}\begin{flushleft}
{\small \Cb{\hspace*{0.5cm}$\bullet$~~\texttt{[sprite]{\comma}\_x[\%]{\comma}\_y[\%]{\comma}\_z[\%]{\comma}\_c[\%]{\comma}\_opacity{\comma}\_[sprite\_mask]{\comma}\_ma\-x\_opacity\_mask}}}\end{flushleft}
Draw specified sprite image on selected images.
~\\(\emph{eq. to} {\small \texttt{'j'}}).
\begin{flushleft}\Cc{\textbf{Default values}:\\~\\\hspace*{0.5cm}{\small $\bullet$~~\texttt{'x=y=z=c=0'{\comma} 'opacity=1'{\comma} 'sprite\_mask=(undefined)'} and \texttt{'max\_opacity\_mask=1'.}}}\end{flushleft}
\begin{center}\includegraphics[keepaspectratio=true,height=6cm,width=\textwidth]{img/gmic_stdlib421.jpg}\\
{\footnotesize \textbf{Example 421~:} \texttt{image.jpg --crop 40\%{\comma}40\%{\comma}60\%{\comma}60\% resize[-1] 200\%{\comma}200\%{\comma}1{\comma}3{\comma}5 frame[-1] 2{\comma}2{\comma}0 image[0] [-1]{\comma}30\%{\comma}30\% keep[0]}}
\end{center}

\subsection{\emph{line\index{line}} (+)}\vspace*{-0.7em}
~\\\textbf{\Cb{Arguments: }}\begin{flushleft}
{\small \Cb{\hspace*{0.5cm}$\bullet$~~\texttt{x0[\%]{\comma}y0[\%]{\comma}x1[\%]{\comma}y1[\%]{\comma}\_opacity{\comma}\_pattern{\comma}\_color1{\comma}...}}}\end{flushleft}
Draw specified colored line on selected images.
~\\'pattern' is an hexadecimal number starting with '0x' which can be omitted
even if a color is specified.
\begin{flushleft}\Cc{\textbf{Default values}:\\~\\\hspace*{0.5cm}{\small $\bullet$~~\texttt{'opacity=1'{\comma} 'pattern=(undefined)'} and \texttt{'color1=0'.}}}\end{flushleft}
\begin{center}\includegraphics[keepaspectratio=true,height=6cm,width=\textwidth]{img/gmic_stdlib422.jpg}\\
{\footnotesize \textbf{Example 422~:} \texttt{image.jpg repeat 500 line 50\%{\comma}50\%{\comma}\{u(w)\}{\comma}\{u(h)\}{\comma}0.5{\comma}\$\{-RGB\} done line 0{\comma}0{\comma}100\%{\comma}100\%{\comma}1{\comma}0xCCCCCCCC{\comma}255 line 100\%{\comma}0{\comma}0{\comma}100\%{\comma}1{\comma}0xCCCCCCCC{\comma}255}}
\end{center}

\subsection{\emph{mandelbrot\index{mandelbrot}} (+)}\vspace*{-0.7em}
~\\\textbf{\Cb{Arguments: }}\begin{flushleft}
{\small \Cb{\hspace*{0.5cm}$\bullet$~~\texttt{z0r{\comma}z0i{\comma}z1r{\comma}z1i{\comma}\_iteration\_max$>$=0{\comma}\_is\_julia=\{ 0 ~$|$~ 1 \}{\comma}\_c0r{\comma}\_\-c0i{\comma}\_opacity}}}\end{flushleft}
Draw mandelbrot/julia fractal on selected images.
\begin{flushleft}\Cc{\textbf{Default values}:\\~\\\hspace*{0.5cm}{\small $\bullet$~~\texttt{'iteration\_max=100'{\comma} 'is\_julia=0'{\comma} 'c0r=c0i=0'} and \texttt{'opacity=1'.}}}\end{flushleft}
\begin{center}\includegraphics[keepaspectratio=true,height=6cm,width=\textwidth]{img/gmic_stdlib423.jpg}\\
{\footnotesize \textbf{Example 423~:} \texttt{400{\comma}400 mandelbrot -2.5{\comma}-2{\comma}2{\comma}2{\comma}1024 map 0 --blur 2 elevation3d[-1] -0.2}}
\end{center}

\subsection{\emph{marble\index{marble}} }\vspace*{-0.7em}
~\\\textbf{\Cb{Arguments: }}\begin{flushleft}
{\small \Cb{\hspace*{0.5cm}$\bullet$~~\texttt{\_image\_weight{\comma}\_pattern\_weight{\comma}\_angle{\comma}\_amplitude{\comma}\_sharpness$>$=\-0{\comma}\_anisotropy$>$=0{\comma}\_alpha{\comma}\_sigma{\comma}\_cut\_low$>$=0{\comma}\_cut\_high$>$=0}}}\end{flushleft}
Render marble like pattern on selected images.
\begin{flushleft}\Cc{\textbf{Default values}:\\~\\\hspace*{0.5cm}{\small $\bullet$~~\texttt{'image\_weight=0.2'{\comma} 'pattern\_weight=0.1'{\comma} 'angle=45'{\comma} 'amplitude=0'{\comma} 'sharpness=0.4'{\comma} 'anisotropy=0.8'{\comma}}}}\end{flushleft}
~\\'alpha=0.6'{\comma} 'sigma=1.1' and 'cut\_low=cut\_high=0'.
\begin{center}\includegraphics[keepaspectratio=true,height=6cm,width=\textwidth]{img/gmic_stdlib424.jpg}\\
{\footnotesize \textbf{Example 424~:} \texttt{image.jpg --marble {\comma}}}
\end{center}

\subsection{\emph{maze\index{maze}} }\vspace*{-0.7em}
~\\\textbf{\Cb{Arguments: }}\begin{flushleft}
{\small \Cb{\hspace*{0.5cm}$\bullet$~~\texttt{\_width$>$0{\comma}\_height$>$0{\comma}\_cell\_size$>$0}}}\end{flushleft}
Input maze with specified size.
\begin{center}\includegraphics[keepaspectratio=true,height=6cm,width=\textwidth]{img/gmic_stdlib425.jpg}\\
{\footnotesize \textbf{Example 425~:} \texttt{maze 30{\comma}20 negate normalize 0{\comma}255}}
\end{center}

\subsection{\emph{maze\_mask\index{maze\_mask}} }\vspace*{-0.7em}
~\\\textbf{\Cb{Arguments: }}\begin{flushleft}
{\small \Cb{\hspace*{0.5cm}$\bullet$~~\texttt{\_cellsize$>$0}}}\end{flushleft}
Input maze according to size and shape of selected mask images.
~\\Mask may contain disconnected shapes.
\begin{center}\includegraphics[keepaspectratio=true,height=6cm,width=\textwidth]{img/gmic_stdlib426.jpg}\\
{\footnotesize \textbf{Example 426~:} \texttt{0 text "G'MIC"{\comma}0{\comma}0{\comma}53{\comma}1{\comma}1 dilate 3 autocrop 0 frame 1{\comma}1{\comma}0 maze\_mask 8 dilate 3 negate mul 255}}
\end{center}

\subsection{\emph{object3d\index{object3d}} (+)}\vspace*{-0.7em}
~\\\textbf{\Cb{Arguments: }}\begin{flushleft}
{\small \Cb{\hspace*{0.5cm}$\bullet$~~\texttt{[object3d]{\comma}\_x[\%]{\comma}\_y[\%]{\comma}\_z{\comma}\_opacity{\comma}\_rendering\_mode{\comma}\_is\_doubl\-e\_sided=\{ 0 ~$|$~ 1 \}{\comma}\_is\_zbuffer=\{ 0 ~$|$~ 1 \}{\comma}\_focale{\comma}\_light\_x{\comma}\_li\-ght\_y{\comma}\_light\_z{\comma}\_specular\_lightness{\comma}\_specular\_shininess}}}\end{flushleft}
Draw specified 3d object on selected images.
~\\(\emph{eq. to} {\small \texttt{'j3d').\textbackslash n}}).
~\\'rendering\_mode' can be \{ 0=dots ~$|$~ 1=wireframe ~$|$~ 2=flat ~$|$~ 3=flat-shaded ~$|$~ 4=gouraud-shaded ~$|$~ 5=phong-shaded \}.
\begin{flushleft}\Cc{\textbf{Default values}:\\~\\\hspace*{0.5cm}{\small $\bullet$~~\texttt{'x=y=z=0'{\comma} 'opacity=1'} and \texttt{'is\_zbuffer=1'. All other arguments take their default values from the 3d environment variables.}}}\end{flushleft}
\begin{center}\includegraphics[keepaspectratio=true,height=6cm,width=\textwidth]{img/gmic_stdlib427.jpg}\\
{\footnotesize \textbf{Example 427~:} \texttt{image.jpg torus3d 100{\comma}10 cone3d 30{\comma}-120 add3d[-2{\comma}-1] rotate3d. 1{\comma}1{\comma}0{\comma}60 object3d[0] [-1]{\comma}50\%{\comma}50\% keep[0]}}
\end{center}

\subsection{\emph{pack\_sprites\index{pack\_sprites}} }\vspace*{-0.7em}
~\\\textbf{\Cb{Arguments: }}\begin{flushleft}
{\small \Cb{\hspace*{0.5cm}$\bullet$~~\texttt{\_nb\_scales$>$=0{\comma}0$<$=\_min\_scale$<$=100{\comma}\_allow\_rotation=\{ 0=0 deg. \-~$|$~ 1=180 deg. ~$|$~ 2=90 deg. ~$|$~ 3=any \}{\comma}\_spacing{\comma}\_precision$>$=0{\comma}ma\-x\_iterations$>$=0}}}\end{flushleft}
Try to randomly pack as many sprites as possible onto the 'empty' areas of an image.
~\\Sprites can be eventually rotated and scaled during the packing process.
~\\First selected image is the canvas that will be filled with the sprites.
~\\Its last channel must be a binary mask whose zero values represent potential locations for drawing the sprites.
~\\All other selected images represent the sprites considered for packing.
~\\Their last channel must be a binary mask that represents the sprite shape (i.e. a 8-connected component).
~\\The order of sprite packing follows the order of specified sprites in the image list.
~\\Sprite packing is done on random locations and iteratively with decreasing scales.
~\\'nb\_scales' sets the number of decreasing scales considered for all specified sprites to be packed.
~\\'min\_scale' (in \%) sets the minimal size considered for packing (specified as a percentage of the original sprite size).
~\\'spacing' can be positive or negative.
~\\'precision' tells about the desired number of failed trials before ending the filling process.
\begin{flushleft}\Cc{\textbf{Default values}:\\~\\\hspace*{0.5cm}{\small $\bullet$~~\texttt{'nb\_scales=5'{\comma} 'min\_scale=25'{\comma} 'allow\_rotation=3'{\comma} 'spacing=1'{\comma} 'precision=7'} and \texttt{'max\_iterations=256'.}}}\end{flushleft}
\begin{center}\includegraphics[keepaspectratio=true,height=6cm,width=\textwidth]{img/gmic_stdlib428.jpg}\\
{\footnotesize \textbf{Example 428~:} \texttt{512{\comma}512{\comma}1{\comma}3{\comma}"min(255{\comma}y*c/2)" 100\%{\comma}100\% circle 50\%{\comma}50\%{\comma}100{\comma}1{\comma}255 append c image.jpg resize2dy[-1] 24 to\_rgba pack\_sprites 3{\comma}25}}
\end{center}

\subsection{\emph{piechart\index{piechart}} }\vspace*{-0.7em}
~\\\textbf{\Cb{Arguments: }}\begin{flushleft}
{\small \Cb{\hspace*{0.5cm}$\bullet$~~\texttt{label\_height$>$=0{\comma}label\_R{\comma}label\_G{\comma}label\_B{\comma}"label1"{\comma}value1{\comma}R1{\comma}G\-1{\comma}B1{\comma}...{\comma}"labelN"{\comma}valueN{\comma}RN{\comma}GN{\comma}BN}}}\end{flushleft}
Draw pie chart on selected (RGB) images.
\begin{center}\includegraphics[keepaspectratio=true,height=6cm,width=\textwidth]{img/gmic_stdlib429.jpg}\\
{\footnotesize \textbf{Example 429~:} \texttt{image.jpg piechart 25{\comma}0{\comma}0{\comma}0{\comma}"Red"{\comma}55{\comma}255{\comma}0{\comma}0{\comma}"Green"{\comma}40{\comma}0{\comma}255{\comma}0{\comma}"Blue"{\comma}30{\comma}128{\comma}128{\comma}255{\comma}"Other"{\comma}5{\comma}128{\comma}128{\comma}128}}
\end{center}

\subsection{\emph{plasma\index{plasma}} (+)}\vspace*{-0.7em}
~\\\textbf{\Cb{Arguments: }}\begin{flushleft}
{\small \Cb{\hspace*{0.5cm}$\bullet$~~\texttt{\_alpha{\comma}\_beta{\comma}\_scale$>$=0}}}\end{flushleft}
Draw a random colored plasma fractal on selected images.
~\\This command implements the so-called 'Diamond-Square' algorithm.
\begin{flushleft}\Cc{\textbf{Default values}:\\~\\\hspace*{0.5cm}{\small $\bullet$~~\texttt{'alpha=1'{\comma} 'beta=1'} and \texttt{'scale=8'.}}}\end{flushleft}
\begin{center}\includegraphics[keepaspectratio=true,height=6cm,width=\textwidth]{img/gmic_stdlib430.jpg}\\
{\footnotesize \textbf{Example 430~:} \texttt{400{\comma}400{\comma}1{\comma}3 plasma}}
\end{center}
~\\
~\textbf{Tutorial page: }\\\url{http://gmic.eu/tutorial/\_plasma.shtml}


\subsection{\emph{point\index{point}} (+)}\vspace*{-0.7em}
~\\\textbf{\Cb{Arguments: }}\begin{flushleft}
{\small \Cb{\hspace*{0.5cm}$\bullet$~~\texttt{x[\%]{\comma}y[\%]{\comma}\_z[\%]{\comma}\_opacity{\comma}\_color1{\comma}...}}}\end{flushleft}
Set specified colored pixel on selected images.
\begin{flushleft}\Cc{\textbf{Default values}:\\~\\\hspace*{0.5cm}{\small $\bullet$~~\texttt{'z=0'{\comma} 'opacity=1'} and \texttt{'color1=0'.}}}\end{flushleft}
\begin{center}\includegraphics[keepaspectratio=true,height=6cm,width=\textwidth]{img/gmic_stdlib431.jpg}\\
{\footnotesize \textbf{Example 431~:} \texttt{image.jpg repeat 10000 point \{u(100)\}\%{\comma}\{u(100)\}\%{\comma}0{\comma}1{\comma}\$\{-RGB\} done}}
\end{center}

\subsection{\emph{polka\_dots\index{polka\_dots}} }\vspace*{-0.7em}
~\\\textbf{\Cb{Arguments: }}\begin{flushleft}
{\small \Cb{\hspace*{0.5cm}$\bullet$~~\texttt{diameter$>$=0{\comma}\_density{\comma}\_offset1{\comma}\_offset2{\comma}\_angle{\comma}\_aliasing{\comma}\_sha\-ding{\comma}\_opacity{\comma}\_color{\comma}...}}}\end{flushleft}
Draw dots pattern on selected images.
\begin{flushleft}\Cc{\textbf{Default values}:\\~\\\hspace*{0.5cm}{\small $\bullet$~~\texttt{'density=20'{\comma} 'offset1=offset2=50'{\comma} 'angle=0'{\comma} 'aliasing=10'{\comma} 'shading=1'{\comma} 'opacity=1'} and \texttt{'color=255'.}}}\end{flushleft}
\begin{center}\includegraphics[keepaspectratio=true,height=6cm,width=\textwidth]{img/gmic_stdlib432.jpg}\\
{\footnotesize \textbf{Example 432~:} \texttt{image.jpg polka\_dots 10{\comma}15{\comma}0{\comma}0{\comma}20{\comma}10{\comma}1{\comma}0.5{\comma}0{\comma}128{\comma}255}}
\end{center}

\subsection{\emph{polygon\index{polygon}} (+)}\vspace*{-0.7em}
~\\\textbf{\Cb{Arguments: }}\begin{flushleft}
{\small \Cb{\hspace*{0.5cm}$\bullet$~~\texttt{N$>$=1{\comma}x1[\%]{\comma}y1[\%]{\comma}...{\comma}xN[\%]{\comma}yN[\%]{\comma}\_opacity{\comma}\_pattern{\comma}\_color1{\comma}.\-..}}}\end{flushleft}
Draw specified colored N-vertices polygon on selected images.
~\\'pattern' is an hexadecimal number starting with '0x' which can be omitted
even if a color is specified. If a pattern is specified{\comma} the polygon is
drawn outlined instead of filled.
\begin{flushleft}\Cc{\textbf{Default values}:\\~\\\hspace*{0.5cm}{\small $\bullet$~~\texttt{'opacity=1'{\comma} 'pattern=(undefined)'} and \texttt{'color1=0'.}}}\end{flushleft}
\begin{center}\includegraphics[keepaspectratio=true,height=6cm,width=\textwidth]{img/gmic_stdlib433.jpg}\\
{\footnotesize \textbf{Example 433~:} \texttt{image.jpg polygon 4{\comma}20\%{\comma}20\%{\comma}80\%{\comma}30\%{\comma}80\%{\comma}70\%{\comma}20\%{\comma}80\%{\comma}0.3{\comma}0{\comma}255{\comma}0 polygon 4{\comma}20\%{\comma}20\%{\comma}80\%{\comma}30\%{\comma}80\%{\comma}70\%{\comma}20\%{\comma}80\%{\comma}1{\comma}0xCCCCCCCC{\comma}255}}
\\\includegraphics[keepaspectratio=true,height=6cm,width=\textwidth]{img/gmic_stdlib434.jpg}\\
{\footnotesize \textbf{Example 434~:} \texttt{image.jpg 2{\comma}16{\comma}1{\comma}1{\comma}'u(if(x{\comma}\{h\}{\comma}\{w\}))' polygon[-2] \{h\}{\comma}\{\textasciicircum \}{\comma}0.6{\comma}255{\comma}0{\comma}255 remove[-1]}}
\end{center}

\subsection{\emph{quiver\index{quiver}} }\vspace*{-0.7em}
~\\\textbf{\Cb{Arguments: }}\begin{flushleft}
{\small \Cb{\hspace*{0.5cm}$\bullet$~~\texttt{[function\_image]{\comma}\_sampling[\%]$>$0{\comma}\_factor$>$=0{\comma}\_is\_arrow=\{ 0 ~$|$~ 1\- \}{\comma}\_opacity{\comma}\_color1{\comma}...}}}\end{flushleft}
Draw specified 2d vector/orientation field on selected images.
\begin{flushleft}\Cc{\textbf{Default values}:\\~\\\hspace*{0.5cm}{\small $\bullet$~~\texttt{'sampling=5\%'{\comma} 'factor=1'{\comma} 'is\_arrow=1'{\comma} 'opacity=1'{\comma} 'pattern=(undefined)'}}}\end{flushleft}
and 'color1=0'.
\begin{center}\includegraphics[keepaspectratio=true,height=6cm,width=\textwidth]{img/gmic_stdlib435.jpg}\\
{\footnotesize \textbf{Example 435~:} \texttt{100{\comma}100{\comma}1{\comma}2{\comma}'if(c==0{\comma}x-w/2{\comma}y-h/2)' 500{\comma}500{\comma}1{\comma}3{\comma}255 quiver[-1] [-2]{\comma}10}}
\\\includegraphics[keepaspectratio=true,height=6cm,width=\textwidth]{img/gmic_stdlib436.jpg}\\
{\footnotesize \textbf{Example 436~:} \texttt{image.jpg --resize2dy 600 luminance[0] gradient[0] mul[1] -1 reverse[0{\comma}1] append[0{\comma}1] c blur[0] 8 orientation[0] quiver[1] [0]{\comma}20{\comma}1{\comma}1{\comma}0.8{\comma}255}}
\end{center}

\subsection{\emph{rectangle\index{rectangle}} }\vspace*{-0.7em}
~\\\textbf{\Cb{Arguments: }}\begin{flushleft}
{\small \Cb{\hspace*{0.5cm}$\bullet$~~\texttt{x0[\%]{\comma}y0[\%]{\comma}x1[\%]{\comma}y1[\%]{\comma}\_opacity{\comma}\_pattern{\comma}\_color1{\comma}...}}}\end{flushleft}
Draw specified colored rectangle on selected images.
~\\'pattern' is an hexadecimal number starting with '0x' which can be omitted
even if a color is specified. If a pattern is specified{\comma} the rectangle is
drawn outlined instead of filled.
\begin{flushleft}\Cc{\textbf{Default values}:\\~\\\hspace*{0.5cm}{\small $\bullet$~~\texttt{'opacity=1'{\comma} 'pattern=(undefined)'} and \texttt{'color1=0'.}}}\end{flushleft}
\begin{center}\includegraphics[keepaspectratio=true,height=6cm,width=\textwidth]{img/gmic_stdlib437.jpg}\\
{\footnotesize \textbf{Example 437~:} \texttt{image.jpg repeat 30 rectangle \{u(100)\}\%{\comma}\{u(100)\}\%{\comma}\{u(100)\}\%{\comma}\{u(100)\}\%{\comma}0.3{\comma}\$\{-RGB\} done}}
\end{center}

\subsection{\emph{rorschach\index{rorschach}} }\vspace*{-0.7em}
~\\\textbf{\Cb{Arguments: }}\begin{flushleft}
{\small \Cb{\hspace*{0.5cm}$\bullet$~~\texttt{'smoothness[\%]$>$=0'{\comma}'mirroring=\{ 0=none ~$|$~ 1=x ~$|$~ 2=y ~$|$~ 3=xy \}}}}\end{flushleft}
Render rorschach-like inkblots on selected images.
\begin{flushleft}\Cc{\textbf{Default values}:\\~\\\hspace*{0.5cm}{\small $\bullet$~~\texttt{'smoothness=5\%'} and \texttt{'mirroring=1'.}}}\end{flushleft}
\begin{center}\includegraphics[keepaspectratio=true,height=6cm,width=\textwidth]{img/gmic_stdlib438.jpg}\\
{\footnotesize \textbf{Example 438~:} \texttt{400{\comma}400 rorschach 3\%}}
\end{center}

\subsection{\emph{sierpinski\index{sierpinski}} }\vspace*{-0.7em}
~\\\textbf{\Cb{Arguments: }}\begin{flushleft}
{\small \Cb{\hspace*{0.5cm}$\bullet$~~\texttt{recursion\_level$>$=0}}}\end{flushleft}
Draw Sierpinski triangle on selected images.
\begin{flushleft}\Cc{\textbf{Default value}:\\~\\\hspace*{0.5cm}{\small $\bullet$~~\texttt{'recursion\_level=7'.}}}\end{flushleft}
\begin{center}\includegraphics[keepaspectratio=true,height=6cm,width=\textwidth]{img/gmic_stdlib439.jpg}\\
{\footnotesize \textbf{Example 439~:} \texttt{image.jpg sierpinski 7}}
\end{center}

\subsection{\emph{spiralbw\index{spiralbw}} }\vspace*{-0.7em}
Draw (squared) spiral on selected images.
\begin{center}\includegraphics[keepaspectratio=true,height=6cm,width=\textwidth]{img/gmic_stdlib440.jpg}\\
{\footnotesize \textbf{Example 440~:} \texttt{16{\comma}16 spiralbw}}
\end{center}

\subsection{\emph{spline\index{spline}} }\vspace*{-0.7em}
~\\\textbf{\Cb{Arguments: }}\begin{flushleft}
{\small \Cb{\hspace*{0.5cm}$\bullet$~~\texttt{x0[\%]{\comma}y0[\%]{\comma}u0[\%]{\comma}v0[\%]{\comma}x1[\%]{\comma}y1[\%]{\comma}u1[\%]{\comma}v1[\%]{\comma}\_opacity{\comma}\_co\-lor1{\comma}...}}}\end{flushleft}
Draw specified colored spline curve on selected images (cubic hermite spline).
\begin{flushleft}\Cc{\textbf{Default values}:\\~\\\hspace*{0.5cm}{\small $\bullet$~~\texttt{'opacity=1'} and \texttt{'color1=0'.}}}\end{flushleft}
\begin{center}\includegraphics[keepaspectratio=true,height=6cm,width=\textwidth]{img/gmic_stdlib441.jpg}\\
{\footnotesize \textbf{Example 441~:} \texttt{image.jpg repeat 30 spline \{u(100)\}\%{\comma}\{u(100)\}\%{\comma}\{u(-600{\comma}600)\}{\comma}\{u(-600{\comma}600)\}{\comma}\{u(100)\}\%{\comma}\{u(100)\}\%{\comma}\{u(-600{\comma}600)\}{\comma}\{u(-600{\comma}600)\}{\comma}0.6{\comma}255 done}}
\end{center}

\subsection{\emph{tetraedron\_shade\index{tetraedron\_shade}} }\vspace*{-0.7em}
~\\\textbf{\Cb{Arguments: }}\begin{flushleft}
{\small \Cb{\hspace*{0.5cm}$\bullet$~~\texttt{x0{\comma}y0{\comma}z0{\comma}x1{\comma}y1{\comma}z1{\comma}x2{\comma}y2{\comma}z2{\comma}x3{\comma}y3{\comma}z3{\comma}R0{\comma}G0{\comma}B0{\comma}...{\comma}R1{\comma}G1{\comma}B1{\comma}..\-.{\comma}R2{\comma}G2{\comma}B2{\comma}...{\comma}R3{\comma}G3{\comma}B3{\comma}...}}}\end{flushleft}
Draw tetraedron with interpolated colors on selected (volumetric) images.


\subsection{\emph{text\index{text}} (+)}\vspace*{-0.7em}
~\\\textbf{\Cb{Arguments: }}\begin{flushleft}
{\small \Cb{\hspace*{0.5cm}$\bullet$~~\texttt{text{\comma}\_x[\%]{\comma}\_y[\%]{\comma}\_font\_height[\%]$>$=0{\comma}\_opacity{\comma}\_color1{\comma}...}}}\end{flushleft}
Draw specified colored text string on selected images.
~\\(\emph{eq. to} {\small \texttt{'t').\textbackslash n}}).
~\\Exact pre-defined sizes are '13'{\comma}'23'{\comma}'53' and '103'. Using these sizes ensures you draw binary letters without anti-aliasing.
~\\Any other font size is interpolated from an exact size (the upper when possible).
~\\Specifying an empty target image resizes it to new dimensions such that the image contains
the entire text string.
\begin{flushleft}\Cc{\textbf{Default values}:\\~\\\hspace*{0.5cm}{\small $\bullet$~~\texttt{'opacity=1'} and \texttt{'color1=0'.}}}\end{flushleft}
\begin{center}\includegraphics[keepaspectratio=true,height=6cm,width=\textwidth]{img/gmic_stdlib442.jpg}\\
{\footnotesize \textbf{Example 442~:} \texttt{image.jpg resize2dy 600 y=0 repeat 30 text \{2*\$$>$\}" : This is a nice text{\comma} isn't it ?"{\comma}10{\comma}\$y{\comma}\{2*\$$>$\}{\comma}0.9{\comma}255 y+=\{2*\$$>$\} done}}
\\\includegraphics[keepaspectratio=true,height=6cm,width=\textwidth]{img/gmic_stdlib443.jpg}\\
{\footnotesize \textbf{Example 443~:} \texttt{0 text "G'MIC"{\comma}0{\comma}0{\comma}23{\comma}1{\comma}255}}
\end{center}

\subsection{\emph{text\_outline\index{text\_outline}} }\vspace*{-0.7em}
~\\\textbf{\Cb{Arguments: }}\begin{flushleft}
{\small \Cb{\hspace*{0.5cm}$\bullet$~~\texttt{text{\comma}\_x[\%]{\comma}\_y[\%]{\comma}\_font\_height[\%]$>$0{\comma}\_outline$>$=0{\comma}\_opacity{\comma}\_col\-or1{\comma}...}}}\end{flushleft}
Draw specified colored and outlined text string on selected images.
\begin{flushleft}\Cc{\textbf{Default values}:\\~\\\hspace*{0.5cm}{\small $\bullet$~~\texttt{'x=y=1\%'{\comma} 'font\_height=7.5\%'{\comma} 'outline=2'{\comma} 'opacity=1'} and \texttt{'color1=255'.}}}\end{flushleft}
\begin{center}\includegraphics[keepaspectratio=true,height=6cm,width=\textwidth]{img/gmic_stdlib444.jpg}\\
{\footnotesize \textbf{Example 444~:} \texttt{image.jpg text\_outline "Hi there!"{\comma}10{\comma}10{\comma}63{\comma}3}}
\end{center}

\subsection{\emph{triangle\_shade\index{triangle\_shade}} }\vspace*{-0.7em}
~\\\textbf{\Cb{Arguments: }}\begin{flushleft}
{\small \Cb{\hspace*{0.5cm}$\bullet$~~\texttt{x0{\comma}y0{\comma}x1{\comma}y1{\comma}x2{\comma}y2{\comma}R0{\comma}G0{\comma}B0{\comma}...{\comma}R1{\comma}G1{\comma}B1{\comma}...{\comma}R2{\comma}G2{\comma}B2{\comma}...}}}\end{flushleft}
Draw triangle with interpolated colors on selected images.
\begin{center}\includegraphics[keepaspectratio=true,height=6cm,width=\textwidth]{img/gmic_stdlib445.jpg}\\
{\footnotesize \textbf{Example 445~:} \texttt{image.jpg triangle\_shade 20{\comma}20{\comma}400{\comma}100{\comma}120{\comma}200{\comma}255{\comma}0{\comma}0{\comma}0{\comma}255{\comma}0{\comma}0{\comma}0{\comma}255}}
\end{center}

\subsection{\emph{truchet\index{truchet}} }\vspace*{-0.7em}
~\\\textbf{\Cb{Arguments: }}\begin{flushleft}
{\small \Cb{\hspace*{0.5cm}$\bullet$~~\texttt{\_scale$>$0{\comma}\_radius$>$=0{\comma}\_pattern\_type=\{ 0=straight ~$|$~ 1=curved \}}}}\end{flushleft}
Fill selected images with random truchet patterns.
\begin{flushleft}\Cc{\textbf{Default values}:\\~\\\hspace*{0.5cm}{\small $\bullet$~~\texttt{'scale=32'{\comma} 'radius=5'} and \texttt{'pattern\_type=1'.}}}\end{flushleft}
\begin{center}\includegraphics[keepaspectratio=true,height=6cm,width=\textwidth]{img/gmic_stdlib446.jpg}\\
{\footnotesize \textbf{Example 446~:} \texttt{400{\comma}300 truchet {\comma}}}
\end{center}

\subsection{\emph{turbulence\index{turbulence}} }\vspace*{-0.7em}
~\\\textbf{\Cb{Arguments: }}\begin{flushleft}
{\small \Cb{\hspace*{0.5cm}$\bullet$~~\texttt{\_radius$>$0{\comma}\_octaves=\{1{\comma}2{\comma}3...{\comma}12\}{\comma}\_alpha$>$0{\comma}\_difference=\{-10{\comma}1\-0\}{\comma}\_mode=\{0{\comma}1{\comma}2{\comma}3\}}}}\end{flushleft}
Render fractal noise or turbulence on selected images.
\begin{flushleft}\Cc{\textbf{Default values}:\\~\\\hspace*{0.5cm}{\small $\bullet$~~\texttt{'radius=32'{\comma} 'octaves=6'{\comma} 'alpha=3'{\comma} 'difference=0'} and \texttt{'mode=0'.}}}\end{flushleft}
\begin{center}\includegraphics[keepaspectratio=true,height=6cm,width=\textwidth]{img/gmic_stdlib447.jpg}\\
{\footnotesize \textbf{Example 447~:} \texttt{400{\comma}400{\comma}1{\comma}3 turbulence 16}}
\end{center}
~\\
~\textbf{Tutorial page: }\\\url{http://gmic.eu/tutorial/\_turbulence.shtml}


\subsection{\emph{yinyang\index{yinyang}} }\vspace*{-0.7em}
Draw a yin-yang symbol on selected images.
\begin{center}\includegraphics[keepaspectratio=true,height=6cm,width=\textwidth]{img/gmic_stdlib448.jpg}\\
{\footnotesize \textbf{Example 448~:} \texttt{400{\comma}400 yinyang}}
\end{center}
\section{Matrix computation}


\subsection{\emph{dijkstra\index{dijkstra}} (+)}\vspace*{-0.7em}
~\\\textbf{\Cb{Arguments: }}\begin{flushleft}
{\small \Cb{\hspace*{0.5cm}$\bullet$~~\texttt{starting\_node$>$=0{\comma}ending\_node$>$=0}}}\end{flushleft}
Compute minimal distances and pathes from specified adjacency matrices by the Dijkstra algorithm.


\subsection{\emph{eigen\index{eigen}} (+)}\vspace*{-0.7em}
Compute the eigenvalues and eigenvectors of selected symmetric matrices or matrix fields.
~\\If one selected image has 3 or 6 channels{\comma} it is regarded as a field of 2x2 or 3x3 symmetric matrices{\comma}
whose eigen elements are computed at each point of the field.
\begin{center}\includegraphics[keepaspectratio=true,height=6cm,width=\textwidth]{img/gmic_stdlib449.jpg}\\
{\footnotesize \textbf{Example 449~:} \texttt{(1{\comma}0{\comma}0;0{\comma}2{\comma}0;0{\comma}0{\comma}3) --eigen}}
\\\includegraphics[keepaspectratio=true,height=6cm,width=\textwidth]{img/gmic_stdlib450.jpg}\\
{\footnotesize \textbf{Example 450~:} \texttt{image.jpg structuretensors blur 2 eigen split[0] c}}
\end{center}
~\\
~\textbf{Tutorial page: }\\\url{http://gmic.eu/tutorial/\_eigen.shtml}


\subsection{\emph{invert\index{invert}} (+)}\vspace*{-0.7em}
Compute the inverse of the selected matrices.
\begin{center}\includegraphics[keepaspectratio=true,height=6cm,width=\textwidth]{img/gmic_stdlib451.jpg}\\
{\footnotesize \textbf{Example 451~:} \texttt{(0{\comma}1{\comma}0;0{\comma}0{\comma}1;1{\comma}0{\comma}0) --invert}}
\end{center}

\subsection{\emph{solve\index{solve}} (+)}\vspace*{-0.7em}
~\\\textbf{\Cb{Arguments: }}\begin{flushleft}
{\small \Cb{\hspace*{0.5cm}$\bullet$~~\texttt{[image]}}}\end{flushleft}
Solve linear system AX = B for selected B-matrices and specified A-matrix.
~\\If the system is under- or over-determined{\comma} the least square solution is returned.
\begin{center}\includegraphics[keepaspectratio=true,height=6cm,width=\textwidth]{img/gmic_stdlib452.jpg}\\
{\footnotesize \textbf{Example 452~:} \texttt{(0{\comma}1{\comma}0;1{\comma}0{\comma}0;0{\comma}0{\comma}1) (1;2;3) --solve[-1] [-2]}}
\end{center}

\subsection{\emph{svd\index{svd}} (+)}\vspace*{-0.7em}
Compute SVD decomposition of selected matrices.
\begin{center}\includegraphics[keepaspectratio=true,height=6cm,width=\textwidth]{img/gmic_stdlib453.jpg}\\
{\footnotesize \textbf{Example 453~:} \texttt{10{\comma}10{\comma}1{\comma}1{\comma}'if(x==y{\comma}x+u(-0.2{\comma}0.2){\comma}0)' --svd}}
\end{center}

\subsection{\emph{transpose\index{transpose}} }\vspace*{-0.7em}
Transpose selected matrices.
\begin{center}\includegraphics[keepaspectratio=true,height=6cm,width=\textwidth]{img/gmic_stdlib454.jpg}\\
{\footnotesize \textbf{Example 454~:} \texttt{image.jpg --transpose}}
\end{center}

\subsection{\emph{trisolve\index{trisolve}} (+)}\vspace*{-0.7em}
~\\\textbf{\Cb{Arguments: }}\begin{flushleft}
{\small \Cb{\hspace*{0.5cm}$\bullet$~~\texttt{[image]}}}\end{flushleft}
Solve tridiagonal system AX = B for selected B-vectors and specified tridiagonal A-matrix.
~\\Tridiagonal matrix must be stored as a 3 column vector{\comma} where 2nd column contains the
diagonal coefficients{\comma} while 1st and 3rd columns contain the left and right coefficients.
\begin{center}\includegraphics[keepaspectratio=true,height=6cm,width=\textwidth]{img/gmic_stdlib455.jpg}\\
{\footnotesize \textbf{Example 455~:} \texttt{(0{\comma}0{\comma}1;1{\comma}0{\comma}0;0{\comma}1{\comma}0) (1;2;3) --trisolve[-1] [-2]}}
\end{center}
\section{3d rendering}


\subsection{\emph{add3d\index{add3d}} (+)}\vspace*{-0.7em}
~\\\textbf{\Cb{Arguments: }}\begin{flushleft}
{\small \Cb{\hspace*{0.5cm}$\bullet$~~\texttt{tx{\comma}\_ty{\comma}\_tz}}}~~~\\
{\small \Cb{\hspace*{0.5cm}$\bullet$~~\texttt{[object3d]}}}~~~\\
{\small \Cb{\hspace*{0.5cm}$\bullet$~~\texttt{(no arg)}}}\end{flushleft}
Shift selected 3d objects with specified displacement vector{\comma} or merge them with specified
3d object{\comma} or merge all selected 3d objects together.
~\\(\emph{eq. to} {\small \texttt{'+3d'}}).
\begin{flushleft}\Cc{\textbf{Default values}:\\~\\\hspace*{0.5cm}{\small $\bullet$~~\texttt{'ty=tz=0'.}}}\end{flushleft}
\begin{center}\includegraphics[keepaspectratio=true,height=6cm,width=\textwidth]{img/gmic_stdlib456.jpg}\\
{\footnotesize \textbf{Example 456~:} \texttt{sphere3d 10 repeat 5 --add3d[-1] 10{\comma}\{u(-10{\comma}10)\}{\comma}0 color3d[-1] \$\{-RGB\} done add3d}}
\\\includegraphics[keepaspectratio=true,height=6cm,width=\textwidth]{img/gmic_stdlib457.jpg}\\
{\footnotesize \textbf{Example 457~:} \texttt{repeat 20 torus3d 15{\comma}2 color3d[-1] \$\{-RGB\} mul3d[-1] 0.5{\comma}1 if \{\$$>$\%2\} rotate3d[-1] 0{\comma}1{\comma}0{\comma}90 endif add3d[-1] 70 add3d rotate3d[-1] 0{\comma}0{\comma}1{\comma}18 done double3d 0}}
\end{center}

\subsection{\emph{animate3d\index{animate3d}} }\vspace*{-0.7em}
~\\\textbf{\Cb{Arguments: }}\begin{flushleft}
{\small \Cb{\hspace*{0.5cm}$\bullet$~~\texttt{\_width$>$0{\comma}\_height$>$0{\comma}\_angle\_dx{\comma}\_angle\_dy{\comma}\_angle\_dz{\comma}\_zoom\_facto\-r$>$=0{\comma}\_filename}}}\end{flushleft}
Animate selected 3d objects in a window.
~\\If argument 'filename' is provided{\comma} each frame of the animation is saved as a numbered filename.
\begin{flushleft}\Cc{\textbf{Default values}:\\~\\\hspace*{0.5cm}{\small $\bullet$~~\texttt{'width=640'{\comma} 'height=480'{\comma} 'angle\_dx=0'{\comma} 'angle\_dy=1'{\comma} 'angle\_dz=0'{\comma} 'zoom\_factor=1'} and \texttt{'filename=(undefined)'.}}}\end{flushleft}


\subsection{\emph{apply\_camera3d\index{apply\_camera3d}} }\vspace*{-0.7em}
~\\\textbf{\Cb{Arguments: }}\begin{flushleft}
{\small \Cb{\hspace*{0.5cm}$\bullet$~~\texttt{pos\_x{\comma}pos\_y{\comma}pos\_z{\comma}target\_x{\comma}target\_y{\comma}target\_z{\comma}up\_x{\comma}up\_y{\comma}up\_z}}}\end{flushleft}
Apply 3d camera matrix to selected 3d objects.
\begin{flushleft}\Cc{\textbf{Default values}:\\~\\\hspace*{0.5cm}{\small $\bullet$~~\texttt{'target\_x=0'{\comma} 'target\_y=0'{\comma} 'target\_z=0'{\comma} 'up\_x=0'{\comma} 'up\_y=-1'} and \texttt{'up\_z=0'.}}}\end{flushleft}


\subsection{\emph{apply\_matrix3d\index{apply\_matrix3d}} }\vspace*{-0.7em}
~\\\textbf{\Cb{Arguments: }}\begin{flushleft}
{\small \Cb{\hspace*{0.5cm}$\bullet$~~\texttt{a11{\comma}a12{\comma}a13{\comma}...{\comma}a31{\comma}a32{\comma}a33}}}\end{flushleft}
Apply specified 3d rotation matrix to selected 3d objects.
\begin{center}\includegraphics[keepaspectratio=true,height=6cm,width=\textwidth]{img/gmic_stdlib458.jpg}\\
{\footnotesize \textbf{Example 458~:} \texttt{torus3d 10{\comma}1 --apply\_matrix3d \{mul(rot(1{\comma}0{\comma}1{\comma}-15){\comma}[1{\comma}0{\comma}0{\comma}0{\comma}2{\comma}0{\comma}0{\comma}0{\comma}8]{\comma}3)\} double3d 0}}
\end{center}

\subsection{\emph{array3d\index{array3d}} }\vspace*{-0.7em}
~\\\textbf{\Cb{Arguments: }}\begin{flushleft}
{\small \Cb{\hspace*{0.5cm}$\bullet$~~\texttt{size\_x$>$=1{\comma}\_size\_y$>$=1{\comma}\_size\_z$>$=1{\comma}\_offset\_x[\%]{\comma}\_offset\_y[\%]{\comma}\_o\-ffset\_y[\%]}}}\end{flushleft}
Duplicate a 3d object along the X{\comma}Y and Z axes.
\begin{flushleft}\Cc{\textbf{Default values}:\\~\\\hspace*{0.5cm}{\small $\bullet$~~\texttt{'size\_y=1'{\comma} 'size\_z=1'} and \texttt{'offset\_x=offset\_y=offset\_z=100\%'.}}}\end{flushleft}
\begin{center}\includegraphics[keepaspectratio=true,height=6cm,width=\textwidth]{img/gmic_stdlib459.jpg}\\
{\footnotesize \textbf{Example 459~:} \texttt{torus3d 10{\comma}1 --array3d 5{\comma}5{\comma}5{\comma}110\%{\comma}110\%{\comma}300\%}}
\end{center}

\subsection{\emph{arrow3d\index{arrow3d}} }\vspace*{-0.7em}
~\\\textbf{\Cb{Arguments: }}\begin{flushleft}
{\small \Cb{\hspace*{0.5cm}$\bullet$~~\texttt{x0{\comma}y0{\comma}z0{\comma}x1{\comma}y1{\comma}z1{\comma}\_radius[\%]$>$=0{\comma}\_head\_length[\%]$>$=0{\comma}\_head\_rad\-ius[\%]$>$=0}}}\end{flushleft}
Input 3d arrow with specified starting and ending 3d points.
\begin{flushleft}\Cc{\textbf{Default values}:\\~\\\hspace*{0.5cm}{\small $\bullet$~~\texttt{'radius=5\%'{\comma} 'head\_length=25\%'} and \texttt{'head\_radius=15\%'.}}}\end{flushleft}
\begin{center}\includegraphics[keepaspectratio=true,height=6cm,width=\textwidth]{img/gmic_stdlib460.jpg}\\
{\footnotesize \textbf{Example 460~:} \texttt{repeat 10 a=\{\$$>$*2*pi/10\} arrow3d 0{\comma}0{\comma}0{\comma}\{cos(\$a)\}{\comma}\{sin(\$a)\}{\comma}-0.5 done +3d}}
\end{center}

\subsection{\emph{axes3d\index{axes3d}} }\vspace*{-0.7em}
~\\\textbf{\Cb{Arguments: }}\begin{flushleft}
{\small \Cb{\hspace*{0.5cm}$\bullet$~~\texttt{\_size\_x{\comma}\_size\_y{\comma}\_size\_z{\comma}\_font\_size$>$0{\comma}\_label\_x{\comma}\_label\_y{\comma}\_labe\-l\_z}}}\end{flushleft}
Input 3d axes with specified sizes along the x{\comma}y and z orientations.
\begin{flushleft}\Cc{\textbf{Default values}:\\~\\\hspace*{0.5cm}{\small $\bullet$~~\texttt{'size\_x=size\_y=size\_z=1'{\comma} 'font\_size=23'{\comma} 'label\_x=X'{\comma} 'label\_y=Y'} and \texttt{'label\_z=Z'.}}}\end{flushleft}
\begin{center}\includegraphics[keepaspectratio=true,height=6cm,width=\textwidth]{img/gmic_stdlib461.jpg}\\
{\footnotesize \textbf{Example 461~:} \texttt{axes3d {\comma}}}
\end{center}

\subsection{\emph{box3d\index{box3d}} }\vspace*{-0.7em}
~\\\textbf{\Cb{Arguments: }}\begin{flushleft}
{\small \Cb{\hspace*{0.5cm}$\bullet$~~\texttt{\_size\_x{\comma}\_size\_y{\comma}\_size\_z}}}\end{flushleft}
Input 3d box at (0{\comma}0{\comma}0){\comma} with specified geometry.
\begin{flushleft}\Cc{\textbf{Default values}:\\~\\\hspace*{0.5cm}{\small $\bullet$~~\texttt{'size\_x=1'} and \texttt{'size\_z=size\_y=size\_x'.}}}\end{flushleft}
\begin{center}\includegraphics[keepaspectratio=true,height=6cm,width=\textwidth]{img/gmic_stdlib462.jpg}\\
{\footnotesize \textbf{Example 462~:} \texttt{box3d 100{\comma}40{\comma}30 --primitives3d 1 color3d[-2] \$\{-RGB\}}}
\end{center}

\subsection{\emph{center3d\index{center3d}} }\vspace*{-0.7em}
Center selected 3d objects at (0{\comma}0{\comma}0).
~\\(\emph{eq. to} {\small \texttt{'c3d'}}).
\begin{center}\includegraphics[keepaspectratio=true,height=6cm,width=\textwidth]{img/gmic_stdlib463.jpg}\\
{\footnotesize \textbf{Example 463~:} \texttt{repeat 100 circle3d \{u(100)\}{\comma}\{u(100)\}{\comma}\{u(100)\}{\comma}2 done add3d color3d[-1] 255{\comma}0{\comma}0 --center3d color3d[-1] 0{\comma}255{\comma}0 add3d}}
\end{center}

\subsection{\emph{circle3d\index{circle3d}} }\vspace*{-0.7em}
~\\\textbf{\Cb{Arguments: }}\begin{flushleft}
{\small \Cb{\hspace*{0.5cm}$\bullet$~~\texttt{\_x0{\comma}\_y0{\comma}\_z0{\comma}\_radius$>$=0}}}\end{flushleft}
Input 3d circle at specified coordinates.
\begin{flushleft}\Cc{\textbf{Default values}:\\~\\\hspace*{0.5cm}{\small $\bullet$~~\texttt{'x0=y0=z0=0'} and \texttt{'radius=1'.}}}\end{flushleft}
\begin{center}\includegraphics[keepaspectratio=true,height=6cm,width=\textwidth]{img/gmic_stdlib464.jpg}\\
{\footnotesize \textbf{Example 464~:} \texttt{repeat 500 a=\{\$$>$*pi/250\} circle3d \{cos(3*\$a)\}{\comma}\{sin(2*\$a)\}{\comma}0{\comma}\{\$a/50\} color3d[-1] \$\{-RGB\}{\comma}0.4 done add3d}}
\end{center}

\subsection{\emph{circles3d\index{circles3d}} }\vspace*{-0.7em}
~\\\textbf{\Cb{Arguments: }}\begin{flushleft}
{\small \Cb{\hspace*{0.5cm}$\bullet$~~\texttt{\_radius$>$=0{\comma}\_is\_filled=\{ 0 ~$|$~ 1 \}}}}\end{flushleft}
Convert specified 3d objects to sets of 3d circles with specified radius.
\begin{flushleft}\Cc{\textbf{Default values}:\\~\\\hspace*{0.5cm}{\small $\bullet$~~\texttt{'radius=1'} and \texttt{'is\_filled=1'.}}}\end{flushleft}
\begin{center}\includegraphics[keepaspectratio=true,height=6cm,width=\textwidth]{img/gmic_stdlib465.jpg}\\
{\footnotesize \textbf{Example 465~:} \texttt{image.jpg luminance resize2dy 40 threshold 50\% * 255 pointcloud3d color3d[-1] 255{\comma}255{\comma}255 circles3d 0.7}}
\end{center}

\subsection{\emph{color3d\index{color3d}} (+)}\vspace*{-0.7em}
~\\\textbf{\Cb{Arguments: }}\begin{flushleft}
{\small \Cb{\hspace*{0.5cm}$\bullet$~~\texttt{R{\comma}\_G{\comma}\_B{\comma}\_opacity}}}\end{flushleft}
Set color and opacity of selected 3d objects.
~\\(\emph{eq. to} {\small \texttt{'col3d'}}).
\begin{flushleft}\Cc{\textbf{Default value}:\\~\\\hspace*{0.5cm}{\small $\bullet$~~\texttt{'B=G=R'} and \texttt{'opacity=(undefined)'.}}}\end{flushleft}
\begin{center}\includegraphics[keepaspectratio=true,height=6cm,width=\textwidth]{img/gmic_stdlib466.jpg}\\
{\footnotesize \textbf{Example 466~:} \texttt{torus3d 100{\comma}10 double3d 0 repeat 7 --rotate3d[-1] 1{\comma}0{\comma}0{\comma}20 color3d[-1] \$\{-RGB\} done add3d}}
\end{center}

\subsection{\emph{colorcube3d\index{colorcube3d}} }\vspace*{-0.7em}
Input 3d color cube.
\begin{center}\includegraphics[keepaspectratio=true,height=6cm,width=\textwidth]{img/gmic_stdlib467.jpg}\\
{\footnotesize \textbf{Example 467~:} \texttt{colorcube3d mode3d 2 --primitives3d 1}}
\end{center}

\subsection{\emph{cone3d\index{cone3d}} }\vspace*{-0.7em}
~\\\textbf{\Cb{Arguments: }}\begin{flushleft}
{\small \Cb{\hspace*{0.5cm}$\bullet$~~\texttt{\_radius{\comma}\_height{\comma}\_nb\_subdivisions$>$0}}}\end{flushleft}
Input 3d cone at (0{\comma}0{\comma}0){\comma} with specified geometry.
\begin{flushleft}\Cc{\textbf{Default value}:\\~\\\hspace*{0.5cm}{\small $\bullet$~~\texttt{'radius=1'{\comma}'height=1'} and \texttt{'nb\_subdivisions=24'.}}}\end{flushleft}
\begin{center}\includegraphics[keepaspectratio=true,height=6cm,width=\textwidth]{img/gmic_stdlib468.jpg}\\
{\footnotesize \textbf{Example 468~:} \texttt{cone3d 10{\comma}40 --primitives3d 1 color3d[-2] \$\{-RGB\}}}
\end{center}

\subsection{\emph{cubes3d\index{cubes3d}} }\vspace*{-0.7em}
~\\\textbf{\Cb{Arguments: }}\begin{flushleft}
{\small \Cb{\hspace*{0.5cm}$\bullet$~~\texttt{\_size$>$=0}}}\end{flushleft}
Convert specified 3d objects to sets of 3d cubes with specified size.
\begin{flushleft}\Cc{\textbf{Default value}:\\~\\\hspace*{0.5cm}{\small $\bullet$~~\texttt{'size=1'.}}}\end{flushleft}
\begin{center}\includegraphics[keepaspectratio=true,height=6cm,width=\textwidth]{img/gmic_stdlib469.jpg}\\
{\footnotesize \textbf{Example 469~:} \texttt{image.jpg luminance resize2dy 40 threshold 50\% * 255 pointcloud3d color3d[-1] 255{\comma}255{\comma}255 cubes3d 1}}
\end{center}

\subsection{\emph{cup3d\index{cup3d}} }\vspace*{-0.7em}
~\\\textbf{\Cb{Arguments: }}\begin{flushleft}
{\small \Cb{\hspace*{0.5cm}$\bullet$~~\texttt{\_resolution$>$0}}}\end{flushleft}
Input 3d cup object.
\begin{center}\includegraphics[keepaspectratio=true,height=6cm,width=\textwidth]{img/gmic_stdlib470.jpg}\\
{\footnotesize \textbf{Example 470~:} \texttt{cup3d {\comma}}}
\end{center}

\subsection{\emph{cylinder3d\index{cylinder3d}} }\vspace*{-0.7em}
~\\\textbf{\Cb{Arguments: }}\begin{flushleft}
{\small \Cb{\hspace*{0.5cm}$\bullet$~~\texttt{\_radius{\comma}\_height{\comma}\_nb\_subdivisions$>$0}}}\end{flushleft}
Input 3d cylinder at (0{\comma}0{\comma}0){\comma} with specified geometry.
\begin{flushleft}\Cc{\textbf{Default value}:\\~\\\hspace*{0.5cm}{\small $\bullet$~~\texttt{'radius=1'{\comma}'height=1'} and \texttt{'nb\_subdivisions=24'.}}}\end{flushleft}
\begin{center}\includegraphics[keepaspectratio=true,height=6cm,width=\textwidth]{img/gmic_stdlib471.jpg}\\
{\footnotesize \textbf{Example 471~:} \texttt{cylinder3d 10{\comma}40 --primitives3d 1 color3d[-2] \$\{-RGB\}}}
\end{center}

\subsection{\emph{delaunay3d\index{delaunay3d}} }\vspace*{-0.7em}
Generate 3d delaunay triangulations from selected images.
~\\One assumes that the selected input images are binary images containing the set of points to mesh.
~\\The output 3d object is a mesh composed of non-oriented triangles.
\begin{center}\includegraphics[keepaspectratio=true,height=6cm,width=\textwidth]{img/gmic_stdlib472.jpg}\\
{\footnotesize \textbf{Example 472~:} \texttt{500{\comma}500 noise 0.05{\comma}2 * 255 --delaunay3d color3d[1] 255{\comma}128{\comma}0 dilate\_circ[0] 5 to\_rgb[0] --object3d[0] [1]{\comma}0{\comma}0{\comma}0{\comma}1{\comma}1 max[-1] [0]}}
\end{center}

\subsection{\emph{distribution3d\index{distribution3d}} }\vspace*{-0.7em}
Get 3d color distribution of selected images.
\begin{center}\includegraphics[keepaspectratio=true,height=6cm,width=\textwidth]{img/gmic_stdlib473.jpg}\\
{\footnotesize \textbf{Example 473~:} \texttt{image.jpg distribution3d colorcube3d primitives3d[-1] 1 add3d}}
\end{center}

\subsection{\emph{div3d\index{div3d}} (+)}\vspace*{-0.7em}
~\\\textbf{\Cb{Arguments: }}\begin{flushleft}
{\small \Cb{\hspace*{0.5cm}$\bullet$~~\texttt{factor}}}~~~\\
{\small \Cb{\hspace*{0.5cm}$\bullet$~~\texttt{factor\_x{\comma}factor\_y{\comma}\_factor\_z}}}\end{flushleft}
Scale selected 3d objects isotropically or anisotropically{\comma} with the inverse of specified
factors.
~\\(\emph{eq. to} {\small \texttt{'/3d'}}).
\begin{flushleft}\Cc{\textbf{Default value}:\\~\\\hspace*{0.5cm}{\small $\bullet$~~\texttt{'factor\_z=0'.}}}\end{flushleft}
\begin{center}\includegraphics[keepaspectratio=true,height=6cm,width=\textwidth]{img/gmic_stdlib474.jpg}\\
{\footnotesize \textbf{Example 474~:} \texttt{torus3d 5{\comma}2 repeat 5 --add3d[-1] 12{\comma}0{\comma}0 div3d[-1] 1.2 color3d[-1] \$\{-RGB\} done add3d}}
\end{center}

\subsection{\emph{double3d\index{double3d}} (+)}\vspace*{-0.7em}
~\\\textbf{\Cb{Arguments: }}\begin{flushleft}
{\small \Cb{\hspace*{0.5cm}$\bullet$~~\texttt{\_is\_double\_sided=\{ 0 ~$|$~ 1 \}}}}\end{flushleft}
Enable/disable double-sided mode for 3d rendering.
~\\(\emph{eq. to} {\small \texttt{'db3d'}}).
\begin{flushleft}\Cc{\textbf{Default value}:\\~\\\hspace*{0.5cm}{\small $\bullet$~~\texttt{'is\_double\_sided=1'.}}}\end{flushleft}
\begin{center}\includegraphics[keepaspectratio=true,height=6cm,width=\textwidth]{img/gmic_stdlib475.jpg}\\
{\footnotesize \textbf{Example 475~:} \texttt{mode3d 1 repeat 2 torus3d 100{\comma}30 rotate3d[-1] 1{\comma}1{\comma}0{\comma}60 double3d \$$>$ snapshot3d[-1] 400 done}}
\end{center}

\subsection{\emph{elevation3d\index{elevation3d}} (+)}\vspace*{-0.7em}
~\\\textbf{\Cb{Arguments: }}\begin{flushleft}
{\small \Cb{\hspace*{0.5cm}$\bullet$~~\texttt{z-factor}}}~~~\\
{\small \Cb{\hspace*{0.5cm}$\bullet$~~\texttt{[elevation\_map]}}}~~~\\
{\small \Cb{\hspace*{0.5cm}$\bullet$~~\texttt{'formula'}}}~~~\\
{\small \Cb{\hspace*{0.5cm}$\bullet$~~\texttt{(no arg)}}}\end{flushleft}
Build 3d elevation of selected images{\comma} with a specified elevation map.
~\\When invoked with (no arg) or 'z-factor'{\comma} the elevation map is computed as the pointwise L2 norm of the
pixel values. Otherwise{\comma} the elevation map is taken from the specified image or formula.
\begin{center}\includegraphics[keepaspectratio=true,height=6cm,width=\textwidth]{img/gmic_stdlib476.jpg}\\
{\footnotesize \textbf{Example 476~:} \texttt{image.jpg blur 5 elevation3d 0.5}}
\\\includegraphics[keepaspectratio=true,height=6cm,width=\textwidth]{img/gmic_stdlib477.jpg}\\
{\footnotesize \textbf{Example 477~:} \texttt{128{\comma}128{\comma}1{\comma}3{\comma}u(255) plasma 10{\comma}3 blur 4 sharpen 10000 elevation3d[-1] 'X=(x-64)/6;Y=(y-64)/6;-100*exp(-(X\textasciicircum 2+Y\textasciicircum 2)/30)*abs(cos(X)*sin(Y))'}}
\end{center}

\subsection{\emph{empty3d\index{empty3d}} }\vspace*{-0.7em}
Input empty 3d object.
\begin{center}\includegraphics[keepaspectratio=true,height=6cm,width=\textwidth]{img/gmic_stdlib478.jpg}\\
{\footnotesize \textbf{Example 478~:} \texttt{empty3d}}
\end{center}

\subsection{\emph{extrude3d\index{extrude3d}} }\vspace*{-0.7em}
~\\\textbf{\Cb{Arguments: }}\begin{flushleft}
{\small \Cb{\hspace*{0.5cm}$\bullet$~~\texttt{\_depth$>$0{\comma}\_resolution$>$0{\comma}\_smoothness[\%]$>$=0}}}\end{flushleft}
Generate extruded 3d object from selected binary XY-profiles.
\begin{flushleft}\Cc{\textbf{Default values}:\\~\\\hspace*{0.5cm}{\small $\bullet$~~\texttt{'depth=16'{\comma} 'resolution=1024'} and \texttt{'smoothness=0.5\%'.}}}\end{flushleft}
\begin{center}\includegraphics[keepaspectratio=true,height=6cm,width=\textwidth]{img/gmic_stdlib479.jpg}\\
{\footnotesize \textbf{Example 479~:} \texttt{image.jpg threshold 50\% extrude3d 16}}
\end{center}

\subsection{\emph{focale3d\index{focale3d}} (+)}\vspace*{-0.7em}
~\\\textbf{\Cb{Arguments: }}\begin{flushleft}
{\small \Cb{\hspace*{0.5cm}$\bullet$~~\texttt{focale}}}\end{flushleft}
Set 3d focale.
~\\(\emph{eq. to} {\small \texttt{'f3d').\textbackslash n}}).
~\\Set 'focale' to 0 to enable parallel projection (instead of perspective).
~\\Set negative 'focale' will disable 3d sprite zooming.
\begin{flushleft}\Cc{\textbf{Default value}:\\~\\\hspace*{0.5cm}{\small $\bullet$~~\texttt{'focale=700'.}}}\end{flushleft}
\begin{center}\includegraphics[keepaspectratio=true,height=6cm,width=\textwidth]{img/gmic_stdlib480.jpg}\\
{\footnotesize \textbf{Example 480~:} \texttt{repeat 5 torus3d 100{\comma}30 rotate3d[-1] 1{\comma}1{\comma}0{\comma}60 focale3d \{\$$<$*90\} snapshot3d[-1] 400 done remove[0]}}
\end{center}

\subsection{\emph{gaussians3d\index{gaussians3d}} }\vspace*{-0.7em}
~\\\textbf{\Cb{Arguments: }}\begin{flushleft}
{\small \Cb{\hspace*{0.5cm}$\bullet$~~\texttt{\_size$>$0{\comma}\_opacity}}}\end{flushleft}
Convert selected 3d objects into set of 3d gaussian-shaped sprites.
\begin{center}\includegraphics[keepaspectratio=true,height=6cm,width=\textwidth]{img/gmic_stdlib481.jpg}\\
{\footnotesize \textbf{Example 481~:} \texttt{image.jpg r2dy 32 distribution3d gaussians3d 20 colorcube3d primitives3d[-1] 1 +3d}}
\end{center}

\subsection{\emph{gmic3d\index{gmic3d}} }\vspace*{-0.7em}
Input a 3d G'MIC logo.
\begin{center}\includegraphics[keepaspectratio=true,height=6cm,width=\textwidth]{img/gmic_stdlib482.jpg}\\
{\footnotesize \textbf{Example 482~:} \texttt{gmic3d --primitives3d 1}}
\end{center}

\subsection{\emph{gyroid3d\index{gyroid3d}} }\vspace*{-0.7em}
~\\\textbf{\Cb{Arguments: }}\begin{flushleft}
{\small \Cb{\hspace*{0.5cm}$\bullet$~~\texttt{\_resolution$>$0{\comma}\_zoom}}}\end{flushleft}
Input 3d gyroid at (0{\comma}0{\comma}0){\comma} with specified resolution.
\begin{flushleft}\Cc{\textbf{Default values}:\\~\\\hspace*{0.5cm}{\small $\bullet$~~\texttt{'resolution=32'} and \texttt{'zoom=5'.}}}\end{flushleft}
\begin{center}\includegraphics[keepaspectratio=true,height=6cm,width=\textwidth]{img/gmic_stdlib483.jpg}\\
{\footnotesize \textbf{Example 483~:} \texttt{gyroid3d 48 --primitives3d 1}}
\end{center}

\subsection{\emph{histogram3d\index{histogram3d}} }\vspace*{-0.7em}
Get 3d color histogram of selected images.
\begin{center}\includegraphics[keepaspectratio=true,height=6cm,width=\textwidth]{img/gmic_stdlib484.jpg}\\
{\footnotesize \textbf{Example 484~:} \texttt{image.jpg histogram3d colorcube3d primitives3d[-1] 1 add3d}}
\end{center}

\subsection{\emph{image6cube3d\index{image6cube3d}} }\vspace*{-0.7em}
Generate 3d mapped cubes from 6-sets of selected images.
\begin{center}\includegraphics[keepaspectratio=true,height=6cm,width=\textwidth]{img/gmic_stdlib485.jpg}\\
{\footnotesize \textbf{Example 485~:} \texttt{image.jpg animate flower{\comma}"30{\comma}0"{\comma}"30{\comma}5"{\comma}6 image6cube3d}}
\end{center}

\subsection{\emph{imageblocks3d\index{imageblocks3d}} }\vspace*{-0.7em}
~\\\textbf{\Cb{Arguments: }}\begin{flushleft}
{\small \Cb{\hspace*{0.5cm}$\bullet$~~\texttt{\_maximum\_elevation{\comma}\_smoothness[\%]$>$=0}}}\end{flushleft}
Generate 3d blocks from selected images.
~\\Transparency of selected images is taken into account.
\begin{flushleft}\Cc{\textbf{Default values}:\\~\\\hspace*{0.5cm}{\small $\bullet$~~\texttt{'maximum\_elevation=10'} and \texttt{'smoothness=0'.}}}\end{flushleft}
\begin{center}\includegraphics[keepaspectratio=true,height=6cm,width=\textwidth]{img/gmic_stdlib486.jpg}\\
{\footnotesize \textbf{Example 486~:} \texttt{image.jpg resize2dy 32 imageblocks3d -20 mode3d 3}}
\end{center}

\subsection{\emph{imagecube3d\index{imagecube3d}} }\vspace*{-0.7em}
Generate 3d mapped cubes from selected images.
\begin{center}\includegraphics[keepaspectratio=true,height=6cm,width=\textwidth]{img/gmic_stdlib487.jpg}\\
{\footnotesize \textbf{Example 487~:} \texttt{image.jpg imagecube3d}}
\end{center}

\subsection{\emph{imageplane3d\index{imageplane3d}} }\vspace*{-0.7em}
Generate 3d mapped planes from selected images.
\begin{center}\includegraphics[keepaspectratio=true,height=6cm,width=\textwidth]{img/gmic_stdlib488.jpg}\\
{\footnotesize \textbf{Example 488~:} \texttt{image.jpg imageplane3d}}
\end{center}

\subsection{\emph{imagepyramid3d\index{imagepyramid3d}} }\vspace*{-0.7em}
Generate 3d mapped pyramides from selected images.
\begin{center}\includegraphics[keepaspectratio=true,height=6cm,width=\textwidth]{img/gmic_stdlib489.jpg}\\
{\footnotesize \textbf{Example 489~:} \texttt{image.jpg imagepyramid3d}}
\end{center}

\subsection{\emph{imagerubik3d\index{imagerubik3d}} }\vspace*{-0.7em}
~\\\textbf{\Cb{Arguments: }}\begin{flushleft}
{\small \Cb{\hspace*{0.5cm}$\bullet$~~\texttt{\_xy\_tiles$>$=1{\comma}0$<$=xy\_shift$<$=100{\comma}0$<$=z\_shift$<$=100}}}\end{flushleft}
Generate 3d mapped rubik's cubes from selected images.
\begin{flushleft}\Cc{\textbf{Default values}:\\~\\\hspace*{0.5cm}{\small $\bullet$~~\texttt{'xy\_tiles=3'{\comma} 'xy\_shift=5'} and \texttt{'z\_shift=5'.}}}\end{flushleft}
\begin{center}\includegraphics[keepaspectratio=true,height=6cm,width=\textwidth]{img/gmic_stdlib490.jpg}\\
{\footnotesize \textbf{Example 490~:} \texttt{image.jpg imagerubik3d {\comma}}}
\end{center}

\subsection{\emph{imagesphere3d\index{imagesphere3d}} }\vspace*{-0.7em}
~\\\textbf{\Cb{Arguments: }}\begin{flushleft}
{\small \Cb{\hspace*{0.5cm}$\bullet$~~\texttt{\_resolution1$>$=3{\comma}\_resolution2$>$=3}}}\end{flushleft}
Generate 3d mapped sphere from selected images.
\begin{flushleft}\Cc{\textbf{Default values}:\\~\\\hspace*{0.5cm}{\small $\bullet$~~\texttt{'resolution1=32'} and \texttt{'resolutions2=16'.}}}\end{flushleft}
\begin{center}\includegraphics[keepaspectratio=true,height=6cm,width=\textwidth]{img/gmic_stdlib491.jpg}\\
{\footnotesize \textbf{Example 491~:} \texttt{image.jpg imagesphere3d 32{\comma}16}}
\end{center}

\subsection{\emph{isoline3d\index{isoline3d}} (+)}\vspace*{-0.7em}
~\\\textbf{\Cb{Arguments: }}\begin{flushleft}
{\small \Cb{\hspace*{0.5cm}$\bullet$~~\texttt{isovalue[\%]}}}~~~\\
{\small \Cb{\hspace*{0.5cm}$\bullet$~~\texttt{'formula'{\comma}value{\comma}\_x0{\comma}\_y0{\comma}\_x1{\comma}\_y1{\comma}\_size\_x$>$0[\%]{\comma}\_size\_y$>$0[\%]}}}\end{flushleft}
Extract 3d isolines with specified value from selected images or from specified formula.
\begin{flushleft}\Cc{\textbf{Default values}:\\~\\\hspace*{0.5cm}{\small $\bullet$~~\texttt{'x0=y0=-3'{\comma} 'x1=y1=3'} and \texttt{'size\_x=size\_y=256'.}}}\end{flushleft}
\begin{center}\includegraphics[keepaspectratio=true,height=6cm,width=\textwidth]{img/gmic_stdlib492.jpg}\\
{\footnotesize \textbf{Example 492~:} \texttt{image.jpg blur 1 isoline3d 50\%}}
\\\includegraphics[keepaspectratio=true,height=6cm,width=\textwidth]{img/gmic_stdlib493.jpg}\\
{\footnotesize \textbf{Example 493~:} \texttt{isoline3d 'X=x-w/2;Y=y-h/2;(X\textasciicircum 2+Y\textasciicircum 2)\%20'{\comma}10{\comma}-10{\comma}-10{\comma}10{\comma}10}}
\end{center}

\subsection{\emph{isosurface3d\index{isosurface3d}} (+)}\vspace*{-0.7em}
~\\\textbf{\Cb{Arguments: }}\begin{flushleft}
{\small \Cb{\hspace*{0.5cm}$\bullet$~~\texttt{isovalue[\%]}}}~~~\\
{\small \Cb{\hspace*{0.5cm}$\bullet$~~\texttt{'formula'{\comma}value{\comma}\_x0{\comma}\_y0{\comma}\_z0{\comma}\_x1{\comma}\_y1{\comma}\_z1{\comma}\_size\_x$>$0[\%]{\comma}\_size\_y\-$>$0[\%]{\comma}\_size\_z$>$0[\%]}}}\end{flushleft}
Extract 3d isosurfaces with specified value from selected images or from specified formula.
\begin{flushleft}\Cc{\textbf{Default values}:\\~\\\hspace*{0.5cm}{\small $\bullet$~~\texttt{'x0=y0=z0=-3'{\comma} 'x1=y1=z1=3'} and \texttt{'size\_x=size\_y=size\_z=32'.}}}\end{flushleft}
\begin{center}\includegraphics[keepaspectratio=true,height=6cm,width=\textwidth]{img/gmic_stdlib494.jpg}\\
{\footnotesize \textbf{Example 494~:} \texttt{image.jpg resize2dy 128 luminance threshold 50\% expand\_z 2{\comma}0 blur 1 isosurface3d 50\% mul3d 1{\comma}1{\comma}30}}
\\\includegraphics[keepaspectratio=true,height=6cm,width=\textwidth]{img/gmic_stdlib495.jpg}\\
{\footnotesize \textbf{Example 495~:} \texttt{isosurface3d 'x\textasciicircum 2+y\textasciicircum 2+abs(z)\textasciicircum abs(4*cos(x*y*z*3))'{\comma}3}}
\end{center}

\subsection{\emph{label3d\index{label3d}} }\vspace*{-0.7em}
~\\\textbf{\Cb{Arguments: }}\begin{flushleft}
{\small \Cb{\hspace*{0.5cm}$\bullet$~~\texttt{"text"{\comma}font\_height$>$=0{\comma}\_opacity{\comma}\_color1{\comma}...}}}\end{flushleft}
Generate 3d text label.
\begin{flushleft}\Cc{\textbf{Default values}:\\~\\\hspace*{0.5cm}{\small $\bullet$~~\texttt{'font\_height=13'{\comma} 'opacity=1'} and \texttt{'color=255{\comma}255{\comma}255'.}}}\end{flushleft}


\subsection{\emph{label\_points3d\index{label\_points3d}} }\vspace*{-0.7em}
~\\\textbf{\Cb{Arguments: }}\begin{flushleft}
{\small \Cb{\hspace*{0.5cm}$\bullet$~~\texttt{\_label\_size$>$0{\comma}\_opacity}}}\end{flushleft}
Add a numbered label to all vertices of selected 3d objects.
\begin{flushleft}\Cc{\textbf{Default values}:\\~\\\hspace*{0.5cm}{\small $\bullet$~~\texttt{'label\_size=13'} and \texttt{'opacity=0.8'.}}}\end{flushleft}
\begin{center}\includegraphics[keepaspectratio=true,height=6cm,width=\textwidth]{img/gmic_stdlib496.jpg}\\
{\footnotesize \textbf{Example 496~:} \texttt{torus3d 100{\comma}40{\comma}6{\comma}6 label\_points3d 23{\comma}1 mode3d 1}}
\end{center}

\subsection{\emph{lathe3d\index{lathe3d}} }\vspace*{-0.7em}
~\\\textbf{\Cb{Arguments: }}\begin{flushleft}
{\small \Cb{\hspace*{0.5cm}$\bullet$~~\texttt{\_resolution$>$0{\comma}\_smoothness[\%]$>$=0{\comma}\_max\_angle$>$=0}}}\end{flushleft}
Generate 3d object from selected binary XY-profiles.
\begin{flushleft}\Cc{\textbf{Default values}:\\~\\\hspace*{0.5cm}{\small $\bullet$~~\texttt{'resolution=128'{\comma} 'smoothness=0.5\%'} and \texttt{'max\_angle=361'.}}}\end{flushleft}
\begin{center}\includegraphics[keepaspectratio=true,height=6cm,width=\textwidth]{img/gmic_stdlib497.jpg}\\
{\footnotesize \textbf{Example 497~:} \texttt{300{\comma}300 rand -1{\comma}1 blur 40 sign normalize 0{\comma}255 lathe3d {\comma}}}
\end{center}

\subsection{\emph{light3d\index{light3d}} (+)}\vspace*{-0.7em}
~\\\textbf{\Cb{Arguments: }}\begin{flushleft}
{\small \Cb{\hspace*{0.5cm}$\bullet$~~\texttt{position\_x{\comma}position\_y{\comma}position\_z}}}~~~\\
{\small \Cb{\hspace*{0.5cm}$\bullet$~~\texttt{[texture]}}}~~~\\
{\small \Cb{\hspace*{0.5cm}$\bullet$~~\texttt{(no arg)}}}\end{flushleft}
Set the light coordinates or the light texture for 3d rendering.
~\\(\emph{eq. to} {\small \texttt{'l3d').\textbackslash n}}).
~\\(no arg) resets the 3d light to default.
\begin{center}\includegraphics[keepaspectratio=true,height=6cm,width=\textwidth]{img/gmic_stdlib498.jpg}\\
{\footnotesize \textbf{Example 498~:} \texttt{torus3d 100{\comma}30 double3d 0 specs3d 1.2 repeat 5 light3d \{\$$>$*100\}{\comma}0{\comma}-300 --snapshot3d[0] 400 done remove[0]}}
\end{center}

\subsection{\emph{line3d\index{line3d}} }\vspace*{-0.7em}
~\\\textbf{\Cb{Arguments: }}\begin{flushleft}
{\small \Cb{\hspace*{0.5cm}$\bullet$~~\texttt{x0{\comma}y0{\comma}z0{\comma}x1{\comma}y1{\comma}z1}}}\end{flushleft}
Input 3d line at specified coordinates.
\begin{center}\includegraphics[keepaspectratio=true,height=6cm,width=\textwidth]{img/gmic_stdlib499.jpg}\\
{\footnotesize \textbf{Example 499~:} \texttt{repeat 100 a=\{\$$>$*pi/50\} line3d 0{\comma}0{\comma}0{\comma}\{cos(3*\$a)\}{\comma}\{sin(2*\$a)\}{\comma}0 color3d. \$\{-RGB\} done add3d}}
\end{center}

\subsection{\emph{lissajous3d\index{lissajous3d}} }\vspace*{-0.7em}
~\\\textbf{\Cb{Arguments: }}\begin{flushleft}
{\small \Cb{\hspace*{0.5cm}$\bullet$~~\texttt{resolution$>$1{\comma}a{\comma}A{\comma}b{\comma}B{\comma}c{\comma}C}}}\end{flushleft}
Input 3d lissajous curves (x(t)=sin(a*t+A*2*pi){\comma}y(t)=sin(b*t+B*2*pi){\comma}z(t)=sin(c*t+C*2*pi)).
\begin{flushleft}\Cc{\textbf{Default values}:\\~\\\hspace*{0.5cm}{\small $\bullet$~~\texttt{'resolution=1024'{\comma} 'a=2'{\comma} 'A=0'{\comma} 'b=1'{\comma} 'B=0'{\comma} 'c=0'} and \texttt{'C=0'.}}}\end{flushleft}
\begin{center}\includegraphics[keepaspectratio=true,height=6cm,width=\textwidth]{img/gmic_stdlib500.jpg}\\
{\footnotesize \textbf{Example 500~:} \texttt{lissajous3d {\comma}}}
\end{center}

\subsection{\emph{mode3d\index{mode3d}} (+)}\vspace*{-0.7em}
~\\\textbf{\Cb{Arguments: }}\begin{flushleft}
{\small \Cb{\hspace*{0.5cm}$\bullet$~~\texttt{\_mode}}}\end{flushleft}
Set static 3d rendering mode.
~\\(\emph{eq. to} {\small \texttt{'m3d').\textbackslash n}}).
~\\'mode' can be \{ -1=bounding-box ~$|$~ 0=dots ~$|$~ 1=wireframe ~$|$~ 2=flat ~$|$~ 3=flat-shaded ~$|$~ 4=gouraud-shaded ~$|$~ 5=phong-shaded \}.");
~\\Bounding-box mode ('mode==-1') is active only for the interactive 3d viewer.
\begin{flushleft}\Cc{\textbf{Default value}:\\~\\\hspace*{0.5cm}{\small $\bullet$~~\texttt{'mode=4'.}}}\end{flushleft}
\begin{center}\includegraphics[keepaspectratio=true,height=6cm,width=\textwidth]{img/gmic_stdlib501.jpg}\\
{\footnotesize \textbf{Example 501~:} \texttt{(0{\comma}1{\comma}2{\comma}3{\comma}4{\comma}5) double3d 0 repeat \{w\} torus3d 100{\comma}30 rotate3d[-1] 1{\comma}1{\comma}0{\comma}60 mode3d \{0{\comma}@\$$>$\} snapshot3d[-1] 300 done remove[0]}}
\end{center}

\subsection{\emph{moded3d\index{moded3d}} (+)}\vspace*{-0.7em}
~\\\textbf{\Cb{Arguments: }}\begin{flushleft}
{\small \Cb{\hspace*{0.5cm}$\bullet$~~\texttt{\_mode}}}\end{flushleft}
Set dynamic 3d rendering mode for interactive 3d viewer.
~\\(\emph{eq. to} {\small \texttt{'md3d').\textbackslash n}}).
~\\'mode' can be \{ -1=bounding-box ~$|$~ 0=dots ~$|$~ 1=wireframe ~$|$~ 2=flat ~$|$~ 3=flat-shaded ~$|$~ 4=gouraud-shaded ~$|$~ 5=phong-shaded \}.
\begin{flushleft}\Cc{\textbf{Default value}:\\~\\\hspace*{0.5cm}{\small $\bullet$~~\texttt{'mode=-1'.}}}\end{flushleft}


\subsection{\emph{mul3d\index{mul3d}} (+)}\vspace*{-0.7em}
~\\\textbf{\Cb{Arguments: }}\begin{flushleft}
{\small \Cb{\hspace*{0.5cm}$\bullet$~~\texttt{factor}}}~~~\\
{\small \Cb{\hspace*{0.5cm}$\bullet$~~\texttt{factor\_x{\comma}factor\_y{\comma}\_factor\_z}}}\end{flushleft}
Scale selected 3d objects isotropically or anisotropically{\comma} with specified factors.
~\\(\emph{eq. to} {\small \texttt{'*3d'}}).
\begin{flushleft}\Cc{\textbf{Default value}:\\~\\\hspace*{0.5cm}{\small $\bullet$~~\texttt{'factor\_z=0'.}}}\end{flushleft}
\begin{center}\includegraphics[keepaspectratio=true,height=6cm,width=\textwidth]{img/gmic_stdlib502.jpg}\\
{\footnotesize \textbf{Example 502~:} \texttt{torus3d 5{\comma}2 repeat 5 --add3d[-1] 10{\comma}0{\comma}0 mul3d[-1] 1.2 color3d[-1] \$\{-RGB\} done add3d}}
\end{center}

\subsection{\emph{normalize3d\index{normalize3d}} }\vspace*{-0.7em}
Normalize selected 3d objects to unit size.
~\\(\emph{eq. to} {\small \texttt{'n3d'}}).
\begin{center}\includegraphics[keepaspectratio=true,height=6cm,width=\textwidth]{img/gmic_stdlib503.jpg}\\
{\footnotesize \textbf{Example 503~:} \texttt{repeat 100 circle3d \{u(3)\}{\comma}\{u(3)\}{\comma}\{u(3)\}{\comma}0.1 done add3d color3d[-1] 255{\comma}0{\comma}0 --normalize3d[-1] color3d[-1] 0{\comma}255{\comma}0 add3d}}
\end{center}

\subsection{\emph{opacity3d\index{opacity3d}} (+)}\vspace*{-0.7em}
~\\\textbf{\Cb{Arguments: }}\begin{flushleft}
{\small \Cb{\hspace*{0.5cm}$\bullet$~~\texttt{\_opacity}}}\end{flushleft}
Set opacity of selected 3d objects.
~\\(\emph{eq. to} {\small \texttt{'o3d'}}).
\begin{flushleft}\Cc{\textbf{Default value}:\\~\\\hspace*{0.5cm}{\small $\bullet$~~\texttt{'opacity=1'.}}}\end{flushleft}
\begin{center}\includegraphics[keepaspectratio=true,height=6cm,width=\textwidth]{img/gmic_stdlib504.jpg}\\
{\footnotesize \textbf{Example 504~:} \texttt{torus3d 100{\comma}10 double3d 0 repeat 7 --rotate3d[-1] 1{\comma}0{\comma}0{\comma}20 opacity3d[-1] \{u\} done add3d}}
\end{center}

\subsection{\emph{parametric3d\index{parametric3d}} }\vspace*{-0.7em}
~\\\textbf{\Cb{Arguments: }}\begin{flushleft}
{\small \Cb{\hspace*{0.5cm}$\bullet$~~\texttt{\_x(a{\comma}b){\comma}\_y(a{\comma}b){\comma}\_z(a{\comma}b){\comma}\_amin{\comma}\_amax{\comma}\_bmin{\comma}\_bmax{\comma}\_res\_a$>$0{\comma}\_re\-s\_b$>$0{\comma}\_res\_x$>$0{\comma}\_res\_y$>$0{\comma}\_res\_z$>$0{\comma}\_smoothness$>$=0{\comma}\_isovalue$>$=0}}}\end{flushleft}
Input 3d object from specified parametric surface (x(a{\comma}b){\comma}y(a{\comma}b){\comma}z(a{\comma}b)).
\begin{flushleft}\Cc{\textbf{Default values}:\\~\\\hspace*{0.5cm}{\small $\bullet$~~\texttt{'x=(2+cos(b))*sin(a)'{\comma} 'y=(2+cos(b))*cos(a)'{\comma} 'c=sin(b)'{\comma} 'amin=-pi'{\comma} 'amax='pi'{\comma} 'bmin=-pi'{\comma} 'bmax='pi'{\comma}}}}\end{flushleft}
~\\'res\_a=512'{\comma} 'res\_b=res\_a'{\comma} 'res\_x=64'{\comma} 'res\_y=res\_x'{\comma} 'res\_z=res\_y'{\comma} 'smoothness=2\%' and 'isovalue=10\%'.
\begin{center}\includegraphics[keepaspectratio=true,height=6cm,width=\textwidth]{img/gmic_stdlib505.jpg}\\
{\footnotesize \textbf{Example 505~:} \texttt{parametric3d {\comma}}}
\end{center}

\subsection{\emph{pca\_patch3d\index{pca\_patch3d}} }\vspace*{-0.7em}
~\\\textbf{\Cb{Arguments: }}\begin{flushleft}
{\small \Cb{\hspace*{0.5cm}$\bullet$~~\texttt{\_patch\_size$>$0{\comma}\_M$>$0{\comma}\_N$>$0{\comma}\_normalize\_input=\{ 0 ~$|$~ 1 \}{\comma}\_normaliz\-e\_output=\{ 0 ~$|$~ 1 \}{\comma}\_lambda\_xy}}}\end{flushleft}
Get 3d patch-pca representation of selected images.
~\\The 3d patch-pca is estimated from M patches on the input image{\comma} and displayed as a cloud of N 3d points.
\begin{flushleft}\Cc{\textbf{Default values}:\\~\\\hspace*{0.5cm}{\small $\bullet$~~\texttt{'patch\_size=7'{\comma} 'M=1000'{\comma} 'N=3000'{\comma} 'normalize\_input=1'{\comma} 'normalize\_output=0'{\comma}} and \texttt{'lambda\_xy=0'.}}}\end{flushleft}
\begin{center}\includegraphics[keepaspectratio=true,height=6cm,width=\textwidth]{img/gmic_stdlib506.jpg}\\
{\footnotesize \textbf{Example 506~:} \texttt{image.jpg pca\_patch3d 7}}
\end{center}

\subsection{\emph{plane3d\index{plane3d}} }\vspace*{-0.7em}
~\\\textbf{\Cb{Arguments: }}\begin{flushleft}
{\small \Cb{\hspace*{0.5cm}$\bullet$~~\texttt{\_size\_x{\comma}\_size\_y{\comma}\_nb\_subdivisions\_x$>$0{\comma}\_nb\_subdisivions\_y$>$0}}}\end{flushleft}
Input 3d plane at (0{\comma}0{\comma}0){\comma} with specified geometry.
\begin{flushleft}\Cc{\textbf{Default values}:\\~\\\hspace*{0.5cm}{\small $\bullet$~~\texttt{'size\_x=1'{\comma} 'size\_y=size\_x'} and \texttt{'nb\_subdivisions\_x=nb\_subdivisions\_y=24'.}}}\end{flushleft}
\begin{center}\includegraphics[keepaspectratio=true,height=6cm,width=\textwidth]{img/gmic_stdlib507.jpg}\\
{\footnotesize \textbf{Example 507~:} \texttt{plane3d 50{\comma}30 --primitives3d 1 color3d[-2] \$\{-RGB\}}}
\end{center}

\subsection{\emph{point3d\index{point3d}} }\vspace*{-0.7em}
~\\\textbf{\Cb{Arguments: }}\begin{flushleft}
{\small \Cb{\hspace*{0.5cm}$\bullet$~~\texttt{x0{\comma}y0{\comma}z0}}}\end{flushleft}
Input 3d point at specified coordinates.
\begin{center}\includegraphics[keepaspectratio=true,height=6cm,width=\textwidth]{img/gmic_stdlib508.jpg}\\
{\footnotesize \textbf{Example 508~:} \texttt{repeat 1000 a=\{\$$>$*pi/500\} point3d \{cos(3*\$a)\}{\comma}\{sin(2*\$a)\}{\comma}0 color3d[-1] \$\{-RGB\} done add3d}}
\end{center}

\subsection{\emph{pointcloud3d\index{pointcloud3d}} }\vspace*{-0.7em}
Convert selected planar or volumetric images to 3d point clouds.
\begin{center}\includegraphics[keepaspectratio=true,height=6cm,width=\textwidth]{img/gmic_stdlib509.jpg}\\
{\footnotesize \textbf{Example 509~:} \texttt{image.jpg luminance resize2dy 100 threshold 50\% mul 255 pointcloud3d color3d[-1] 255{\comma}255{\comma}255}}
\end{center}

\subsection{\emph{pose3d\index{pose3d}} }\vspace*{-0.7em}
~\\\textbf{\Cb{Arguments: }}\begin{flushleft}
{\small \Cb{\hspace*{0.5cm}$\bullet$~~\texttt{p1{\comma}...{\comma}p12}}}\end{flushleft}
Apply 3d pose matrix to selected 3d objects.
\begin{center}\includegraphics[keepaspectratio=true,height=6cm,width=\textwidth]{img/gmic_stdlib510.jpg}\\
{\footnotesize \textbf{Example 510~:} \texttt{torus3d 100{\comma}20 pose3d 0.152437{\comma}1.20666{\comma}-0.546366{\comma}0{\comma}-0.535962{\comma}0.559129{\comma}1.08531{\comma}0{\comma}1.21132{\comma}0.0955431{\comma}0.548966{\comma}0{\comma}0{\comma}0{\comma}-206{\comma}1 snapshot3d 400}}
\end{center}

\subsection{\emph{primitives3d\index{primitives3d}} (+)}\vspace*{-0.7em}
~\\\textbf{\Cb{Arguments: }}\begin{flushleft}
{\small \Cb{\hspace*{0.5cm}$\bullet$~~\texttt{mode}}}\end{flushleft}
Convert primitives of selected 3d objects.
~\\(\emph{eq. to} {\small \texttt{'p3d').\textbackslash n}}).
~\\'mode' can be \{ 0=points ~$|$~ 1=segments ~$|$~ 2=non-textured \}.
\begin{center}\includegraphics[keepaspectratio=true,height=6cm,width=\textwidth]{img/gmic_stdlib511.jpg}\\
{\footnotesize \textbf{Example 511~:} \texttt{sphere3d 30 primitives3d 1 torus3d 50{\comma}10 color3d[-1] \$\{-RGB\} add3d}}
\end{center}

\subsection{\emph{projections3d\index{projections3d}} }\vspace*{-0.7em}
~\\\textbf{\Cb{Arguments: }}\begin{flushleft}
{\small \Cb{\hspace*{0.5cm}$\bullet$~~\texttt{\_x[\%]{\comma}\_y[\%]{\comma}\_z[\%]{\comma}\_is\_bounding\_box=\{ 0 ~$|$~ 1 \}}}}\end{flushleft}
Generate 3d xy{\comma}xz{\comma}yz projection planes from specified volumetric images.


\subsection{\emph{pyramid3d\index{pyramid3d}} }\vspace*{-0.7em}
~\\\textbf{\Cb{Arguments: }}\begin{flushleft}
{\small \Cb{\hspace*{0.5cm}$\bullet$~~\texttt{width{\comma}height}}}\end{flushleft}
Input 3d pyramid at (0{\comma}0{\comma}0){\comma} with specified geometry.
\begin{center}\includegraphics[keepaspectratio=true,height=6cm,width=\textwidth]{img/gmic_stdlib512.jpg}\\
{\footnotesize \textbf{Example 512~:} \texttt{pyramid3d 100{\comma}-100 --primitives3d 1 color3d[-2] \$\{-RGB\}}}
\end{center}

\subsection{\emph{quadrangle3d\index{quadrangle3d}} }\vspace*{-0.7em}
~\\\textbf{\Cb{Arguments: }}\begin{flushleft}
{\small \Cb{\hspace*{0.5cm}$\bullet$~~\texttt{x0{\comma}y0{\comma}z0{\comma}x1{\comma}y1{\comma}z1{\comma}x2{\comma}y2{\comma}z2{\comma}x3{\comma}y3{\comma}z3}}}\end{flushleft}
Input 3d quadrangle at specified coordinates.
\begin{center}\includegraphics[keepaspectratio=true,height=6cm,width=\textwidth]{img/gmic_stdlib513.jpg}\\
{\footnotesize \textbf{Example 513~:} \texttt{quadrangle3d -10{\comma}-10{\comma}10{\comma}10{\comma}-10{\comma}10{\comma}10{\comma}10{\comma}10{\comma}-10{\comma}10{\comma}10 repeat 10 --rotate3d[-1] 0{\comma}1{\comma}0{\comma}30 color3d[-1] \$\{-RGB\}{\comma}0.6 done add3d mode3d 2}}
\end{center}

\subsection{\emph{random3d\index{random3d}} }\vspace*{-0.7em}
~\\\textbf{\Cb{Arguments: }}\begin{flushleft}
{\small \Cb{\hspace*{0.5cm}$\bullet$~~\texttt{nb\_points$>$=0}}}\end{flushleft}
Input random 3d point cloud in [0{\comma}1]\textasciicircum 3.
\begin{center}\includegraphics[keepaspectratio=true,height=6cm,width=\textwidth]{img/gmic_stdlib514.jpg}\\
{\footnotesize \textbf{Example 514~:} \texttt{random3d 100 circles3d 0.1 opacity3d 0.5}}
\end{center}

\subsection{\emph{reverse3d\index{reverse3d}} (+)}\vspace*{-0.7em}
Reverse primitive orientations of selected 3d objects.
~\\(\emph{eq. to} {\small \texttt{'rv3d'}}).
\begin{center}\includegraphics[keepaspectratio=true,height=6cm,width=\textwidth]{img/gmic_stdlib515.jpg}\\
{\footnotesize \textbf{Example 515~:} \texttt{torus3d 100{\comma}40 double3d 0 --reverse3d}}
\end{center}

\subsection{\emph{rotate3d\index{rotate3d}} (+)}\vspace*{-0.7em}
~\\\textbf{\Cb{Arguments: }}\begin{flushleft}
{\small \Cb{\hspace*{0.5cm}$\bullet$~~\texttt{u{\comma}v{\comma}w{\comma}angle}}}\end{flushleft}
Rotate selected 3d objects around specified axis with specified angle (in deg.).
~\\(\emph{eq. to} {\small \texttt{'r3d'}}).
\begin{center}\includegraphics[keepaspectratio=true,height=6cm,width=\textwidth]{img/gmic_stdlib516.jpg}\\
{\footnotesize \textbf{Example 516~:} \texttt{torus3d 100{\comma}10 double3d 0 repeat 7 --rotate3d[-1] 1{\comma}0{\comma}0{\comma}20 done add3d}}
\end{center}

\subsection{\emph{rotation3d\index{rotation3d}} }\vspace*{-0.7em}
~\\\textbf{\Cb{Arguments: }}\begin{flushleft}
{\small \Cb{\hspace*{0.5cm}$\bullet$~~\texttt{u{\comma}v{\comma}w{\comma}angle}}}\end{flushleft}
Input 3x3 rotation matrix with specified axis and angle (in deg).
\begin{center}\includegraphics[keepaspectratio=true,height=6cm,width=\textwidth]{img/gmic_stdlib517.jpg}\\
{\footnotesize \textbf{Example 517~:} \texttt{rotation3d 1{\comma}0{\comma}0{\comma}0 rotation3d 1{\comma}0{\comma}0{\comma}90 rotation3d 1{\comma}0{\comma}0{\comma}180}}
\end{center}

\subsection{\emph{sierpinski3d\index{sierpinski3d}} }\vspace*{-0.7em}
~\\\textbf{\Cb{Arguments: }}\begin{flushleft}
{\small \Cb{\hspace*{0.5cm}$\bullet$~~\texttt{\_recursion\_level$>$=0{\comma}\_width{\comma}\_height}}}\end{flushleft}
Input 3d Sierpinski pyramid.
\begin{center}\includegraphics[keepaspectratio=true,height=6cm,width=\textwidth]{img/gmic_stdlib518.jpg}\\
{\footnotesize \textbf{Example 518~:} \texttt{sierpinski3d 3{\comma}100{\comma}-100 --primitives3d 1 color3d[-2] \$\{-RGB\}}}
\end{center}

\subsection{\emph{size3d\index{size3d}} }\vspace*{-0.7em}
Return bounding box size of the last selected 3d object.


\subsection{\emph{skeleton3d\index{skeleton3d}} }\vspace*{-0.7em}
~\\\textbf{\Cb{Arguments: }}\begin{flushleft}
{\small \Cb{\hspace*{0.5cm}$\bullet$~~\texttt{\_metric{\comma}\_frame\_type=\{ 0=squares ~$|$~ 1=diamonds ~$|$~ 2=circles ~$|$~ 3\-=auto \}{\comma}\_skeleton\_opacity{\comma}\_frame\_opacity{\comma}\_is\_frame\_wireframe\-=\{ 0 ~$|$~ 1 \}}}}\end{flushleft}
Build 3d skeletal structure object from 2d binary shapes located in selected images.
~\\'metric' can be \{ 0=chebyshev ~$|$~ 1=manhattan ~$|$~ 2=euclidean \}.
\begin{flushleft}\Cc{\textbf{Default values}:\\~\\\hspace*{0.5cm}{\small $\bullet$~~\texttt{'metric=2'{\comma} 'bones\_type=3'{\comma} 'skeleton\_opacity=1'} and \texttt{'frame\_opacity=0.1'.}}}\end{flushleft}
\begin{center}\includegraphics[keepaspectratio=true,height=6cm,width=\textwidth]{img/gmic_stdlib519.jpg}\\
{\footnotesize \textbf{Example 519~:} \texttt{shape\_cupid 480 --skeleton3d {\comma}}}
\end{center}

\subsection{\emph{snapshot3d\index{snapshot3d}} }\vspace*{-0.7em}
~\\\textbf{\Cb{Arguments: }}\begin{flushleft}
{\small \Cb{\hspace*{0.5cm}$\bullet$~~\texttt{\_size$>$0{\comma}\_zoom$>$=0{\comma}\_backgroundR{\comma}\_backgroundG{\comma}\_backgroundB{\comma}\_bac\-kgroundA}}}~~~\\
{\small \Cb{\hspace*{0.5cm}$\bullet$~~\texttt{[background\_image]{\comma}zoom$>$=0}}}\end{flushleft}
Take 2d snapshots of selected 3d objects.
~\\Set 'zoom' to 0 to disable object auto-scaling.
\begin{flushleft}\Cc{\textbf{Default values}:\\~\\\hspace*{0.5cm}{\small $\bullet$~~\texttt{'size=512'{\comma} 'zoom=1'} and \texttt{'[background\_image]=(default)'.}}}\end{flushleft}
\begin{center}\includegraphics[keepaspectratio=true,height=6cm,width=\textwidth]{img/gmic_stdlib520.jpg}\\
{\footnotesize \textbf{Example 520~:} \texttt{torus3d 100{\comma}20 rotate3d 1{\comma}1{\comma}0{\comma}60 snapshot3d 400{\comma}1.2{\comma}128{\comma}64{\comma}32}}
\\\includegraphics[keepaspectratio=true,height=6cm,width=\textwidth]{img/gmic_stdlib521.jpg}\\
{\footnotesize \textbf{Example 521~:} \texttt{torus3d 100{\comma}20 rotate3d 1{\comma}1{\comma}0{\comma}60 sample ? --snapshot3d[0] [1]{\comma}1.2}}
\end{center}

\subsection{\emph{specl3d\index{specl3d}} (+)}\vspace*{-0.7em}
~\\\textbf{\Cb{Arguments: }}\begin{flushleft}
{\small \Cb{\hspace*{0.5cm}$\bullet$~~\texttt{value$>$=0}}}\end{flushleft}
Set lightness of 3d specular light.
~\\(\emph{eq. to} {\small \texttt{'sl3d'}}).
\begin{flushleft}\Cc{\textbf{Default value}:\\~\\\hspace*{0.5cm}{\small $\bullet$~~\texttt{'value=0.15'.}}}\end{flushleft}
\begin{center}\includegraphics[keepaspectratio=true,height=6cm,width=\textwidth]{img/gmic_stdlib522.jpg}\\
{\footnotesize \textbf{Example 522~:} \texttt{(0{\comma}0.3{\comma}0.6{\comma}0.9{\comma}1.2) repeat \{w\} torus3d 100{\comma}30 rotate3d[-1] 1{\comma}1{\comma}0{\comma}60 color3d[-1] 255{\comma}0{\comma}0 specl3d \{0{\comma}@\$$>$\} snapshot3d[-1] 400 done remove[0]}}
\end{center}

\subsection{\emph{specs3d\index{specs3d}} (+)}\vspace*{-0.7em}
~\\\textbf{\Cb{Arguments: }}\begin{flushleft}
{\small \Cb{\hspace*{0.5cm}$\bullet$~~\texttt{value$>$=0}}}\end{flushleft}
Set shininess of 3d specular light.
~\\(\emph{eq. to} {\small \texttt{'ss3d'}}).
\begin{flushleft}\Cc{\textbf{Default value}:\\~\\\hspace*{0.5cm}{\small $\bullet$~~\texttt{'value=0.8'.}}}\end{flushleft}
\begin{center}\includegraphics[keepaspectratio=true,height=6cm,width=\textwidth]{img/gmic_stdlib523.jpg}\\
{\footnotesize \textbf{Example 523~:} \texttt{(0{\comma}0.3{\comma}0.6{\comma}0.9{\comma}1.2) repeat \{w\} torus3d 100{\comma}30 rotate3d[-1] 1{\comma}1{\comma}0{\comma}60 color3d[-1] 255{\comma}0{\comma}0 specs3d \{0{\comma}@\$$>$\} snapshot3d[-1] 400 done remove[0]}}
\end{center}

\subsection{\emph{sphere3d\index{sphere3d}} (+)}\vspace*{-0.7em}
~\\\textbf{\Cb{Arguments: }}\begin{flushleft}
{\small \Cb{\hspace*{0.5cm}$\bullet$~~\texttt{radius{\comma}\_nb\_recursions$>$=0}}}\end{flushleft}
Input 3d sphere at (0{\comma}0{\comma}0){\comma} with specified geometry.
\begin{flushleft}\Cc{\textbf{Default value}:\\~\\\hspace*{0.5cm}{\small $\bullet$~~\texttt{'nb\_recursions=3'.}}}\end{flushleft}
\begin{center}\includegraphics[keepaspectratio=true,height=6cm,width=\textwidth]{img/gmic_stdlib524.jpg}\\
{\footnotesize \textbf{Example 524~:} \texttt{sphere3d 100 --primitives3d 1 color3d[-2] \$\{-RGB\}}}
\end{center}

\subsection{\emph{spherical3d\index{spherical3d}} }\vspace*{-0.7em}
~\\\textbf{\Cb{Arguments: }}\begin{flushleft}
{\small \Cb{\hspace*{0.5cm}$\bullet$~~\texttt{\_nb\_azimuth$>$=3{\comma}\_nb\_zenith$>$=3{\comma}\_radius\_function(phi{\comma}theta)}}}\end{flushleft}
Input 3d spherical object at (0{\comma}0{\comma}0){\comma} with specified geometry.
\begin{flushleft}\Cc{\textbf{Default values}:\\~\\\hspace*{0.5cm}{\small $\bullet$~~\texttt{'nb\_zenith=nb\_azimut=64'} and \texttt{'radius\_function="abs(1+0.5*cos(3*phi)*sin(4*theta))"'.}}}\end{flushleft}
\begin{center}\includegraphics[keepaspectratio=true,height=6cm,width=\textwidth]{img/gmic_stdlib525.jpg}\\
{\footnotesize \textbf{Example 525~:} \texttt{spherical3d 64 --primitives3d 1}}
\end{center}

\subsection{\emph{spline3d\index{spline3d}} }\vspace*{-0.7em}
~\\\textbf{\Cb{Arguments: }}\begin{flushleft}
{\small \Cb{\hspace*{0.5cm}$\bullet$~~\texttt{x0[\%]{\comma}y0[\%]{\comma}z0[\%]{\comma}u0[\%]{\comma}v0[\%]{\comma}w0[\%]{\comma}x1[\%]{\comma}y1[\%]{\comma}z1[\%]{\comma}u1[\%]{\comma}\-v1[\%]{\comma}w1[\%]{\comma}\_nb\_vertices$>$=2}}}\end{flushleft}
Input 3d spline with specified geometry.
\begin{flushleft}\Cc{\textbf{Default values}:\\~\\\hspace*{0.5cm}{\small $\bullet$~~\texttt{'nb\_vertices=128'.}}}\end{flushleft}
\begin{center}\includegraphics[keepaspectratio=true,height=6cm,width=\textwidth]{img/gmic_stdlib526.jpg}\\
{\footnotesize \textbf{Example 526~:} \texttt{repeat 100 spline3d \{u\}{\comma}\{u\}{\comma}\{u\}{\comma}\{u\}{\comma}\{u\}{\comma}\{u\}{\comma}\{u\}{\comma}\{u\}{\comma}\{u\}{\comma}\{u\}{\comma}\{u\}{\comma}\{u\}{\comma}128 color3d[-1] \$\{-RGB\} done box3d 1 primitives3d[-1] 1 add3d}}
\end{center}

\subsection{\emph{split3d\index{split3d}} (+)}\vspace*{-0.7em}
~\\\textbf{\Cb{Arguments: }}\begin{flushleft}
{\small \Cb{\hspace*{0.5cm}$\bullet$~~\texttt{\_keep\_shared\_data=\{ 0 ~$|$~ 1 \}}}}\end{flushleft}
Split selected 3d objects into 6 feature vectors :
\{ header{\comma} sizes{\comma} vertices{\comma} primitives{\comma} colors{\comma} opacities \}.
~\\(\emph{eq. to} {\small \texttt{'s3d').\textbackslash n}}).
~\\To recreate the 3d object{\comma} append these 6 images along the y-axis.
\begin{flushleft}\Cc{\textbf{Default value}:\\~\\\hspace*{0.5cm}{\small $\bullet$~~\texttt{'keep\_shared\_data=1'.}}}\end{flushleft}
\begin{center}\includegraphics[keepaspectratio=true,height=6cm,width=\textwidth]{img/gmic_stdlib527.jpg}\\
{\footnotesize \textbf{Example 527~:} \texttt{box3d 100 --split3d}}
\end{center}

\subsection{\emph{sprite3d\index{sprite3d}} }\vspace*{-0.7em}
Convert selected images as 3d sprites.
~\\Selected image with alpha channels are managed.
\begin{center}\includegraphics[keepaspectratio=true,height=6cm,width=\textwidth]{img/gmic_stdlib528.jpg}\\
{\footnotesize \textbf{Example 528~:} \texttt{image.jpg sprite3d}}
\end{center}

\subsection{\emph{sprites3d\index{sprites3d}} }\vspace*{-0.7em}
~\\\textbf{\Cb{Arguments: }}\begin{flushleft}
{\small \Cb{\hspace*{0.5cm}$\bullet$~~\texttt{[sprite]{\comma}\_sprite\_has\_alpha\_channel=\{ 0 ~$|$~ 1 \}}}}\end{flushleft}
Convert selected 3d objects as a sprite cloud.
~\\Set 'sprite\_has\_alpha\_channel' to 1 to make the last channel of the selected sprite be a transparency mask.
\begin{flushleft}\Cc{\textbf{Default value}:\\~\\\hspace*{0.5cm}{\small $\bullet$~~\texttt{'mask\_has\_alpha\_channel=0'.}}}\end{flushleft}
\begin{center}\includegraphics[keepaspectratio=true,height=6cm,width=\textwidth]{img/gmic_stdlib529.jpg}\\
{\footnotesize \textbf{Example 529~:} \texttt{torus3d 100{\comma}20 image.jpg resize2dy[-1] 64 100\%{\comma}100\% gaussian[-1] 30\%{\comma}30\% *[-1] 255 append[-2{\comma}-1] c --sprites3d[0] [1]{\comma}1 display\_rgba[-2]}}
\end{center}

\subsection{\emph{star3d\index{star3d}} }\vspace*{-0.7em}
~\\\textbf{\Cb{Arguments: }}\begin{flushleft}
{\small \Cb{\hspace*{0.5cm}$\bullet$~~\texttt{\_nb\_branches$>$0{\comma}0$<$=\_thickness$<$=1}}}\end{flushleft}
Input 3d star at (0{\comma}0{\comma}0){\comma} with specified geometry.
\begin{flushleft}\Cc{\textbf{Default values}:\\~\\\hspace*{0.5cm}{\small $\bullet$~~\texttt{'nb\_branches=5'} and \texttt{'thickness=0.38'.}}}\end{flushleft}
\begin{center}\includegraphics[keepaspectratio=true,height=6cm,width=\textwidth]{img/gmic_stdlib530.jpg}\\
{\footnotesize \textbf{Example 530~:} \texttt{star3d {\comma} --primitives3d 1 color3d[-2] \$\{-RGB\}}}
\end{center}

\subsection{\emph{streamline3d\index{streamline3d}} (+)}\vspace*{-0.7em}
~\\\textbf{\Cb{Arguments: }}\begin{flushleft}
{\small \Cb{\hspace*{0.5cm}$\bullet$~~\texttt{x[\%]{\comma}y[\%]{\comma}z[\%]{\comma}\_L$>$=0{\comma}\_dl$>$0{\comma}\_interpolation{\comma}\_is\_backward=\{ 0 ~$|$~\- 1 \}{\comma}\_is\_oriented=\{ 0 ~$|$~ 1 \}}}}~~~\\
{\small \Cb{\hspace*{0.5cm}$\bullet$~~\texttt{'formula'{\comma}x{\comma}y{\comma}z{\comma}\_L$>$=0{\comma}\_dl$>$0{\comma}\_interpolation{\comma}\_is\_backward=\{ 0 \-~$|$~ 1 \}{\comma}\_is\_oriented=\{ 0 ~$|$~ 1 \}}}}\end{flushleft}
Extract 3d streamlines from selected vector fields or from specified formula.
~\\'interpolation' can be \{ 0=nearest integer ~$|$~ 1=1st-order ~$|$~ 2=2nd-order ~$|$~ 3=4th-order \}.
\begin{flushleft}\Cc{\textbf{Default values}:\\~\\\hspace*{0.5cm}{\small $\bullet$~~\texttt{'dl=0.1'{\comma} 'interpolation=2'{\comma} 'is\_backward=0'} and \texttt{'is\_oriented=0'.}}}\end{flushleft}
\begin{center}\includegraphics[keepaspectratio=true,height=6cm,width=\textwidth]{img/gmic_stdlib531.jpg}\\
{\footnotesize \textbf{Example 531~:} \texttt{100{\comma}100{\comma}100{\comma}3 rand -10{\comma}10 blur 3 repeat 300 --streamline3d[0] \{u(100)\}{\comma}\{u(100)\}{\comma}\{u(100)\}{\comma}1000{\comma}1{\comma}1 color3d[-1] \$\{-RGB\} done remove[0] box3d 100 primitives3d[-1] 1 add3d}}
\end{center}

\subsection{\emph{sub3d\index{sub3d}} (+)}\vspace*{-0.7em}
~\\\textbf{\Cb{Arguments: }}\begin{flushleft}
{\small \Cb{\hspace*{0.5cm}$\bullet$~~\texttt{tx{\comma}\_ty{\comma}\_tz}}}\end{flushleft}
Shift selected 3d objects with the opposite of specified displacement vector.
~\\(\emph{eq. to} {\small \texttt{'-3d'}}).
\begin{flushleft}\Cc{\textbf{Default values}:\\~\\\hspace*{0.5cm}{\small $\bullet$~~\texttt{'ty=tz=0'.}}}\end{flushleft}
\begin{center}\includegraphics[keepaspectratio=true,height=6cm,width=\textwidth]{img/gmic_stdlib532.jpg}\\
{\footnotesize \textbf{Example 532~:} \texttt{sphere3d 10 repeat 5 --sub3d[-1] 10{\comma}\{u(-10{\comma}10)\}{\comma}0 color3d[-1] \$\{-RGB\} done add3d}}
\end{center}

\subsection{\emph{superformula3d\index{superformula3d}} }\vspace*{-0.7em}
~\\\textbf{\Cb{Arguments: }}\begin{flushleft}
{\small \Cb{\hspace*{0.5cm}$\bullet$~~\texttt{resolution$>$1{\comma}m$>$=1{\comma}n1{\comma}n2{\comma}n3}}}\end{flushleft}
Input 2d superformula curve as a 3d object.
\begin{flushleft}\Cc{\textbf{Default values}:\\~\\\hspace*{0.5cm}{\small $\bullet$~~\texttt{'resolution=1024'{\comma} 'm=8'{\comma} 'n1=1'{\comma} 'n2=5'} and \texttt{'n3=8'.}}}\end{flushleft}
\begin{center}\includegraphics[keepaspectratio=true,height=6cm,width=\textwidth]{img/gmic_stdlib533.jpg}\\
{\footnotesize \textbf{Example 533~:} \texttt{superformula3d {\comma}}}
\end{center}

\subsection{\emph{tensors3d\index{tensors3d}} }\vspace*{-0.7em}
~\\\textbf{\Cb{Arguments: }}\begin{flushleft}
{\small \Cb{\hspace*{0.5cm}$\bullet$~~\texttt{\_radius\_factor$>$=0{\comma}\_shape=\{ 0=box ~$|$~ $>$=N=ellipsoid \}{\comma}\_radius\_m\-in$>$=0}}}\end{flushleft}
Generate 3d tensor fields from selected images.
when 'shape'$>$0{\comma} it gives the ellipsoid shape precision.
\begin{flushleft}\Cc{\textbf{Default values}:\\~\\\hspace*{0.5cm}{\small $\bullet$~~\texttt{'radius\_factor=1'{\comma} 'shape=2'} and \texttt{'radius\_min=0.05'.}}}\end{flushleft}
\begin{center}\includegraphics[keepaspectratio=true,height=6cm,width=\textwidth]{img/gmic_stdlib534.jpg}\\
{\footnotesize \textbf{Example 534~:} \texttt{6{\comma}6{\comma}6{\comma}9{\comma}"U = [x{\comma}y{\comma}z] - [w{\comma}h{\comma}d]/2; U/=norm(U); mul(U{\comma}U{\comma}3) + 0.3*eye(3)" tensors3d 0.8}}
\end{center}

\subsection{\emph{text\_pointcloud3d\index{text\_pointcloud3d}} }\vspace*{-0.7em}
~\\\textbf{\Cb{Arguments: }}\begin{flushleft}
{\small \Cb{\hspace*{0.5cm}$\bullet$~~\texttt{\_"text1"{\comma}\_"text2"{\comma}\_smoothness}}}\end{flushleft}
Input 3d text pointcloud from the two specified strings.
\begin{flushleft}\Cc{\textbf{Default values}:\\~\\\hspace*{0.5cm}{\small $\bullet$~~\texttt{'text1="text1"'{\comma} 'text2="text2"'} and \texttt{'smoothness=1'.}}}\end{flushleft}
\begin{center}\includegraphics[keepaspectratio=true,height=6cm,width=\textwidth]{img/gmic_stdlib535.jpg}\\
{\footnotesize \textbf{Example 535~:} \texttt{text\_pointcloud3d "G'MIC"{\comma}"Rocks!"}}
\end{center}

\subsection{\emph{text3d\index{text3d}} }\vspace*{-0.7em}
~\\\textbf{\Cb{Arguments: }}\begin{flushleft}
{\small \Cb{\hspace*{0.5cm}$\bullet$~~\texttt{text{\comma}\_font\_height$>$0{\comma}\_depth$>$0{\comma}\_smoothness}}}\end{flushleft}
Input a 3d text object from specified text.
\begin{flushleft}\Cc{\textbf{Default values}:\\~\\\hspace*{0.5cm}{\small $\bullet$~~\texttt{'font\_height=53'{\comma} 'depth=10'} and \texttt{'smoothness=1.5'.}}}\end{flushleft}
\begin{center}\includegraphics[keepaspectratio=true,height=6cm,width=\textwidth]{img/gmic_stdlib536.jpg}\\
{\footnotesize \textbf{Example 536~:} \texttt{text3d "G'MIC as a\textbackslash n3D logo!"}}
\end{center}

\subsection{\emph{texturize3d\index{texturize3d}} (+)}\vspace*{-0.7em}
~\\\textbf{\Cb{Arguments: }}\begin{flushleft}
{\small \Cb{\hspace*{0.5cm}$\bullet$~~\texttt{[ind\_texture]{\comma}\_[ind\_coords]}}}\end{flushleft}
Texturize selected 3d objects with specified texture and coordinates.
~\\(\emph{eq. to} {\small \texttt{'t3d').\textbackslash n}}).
~\\When '[ind\_coords]' is omitted{\comma} default XY texture projection is performed.
\begin{flushleft}\Cc{\textbf{Default value}:\\~\\\hspace*{0.5cm}{\small $\bullet$~~\texttt{'ind\_coords=(undefined)'.}}}\end{flushleft}
\begin{center}\includegraphics[keepaspectratio=true,height=6cm,width=\textwidth]{img/gmic_stdlib537.jpg}\\
{\footnotesize \textbf{Example 537~:} \texttt{image.jpg torus3d 100{\comma}30 texturize3d[-1] [-2] keep[-1]}}
\end{center}

\subsection{\emph{torus3d\index{torus3d}} }\vspace*{-0.7em}
~\\\textbf{\Cb{Arguments: }}\begin{flushleft}
{\small \Cb{\hspace*{0.5cm}$\bullet$~~\texttt{\_radius1{\comma}\_radius2{\comma}\_nb\_subdivisions1$>$2{\comma}\_nb\_subdivisions2$>$2}}}\end{flushleft}
Input 3d torus at (0{\comma}0{\comma}0){\comma} with specified geometry.
\begin{flushleft}\Cc{\textbf{Default values}:\\~\\\hspace*{0.5cm}{\small $\bullet$~~\texttt{'radius1=1'{\comma} 'radius2=0.3'{\comma} 'nb\_subdivisions1=24'} and \texttt{'nb\_subdivisions2=12'.}}}\end{flushleft}
\begin{center}\includegraphics[keepaspectratio=true,height=6cm,width=\textwidth]{img/gmic_stdlib538.jpg}\\
{\footnotesize \textbf{Example 538~:} \texttt{torus3d 10{\comma}3 --primitives3d 1 color3d[-2] \$\{-RGB\}}}
\end{center}

\subsection{\emph{triangle3d\index{triangle3d}} }\vspace*{-0.7em}
~\\\textbf{\Cb{Arguments: }}\begin{flushleft}
{\small \Cb{\hspace*{0.5cm}$\bullet$~~\texttt{x0{\comma}y0{\comma}z0{\comma}x1{\comma}y1{\comma}z1{\comma}x2{\comma}y2{\comma}z2}}}\end{flushleft}
Input 3d triangle at specified coordinates.
\begin{center}\includegraphics[keepaspectratio=true,height=6cm,width=\textwidth]{img/gmic_stdlib539.jpg}\\
{\footnotesize \textbf{Example 539~:} \texttt{repeat 100 a=\{\$$>$*pi/50\} triangle3d 0{\comma}0{\comma}0{\comma}0{\comma}0{\comma}3{\comma}\{cos(3*\$a)\}{\comma}\{sin(2*\$a)\}{\comma}0 color3d[-1] \$\{-RGB\} done add3d}}
\end{center}

\subsection{\emph{volume3d\index{volume3d}} }\vspace*{-0.7em}
Transform selected 3d volumetric images as 3d parallelepipedic objects.
\begin{center}\includegraphics[keepaspectratio=true,height=6cm,width=\textwidth]{img/gmic_stdlib540.jpg}\\
{\footnotesize \textbf{Example 540~:} \texttt{image.jpg animate blur{\comma}0{\comma}5{\comma}30 append z volume3d}}
\end{center}

\subsection{\emph{weird3d\index{weird3d}} }\vspace*{-0.7em}
~\\\textbf{\Cb{Arguments: }}\begin{flushleft}
{\small \Cb{\hspace*{0.5cm}$\bullet$~~\texttt{\_resolution$>$0}}}\end{flushleft}
Input 3d weird object at (0{\comma}0{\comma}0){\comma} with specified resolution.
\begin{flushleft}\Cc{\textbf{Default value}:\\~\\\hspace*{0.5cm}{\small $\bullet$~~\texttt{'resolution=32'.}}}\end{flushleft}
\begin{center}\includegraphics[keepaspectratio=true,height=6cm,width=\textwidth]{img/gmic_stdlib541.jpg}\\
{\footnotesize \textbf{Example 541~:} \texttt{weird3d 48 --primitives3d 1 color3d[-2] \$\{-RGB\}}}
\end{center}
\section{Program control}


\subsection{\emph{apply\_parallel\index{apply\_parallel}} }\vspace*{-0.7em}
~\\\textbf{\Cb{Arguments: }}\begin{flushleft}
{\small \Cb{\hspace*{0.5cm}$\bullet$~~\texttt{"command"}}}\end{flushleft}
Apply specified command on each of the selected images{\comma} by parallelizing it for all image of the list.
~\\(\emph{eq. to} {\small \texttt{'ap'}}).
\begin{center}\includegraphics[keepaspectratio=true,height=6cm,width=\textwidth]{img/gmic_stdlib542.jpg}\\
{\footnotesize \textbf{Example 542~:} \texttt{image.jpg --mirror x --mirror y apply\_parallel "blur 3"}}
\end{center}

\subsection{\emph{apply\_parallel\_channels\index{apply\_parallel\_channels}} }\vspace*{-0.7em}
~\\\textbf{\Cb{Arguments: }}\begin{flushleft}
{\small \Cb{\hspace*{0.5cm}$\bullet$~~\texttt{"command"}}}\end{flushleft}
Apply specified command on each of the selected images{\comma} by parallelizing it for all channel of the images independently.
~\\(\emph{eq. to} {\small \texttt{'apc'}}).
\begin{center}\includegraphics[keepaspectratio=true,height=6cm,width=\textwidth]{img/gmic_stdlib543.jpg}\\
{\footnotesize \textbf{Example 543~:} \texttt{image.jpg apply\_parallel\_channels "blur 3"}}
\end{center}

\subsection{\emph{apply\_parallel\_overlap\index{apply\_parallel\_overlap}} }\vspace*{-0.7em}
~\\\textbf{\Cb{Arguments: }}\begin{flushleft}
{\small \Cb{\hspace*{0.5cm}$\bullet$~~\texttt{"command"{\comma}overlap[\%]{\comma}nb\_threads=\{ 0=auto ~$|$~ 1 ~$|$~ 2 ~$|$~ 4 ~$|$~ 8 ~$|$~ 1\-6 \}}}}\end{flushleft}
Apply specified command on each of the selected images{\comma} by parallelizing it on 'nb\_threads' overlapped sub-images.
~\\(\emph{eq. to} {\small \texttt{'apo').\textbackslash n}}).
~\\'nb\_threads' must be a power of 2.
\begin{flushleft}\Cc{\textbf{Default values}:\\~\\\hspace*{0.5cm}{\small $\bullet$~~\texttt{'overlap=0'{\comma}'nb\_threads=0'.}}}\end{flushleft}
\begin{center}\includegraphics[keepaspectratio=true,height=6cm,width=\textwidth]{img/gmic_stdlib544.jpg}\\
{\footnotesize \textbf{Example 544~:} \texttt{image.jpg --apply\_parallel\_overlap "smooth 500{\comma}0{\comma}1"{\comma}1}}
\end{center}

\subsection{\emph{apply\_timeout\index{apply\_timeout}} }\vspace*{-0.7em}
~\\\textbf{\Cb{Arguments: }}\begin{flushleft}
{\small \Cb{\hspace*{0.5cm}$\bullet$~~\texttt{"command"{\comma}\_timeout=\{ 0=no timeout ~$|$~ $>$0=with specified timeou\-t (in seconds) \}}}}\end{flushleft}
Apply a command with a timeout.


\subsection{\emph{check\index{check}} (+)}\vspace*{-0.7em}
~\\\textbf{\Cb{Arguments: }}\begin{flushleft}
{\small \Cb{\hspace*{0.5cm}$\bullet$~~\texttt{expression}}}\end{flushleft}
Evaluate specified expression and display an error message if evaluated to false.
~\\If 'expression' is not a math expression{\comma} it is regarded as a filename and checked if it exists.


\subsection{\emph{check3d\index{check3d}} (+)}\vspace*{-0.7em}
~\\\textbf{\Cb{Arguments: }}\begin{flushleft}
{\small \Cb{\hspace*{0.5cm}$\bullet$~~\texttt{\_is\_full\_check=\{ 0 ~$|$~ 1 \}}}}\end{flushleft}
Check validity of selected 3d vector objects{\comma} and display an error message
if one of the selected image is not a valid 3d vector object.
~\\Full 3d object check is slower but more precise.
\begin{flushleft}\Cc{\textbf{Default value}:\\~\\\hspace*{0.5cm}{\small $\bullet$~~\texttt{'is\_full\_check=1'.}}}\end{flushleft}


\subsection{\emph{continue\index{continue}} (+)}\vspace*{-0.7em}
Go to end of current 'repeat...done'{\comma} 'do...while' or 'local...endlocal' block.
\begin{center}\includegraphics[keepaspectratio=true,height=6cm,width=\textwidth]{img/gmic_stdlib545.jpg}\\
{\footnotesize \textbf{Example 545~:} \texttt{image.jpg repeat 10 blur 1 if \{1==1\} continue endif deform 10 done}}
\end{center}

\subsection{\emph{break\index{break}} (+)}\vspace*{-0.7em}
Break current 'repeat...done'{\comma} 'do...while' or 'local...endlocal' block.
\begin{center}\includegraphics[keepaspectratio=true,height=6cm,width=\textwidth]{img/gmic_stdlib546.jpg}\\
{\footnotesize \textbf{Example 546~:} \texttt{image.jpg repeat 10 blur 1 if \{1==1\} break endif deform 10 done}}
\end{center}

\subsection{\emph{do\index{do}} (+)}\vspace*{-0.7em}
Start a 'do...while' block.
\begin{center}\includegraphics[keepaspectratio=true,height=6cm,width=\textwidth]{img/gmic_stdlib547.jpg}\\
{\footnotesize \textbf{Example 547~:} \texttt{image.jpg luminance i=\{ia+2\} do set 255{\comma}\{u(100)\}\%{\comma}\{u(100)\}\% while \{ia$<$\$i\}}}
\end{center}

\subsection{\emph{done\index{done}} (+)}\vspace*{-0.7em}
End a 'repeat/for...done' block{\comma} and go to associated 'repeat/for' position{\comma} if iterations remain.


\subsection{\emph{elif\index{elif}} (+)}\vspace*{-0.7em}
~\\\textbf{\Cb{Arguments: }}\begin{flushleft}
{\small \Cb{\hspace*{0.5cm}$\bullet$~~\texttt{boolean}}}~~~\\
{\small \Cb{\hspace*{0.5cm}$\bullet$~~\texttt{filename}}}\end{flushleft}
Start a 'elif...[else]...endif' block if previous 'if' was not verified
and test if specified boolean is true{\comma} or if specified filename exists.
~\\'boolean' can be a float number standing for \{ 0=false ~$|$~ other=true \}.


\subsection{\emph{else\index{else}} (+)}\vspace*{-0.7em}
Execute following commands if previous 'if' or 'elif' conditions failed.


\subsection{\emph{endif\index{endif}} (+)}\vspace*{-0.7em}
End a 'if...[elif]...[else]...endif' block.


\subsection{\emph{endlocal\index{endlocal}} (+)}\vspace*{-0.7em}
End a 'local...endlocal' block.
~\\(\emph{eq. to} {\small \texttt{'endl'}}).


\subsection{\emph{error\index{error}} (+)}\vspace*{-0.7em}
~\\\textbf{\Cb{Arguments: }}\begin{flushleft}
{\small \Cb{\hspace*{0.5cm}$\bullet$~~\texttt{message}}}\end{flushleft}
Print specified error message on the standard error (stderr) and exit interpreter{\comma} except
if error is caught by a '-onfail' command.
~\\Command selection (if any) stands for displayed call stack subset instead of image indices.


\subsection{\emph{eval\index{eval}} (+)}\vspace*{-0.7em}
~\\\textbf{\Cb{Arguments: }}\begin{flushleft}
{\small \Cb{\hspace*{0.5cm}$\bullet$~~\texttt{expression}}}\end{flushleft}
Evaluate specified math expression and return its result.


\subsection{\emph{exec\index{exec}} (+)}\vspace*{-0.7em}
~\\\textbf{\Cb{Arguments: }}\begin{flushleft}
{\small \Cb{\hspace*{0.5cm}$\bullet$~~\texttt{command}}}\end{flushleft}
Execute external command using a system call.
~\\The status value is then set to the error code returned by the system call.
~\\(\emph{eq. to} {\small \texttt{'x'}}).


\subsection{\emph{for\index{for}} (+)}\vspace*{-0.7em}
~\\\textbf{\Cb{Arguments: }}\begin{flushleft}
{\small \Cb{\hspace*{0.5cm}$\bullet$~~\texttt{condition}}}\end{flushleft}
Start a 'for...done' block.
\begin{center}\includegraphics[keepaspectratio=true,height=6cm,width=\textwidth]{img/gmic_stdlib548.jpg}\\
{\footnotesize \textbf{Example 548~:} \texttt{image.jpg resize2dy 32 400{\comma}400{\comma}1{\comma}3 x=0 for \{\$x$<$400\} image[1] [0]{\comma}\$x{\comma}\$x x+=40 done}}
\end{center}

\subsection{\emph{if\index{if}} (+)}\vspace*{-0.7em}
~\\\textbf{\Cb{Arguments: }}\begin{flushleft}
{\small \Cb{\hspace*{0.5cm}$\bullet$~~\texttt{boolean}}}~~~\\
{\small \Cb{\hspace*{0.5cm}$\bullet$~~\texttt{filename}}}\end{flushleft}
Start a 'if...[elif]...[else]...endif' block and test if specified boolean is true{\comma}
or if specified filename exists.
~\\'boolean' can be a float number standing for \{ 0=false ~$|$~ other=true \}.
\begin{center}\includegraphics[keepaspectratio=true,height=6cm,width=\textwidth]{img/gmic_stdlib549.jpg}\\
{\footnotesize \textbf{Example 549~:} \texttt{image.jpg if \{ia$<$64\} add 50\% elif \{ia$<$128\} add 25\% elif \{ia$<$192\} sub 25\% else sub 50\% endif cut 0{\comma}255}}
\end{center}

\subsection{\emph{local\index{local}} (+)}\vspace*{-0.7em}
Start a 'local...[onfail]...endlocal' block{\comma} with selected images.
~\\(\emph{eq. to} {\small \texttt{'l'}}).
\begin{center}\includegraphics[keepaspectratio=true,height=6cm,width=\textwidth]{img/gmic_stdlib550.jpg}\\
{\footnotesize \textbf{Example 550~:} \texttt{image.jpg local[] 300{\comma}300{\comma}1{\comma}3 rand[0] 0{\comma}255 blur 4 sharpen 1000 endlocal}}
\\\includegraphics[keepaspectratio=true,height=6cm,width=\textwidth]{img/gmic_stdlib551.jpg}\\
{\footnotesize \textbf{Example 551~:} \texttt{image.jpg --local repeat 3 deform 20 done endlocal}}
\end{center}
~\\
~\textbf{Tutorial page: }\\\url{http://gmic.eu/tutorial/\_local.shtml}


\subsection{\emph{mutex\index{mutex}} (+)}\vspace*{-0.7em}
~\\\textbf{\Cb{Arguments: }}\begin{flushleft}
{\small \Cb{\hspace*{0.5cm}$\bullet$~~\texttt{indice{\comma}\_action=\{ 0=unlock ~$|$~ 1=lock \}}}}\end{flushleft}
Lock or unlock specified mutex for multi-threaded programming.
~\\A locked mutex can be unlocked only by the same thread. All mutexes are unlocked by default.
~\\'indice' designates the mutex indice{\comma} in [0{\comma}255].
\begin{flushleft}\Cc{\textbf{Default value}:\\~\\\hspace*{0.5cm}{\small $\bullet$~~\texttt{'action=1'.}}}\end{flushleft}


\subsection{\emph{noarg\index{noarg}} (+)}\vspace*{-0.7em}
Used in a custom command{\comma} '-noarg' tells the command that its argument list have not been used
finally{\comma} and so they must be evaluated next in the G'MIC pipeline{\comma} just as if the custom
command takes no arguments at all.
~\\Use this command to write a custom command which can decide if it takes arguments or not.


\subsection{\emph{onfail\index{onfail}} (+)}\vspace*{-0.7em}
Execute following commands when an error is encountered in the body of the 'local...endlocal' block.
~\\The status value is set with the corresponding error message.
\begin{center}\includegraphics[keepaspectratio=true,height=6cm,width=\textwidth]{img/gmic_stdlib552.jpg}\\
{\footnotesize \textbf{Example 552~:} \texttt{image.jpg --local blur -3 onfail mirror x endlocal}}
\end{center}

\subsection{\emph{parallel\index{parallel}} (+)}\vspace*{-0.7em}
~\\\textbf{\Cb{Arguments: }}\begin{flushleft}
{\small \Cb{\hspace*{0.5cm}$\bullet$~~\texttt{\_wait\_threads{\comma}"command1"{\comma}"command2"{\comma}...}}}\end{flushleft}
Execute specified commands in parallel{\comma} each in a different thread.
~\\Parallel threads share the list of images.
~\\'wait\_threads' can be \{ 0=when current environment ends ~$|$~ 1=immediately \}.
\begin{flushleft}\Cc{\textbf{Default value}:\\~\\\hspace*{0.5cm}{\small $\bullet$~~\texttt{'wait\_threads=1'.}}}\end{flushleft}
\begin{center}\includegraphics[keepaspectratio=true,height=6cm,width=\textwidth]{img/gmic_stdlib553.jpg}\\
{\footnotesize \textbf{Example 553~:} \texttt{image.jpg [0] parallel "blur[0] 3"{\comma}"mirror[1] c"}}
\end{center}

\subsection{\emph{progress\index{progress}} (+)}\vspace*{-0.7em}
~\\\textbf{\Cb{Arguments: }}\begin{flushleft}
{\small \Cb{\hspace*{0.5cm}$\bullet$~~\texttt{0$<$=value$<$=100}}}~~~\\
{\small \Cb{\hspace*{0.5cm}$\bullet$~~\texttt{-1}}}\end{flushleft}
Set the progress indice of the current processing pipeline.
~\\This command is useful only when G'MIC is used by an embedding application.


\subsection{\emph{quit\index{quit}} (+)}\vspace*{-0.7em}
Quit G'MIC interpreter.
~\\(\emph{eq. to} {\small \texttt{'q'}}).


\subsection{\emph{repeat\index{repeat}} (+)}\vspace*{-0.7em}
~\\\textbf{\Cb{Arguments: }}\begin{flushleft}
{\small \Cb{\hspace*{0.5cm}$\bullet$~~\texttt{nb\_iterations{\comma}\_variable\_name}}}\end{flushleft}
Start iterations of a 'repeat...done' block.
\begin{center}\includegraphics[keepaspectratio=true,height=6cm,width=\textwidth]{img/gmic_stdlib554.jpg}\\
{\footnotesize \textbf{Example 554~:} \texttt{image.jpg split y repeat \$!{\comma}n shift[\$n] \$$<${\comma}0{\comma}0{\comma}0{\comma}2 done append y}}
\\\includegraphics[keepaspectratio=true,height=6cm,width=\textwidth]{img/gmic_stdlib555.jpg}\\
{\footnotesize \textbf{Example 555~:} \texttt{image.jpg mode3d 2 repeat 4 imagecube3d rotate3d 1{\comma}1{\comma}0{\comma}40 snapshot3d 400{\comma}1.4 done}}
\end{center}
~\\
~\textbf{Tutorial page: }\\\url{http://gmic.eu/tutorial/\_repeat.shtml}


\subsection{\emph{return\index{return}} (+)}\vspace*{-0.7em}
Return from current custom command.


\subsection{\emph{rprogress\index{rprogress}} }\vspace*{-0.7em}
~\\\textbf{\Cb{Arguments: }}\begin{flushleft}
{\small \Cb{\hspace*{0.5cm}$\bullet$~~\texttt{0$<$=value$<$=100 ~$|$~ -1 ~$|$~ "command"{\comma}0$<$=value\_min$<$=100{\comma}0$<$=value\_ma\-x$<$=100}}}\end{flushleft}
Set the progress indice of the current processing pipeline (relatively to
previously defined progress bounds){\comma} or call the specified command with
specified progress bounds.


\subsection{\emph{skip\index{skip}} (+)}\vspace*{-0.7em}
~\\\textbf{\Cb{Arguments: }}\begin{flushleft}
{\small \Cb{\hspace*{0.5cm}$\bullet$~~\texttt{item}}}\end{flushleft}
Do nothing but skip specified item.


\subsection{\emph{status\index{status}} (+)}\vspace*{-0.7em}
~\\\textbf{\Cb{Arguments: }}\begin{flushleft}
{\small \Cb{\hspace*{0.5cm}$\bullet$~~\texttt{status\_string}}}\end{flushleft}
Set the current status. Used to define a returning value from a function.
~\\(\emph{eq. to} {\small \texttt{'u'}}).
\begin{center}\includegraphics[keepaspectratio=true,height=6cm,width=\textwidth]{img/gmic_stdlib556.jpg}\\
{\footnotesize \textbf{Example 556~:} \texttt{image.jpg command "foo : u0=Dark u1=Bright status \$\{u\{ia$>$=128\}\}" text\_outline \$\{-foo\}{\comma}2{\comma}2{\comma}23{\comma}2{\comma}1{\comma}255}}
\end{center}

\subsection{\emph{while\index{while}} (+)}\vspace*{-0.7em}
~\\\textbf{\Cb{Arguments: }}\begin{flushleft}
{\small \Cb{\hspace*{0.5cm}$\bullet$~~\texttt{boolean}}}~~~\\
{\small \Cb{\hspace*{0.5cm}$\bullet$~~\texttt{filename}}}\end{flushleft}
End a 'do...while' block and go back to associated '-do'
if specified boolean is true or if specified filename exists.
~\\'boolean' can be a float number standing for \{ 0=false ~$|$~ other=true \}.

\section{Arrays{\comma} tiles and frames}


\subsection{\emph{array\index{array}} }\vspace*{-0.7em}
~\\\textbf{\Cb{Arguments: }}\begin{flushleft}
{\small \Cb{\hspace*{0.5cm}$\bullet$~~\texttt{M$>$0{\comma}\_N$>$0{\comma}\_expand\_type=\{ 0=min ~$|$~ 1=max ~$|$~ 2=all \}}}}\end{flushleft}
Create MxN array from selected images.
\begin{flushleft}\Cc{\textbf{Default values}:\\~\\\hspace*{0.5cm}{\small $\bullet$~~\texttt{'N=M'} and \texttt{'expand\_type=0'.}}}\end{flushleft}
\begin{center}\includegraphics[keepaspectratio=true,height=6cm,width=\textwidth]{img/gmic_stdlib557.jpg}\\
{\footnotesize \textbf{Example 557~:} \texttt{image.jpg array 3{\comma}2{\comma}2}}
\end{center}

\subsection{\emph{array\_fade\index{array\_fade}} }\vspace*{-0.7em}
~\\\textbf{\Cb{Arguments: }}\begin{flushleft}
{\small \Cb{\hspace*{0.5cm}$\bullet$~~\texttt{M$>$0{\comma}\_N$>$0{\comma}0$<$=\_fade\_start$<$=100{\comma}0$<$=\_fade\_end$<$=100{\comma}\_expand\_type=\-\{0=min ~$|$~ 1=max ~$|$~ 2=all\}}}}\end{flushleft}
Create MxN array from selected images.
\begin{flushleft}\Cc{\textbf{Default values}:\\~\\\hspace*{0.5cm}{\small $\bullet$~~\texttt{'N=M'{\comma} 'fade\_start=60'{\comma} 'fade\_end=90'} and \texttt{'expand\_type=1'.}}}\end{flushleft}
\begin{center}\includegraphics[keepaspectratio=true,height=6cm,width=\textwidth]{img/gmic_stdlib558.jpg}\\
{\footnotesize \textbf{Example 558~:} \texttt{image.jpg array\_fade 3{\comma}2}}
\end{center}

\subsection{\emph{array\_mirror\index{array\_mirror}} }\vspace*{-0.7em}
~\\\textbf{\Cb{Arguments: }}\begin{flushleft}
{\small \Cb{\hspace*{0.5cm}$\bullet$~~\texttt{N$>$=0{\comma}\_dir=\{ 0=x ~$|$~ 1=y ~$|$~ 2=xy ~$|$~ 3=tri-xy \}{\comma}\_expand\_type=\{ 0 ~$|$~\- 1 \}}}}\end{flushleft}
Create 2\textasciicircum Nx2\textasciicircum N array from selected images.
\begin{flushleft}\Cc{\textbf{Default values}:\\~\\\hspace*{0.5cm}{\small $\bullet$~~\texttt{'dir=2'} and \texttt{'expand\_type=0'.}}}\end{flushleft}
\begin{center}\includegraphics[keepaspectratio=true,height=6cm,width=\textwidth]{img/gmic_stdlib559.jpg}\\
{\footnotesize \textbf{Example 559~:} \texttt{image.jpg array\_mirror 2}}
\end{center}

\subsection{\emph{array\_random\index{array\_random}} }\vspace*{-0.7em}
~\\\textbf{\Cb{Arguments: }}\begin{flushleft}
{\small \Cb{\hspace*{0.5cm}$\bullet$~~\texttt{Ms$>$0{\comma}\_Ns$>$0{\comma}\_Md$>$0{\comma}\_Nd$>$0}}}\end{flushleft}
Create MdxNd array of tiles from selected MsxNs source arrays.
\begin{flushleft}\Cc{\textbf{Default values}:\\~\\\hspace*{0.5cm}{\small $\bullet$~~\texttt{'Ns=Ms'{\comma} 'Md=Ms'} and \texttt{'Nd=Ns'.}}}\end{flushleft}
\begin{center}\includegraphics[keepaspectratio=true,height=6cm,width=\textwidth]{img/gmic_stdlib560.jpg}\\
{\footnotesize \textbf{Example 560~:} \texttt{image.jpg --array\_random 8{\comma}8{\comma}15{\comma}10}}
\end{center}

\subsection{\emph{frame\_blur\index{frame\_blur}} }\vspace*{-0.7em}
~\\\textbf{\Cb{Arguments: }}\begin{flushleft}
{\small \Cb{\hspace*{0.5cm}$\bullet$~~\texttt{\_sharpness$>$0{\comma}\_size$>$=0{\comma}\_smoothness{\comma}\_shading{\comma}\_blur}}}\end{flushleft}
Draw RGBA-colored round frame in selected images.
\begin{flushleft}\Cc{\textbf{Default values}:\\~\\\hspace*{0.5cm}{\small $\bullet$~~\texttt{'sharpness=10'{\comma} 'size=30'{\comma} 'smoothness=0'{\comma} 'shading=1'} and \texttt{'blur=3\%'.}}}\end{flushleft}
\begin{center}\includegraphics[keepaspectratio=true,height=6cm,width=\textwidth]{img/gmic_stdlib561.jpg}\\
{\footnotesize \textbf{Example 561~:} \texttt{image.jpg frame\_blur 3{\comma}30{\comma}8{\comma}10\%}}
\end{center}

\subsection{\emph{frame\_cube\index{frame\_cube}} }\vspace*{-0.7em}
~\\\textbf{\Cb{Arguments: }}\begin{flushleft}
{\small \Cb{\hspace*{0.5cm}$\bullet$~~\texttt{\_depth$>$=0{\comma}\_centering\_x{\comma}\_centering\_y{\comma}\_left\_side=\{0=normal ~$|$~ 1\-=mirror-x ~$|$~ 2=mirror-y ~$|$~ 3=mirror-xy\}{\comma}\_right\_side{\comma}\_lower\_sid\-e{\comma}\_upper\_side}}}\end{flushleft}
Insert 3d frames in selected images.
\begin{flushleft}\Cc{\textbf{Default values}:\\~\\\hspace*{0.5cm}{\small $\bullet$~~\texttt{'depth=1'{\comma} 'centering\_x=centering\_y=0'} and \texttt{'left\_side=right\_side{\comma}lower\_side=upper\_side=0'.}}}\end{flushleft}
\begin{center}\includegraphics[keepaspectratio=true,height=6cm,width=\textwidth]{img/gmic_stdlib562.jpg}\\
{\footnotesize \textbf{Example 562~:} \texttt{image.jpg frame\_cube {\comma}}}
\end{center}

\subsection{\emph{frame\_fuzzy\index{frame\_fuzzy}} }\vspace*{-0.7em}
~\\\textbf{\Cb{Arguments: }}\begin{flushleft}
{\small \Cb{\hspace*{0.5cm}$\bullet$~~\texttt{size\_x[\%]$>$=0{\comma}\_size\_y[\%]$>$=0{\comma}\_fuzzyness$>$=0{\comma}\_smoothness[\%]$>$=0{\comma}\_\-R{\comma}\_G{\comma}\_B{\comma}\_A}}}\end{flushleft}
Draw RGBA-colored fuzzy frame in selected images.
\begin{flushleft}\Cc{\textbf{Default values}:\\~\\\hspace*{0.5cm}{\small $\bullet$~~\texttt{'size\_y=size\_x'{\comma} 'fuzzyness=5'{\comma} 'smoothness=1'} and \texttt{'R=G=B=A=255'.}}}\end{flushleft}
\begin{center}\includegraphics[keepaspectratio=true,height=6cm,width=\textwidth]{img/gmic_stdlib563.jpg}\\
{\footnotesize \textbf{Example 563~:} \texttt{image.jpg frame\_fuzzy 20}}
\end{center}

\subsection{\emph{frame\_painting\index{frame\_painting}} }\vspace*{-0.7em}
~\\\textbf{\Cb{Arguments: }}\begin{flushleft}
{\small \Cb{\hspace*{0.5cm}$\bullet$~~\texttt{\_size[\%]$>$=0{\comma}0$<$=\_contrast$<$=1{\comma}\_profile\_smoothness[\%]$>$=0{\comma}\_R{\comma}\_G{\comma}\-\_B{\comma}\_vignette\_size[\%]$>$=0{\comma}\_vignette\_contrast$>$=0{\comma}\_defects\_contr\-ast$>$=0{\comma}0$<$=\_defects\_density$<$=100{\comma}\_defects\_size$>$=0{\comma}\_defects\_sm\-oothness[\%]$>$=0{\comma}\_serial\_number}}}\end{flushleft}
Add a painting frame to selected images.
\begin{flushleft}\Cc{\textbf{Default values}:\\~\\\hspace*{0.5cm}{\small $\bullet$~~\texttt{'size=10\%'{\comma} 'contrast=0.4'{\comma} 'profile\_smoothness=6\%'{\comma} 'R=225'{\comma} 'G=200'{\comma} 'B=120'{\comma} 'vignette\_size=2\%'{\comma} 'vignette\_contrast=400'{\comma} 'defects\_contrast=50'{\comma} 'defects\_density=10'{\comma} 'defects\_size=1'{\comma} 'defects\_smoothness=0.5\%'} and \texttt{'serial\_number=123456789'.}}}\end{flushleft}
\begin{center}\includegraphics[keepaspectratio=true,height=6cm,width=\textwidth]{img/gmic_stdlib564.jpg}\\
{\footnotesize \textbf{Example 564~:} \texttt{image.jpg frame\_painting {\comma}}}
\end{center}

\subsection{\emph{frame\_pattern\index{frame\_pattern}} }\vspace*{-0.7em}
~\\\textbf{\Cb{Arguments: }}\begin{flushleft}
{\small \Cb{\hspace*{0.5cm}$\bullet$~~\texttt{M$>$=3{\comma}\_constrain\_size=\{ 0 ~$|$~ 1 \}}}}~~~\\
{\small \Cb{\hspace*{0.5cm}$\bullet$~~\texttt{M$>$=3{\comma}\_[frame\_image]{\comma}\_constrain\_size=\{ 0 ~$|$~ 1 \}}}}\end{flushleft}
Insert selected pattern frame in selected images.
\begin{flushleft}\Cc{\textbf{Default values}:\\~\\\hspace*{0.5cm}{\small $\bullet$~~\texttt{'pattern=0'} and \texttt{'constrain\_size=0'.}}}\end{flushleft}
\begin{center}\includegraphics[keepaspectratio=true,height=6cm,width=\textwidth]{img/gmic_stdlib565.jpg}\\
{\footnotesize \textbf{Example 565~:} \texttt{image.jpg frame\_pattern 8}}
\end{center}

\subsection{\emph{frame\_round\index{frame\_round}} }\vspace*{-0.7em}
~\\\textbf{\Cb{Arguments: }}\begin{flushleft}
{\small \Cb{\hspace*{0.5cm}$\bullet$~~\texttt{\_sharpness$>$0{\comma}\_size$>$=0{\comma}\_smoothness{\comma}\_shading{\comma}\_R{\comma}\_G{\comma}\_B{\comma}\_A}}}\end{flushleft}
Draw RGBA-colored round frame in selected images.
\begin{flushleft}\Cc{\textbf{Default values}:\\~\\\hspace*{0.5cm}{\small $\bullet$~~\texttt{'sharpness=10'{\comma} 'size=10'{\comma} 'smoothness=0'{\comma} 'shading=0'} and \texttt{'R=G=B=A=255'.}}}\end{flushleft}
\begin{center}\includegraphics[keepaspectratio=true,height=6cm,width=\textwidth]{img/gmic_stdlib566.jpg}\\
{\footnotesize \textbf{Example 566~:} \texttt{image.jpg frame\_round 10}}
\end{center}

\subsection{\emph{frame\_seamless\index{frame\_seamless}} }\vspace*{-0.7em}
~\\\textbf{\Cb{Arguments: }}\begin{flushleft}
{\small \Cb{\hspace*{0.5cm}$\bullet$~~\texttt{frame\_size$>$=0{\comma}\_patch\_size$>$0{\comma}\_blend\_size$>$=0{\comma}\_frame\_direction=\-\{ 0=inner (preserve image size) ~$|$~ 1=outer \}}}}\end{flushleft}
Insert frame in selected images{\comma} so that tiling the resulting image makes less visible seams.
\begin{flushleft}\Cc{\textbf{Default values}:\\~\\\hspace*{0.5cm}{\small $\bullet$~~\texttt{'patch\_size=7'{\comma} 'blend\_size=5'} and \texttt{'frame\_direction=1'.}}}\end{flushleft}
\begin{center}\includegraphics[keepaspectratio=true,height=6cm,width=\textwidth]{img/gmic_stdlib567.jpg}\\
{\footnotesize \textbf{Example 567~:} \texttt{image.jpg --frame\_seamless 30 array 2{\comma}2}}
\end{center}

\subsection{\emph{frame\_x\index{frame\_x}} }\vspace*{-0.7em}
~\\\textbf{\Cb{Arguments: }}\begin{flushleft}
{\small \Cb{\hspace*{0.5cm}$\bullet$~~\texttt{size\_x[\%]{\comma}\_col1{\comma}...{\comma}\_colN}}}\end{flushleft}
Insert colored frame along the x-axis in selected images.
\begin{flushleft}\Cc{\textbf{Default values}:\\~\\\hspace*{0.5cm}{\small $\bullet$~~\texttt{'col1=col2=col3=255'} and \texttt{'col4=255'.}}}\end{flushleft}
\begin{center}\includegraphics[keepaspectratio=true,height=6cm,width=\textwidth]{img/gmic_stdlib568.jpg}\\
{\footnotesize \textbf{Example 568~:} \texttt{image.jpg frame\_x 20{\comma}255{\comma}0{\comma}255}}
\end{center}

\subsection{\emph{frame\_xy\index{frame\_xy}} }\vspace*{-0.7em}
~\\\textbf{\Cb{Arguments: }}\begin{flushleft}
{\small \Cb{\hspace*{0.5cm}$\bullet$~~\texttt{size\_x[\%]{\comma}\_size\_y[\%]{\comma}\_col1{\comma}...{\comma}\_colN}}}\end{flushleft}
Insert colored frame along the x-axis in selected images.
\begin{flushleft}\Cc{\textbf{Default values}:\\~\\\hspace*{0.5cm}{\small $\bullet$~~\texttt{'size\_y=size\_x'{\comma} 'col1=col2=col3=255'} and \texttt{'col4=255'.}}}\end{flushleft}
~\\(\emph{eq. to} {\small \texttt{'frame'}}).
\begin{center}\includegraphics[keepaspectratio=true,height=6cm,width=\textwidth]{img/gmic_stdlib569.jpg}\\
{\footnotesize \textbf{Example 569~:} \texttt{image.jpg frame\_xy 1{\comma}1{\comma}0 frame\_xy 20{\comma}10{\comma}255{\comma}0{\comma}255}}
\end{center}

\subsection{\emph{frame\_xyz\index{frame\_xyz}} }\vspace*{-0.7em}
~\\\textbf{\Cb{Arguments: }}\begin{flushleft}
{\small \Cb{\hspace*{0.5cm}$\bullet$~~\texttt{size\_x[\%]{\comma}\_size\_y[\%]{\comma}\_size\_z[\%]\_col1{\comma}...{\comma}\_colN}}}\end{flushleft}
Insert colored frame along the x-axis in selected images.
\begin{flushleft}\Cc{\textbf{Default values}:\\~\\\hspace*{0.5cm}{\small $\bullet$~~\texttt{'size\_y=size\_x=size\_z'{\comma} 'col1=col2=col3=255'} and \texttt{'col4=255'.}}}\end{flushleft}


\subsection{\emph{frame\_y\index{frame\_y}} }\vspace*{-0.7em}
~\\\textbf{\Cb{Arguments: }}\begin{flushleft}
{\small \Cb{\hspace*{0.5cm}$\bullet$~~\texttt{size\_y[\%]{\comma}\_col1{\comma}...{\comma}\_colN}}}\end{flushleft}
Insert colored frame along the y-axis in selected images.
\begin{flushleft}\Cc{\textbf{Default values}:\\~\\\hspace*{0.5cm}{\small $\bullet$~~\texttt{'col1=col2=col3=255'} and \texttt{'col4=255'.}}}\end{flushleft}
\begin{center}\includegraphics[keepaspectratio=true,height=6cm,width=\textwidth]{img/gmic_stdlib570.jpg}\\
{\footnotesize \textbf{Example 570~:} \texttt{image.jpg frame\_y 20{\comma}255{\comma}0{\comma}255}}
\end{center}

\subsection{\emph{img2ascii\index{img2ascii}} }\vspace*{-0.7em}
~\\\textbf{\Cb{Arguments: }}\begin{flushleft}
{\small \Cb{\hspace*{0.5cm}$\bullet$~~\texttt{\_charset{\comma}\_analysis\_scale$>$0{\comma}\_analysis\_smoothness[\%]$>$=0{\comma}\_synth\-esis\_scale$>$0{\comma}\_output\_ascii\_filename}}}\end{flushleft}
Render selected images as binary ascii art.
~\\This command returns the corresponding the list of widths and heights (expressed as a number of characters) for each selected image.
\begin{flushleft}\Cc{\textbf{Default values}:\\~\\\hspace*{0.5cm}{\small $\bullet$~~\texttt{'charset=[ascii charset]'{\comma} 'analysis\_scale=16'{\comma} 'analysis\_smoothness=20\%'{\comma} 'synthesis\_scale=16'} and \texttt{'\_output\_ascii\_filename=[undefined]'.}}}\end{flushleft}
\begin{center}\includegraphics[keepaspectratio=true,height=6cm,width=\textwidth]{img/gmic_stdlib571.jpg}\\
{\footnotesize \textbf{Example 571~:} \texttt{image.jpg --img2ascii {\comma} resize[0] [1]{\comma}[1]{\comma}1{\comma}3 --mul}}
\end{center}

\subsection{\emph{imagegrid\index{imagegrid}} }\vspace*{-0.7em}
~\\\textbf{\Cb{Arguments: }}\begin{flushleft}
{\small \Cb{\hspace*{0.5cm}$\bullet$~~\texttt{M$>$0{\comma}\_N$>$0}}}\end{flushleft}
Create MxN image grid from selected images.
\begin{flushleft}\Cc{\textbf{Default value}:\\~\\\hspace*{0.5cm}{\small $\bullet$~~\texttt{'N=M'.}}}\end{flushleft}
\begin{center}\includegraphics[keepaspectratio=true,height=6cm,width=\textwidth]{img/gmic_stdlib572.jpg}\\
{\footnotesize \textbf{Example 572~:} \texttt{image.jpg imagegrid 16}}
\end{center}

\subsection{\emph{imagegrid\_hexagonal\index{imagegrid\_hexagonal}} }\vspace*{-0.7em}
~\\\textbf{\Cb{Arguments: }}\begin{flushleft}
{\small \Cb{\hspace*{0.5cm}$\bullet$~~\texttt{\_resolution$>$0{\comma}0$<$=\_outline$<$=1}}}\end{flushleft}
Create hexagonal grids from selected images.
\begin{flushleft}\Cc{\textbf{Default values}:\\~\\\hspace*{0.5cm}{\small $\bullet$~~\texttt{'resolution=32'{\comma} 'outline=0.1'} and \texttt{'is\_antialiased=1'.}}}\end{flushleft}
\begin{center}\includegraphics[keepaspectratio=true,height=6cm,width=\textwidth]{img/gmic_stdlib573.jpg}\\
{\footnotesize \textbf{Example 573~:} \texttt{image.jpg imagegrid\_hexagonal 24}}
\end{center}

\subsection{\emph{imagegrid\_triangular\index{imagegrid\_triangular}} }\vspace*{-0.7em}
~\\\textbf{\Cb{Arguments: }}\begin{flushleft}
{\small \Cb{\hspace*{0.5cm}$\bullet$~~\texttt{pattern\_width$>$=1{\comma}\_pattern\_height$>$=1{\comma}\_pattern\_type{\comma}0$<$=\_outlin\-e\_opacity$<$=1{\comma}\_outline\_color1{\comma}...}}}\end{flushleft}
Create triangular grids from selected images.
~\\'pattern type' can be \{ 0=horizontal ~$|$~ 1=vertical ~$|$~ 2=crossed ~$|$~ 3=cube ~$|$~ 4=decreasing ~$|$~ 5=increasing \}.
\begin{flushleft}\Cc{\textbf{Default values}:\\~\\\hspace*{0.5cm}{\small $\bullet$~~\texttt{'pattern\_width=24'{\comma} 'pattern\_height=pattern\_width'{\comma} 'pattern\_type=0'{\comma} 'outline\_opacity=0.1'} and \texttt{'outline\_color1=0'.}}}\end{flushleft}
\begin{center}\includegraphics[keepaspectratio=true,height=6cm,width=\textwidth]{img/gmic_stdlib574.jpg}\\
{\footnotesize \textbf{Example 574~:} \texttt{image.jpg imagegrid\_triangular 6{\comma}10{\comma}3{\comma}0.5}}
\end{center}

\subsection{\emph{linearize\_tiles\index{linearize\_tiles}} }\vspace*{-0.7em}
~\\\textbf{\Cb{Arguments: }}\begin{flushleft}
{\small \Cb{\hspace*{0.5cm}$\bullet$~~\texttt{M$>$0{\comma}\_N$>$0}}}\end{flushleft}
Linearize MxN tiles on selected images.
\begin{flushleft}\Cc{\textbf{Default value}:\\~\\\hspace*{0.5cm}{\small $\bullet$~~\texttt{'N=M'.}}}\end{flushleft}
\begin{center}\includegraphics[keepaspectratio=true,height=6cm,width=\textwidth]{img/gmic_stdlib575.jpg}\\
{\footnotesize \textbf{Example 575~:} \texttt{image.jpg --linearize\_tiles 16}}
\end{center}

\subsection{\emph{map\_sprites\index{map\_sprites}} }\vspace*{-0.7em}
~\\\textbf{\Cb{Arguments: }}\begin{flushleft}
{\small \Cb{\hspace*{0.5cm}$\bullet$~~\texttt{\_nb\_sprites$>$=1{\comma}\_allow\_rotation=\{ 0=none ~$|$~ 1=90 deg. ~$|$~ 2=180 \-deg. \}}}}\end{flushleft}
Map set of sprites (defined as the 'nb\_sprites' latest images of the selection) to other selected images{\comma}
according to the luminosity of their pixel values.
\begin{center}\includegraphics[keepaspectratio=true,height=6cm,width=\textwidth]{img/gmic_stdlib576.jpg}\\
{\footnotesize \textbf{Example 576~:} \texttt{image.jpg resize2dy 48 repeat 16 ball \{8+2*\$$>$\}{\comma}\$\{-RGB\} mul[-1] \{(1+\$$>$)/16\} done map\_sprites 16}}
\end{center}

\subsection{\emph{pack\index{pack}} }\vspace*{-0.7em}
~\\\textbf{\Cb{Arguments: }}\begin{flushleft}
{\small \Cb{\hspace*{0.5cm}$\bullet$~~\texttt{is\_ratio\_constraint=\{ 0 ~$|$~ 1 \}{\comma}\_sort\_criterion}}}\end{flushleft}
Pack selected images into a single image.
~\\The returned status contains the list of new (x{\comma}y) offsets for each input image.
~\\Parameter 'is\_ratio\_constraint' tells if the resulting image must tend to a square image.
\begin{flushleft}\Cc{\textbf{Default values}:\\~\\\hspace*{0.5cm}{\small $\bullet$~~\texttt{'is\_ratio\_constraint=0'} and \texttt{'sort\_criterion=max(w{\comma}h)'.}}}\end{flushleft}
\begin{center}\includegraphics[keepaspectratio=true,height=6cm,width=\textwidth]{img/gmic_stdlib577.jpg}\\
{\footnotesize \textbf{Example 577~:} \texttt{image.jpg repeat 10 --resize2dx[-1] 75\% balance\_gamma[-1] \$\{-RGB\} done pack 0}}
\end{center}

\subsection{\emph{puzzle\index{puzzle}} }\vspace*{-0.7em}
~\\\textbf{\Cb{Arguments: }}\begin{flushleft}
{\small \Cb{\hspace*{0.5cm}$\bullet$~~\texttt{\_width$>$0{\comma}\_height$>$0{\comma}\_M$>$=1{\comma}\_N$>$=1{\comma}\_curvature{\comma}\_centering{\comma}\_connec\-tors\_variability{\comma}\_resolution$>$=1}}}\end{flushleft}
Input puzzle binary mask with specified size and geometry.
\begin{flushleft}\Cc{\textbf{Default values}:\\~\\\hspace*{0.5cm}{\small $\bullet$~~\texttt{'width=height=512'{\comma} 'M=N=5'{\comma} 'curvature=0.5'{\comma} 'centering=0.5'{\comma} 'connectors\_variability=0.5'} and \texttt{'resolution=64'.}}}\end{flushleft}
\begin{center}\includegraphics[keepaspectratio=true,height=6cm,width=\textwidth]{img/gmic_stdlib578.jpg}\\
{\footnotesize \textbf{Example 578~:} \texttt{puzzle {\comma}}}
\end{center}

\subsection{\emph{quadratize\_tiles\index{quadratize\_tiles}} }\vspace*{-0.7em}
~\\\textbf{\Cb{Arguments: }}\begin{flushleft}
{\small \Cb{\hspace*{0.5cm}$\bullet$~~\texttt{M$>$0{\comma}\_N$>$0}}}\end{flushleft}
Quadratize MxN tiles on selected images.
\begin{flushleft}\Cc{\textbf{Default value}:\\~\\\hspace*{0.5cm}{\small $\bullet$~~\texttt{'N=M'.}}}\end{flushleft}
\begin{center}\includegraphics[keepaspectratio=true,height=6cm,width=\textwidth]{img/gmic_stdlib579.jpg}\\
{\footnotesize \textbf{Example 579~:} \texttt{image.jpg --quadratize\_tiles 16}}
\end{center}

\subsection{\emph{rotate\_tiles\index{rotate\_tiles}} }\vspace*{-0.7em}
~\\\textbf{\Cb{Arguments: }}\begin{flushleft}
{\small \Cb{\hspace*{0.5cm}$\bullet$~~\texttt{angle{\comma}\_M$>$0{\comma}N$>$0}}}\end{flushleft}
Apply MxN tiled-rotation effect on selected images.
\begin{flushleft}\Cc{\textbf{Default values}:\\~\\\hspace*{0.5cm}{\small $\bullet$~~\texttt{'M=8'} and \texttt{'N=M'.}}}\end{flushleft}
\begin{center}\includegraphics[keepaspectratio=true,height=6cm,width=\textwidth]{img/gmic_stdlib580.jpg}\\
{\footnotesize \textbf{Example 580~:} \texttt{image.jpg to\_rgba rotate\_tiles 10{\comma}8 drop\_shadow 10{\comma}10 display\_rgba}}
\end{center}

\subsection{\emph{shift\_tiles\index{shift\_tiles}} }\vspace*{-0.7em}
~\\\textbf{\Cb{Arguments: }}\begin{flushleft}
{\small \Cb{\hspace*{0.5cm}$\bullet$~~\texttt{M$>$0{\comma}\_N$>$0{\comma}\_amplitude}}}\end{flushleft}
Apply MxN tiled-shift effect on selected images.
\begin{flushleft}\Cc{\textbf{Default values}:\\~\\\hspace*{0.5cm}{\small $\bullet$~~\texttt{'N=M'} and \texttt{'amplitude=20'.}}}\end{flushleft}
\begin{center}\includegraphics[keepaspectratio=true,height=6cm,width=\textwidth]{img/gmic_stdlib581.jpg}\\
{\footnotesize \textbf{Example 581~:} \texttt{image.jpg --shift\_tiles 8{\comma}8{\comma}10}}
\end{center}

\subsection{\emph{taquin\index{taquin}} }\vspace*{-0.7em}
~\\\textbf{\Cb{Arguments: }}\begin{flushleft}
{\small \Cb{\hspace*{0.5cm}$\bullet$~~\texttt{M$>$0{\comma}\_N$>$0{\comma}\_remove\_tile=\{ 0=none ~$|$~ 1=first ~$|$~ 2=last ~$|$~ 3=random\- \}{\comma}\_relief{\comma}\_border\_thickness[\%]{\comma}\_border\_outline[\%]{\comma}\_outline\_\-color}}}\end{flushleft}
Create MxN taquin puzzle from selected images.
\begin{flushleft}\Cc{\textbf{Default value}:\\~\\\hspace*{0.5cm}{\small $\bullet$~~\texttt{'N=M'{\comma} 'relief=50'{\comma} 'border\_thickness=5'{\comma} 'border\_outline=0'} and \texttt{'remove\_tile=0'.}}}\end{flushleft}
\begin{center}\includegraphics[keepaspectratio=true,height=6cm,width=\textwidth]{img/gmic_stdlib582.jpg}\\
{\footnotesize \textbf{Example 582~:} \texttt{image.jpg --taquin 8}}
\end{center}

\subsection{\emph{tunnel\index{tunnel}} }\vspace*{-0.7em}
~\\\textbf{\Cb{Arguments: }}\begin{flushleft}
{\small \Cb{\hspace*{0.5cm}$\bullet$~~\texttt{\_level$>$=0{\comma}\_factor$>$0{\comma}\_centering\_x{\comma}\_centering\_y{\comma}\_opacity{\comma}\_angl\-e}}}\end{flushleft}
Apply tunnel effect on selected images.
\begin{flushleft}\Cc{\textbf{Default values}:\\~\\\hspace*{0.5cm}{\small $\bullet$~~\texttt{'level=9'{\comma} 'factor=80\%'{\comma} 'centering\_x=centering\_y=0.5'{\comma} 'opacity=1'} and \texttt{'angle=0'}}}\end{flushleft}
\begin{center}\includegraphics[keepaspectratio=true,height=6cm,width=\textwidth]{img/gmic_stdlib583.jpg}\\
{\footnotesize \textbf{Example 583~:} \texttt{image.jpg --tunnel 20}}
\end{center}
\section{Artistic}


\subsection{\emph{boxfitting\index{boxfitting}} }\vspace*{-0.7em}
~\\\textbf{\Cb{Arguments: }}\begin{flushleft}
{\small \Cb{\hspace*{0.5cm}$\bullet$~~\texttt{\_min\_box\_size$>$=1{\comma}\_max\_box\_size$>$=0{\comma}\_initial\_density$>$=0{\comma}\_nb\_at\-tempts$>$=1}}}\end{flushleft}
Apply box fitting effect on selected images{\comma} as displayed the web page:
[http://www.complexification.net/gallery/machines/boxFittingImg/]
\begin{flushleft}\Cc{\textbf{Default values}:\\~\\\hspace*{0.5cm}{\small $\bullet$~~\texttt{'min\_box\_size=1'{\comma} 'max\_box\_size=0'{\comma} 'initial\_density=0.1'} and \texttt{'nb\_attempts=3'.}}}\end{flushleft}
\begin{center}\includegraphics[keepaspectratio=true,height=6cm,width=\textwidth]{img/gmic_stdlib584.jpg}\\
{\footnotesize \textbf{Example 584~:} \texttt{image.jpg --boxfitting {\comma}}}
\end{center}

\subsection{\emph{brushify\index{brushify}} }\vspace*{-0.7em}
~\\\textbf{\Cb{Arguments: }}\begin{flushleft}
{\small \Cb{\hspace*{0.5cm}$\bullet$~~\texttt{[brush]{\comma}\_brush\_nb\_sizes$>$=1{\comma}0$<$=\_brush\_min\_size\_factor$<$=1{\comma}\_bru\-sh\_nb\_orientations$>$=1{\comma}\_brush\_light\_type{\comma}0$<$=\_brush\_light\_stre\-ngth$<$=1{\comma}\_brush\_opacity{\comma}\_painting\_density[\%]$>$=0{\comma}0$<$=\_painting\_\-contours\_coherence$<$=1{\comma}0$<$=\_painting\_orientation\_coherence$<$=1{\comma}\-\_painting\_coherence\_alpha[\%]$>$=0{\comma}\_painting\_coherence\_sigma[\%]\-$>$=0{\comma}\_painting\_primary\_angle{\comma}0$<$=\_painting\_angle\_dispersion$<$=1}}}\end{flushleft}
Apply specified brush to create painterly versions of specified images.
~\\'brush\_light\_type' can be \{ 0=none ~$|$~ 1=flat ~$|$~ 2=darken ~$|$~ 3=lighten ~$|$~ 4=full \}.
\begin{flushleft}\Cc{\textbf{Default values}:\\~\\\hspace*{0.5cm}{\small $\bullet$~~\texttt{'brush\_nb\_sizes=3'{\comma} 'brush\_min\_size\_factor=0.66'{\comma} 'brush\_nb\_orientations=12'{\comma} 'brush\_light\_type=0'{\comma} 'brush\_light\_strength=0.25'{\comma} 'brush\_opacity=0.8'{\comma} 'painting\_density=20\%'{\comma} 'painting\_contours\_coherence=0.9'{\comma} 'painting\_orientation\_coherence=0.9'{\comma} 'painting\_coherence\_alpha=1'{\comma} 'painting\_coherence\_sigma=1'{\comma} 'painting\_primary\_angle=0'{\comma} 'painting\_angle\_dispersion=0.2'}}}\end{flushleft}
\begin{center}\includegraphics[keepaspectratio=true,height=6cm,width=\textwidth]{img/gmic_stdlib585.jpg}\\
{\footnotesize \textbf{Example 585~:} \texttt{image.jpg 40{\comma}40 gaussian[-1] 8{\comma}2 spread[-1] 4{\comma}0 --brushify[0] [1]}}
\end{center}

\subsection{\emph{cartoon\index{cartoon}} }\vspace*{-0.7em}
~\\\textbf{\Cb{Arguments: }}\begin{flushleft}
{\small \Cb{\hspace*{0.5cm}$\bullet$~~\texttt{\_smoothness{\comma}\_sharpening{\comma}\_threshold$>$=0{\comma}\_thickness$>$=0{\comma}\_color$>$=\-0{\comma}quantization$>$0}}}\end{flushleft}
Apply cartoon effect on selected images.
\begin{flushleft}\Cc{\textbf{Default values}:\\~\\\hspace*{0.5cm}{\small $\bullet$~~\texttt{'smoothness=3'{\comma} 'sharpening=150'{\comma} 'threshold=20'{\comma} 'thickness=0.25'{\comma} 'color=1.5'} and \texttt{'quantization=8'.}}}\end{flushleft}
\begin{center}\includegraphics[keepaspectratio=true,height=6cm,width=\textwidth]{img/gmic_stdlib586.jpg}\\
{\footnotesize \textbf{Example 586~:} \texttt{image.jpg --cartoon 3{\comma}80{\comma}15}}
\end{center}

\subsection{\emph{color\_ellipses\index{color\_ellipses}} }\vspace*{-0.7em}
~\\\textbf{\Cb{Arguments: }}\begin{flushleft}
{\small \Cb{\hspace*{0.5cm}$\bullet$~~\texttt{\_count$>$0{\comma}\_radius$>$=0{\comma}\_opacity$>$=0}}}\end{flushleft}
Add random color ellipses to selected images.
\begin{flushleft}\Cc{\textbf{Default values}:\\~\\\hspace*{0.5cm}{\small $\bullet$~~\texttt{'count=400'{\comma} 'radius=5'} and \texttt{'opacity=0.1'.}}}\end{flushleft}
\begin{center}\includegraphics[keepaspectratio=true,height=6cm,width=\textwidth]{img/gmic_stdlib587.jpg}\\
{\footnotesize \textbf{Example 587~:} \texttt{image.jpg --color\_ellipses {\comma}{\comma}0.15}}
\end{center}

\subsection{\emph{cubism\index{cubism}} }\vspace*{-0.7em}
~\\\textbf{\Cb{Arguments: }}\begin{flushleft}
{\small \Cb{\hspace*{0.5cm}$\bullet$~~\texttt{\_density$>$=0{\comma}0$<$=\_thickness$<$=50{\comma}\_max\_angle{\comma}\_opacity{\comma}\_smoothnes\-s$>$=0}}}\end{flushleft}
Apply cubism effect on selected images.
\begin{flushleft}\Cc{\textbf{Default values}:\\~\\\hspace*{0.5cm}{\small $\bullet$~~\texttt{'density=50'{\comma} 'thickness=10'{\comma} 'max\_angle=75'{\comma} 'opacity=0.7'} and \texttt{'smoothness=0'.}}}\end{flushleft}
\begin{center}\includegraphics[keepaspectratio=true,height=6cm,width=\textwidth]{img/gmic_stdlib588.jpg}\\
{\footnotesize \textbf{Example 588~:} \texttt{image.jpg --cubism {\comma}}}
\end{center}

\subsection{\emph{draw\_whirl\index{draw\_whirl}} }\vspace*{-0.7em}
~\\\textbf{\Cb{Arguments: }}\begin{flushleft}
{\small \Cb{\hspace*{0.5cm}$\bullet$~~\texttt{\_amplitude$>$=0}}}\end{flushleft}
Apply whirl drawing effect on selected images.
\begin{flushleft}\Cc{\textbf{Default value}:\\~\\\hspace*{0.5cm}{\small $\bullet$~~\texttt{'amplitude=100'.}}}\end{flushleft}
\begin{center}\includegraphics[keepaspectratio=true,height=6cm,width=\textwidth]{img/gmic_stdlib589.jpg}\\
{\footnotesize \textbf{Example 589~:} \texttt{image.jpg --draw\_whirl {\comma}}}
\end{center}

\subsection{\emph{drawing\index{drawing}} }\vspace*{-0.7em}
~\\\textbf{\Cb{Arguments: }}\begin{flushleft}
{\small \Cb{\hspace*{0.5cm}$\bullet$~~\texttt{\_amplitude$>$=0}}}\end{flushleft}
Apply drawing effect on selected images.
\begin{flushleft}\Cc{\textbf{Default value}:\\~\\\hspace*{0.5cm}{\small $\bullet$~~\texttt{'amplitude=200'.}}}\end{flushleft}
\begin{center}\includegraphics[keepaspectratio=true,height=6cm,width=\textwidth]{img/gmic_stdlib590.jpg}\\
{\footnotesize \textbf{Example 590~:} \texttt{image.jpg --drawing {\comma}}}
\end{center}

\subsection{\emph{drop\_shadow\index{drop\_shadow}} }\vspace*{-0.7em}
~\\\textbf{\Cb{Arguments: }}\begin{flushleft}
{\small \Cb{\hspace*{0.5cm}$\bullet$~~\texttt{\_offset\_x[\%]{\comma}\_offset\_y[\%]{\comma}\_smoothness[\%]$>$=0{\comma}0$<$=\_curvature$<$=1\-{\comma}\_expand\_size=\{ 0 ~$|$~ 1 \}}}}\end{flushleft}
Drop shadow behind selected images.
\begin{flushleft}\Cc{\textbf{Default values}:\\~\\\hspace*{0.5cm}{\small $\bullet$~~\texttt{'offset\_x=20'{\comma} 'offset\_y=offset\_x'{\comma} 'smoothness=5'{\comma} 'curvature=0'} and \texttt{'expand\_size=1'.}}}\end{flushleft}
\begin{center}\includegraphics[keepaspectratio=true,height=6cm,width=\textwidth]{img/gmic_stdlib591.jpg}\\
{\footnotesize \textbf{Example 591~:} \texttt{image.jpg drop\_shadow 10{\comma}20{\comma}5{\comma}0.5 expand\_xy 20{\comma}0 display\_rgba}}
\end{center}

\subsection{\emph{ellipsionism\index{ellipsionism}} }\vspace*{-0.7em}
~\\\textbf{\Cb{Arguments: }}\begin{flushleft}
{\small \Cb{\hspace*{0.5cm}$\bullet$~~\texttt{\_R$>$0[\%]{\comma}\_r$>$0[\%]{\comma}\_smoothness$>$=0[\%]{\comma}\_opacity{\comma}\_outline$>$0{\comma}\_densi\-ty$>$0}}}\end{flushleft}
Apply ellipsionism filter to selected images.
\begin{flushleft}\Cc{\textbf{Default values}:\\~\\\hspace*{0.5cm}{\small $\bullet$~~\texttt{'R=10'{\comma} 'r=3'{\comma} 'smoothness=1\%'{\comma} 'opacity=0.7'{\comma} 'outlise=8'} and \texttt{'density=0.6'.}}}\end{flushleft}
\begin{center}\includegraphics[keepaspectratio=true,height=6cm,width=\textwidth]{img/gmic_stdlib592.jpg}\\
{\footnotesize \textbf{Example 592~:} \texttt{image.jpg --ellipsionism {\comma}}}
\end{center}

\subsection{\emph{fire\_edges\index{fire\_edges}} }\vspace*{-0.7em}
~\\\textbf{\Cb{Arguments: }}\begin{flushleft}
{\small \Cb{\hspace*{0.5cm}$\bullet$~~\texttt{\_edges$>$=0{\comma}0$<$=\_attenuation$<$=1{\comma}\_smoothness$>$=0{\comma}\_threshold$>$=0{\comma}\_n\-b\_frames$>$0{\comma}\_starting\_frame$>$=0{\comma}frame\_skip$>$=0}}}\end{flushleft}
Generate fire effect from edges of selected images.
\begin{flushleft}\Cc{\textbf{Default values}:\\~\\\hspace*{0.5cm}{\small $\bullet$~~\texttt{'edges=0.7'{\comma} 'attenuation=0.25'{\comma} 'smoothness=0.5'{\comma} 'threshold=25'{\comma} 'nb\_frames=1'{\comma} 'starting\_frame=20'} and \texttt{'frame\_skip=0'.}}}\end{flushleft}
\begin{center}\includegraphics[keepaspectratio=true,height=6cm,width=\textwidth]{img/gmic_stdlib593.jpg}\\
{\footnotesize \textbf{Example 593~:} \texttt{image.jpg fire\_edges {\comma}}}
\end{center}

\subsection{\emph{fractalize\index{fractalize}} }\vspace*{-0.7em}
~\\\textbf{\Cb{Arguments: }}\begin{flushleft}
{\small \Cb{\hspace*{0.5cm}$\bullet$~~\texttt{0$<$=detail\_level$<$=1}}}\end{flushleft}
Randomly fractalize selected images.
\begin{flushleft}\Cc{\textbf{Default value}:\\~\\\hspace*{0.5cm}{\small $\bullet$~~\texttt{'detail\_level=0.8'}}}\end{flushleft}
\begin{center}\includegraphics[keepaspectratio=true,height=6cm,width=\textwidth]{img/gmic_stdlib594.jpg}\\
{\footnotesize \textbf{Example 594~:} \texttt{image.jpg --fractalize {\comma}}}
\end{center}

\subsection{\emph{glow\index{glow}} }\vspace*{-0.7em}
~\\\textbf{\Cb{Arguments: }}\begin{flushleft}
{\small \Cb{\hspace*{0.5cm}$\bullet$~~\texttt{\_amplitude$>$=0}}}\end{flushleft}
Add soft glow on selected images.
\begin{flushleft}\Cc{\textbf{Default value}:\\~\\\hspace*{0.5cm}{\small $\bullet$~~\texttt{'amplitude=1\%'.}}}\end{flushleft}
\begin{center}\includegraphics[keepaspectratio=true,height=6cm,width=\textwidth]{img/gmic_stdlib595.jpg}\\
{\footnotesize \textbf{Example 595~:} \texttt{image.jpg --glow {\comma}}}
\end{center}

\subsection{\emph{halftone\index{halftone}} }\vspace*{-0.7em}
~\\\textbf{\Cb{Arguments: }}\begin{flushleft}
{\small \Cb{\hspace*{0.5cm}$\bullet$~~\texttt{nb\_levels$>$=2{\comma}\_size\_dark$>$=2{\comma}\_size\_bright$>$=2{\comma}\_shape=\{ 0=square\- ~$|$~ 1=diamond ~$|$~ 2=circle ~$|$~ 3=inv-square ~$|$~ 4=inv-diamond ~$|$~ 5=i\-nv-circle \}{\comma}\_smoothness[\%]$>$=0}}}\end{flushleft}
Apply halftone dithering to selected images.
\begin{flushleft}\Cc{\textbf{Default values}:\\~\\\hspace*{0.5cm}{\small $\bullet$~~\texttt{'nb\_levels=5'{\comma} 'size\_dark=8'{\comma} 'size\_bright=8'{\comma} 'shape=5'} and \texttt{'smoothnesss=0'.}}}\end{flushleft}
\begin{center}\includegraphics[keepaspectratio=true,height=6cm,width=\textwidth]{img/gmic_stdlib596.jpg}\\
{\footnotesize \textbf{Example 596~:} \texttt{image.jpg --halftone {\comma}}}
\end{center}

\subsection{\emph{hardsketchbw\index{hardsketchbw}} }\vspace*{-0.7em}
~\\\textbf{\Cb{Arguments: }}\begin{flushleft}
{\small \Cb{\hspace*{0.5cm}$\bullet$~~\texttt{\_amplitude$>$=0{\comma}\_density$>$=0{\comma}\_opacity{\comma}0$<$=\_edge\_threshold$<$=100{\comma}\_\-is\_fast=\{ 0 ~$|$~ 1 \}}}}\end{flushleft}
Apply hard B\&W sketch effect on selected images.
\begin{flushleft}\Cc{\textbf{Default values}:\\~\\\hspace*{0.5cm}{\small $\bullet$~~\texttt{'amplitude=1000'{\comma} 'sampling=3'{\comma} 'opacity=0.1'{\comma} 'edge\_threshold=20'} and \texttt{'is\_fast=0'.}}}\end{flushleft}
\begin{center}\includegraphics[keepaspectratio=true,height=6cm,width=\textwidth]{img/gmic_stdlib597.jpg}\\
{\footnotesize \textbf{Example 597~:} \texttt{image.jpg --hardsketchbw 200{\comma}70{\comma}0.1{\comma}10 median[-1] 2 --local reverse blur[-1] 3 blend[-2{\comma}-1] overlay endlocal}}
\end{center}

\subsection{\emph{hearts\index{hearts}} }\vspace*{-0.7em}
~\\\textbf{\Cb{Arguments: }}\begin{flushleft}
{\small \Cb{\hspace*{0.5cm}$\bullet$~~\texttt{\_density$>$=0}}}\end{flushleft}
Apply heart effect on selected images.
\begin{flushleft}\Cc{\textbf{Default value}:\\~\\\hspace*{0.5cm}{\small $\bullet$~~\texttt{'density=10'.}}}\end{flushleft}
\begin{center}\includegraphics[keepaspectratio=true,height=6cm,width=\textwidth]{img/gmic_stdlib598.jpg}\\
{\footnotesize \textbf{Example 598~:} \texttt{image.jpg --hearts {\comma}}}
\end{center}

\subsection{\emph{houghsketchbw\index{houghsketchbw}} }\vspace*{-0.7em}
~\\\textbf{\Cb{Arguments: }}\begin{flushleft}
{\small \Cb{\hspace*{0.5cm}$\bullet$~~\texttt{\_density$>$=0{\comma}\_radius$>$0{\comma}0$<$=\_threshold$<$=100{\comma}0$<$=\_opacity$<$=1{\comma}\_vot\-esize[\%]$>$0}}}\end{flushleft}
Apply hough B\&W sketch effect on selected images.
\begin{flushleft}\Cc{\textbf{Default values}:\\~\\\hspace*{0.5cm}{\small $\bullet$~~\texttt{'density=8'{\comma} 'radius=5'{\comma} 'threshold=80'{\comma} 'opacity=0.1'} and \texttt{'votesize=100\%'.}}}\end{flushleft}
\begin{center}\includegraphics[keepaspectratio=true,height=6cm,width=\textwidth]{img/gmic_stdlib599.jpg}\\
{\footnotesize \textbf{Example 599~:} \texttt{image.jpg --houghsketchbw {\comma}}}
\end{center}

\subsection{\emph{lightrays\index{lightrays}} }\vspace*{-0.7em}
~\\\textbf{\Cb{Arguments: }}\begin{flushleft}
{\small \Cb{\hspace*{0.5cm}$\bullet$~~\texttt{100$<$=\_density$<$=0{\comma}\_center\_x[\%]{\comma}\_center\_y[\%]{\comma}\_ray\_length$>$=0{\comma}\_r\-ay\_attenuation$>$=0}}}\end{flushleft}
Generate ray lights from the edges of selected images.
~\\Defaults values : 'density=50\%'{\comma} 'center\_x=50\%'{\comma} 'center\_y=50\%'{\comma} 'ray\_length=0.9' and 'ray\_attenuation=0.5'.
\begin{center}\includegraphics[keepaspectratio=true,height=6cm,width=\textwidth]{img/gmic_stdlib600.jpg}\\
{\footnotesize \textbf{Example 600~:} \texttt{image.jpg --lightrays {\comma} + cut 0{\comma}255}}
\end{center}

\subsection{\emph{light\_relief\index{light\_relief}} }\vspace*{-0.7em}
~\\\textbf{\Cb{Arguments: }}\begin{flushleft}
{\small \Cb{\hspace*{0.5cm}$\bullet$~~\texttt{\_ambient\_light{\comma}\_specular\_lightness{\comma}\_specular\_size{\comma}\_light\_smo\-othness{\comma}\_darkness{\comma}\_xl{\comma}\_yl{\comma}\_zl{\comma}\_zscale{\comma}\_opacity\_is\_heightmap=\-\{ 0 ~$|$~ 1 \}}}}\end{flushleft}
Apply relief light to selected images.
\begin{flushleft}\Cc{\textbf{Default values(s)}:\\~\\\hspace*{0.5cm}{\small $\bullet$~~\texttt{'ambient\_light=0.3'{\comma} 'specular\_lightness=0.5'{\comma} 'specular\_size=0.2'{\comma} 'darkness=0'{\comma} 'xl=0.2'{\comma} 'yl=zl=0.5'{\comma}}}}\end{flushleft}
~\\'zscale=1'{\comma} 'opacity=1' and 'opacity\_is\_heightmap=0'.
\begin{center}\includegraphics[keepaspectratio=true,height=6cm,width=\textwidth]{img/gmic_stdlib601.jpg}\\
{\footnotesize \textbf{Example 601~:} \texttt{image.jpg --blur 2 light\_relief[-1] 0.3{\comma}4{\comma}0.1{\comma}0}}
\end{center}

\subsection{\emph{mosaic\index{mosaic}} }\vspace*{-0.7em}
~\\\textbf{\Cb{Arguments: }}\begin{flushleft}
{\small \Cb{\hspace*{0.5cm}$\bullet$~~\texttt{0$<$=\_density$<$=100}}}\end{flushleft}
Create random mosaic from selected images.
\begin{flushleft}\Cc{\textbf{Default values}:\\~\\\hspace*{0.5cm}{\small $\bullet$~~\texttt{'density=30'.}}}\end{flushleft}
\begin{center}\includegraphics[keepaspectratio=true,height=6cm,width=\textwidth]{img/gmic_stdlib602.jpg}\\
{\footnotesize \textbf{Example 602~:} \texttt{image.jpg --mosaic {\comma}}}
\end{center}

\subsection{\emph{old\_photo\index{old\_photo}} }\vspace*{-0.7em}
Apply old photo effect on selected images.
\begin{center}\includegraphics[keepaspectratio=true,height=6cm,width=\textwidth]{img/gmic_stdlib603.jpg}\\
{\footnotesize \textbf{Example 603~:} \texttt{image.jpg --old\_photo}}
\end{center}

\subsection{\emph{pencilbw\index{pencilbw}} }\vspace*{-0.7em}
~\\\textbf{\Cb{Arguments: }}\begin{flushleft}
{\small \Cb{\hspace*{0.5cm}$\bullet$~~\texttt{\_size$>$=0{\comma}\_amplitude$>$=0}}}\end{flushleft}
Apply B\&W pencil effect on selected images.
\begin{flushleft}\Cc{\textbf{Default values}:\\~\\\hspace*{0.5cm}{\small $\bullet$~~\texttt{'size=0.3'} and \texttt{'amplitude=60'.}}}\end{flushleft}
\begin{center}\includegraphics[keepaspectratio=true,height=6cm,width=\textwidth]{img/gmic_stdlib604.jpg}\\
{\footnotesize \textbf{Example 604~:} \texttt{image.jpg --pencilbw {\comma}}}
\end{center}

\subsection{\emph{pixelsort\index{pixelsort}} }\vspace*{-0.7em}
~\\\textbf{\Cb{Arguments: }}\begin{flushleft}
{\small \Cb{\hspace*{0.5cm}$\bullet$~~\texttt{\_ordering=\{ + ~$|$~ - \}{\comma}\_axis=\{ x ~$|$~ y ~$|$~ z ~$|$~ xy ~$|$~ yx \}{\comma}\_[sorting\_\-criterion]{\comma}\_[mask]}}}\end{flushleft}
Apply a 'pixel sorting' algorithm on selected images{\comma} as described in the page :
http://satyarth.me/articles/pixel-sorting/
\begin{center}\includegraphics[keepaspectratio=true,height=6cm,width=\textwidth]{img/gmic_stdlib605.jpg}\\
{\footnotesize \textbf{Example 605~:} \texttt{image.jpg --norm --ge[-1] 30\% --pixelsort[0] +{\comma}y{\comma}[1]{\comma}[2]}}
\end{center}

\subsection{\emph{polaroid\index{polaroid}} }\vspace*{-0.7em}
~\\\textbf{\Cb{Arguments: }}\begin{flushleft}
{\small \Cb{\hspace*{0.5cm}$\bullet$~~\texttt{\_size1$>$=0{\comma}\_size2$>$=0}}}\end{flushleft}
Create polaroid effect in selected images.
\begin{flushleft}\Cc{\textbf{Default values}:\\~\\\hspace*{0.5cm}{\small $\bullet$~~\texttt{'size1=10'} and \texttt{'size2=20'.}}}\end{flushleft}
\begin{center}\includegraphics[keepaspectratio=true,height=6cm,width=\textwidth]{img/gmic_stdlib606.jpg}\\
{\footnotesize \textbf{Example 606~:} \texttt{image.jpg to\_rgba polaroid 5{\comma}30 rotate 20 drop\_shadow {\comma} display\_rgba}}
\end{center}

\subsection{\emph{polygonize\index{polygonize}} }\vspace*{-0.7em}
~\\\textbf{\Cb{Arguments: }}\begin{flushleft}
{\small \Cb{\hspace*{0.5cm}$\bullet$~~\texttt{\_warp\_amplitude$>$=0{\comma}\_smoothness[\%]$>$=0{\comma}\_min\_area[\%]$>$=0{\comma}\_resolu\-tion\_x[\%]$>$0{\comma}\_resolution\_y[\%]$>$0}}}\end{flushleft}
Apply polygon effect on selected images.
\begin{flushleft}\Cc{\textbf{Default values}:\\~\\\hspace*{0.5cm}{\small $\bullet$~~\texttt{'warp\_amplitude=300'{\comma} 'smoothness=2\%'{\comma} 'min\_area=0.1\%'{\comma} 'resolution\_x=resolution\_y=10\%'.}}}\end{flushleft}
\begin{center}\includegraphics[keepaspectratio=true,height=6cm,width=\textwidth]{img/gmic_stdlib607.jpg}\\
{\footnotesize \textbf{Example 607~:} \texttt{image.jpg --polygonize {\comma}}}
\end{center}

\subsection{\emph{poster\_edges\index{poster\_edges}} }\vspace*{-0.7em}
~\\\textbf{\Cb{Arguments: }}\begin{flushleft}
{\small \Cb{\hspace*{0.5cm}$\bullet$~~\texttt{0$<$=\_edge\_threshold$<$=100{\comma}0$<$=\_edge\_shade$<$=100{\comma}\_edge\_thickness$>$\-=0{\comma}\_edge\_antialiasing$>$=0{\comma}0$<$=\_posterization\_level$<$=15{\comma}\_poster\-ization\_antialiasing$>$=0}}}\end{flushleft}
Apply poster edges effect on selected images.
\begin{flushleft}\Cc{\textbf{Default values}:\\~\\\hspace*{0.5cm}{\small $\bullet$~~\texttt{'edge\_threshold=40'{\comma} 'edge\_shade=5'{\comma} 'edge\_thickness=0.5'{\comma} 'edge\_antialiasing=10'{\comma} 'posterization\_level=12'} and \texttt{'posterization\_antialiasing=0'.}}}\end{flushleft}
\begin{center}\includegraphics[keepaspectratio=true,height=6cm,width=\textwidth]{img/gmic_stdlib608.jpg}\\
{\footnotesize \textbf{Example 608~:} \texttt{image.jpg --poster\_edges {\comma}}}
\end{center}

\subsection{\emph{poster\_hope\index{poster\_hope}} }\vspace*{-0.7em}
~\\\textbf{\Cb{Arguments: }}\begin{flushleft}
{\small \Cb{\hspace*{0.5cm}$\bullet$~~\texttt{\_smoothness$>$=0}}}\end{flushleft}
Apply Hope stencil poster effect on selected images.
\begin{flushleft}\Cc{\textbf{Default value}:\\~\\\hspace*{0.5cm}{\small $\bullet$~~\texttt{'smoothness=3'.}}}\end{flushleft}
\begin{center}\includegraphics[keepaspectratio=true,height=6cm,width=\textwidth]{img/gmic_stdlib609.jpg}\\
{\footnotesize \textbf{Example 609~:} \texttt{image.jpg --poster\_hope {\comma}}}
\end{center}

\subsection{\emph{rodilius\index{rodilius}} }\vspace*{-0.7em}
~\\\textbf{\Cb{Arguments: }}\begin{flushleft}
{\small \Cb{\hspace*{0.5cm}$\bullet$~~\texttt{0$<$=\_amplitude$<$=100{\comma}\_0$<$=thickness$<$=100{\comma}\_sharpness$>$=0{\comma}\_nb\_orie\-ntations$>$0{\comma}\_offset{\comma}\_color\_mode=\{ 0=darker ~$|$~ 1=brighter \}}}}\end{flushleft}
Apply rodilius (fractalius-like) filter on selected images.
\begin{flushleft}\Cc{\textbf{Default values}:\\~\\\hspace*{0.5cm}{\small $\bullet$~~\texttt{'amplitude=10'{\comma} 'thickness=10'{\comma} 'sharpness=400'{\comma} 'nb\_orientations=7'{\comma} 'offset=0'} and \texttt{'color\_mode=1'.}}}\end{flushleft}
\begin{center}\includegraphics[keepaspectratio=true,height=6cm,width=\textwidth]{img/gmic_stdlib610.jpg}\\
{\footnotesize \textbf{Example 610~:} \texttt{image.jpg --rodilius 12{\comma}10{\comma}300{\comma}10 normalize\_local[-1] 10{\comma}6}}
\end{center}

\subsection{\emph{stained\_glass\index{stained\_glass}} }\vspace*{-0.7em}
~\\\textbf{\Cb{Arguments: }}\begin{flushleft}
{\small \Cb{\hspace*{0.5cm}$\bullet$~~\texttt{\_edges[\%]$>$=0{\comma} shading$>$=0{\comma} is\_thin\_separators=\{ 0 ~$|$~ 1 \}}}}\end{flushleft}
Generate stained glass from selected images.
\begin{flushleft}\Cc{\textbf{Default values}:\\~\\\hspace*{0.5cm}{\small $\bullet$~~\texttt{'edges=40\%'{\comma} 'shading=0.2'} and \texttt{'is\_precise=0'.}}}\end{flushleft}
\begin{center}\includegraphics[keepaspectratio=true,height=6cm,width=\textwidth]{img/gmic_stdlib611.jpg}\\
{\footnotesize \textbf{Example 611~:} \texttt{image.jpg --stained\_glass {\comma}}}
\end{center}

\subsection{\emph{stars\index{stars}} }\vspace*{-0.7em}
~\\\textbf{\Cb{Arguments: }}\begin{flushleft}
{\small \Cb{\hspace*{0.5cm}$\bullet$~~\texttt{\_density[\%]$>$=0{\comma}\_depth$>$=0{\comma}\_size$>$0{\comma}\_nb\_branches$>$=1{\comma}0$<$=\_thickne\-ss$<$=1{\comma}\_smoothness[\%]$>$=0{\comma}\_R{\comma}\_G{\comma}\_B{\comma}\_opacity}}}\end{flushleft}
Add random stars to selected images.
\begin{flushleft}\Cc{\textbf{Default values}:\\~\\\hspace*{0.5cm}{\small $\bullet$~~\texttt{'density=10\%'{\comma} 'depth=1'{\comma} 'size=32'{\comma} 'nb\_branches=5'{\comma} 'thickness=0.38'{\comma} 'smoothness=0.5'{\comma} 'R=G=B=200'} and \texttt{'opacity=1'.}}}\end{flushleft}
\begin{center}\includegraphics[keepaspectratio=true,height=6cm,width=\textwidth]{img/gmic_stdlib612.jpg}\\
{\footnotesize \textbf{Example 612~:} \texttt{image.jpg stars {\comma}}}
\end{center}

\subsection{\emph{sketchbw\index{sketchbw}} }\vspace*{-0.7em}
~\\\textbf{\Cb{Arguments: }}\begin{flushleft}
{\small \Cb{\hspace*{0.5cm}$\bullet$~~\texttt{\_nb\_orients$>$0{\comma}\_start\_angle{\comma}\_angle\_range$>$=0{\comma}\_length$>$=0{\comma}\_thres\-hold$>$=0{\comma}\_opacity{\comma}\_bgfactor$>$=0{\comma}\_density$>$0{\comma}\_sharpness$>$=0{\comma}\_anis\-otropy$>$=0{\comma}\_smoothness$>$=0{\comma}\_coherence$>$=0{\comma}\_is\_boost=\{ 0 ~$|$~ 1 \}{\comma}\_\-is\_curved=\{ 0 ~$|$~ 1 \}}}}\end{flushleft}
Apply sketch effect to selected images.
\begin{flushleft}\Cc{\textbf{Default values}:\\~\\\hspace*{0.5cm}{\small $\bullet$~~\texttt{'nb\_orients=2'{\comma} 'start\_angle=45'{\comma} 'angle\_range=180'{\comma} 'length=30'{\comma} 'threshold=1'{\comma} 'opacity=0.03'{\comma}}}}\end{flushleft}
~\\'bgfactor=0'{\comma} 'density=0.6'{\comma} 'sharpness=0.1'{\comma} 'anisotropy=0.6'{\comma} 'smoothness=0.25'{\comma} 'coherence=1'{\comma} 'is\_boost=0' and 'is\_curved=1'.
\begin{center}\includegraphics[keepaspectratio=true,height=6cm,width=\textwidth]{img/gmic_stdlib613.jpg}\\
{\footnotesize \textbf{Example 613~:} \texttt{image.jpg --sketchbw 1 --local reverse blur[-1] 3 blend[-2{\comma}-1] overlay endlocal}}
\end{center}

\subsection{\emph{sponge\index{sponge}} }\vspace*{-0.7em}
~\\\textbf{\Cb{Arguments: }}\begin{flushleft}
{\small \Cb{\hspace*{0.5cm}$\bullet$~~\texttt{\_size$>$0}}}\end{flushleft}
Apply sponge effect on selected images.
\begin{flushleft}\Cc{\textbf{Default value}:\\~\\\hspace*{0.5cm}{\small $\bullet$~~\texttt{'size=13'.}}}\end{flushleft}
\begin{center}\includegraphics[keepaspectratio=true,height=6cm,width=\textwidth]{img/gmic_stdlib614.jpg}\\
{\footnotesize \textbf{Example 614~:} \texttt{image.jpg --sponge {\comma}}}
\end{center}

\subsection{\emph{stencil\index{stencil}} }\vspace*{-0.7em}
~\\\textbf{\Cb{Arguments: }}\begin{flushleft}
{\small \Cb{\hspace*{0.5cm}$\bullet$~~\texttt{\_radius[\%]$>$=0{\comma}\_smoothness$>$=0{\comma}\_iterations$>$=0}}}\end{flushleft}
Apply stencil filter on selected images.
\begin{flushleft}\Cc{\textbf{Default values}:\\~\\\hspace*{0.5cm}{\small $\bullet$~~\texttt{'radius=3'{\comma} 'smoothness=1'} and \texttt{'iterations=8'.}}}\end{flushleft}
\begin{center}\includegraphics[keepaspectratio=true,height=6cm,width=\textwidth]{img/gmic_stdlib615.jpg}\\
{\footnotesize \textbf{Example 615~:} \texttt{image.jpg --stencil 1{\comma}10{\comma}3}}
\end{center}

\subsection{\emph{stencilbw\index{stencilbw}} }\vspace*{-0.7em}
~\\\textbf{\Cb{Arguments: }}\begin{flushleft}
{\small \Cb{\hspace*{0.5cm}$\bullet$~~\texttt{\_edges$>$=0{\comma}\_smoothness$>$=0}}}\end{flushleft}
Apply B\&W stencil effect on selected images.
\begin{flushleft}\Cc{\textbf{Default values}:\\~\\\hspace*{0.5cm}{\small $\bullet$~~\texttt{'edges=15'} and \texttt{'smoothness=10'.}}}\end{flushleft}
\begin{center}\includegraphics[keepaspectratio=true,height=6cm,width=\textwidth]{img/gmic_stdlib616.jpg}\\
{\footnotesize \textbf{Example 616~:} \texttt{image.jpg --stencilbw 40{\comma}4}}
\end{center}

\subsection{\emph{tetris\index{tetris}} }\vspace*{-0.7em}
~\\\textbf{\Cb{Arguments: }}\begin{flushleft}
{\small \Cb{\hspace*{0.5cm}$\bullet$~~\texttt{\_scale$>$0}}}\end{flushleft}
Apply tetris effect on selected images.
\begin{flushleft}\Cc{\textbf{Default value}:\\~\\\hspace*{0.5cm}{\small $\bullet$~~\texttt{'scale=10'.}}}\end{flushleft}
\begin{center}\includegraphics[keepaspectratio=true,height=6cm,width=\textwidth]{img/gmic_stdlib617.jpg}\\
{\footnotesize \textbf{Example 617~:} \texttt{image.jpg --tetris 10}}
\end{center}

\subsection{\emph{warhol\index{warhol}} }\vspace*{-0.7em}
~\\\textbf{\Cb{Arguments: }}\begin{flushleft}
{\small \Cb{\hspace*{0.5cm}$\bullet$~~\texttt{\_M$>$0{\comma}\_N$>$0{\comma}\_smoothness$>$=0{\comma}\_color$>$=0}}}\end{flushleft}
Create MxN Andy Warhol-like artwork from selected images.
\begin{flushleft}\Cc{\textbf{Default values}:\\~\\\hspace*{0.5cm}{\small $\bullet$~~\texttt{'M=3'{\comma} 'N=M'{\comma} 'smoothness=2'} and \texttt{'color=20'.}}}\end{flushleft}
\begin{center}\includegraphics[keepaspectratio=true,height=6cm,width=\textwidth]{img/gmic_stdlib618.jpg}\\
{\footnotesize \textbf{Example 618~:} \texttt{image.jpg --warhol 5{\comma}3{\comma}3{\comma}40}}
\end{center}

\subsection{\emph{weave\index{weave}} }\vspace*{-0.7em}
~\\\textbf{\Cb{Arguments: }}\begin{flushleft}
{\small \Cb{\hspace*{0.5cm}$\bullet$~~\texttt{\_density$>$=0{\comma}0$<$=\_thickness$<$=100{\comma}0$<$=\_shadow$<$=100{\comma}\_shading$>$=0{\comma}\_\-fibers\_amplitude$>$=0{\comma}\_fibers\_smoothness$>$=0{\comma}\_angle{\comma}-1$<$=\_x\_curv\-ature$<$=1{\comma}-1$<$=\_y\_curvature$<$=1}}}\end{flushleft}
Apply weave effect to the selected images.
~\\'angle' can be \{ 0=0 deg. ~$|$~ 1=22.5 deg. ~$|$~ 2=45 deg. ~$|$~ 3=67.5 deg. \}.
\begin{flushleft}\Cc{\textbf{Default values}:\\~\\\hspace*{0.5cm}{\small $\bullet$~~\texttt{'density=6'{\comma} 'thickness=65'{\comma} 'shadow=40'{\comma} 'shading=0.5'{\comma} 'fibers\_amplitude=0'{\comma} 'fibers\_smoothness=0'{\comma} 'angle=0'} and \texttt{'curvature\_x=curvature\_y=0'}}}\end{flushleft}
\begin{center}\includegraphics[keepaspectratio=true,height=6cm,width=\textwidth]{img/gmic_stdlib619.jpg}\\
{\footnotesize \textbf{Example 619~:} \texttt{image.jpg --weave {\comma}}}
\end{center}

\subsection{\emph{whirls\index{whirls}} }\vspace*{-0.7em}
~\\\textbf{\Cb{Arguments: }}\begin{flushleft}
{\small \Cb{\hspace*{0.5cm}$\bullet$~~\texttt{\_texture$>$=0{\comma}\_smoothness$>$=0{\comma}\_darkness$>$=0{\comma}\_lightness$>$=0}}}\end{flushleft}
Add random whirl texture to selected images.
\begin{flushleft}\Cc{\textbf{Default values}:\\~\\\hspace*{0.5cm}{\small $\bullet$~~\texttt{'texture=3'{\comma} 'smoothness=6'{\comma} 'darkness=0.5'} and \texttt{'lightness=1.8'.}}}\end{flushleft}
\begin{center}\includegraphics[keepaspectratio=true,height=6cm,width=\textwidth]{img/gmic_stdlib620.jpg}\\
{\footnotesize \textbf{Example 620~:} \texttt{image.jpg --whirls {\comma}}}
\end{center}
\section{Warpings}


\subsection{\emph{deform\index{deform}} }\vspace*{-0.7em}
~\\\textbf{\Cb{Arguments: }}\begin{flushleft}
{\small \Cb{\hspace*{0.5cm}$\bullet$~~\texttt{\_amplitude$>$=0{\comma}\_interpolation}}}\end{flushleft}
Apply random smooth deformation on selected images.
~\\'interpolation' can be \{ 0=none ~$|$~ 1=linear ~$|$~ 2=bicubic \}.
\begin{flushleft}\Cc{\textbf{Default value}:\\~\\\hspace*{0.5cm}{\small $\bullet$~~\texttt{'amplitude=10'.}}}\end{flushleft}
\begin{center}\includegraphics[keepaspectratio=true,height=6cm,width=\textwidth]{img/gmic_stdlib621.jpg}\\
{\footnotesize \textbf{Example 621~:} \texttt{image.jpg --deform[0] 10 --deform[0] 20}}
\end{center}

\subsection{\emph{euclidean2polar\index{euclidean2polar}} }\vspace*{-0.7em}
~\\\textbf{\Cb{Arguments: }}\begin{flushleft}
{\small \Cb{\hspace*{0.5cm}$\bullet$~~\texttt{\_center\_x[\%]{\comma}\_center\_y[\%]{\comma}\_stretch\_factor$>$0{\comma}\_boundary\_condit\-ions=\{ 0=dirichlet ~$|$~ 1=neumann ~$|$~ 2=periodic ~$|$~ 3=mirror \}}}}\end{flushleft}
Apply euclidean to polar transform on selected images.
\begin{flushleft}\Cc{\textbf{Default values}:\\~\\\hspace*{0.5cm}{\small $\bullet$~~\texttt{'center\_x=center\_y=50\%'{\comma} 'stretch\_factor=1'} and \texttt{'boundary\_conditions=1'.}}}\end{flushleft}
\begin{center}\includegraphics[keepaspectratio=true,height=6cm,width=\textwidth]{img/gmic_stdlib622.jpg}\\
{\footnotesize \textbf{Example 622~:} \texttt{image.jpg --euclidean2polar {\comma}}}
\end{center}

\subsection{\emph{equirectangular2nadirzenith\index{equirectangular2nadirzenith}} }\vspace*{-0.7em}
Transform selected equirectangular images to nadir/zenith rectilinear projections.


\subsection{\emph{fisheye\index{fisheye}} }\vspace*{-0.7em}
~\\\textbf{\Cb{Arguments: }}\begin{flushleft}
{\small \Cb{\hspace*{0.5cm}$\bullet$~~\texttt{\_center\_x{\comma}\_center\_y{\comma}0$<$=\_radius$<$=100{\comma}\_amplitude$>$=0}}}\end{flushleft}
Apply fish-eye deformation on selected images.
\begin{flushleft}\Cc{\textbf{Default values}:\\~\\\hspace*{0.5cm}{\small $\bullet$~~\texttt{'x=y=50'{\comma} 'radius=50'} and \texttt{'amplitude=1.2'.}}}\end{flushleft}
\begin{center}\includegraphics[keepaspectratio=true,height=6cm,width=\textwidth]{img/gmic_stdlib623.jpg}\\
{\footnotesize \textbf{Example 623~:} \texttt{image.jpg --fisheye {\comma}}}
\end{center}

\subsection{\emph{flower\index{flower}} }\vspace*{-0.7em}
~\\\textbf{\Cb{Arguments: }}\begin{flushleft}
{\small \Cb{\hspace*{0.5cm}$\bullet$~~\texttt{\_amplitude{\comma}\_frequency{\comma}\_offset\_r[\%]{\comma}\_angle{\comma}\_center\_x[\%]{\comma}\_cent\-er\_y[\%]{\comma}\_boundary\_conditions=\{ 0=dirichlet ~$|$~ 1=neumann ~$|$~ 2=p\-eriodic ~$|$~ 3=mirror\}}}}\end{flushleft}
Apply flower deformation on selected images.
\begin{flushleft}\Cc{\textbf{Default values}:\\~\\\hspace*{0.5cm}{\small $\bullet$~~\texttt{'amplitude=30'{\comma} 'frequency=6'{\comma} 'offset\_r=0'{\comma} 'angle=0'{\comma} 'center\_x=center\_y=50\%'} and \texttt{'boundary\_conditions=3'.}}}\end{flushleft}
\begin{center}\includegraphics[keepaspectratio=true,height=6cm,width=\textwidth]{img/gmic_stdlib624.jpg}\\
{\footnotesize \textbf{Example 624~:} \texttt{image.jpg flower {\comma}}}
\end{center}

\subsection{\emph{kaleidoscope\index{kaleidoscope}} }\vspace*{-0.7em}
~\\\textbf{\Cb{Arguments: }}\begin{flushleft}
{\small \Cb{\hspace*{0.5cm}$\bullet$~~\texttt{\_center\_x[\%]{\comma}\_center\_y[\%]{\comma}\_radius{\comma}\_angle{\comma}\_boundary\_condition\-s=\{ 0=dirichlet ~$|$~ 1=neumann ~$|$~ 2=periodic ~$|$~ 3=mirror \}}}}\end{flushleft}
Create kaleidoscope effect from selected images.
\begin{flushleft}\Cc{\textbf{Default values}:\\~\\\hspace*{0.5cm}{\small $\bullet$~~\texttt{'center\_x=center\_y=50\%'{\comma} 'radius=100'{\comma} 'angle=30'} and \texttt{'boundary\_conditions=3'.}}}\end{flushleft}
\begin{center}\includegraphics[keepaspectratio=true,height=6cm,width=\textwidth]{img/gmic_stdlib625.jpg}\\
{\footnotesize \textbf{Example 625~:} \texttt{image.jpg --kaleidoscope {\comma}}}
\end{center}

\subsection{\emph{map\_sphere\index{map\_sphere}} }\vspace*{-0.7em}
~\\\textbf{\Cb{Arguments: }}\begin{flushleft}
{\small \Cb{\hspace*{0.5cm}$\bullet$~~\texttt{\_width$>$0{\comma}\_height$>$0{\comma}\_radius{\comma}\_dilation$>$0{\comma}\_fading$>$=0{\comma}\_fading\_po\-wer$>$=0}}}\end{flushleft}
Map selected images on a sphere.
\begin{flushleft}\Cc{\textbf{Default values}:\\~\\\hspace*{0.5cm}{\small $\bullet$~~\texttt{'width=height=512'{\comma} 'radius=100'{\comma} 'dilation=0.5'{\comma} 'fading=0'} and \texttt{'fading\_power=0.5'.}}}\end{flushleft}
\begin{center}\includegraphics[keepaspectratio=true,height=6cm,width=\textwidth]{img/gmic_stdlib626.jpg}\\
{\footnotesize \textbf{Example 626~:} \texttt{image.jpg --map\_sphere {\comma}}}
\end{center}

\subsection{\emph{nadirzenith2equirectangular\index{nadirzenith2equirectangular}} }\vspace*{-0.7em}
Transform selected nadir/zenith rectilinear projections to equirectangular images.


\subsection{\emph{polar2euclidean\index{polar2euclidean}} }\vspace*{-0.7em}
~\\\textbf{\Cb{Arguments: }}\begin{flushleft}
{\small \Cb{\hspace*{0.5cm}$\bullet$~~\texttt{\_center\_x[\%]{\comma}\_center\_y[\%]{\comma}\_stretch\_factor$>$0{\comma}\_boundary\_condit\-ions=\{ 0=dirichlet ~$|$~ 1=neumann ~$|$~ 2=periodic ~$|$~ 3=mirror \}}}}\end{flushleft}
Apply euclidean to polar transform on selected images.
\begin{flushleft}\Cc{\textbf{Default values}:\\~\\\hspace*{0.5cm}{\small $\bullet$~~\texttt{'center\_x=center\_y=50\%'{\comma} 'stretch\_factor=1'} and \texttt{'boundary\_conditions=1'.}}}\end{flushleft}
\begin{center}\includegraphics[keepaspectratio=true,height=6cm,width=\textwidth]{img/gmic_stdlib627.jpg}\\
{\footnotesize \textbf{Example 627~:} \texttt{image.jpg --euclidean2polar {\comma}}}
\end{center}

\subsection{\emph{raindrops\index{raindrops}} }\vspace*{-0.7em}
~\\\textbf{\Cb{Arguments: }}\begin{flushleft}
{\small \Cb{\hspace*{0.5cm}$\bullet$~~\texttt{\_amplitude{\comma}\_density$>$=0{\comma}\_wavelength$>$=0{\comma}\_merging\_steps$>$=0}}}\end{flushleft}
Apply raindrops deformation on selected images.
\begin{flushleft}\Cc{\textbf{Default values}:\\~\\\hspace*{0.5cm}{\small $\bullet$~~\texttt{'amplitude=80'{\comma}'density=0.1'{\comma} 'wavelength=1'} and \texttt{'merging\_steps=0'.}}}\end{flushleft}
\begin{center}\includegraphics[keepaspectratio=true,height=6cm,width=\textwidth]{img/gmic_stdlib628.jpg}\\
{\footnotesize \textbf{Example 628~:} \texttt{image.jpg --raindrops {\comma}}}
\end{center}

\subsection{\emph{ripple\index{ripple}} }\vspace*{-0.7em}
~\\\textbf{\Cb{Arguments: }}\begin{flushleft}
{\small \Cb{\hspace*{0.5cm}$\bullet$~~\texttt{\_amplitude{\comma}\_bandwidth{\comma}\_shape=\{ 0=bloc ~$|$~ 1=triangle ~$|$~ 2=sine \-~$|$~ 3=sine+ ~$|$~ 4=random \}{\comma}\_angle{\comma}\_offset}}}\end{flushleft}
Apply ripple deformation on selected images.
\begin{flushleft}\Cc{\textbf{Default values}:\\~\\\hspace*{0.5cm}{\small $\bullet$~~\texttt{'amplitude=10'{\comma} 'bandwidth=10'{\comma} 'shape=2'{\comma} 'angle=0'} and \texttt{'offset=0'.}}}\end{flushleft}
\begin{center}\includegraphics[keepaspectratio=true,height=6cm,width=\textwidth]{img/gmic_stdlib629.jpg}\\
{\footnotesize \textbf{Example 629~:} \texttt{image.jpg --ripple {\comma}}}
\end{center}

\subsection{\emph{rotoidoscope\index{rotoidoscope}} }\vspace*{-0.7em}
~\\\textbf{\Cb{Arguments: }}\begin{flushleft}
{\small \Cb{\hspace*{0.5cm}$\bullet$~~\texttt{\_center\_x[\%]{\comma}\_center\_y[\%]{\comma}\_tiles$>$0{\comma}\_smoothness[\%]$>$=0{\comma}\_bounda\-ry\_conditions=\{ 0=dirichlet ~$|$~ 1=neumann ~$|$~ 2=periodic ~$|$~ 3=mir\-ror \}}}}\end{flushleft}
Create rotational kaleidoscope effect from selected images.
\begin{flushleft}\Cc{\textbf{Default values}:\\~\\\hspace*{0.5cm}{\small $\bullet$~~\texttt{'cx=cy=50\%'{\comma} 'tiles=10'{\comma} 'smoothness=1'} and \texttt{'boundary\_conditions=3'.}}}\end{flushleft}
\begin{center}\includegraphics[keepaspectratio=true,height=6cm,width=\textwidth]{img/gmic_stdlib630.jpg}\\
{\footnotesize \textbf{Example 630~:} \texttt{image.jpg --rotoidoscope {\comma}}}
\end{center}

\subsection{\emph{spherize\index{spherize}} }\vspace*{-0.7em}
~\\\textbf{\Cb{Arguments: }}\begin{flushleft}
{\small \Cb{\hspace*{0.5cm}$\bullet$~~\texttt{\_radius[\%]$>$=0{\comma}\_strength{\comma}\_smoothness[\%]$>$=0{\comma}\_center\_x[\%]{\comma}\_cent\-er\_y[\%]{\comma}\_ratio\_x/y$>$0{\comma}\_angle{\comma}\_interpolation}}}\end{flushleft}
Apply spherize effect on selected images.
\begin{flushleft}\Cc{\textbf{Default values}:\\~\\\hspace*{0.5cm}{\small $\bullet$~~\texttt{'radius=50\%'{\comma} 'strength=1'{\comma} 'smoothness=0'{\comma} 'center\_x=center\_y=50\%'{\comma} 'ratio\_x/y=1'{\comma} 'angle=0'} and \texttt{'interpolation=1'.}}}\end{flushleft}
\begin{center}\includegraphics[keepaspectratio=true,height=6cm,width=\textwidth]{img/gmic_stdlib631.jpg}\\
{\footnotesize \textbf{Example 631~:} \texttt{image.jpg grid 5\%{\comma}5\%{\comma}0{\comma}0{\comma}0.6{\comma}255 --spherize {\comma}}}
\end{center}

\subsection{\emph{symmetrize\index{symmetrize}} }\vspace*{-0.7em}
~\\\textbf{\Cb{Arguments: }}\begin{flushleft}
{\small \Cb{\hspace*{0.5cm}$\bullet$~~\texttt{\_x[\%]{\comma}\_y[\%]{\comma}\_angle{\comma}\_boundary\_conditions=\{ 0=dirichlet ~$|$~ 1=ne\-umann ~$|$~ 2=periodic ~$|$~ 3=mirror \}{\comma}\_is\_antisymmetry=\{ 0 ~$|$~ 1 \}{\comma}\_\-swap\_sides=\{ 0 ~$|$~ 1 \}}}}\end{flushleft}
Symmetrize selected image regarding specified axis.
\begin{flushleft}\Cc{\textbf{Default values}:\\~\\\hspace*{0.5cm}{\small $\bullet$~~\texttt{'x=y=50\%'{\comma} 'angle=90'{\comma} 'boundary\_conditions=3'{\comma} 'is\_antisymmetry=0'} and \texttt{'swap\_sides=0'.}}}\end{flushleft}
\begin{center}\includegraphics[keepaspectratio=true,height=6cm,width=\textwidth]{img/gmic_stdlib632.jpg}\\
{\footnotesize \textbf{Example 632~:} \texttt{image.jpg --symmetrize 50\%{\comma}50\%{\comma}45 --symmetrize[-1] 50\%{\comma}50\%{\comma}-45}}
\end{center}

\subsection{\emph{transform\_polar\index{transform\_polar}} }\vspace*{-0.7em}
~\\\textbf{\Cb{Arguments: }}\begin{flushleft}
{\small \Cb{\hspace*{0.5cm}$\bullet$~~\texttt{"expr\_radius"{\comma}\_"expr\_angle"{\comma}\_center\_x[\%]{\comma}\_center\_y[\%]{\comma}\_bound\-ary\_conditions=\{ 0=dirichlet ~$|$~ 1=neumann \}}}}\end{flushleft}
Apply user-defined transform on polar representation of selected images.
\begin{flushleft}\Cc{\textbf{Default values}:\\~\\\hspace*{0.5cm}{\small $\bullet$~~\texttt{'expr\_radius=R-r'{\comma} 'expr\_rangle=a'{\comma} 'center\_x=center\_y=50\%'} and \texttt{'boundary\_conditions=1'.}}}\end{flushleft}
\begin{center}\includegraphics[keepaspectratio=true,height=6cm,width=\textwidth]{img/gmic_stdlib633.jpg}\\
{\footnotesize \textbf{Example 633~:} \texttt{image.jpg --transform\_polar[0] R*(r/R)\textasciicircum 2{\comma}a --transform\_polar[0] r{\comma}2*a}}
\end{center}

\subsection{\emph{twirl\index{twirl}} }\vspace*{-0.7em}
~\\\textbf{\Cb{Arguments: }}\begin{flushleft}
{\small \Cb{\hspace*{0.5cm}$\bullet$~~\texttt{\_amplitude{\comma}\_center\_x[\%]{\comma}\_center\_y[\%]{\comma}\_boundary\_conditions=\{ \-0=dirichlet ~$|$~ 1=neumann ~$|$~ 2=periodic ~$|$~ 3=mirror \}}}}\end{flushleft}
Apply twirl deformation on selected images.
\begin{flushleft}\Cc{\textbf{Default values}:\\~\\\hspace*{0.5cm}{\small $\bullet$~~\texttt{'amplitude=1'{\comma} 'center\_x=center\_y=50\%'} and \texttt{'boundary\_conditions=3'.}}}\end{flushleft}
\begin{center}\includegraphics[keepaspectratio=true,height=6cm,width=\textwidth]{img/gmic_stdlib634.jpg}\\
{\footnotesize \textbf{Example 634~:} \texttt{image.jpg --twirl 0.6}}
\end{center}

\subsection{\emph{warp\_perspective\index{warp\_perspective}} }\vspace*{-0.7em}
~\\\textbf{\Cb{Arguments: }}\begin{flushleft}
{\small \Cb{\hspace*{0.5cm}$\bullet$~~\texttt{\_x-angle{\comma}\_y-angle{\comma}\_zoom$>$0{\comma}\_x-center{\comma}\_y-center{\comma}\_boundary\_cond\-itions=\{ 0=dirichlet ~$|$~ 1=neumann ~$|$~ 2=periodic ~$|$~ 3=mirror \}}}}\end{flushleft}
Warp selected images with perspective deformation.
\begin{flushleft}\Cc{\textbf{Default values}:\\~\\\hspace*{0.5cm}{\small $\bullet$~~\texttt{'x-angle=1.5'{\comma} 'y-angle=0'{\comma} 'zoom=1'{\comma} 'x-center=y-center=50'} and \texttt{'boundary\_conditions=2'.}}}\end{flushleft}
\begin{center}\includegraphics[keepaspectratio=true,height=6cm,width=\textwidth]{img/gmic_stdlib635.jpg}\\
{\footnotesize \textbf{Example 635~:} \texttt{image.jpg --warp\_perspective {\comma}}}
\end{center}

\subsection{\emph{water\index{water}} }\vspace*{-0.7em}
~\\\textbf{\Cb{Arguments: }}\begin{flushleft}
{\small \Cb{\hspace*{0.5cm}$\bullet$~~\texttt{\_amplitude{\comma}\_smoothness$>$=0{\comma}\_angle}}}\end{flushleft}
Apply water deformation on selected images.
\begin{flushleft}\Cc{\textbf{Default values}:\\~\\\hspace*{0.5cm}{\small $\bullet$~~\texttt{'amplitude=30'{\comma} 'smoothness=1.5'} and \texttt{'angle=45'.}}}\end{flushleft}
\begin{center}\includegraphics[keepaspectratio=true,height=6cm,width=\textwidth]{img/gmic_stdlib636.jpg}\\
{\footnotesize \textbf{Example 636~:} \texttt{image.jpg --water {\comma}}}
\end{center}

\subsection{\emph{wave\index{wave}} }\vspace*{-0.7em}
~\\\textbf{\Cb{Arguments: }}\begin{flushleft}
{\small \Cb{\hspace*{0.5cm}$\bullet$~~\texttt{\_amplitude$>$=0{\comma}\_frequency$>$=0{\comma}\_center\_x{\comma}\_center\_y}}}\end{flushleft}
Apply wave deformation on selected images.
\begin{flushleft}\Cc{\textbf{Default values}:\\~\\\hspace*{0.5cm}{\small $\bullet$~~\texttt{'amplitude=4'{\comma} 'frequency=0.4'} and \texttt{'center\_x=center\_y=50'.}}}\end{flushleft}
\begin{center}\includegraphics[keepaspectratio=true,height=6cm,width=\textwidth]{img/gmic_stdlib637.jpg}\\
{\footnotesize \textbf{Example 637~:} \texttt{image.jpg --wave {\comma}}}
\end{center}

\subsection{\emph{wind\index{wind}} }\vspace*{-0.7em}
~\\\textbf{\Cb{Arguments: }}\begin{flushleft}
{\small \Cb{\hspace*{0.5cm}$\bullet$~~\texttt{\_amplitude$>$=0{\comma}\_angle{\comma}0$<$=\_attenuation$<$=1{\comma}\_threshold}}}\end{flushleft}
Apply wind effect on selected images.
\begin{flushleft}\Cc{\textbf{Default values}:\\~\\\hspace*{0.5cm}{\small $\bullet$~~\texttt{'amplitude=20'{\comma} 'angle=0'{\comma} 'attenuation=0.7'} and \texttt{'threshold=20'.}}}\end{flushleft}
\begin{center}\includegraphics[keepaspectratio=true,height=6cm,width=\textwidth]{img/gmic_stdlib638.jpg}\\
{\footnotesize \textbf{Example 638~:} \texttt{image.jpg --wind {\comma}}}
\end{center}

\subsection{\emph{zoom\index{zoom}} }\vspace*{-0.7em}
~\\\textbf{\Cb{Arguments: }}\begin{flushleft}
{\small \Cb{\hspace*{0.5cm}$\bullet$~~\texttt{\_factor{\comma}\_cx{\comma}\_cy{\comma}\_cz{\comma}\_boundary\_conditions=\{ 0=dirichlet ~$|$~ 1=n\-eumann ~$|$~ 2=periodic ~$|$~ 3=mirror \}}}}\end{flushleft}
Apply zoom factor to selected images.
\begin{flushleft}\Cc{\textbf{Default values}:\\~\\\hspace*{0.5cm}{\small $\bullet$~~\texttt{'factor=1'{\comma} 'cx=cy=cz=0.5'} and \texttt{'boundary\_conditions=0'.}}}\end{flushleft}
\begin{center}\includegraphics[keepaspectratio=true,height=6cm,width=\textwidth]{img/gmic_stdlib639.jpg}\\
{\footnotesize \textbf{Example 639~:} \texttt{image.jpg --zoom[0] 0.6 --zoom[0] 1.5}}
\end{center}
\section{Degradations}


\subsection{\emph{cracks\index{cracks}} }\vspace*{-0.7em}
~\\\textbf{\Cb{Arguments: }}\begin{flushleft}
{\small \Cb{\hspace*{0.5cm}$\bullet$~~\texttt{0$<$=\_density$<$=100{\comma}\_is\_relief=\{ 0 ~$|$~ 1 \}{\comma}\_opacity{\comma}\_color1{\comma}...}}}\end{flushleft}
Draw random cracks on selected images with specified color.
\begin{flushleft}\Cc{\textbf{Default values}:\\~\\\hspace*{0.5cm}{\small $\bullet$~~\texttt{'density=25'{\comma} 'is\_relief=0'{\comma} 'opacity=1'} and \texttt{'color1=0'.}}}\end{flushleft}
\begin{center}\includegraphics[keepaspectratio=true,height=6cm,width=\textwidth]{img/gmic_stdlib640.jpg}\\
{\footnotesize \textbf{Example 640~:} \texttt{image.jpg --cracks {\comma}}}
\end{center}

\subsection{\emph{light\_patch\index{light\_patch}} }\vspace*{-0.7em}
~\\\textbf{\Cb{Arguments: }}\begin{flushleft}
{\small \Cb{\hspace*{0.5cm}$\bullet$~~\texttt{\_density$>$0{\comma}\_darkness$>$=0{\comma}\_lightness$>$=0}}}\end{flushleft}
Add light patches to selected images.
\begin{flushleft}\Cc{\textbf{Default values}:\\~\\\hspace*{0.5cm}{\small $\bullet$~~\texttt{'density=10'{\comma} 'darkness=0.9'} and \texttt{'lightness=1.7'.}}}\end{flushleft}
\begin{center}\includegraphics[keepaspectratio=true,height=6cm,width=\textwidth]{img/gmic_stdlib641.jpg}\\
{\footnotesize \textbf{Example 641~:} \texttt{image.jpg --light\_patch 20{\comma}0.9{\comma}4}}
\end{center}

\subsection{\emph{noise\_hurl\index{noise\_hurl}} }\vspace*{-0.7em}
~\\\textbf{\Cb{Arguments: }}\begin{flushleft}
{\small \Cb{\hspace*{0.5cm}$\bullet$~~\texttt{\_amplitude$>$=0}}}\end{flushleft}
Add hurl noise to selected images.
\begin{flushleft}\Cc{\textbf{Default value}:\\~\\\hspace*{0.5cm}{\small $\bullet$~~\texttt{'amplitude=10'.}}}\end{flushleft}
\begin{center}\includegraphics[keepaspectratio=true,height=6cm,width=\textwidth]{img/gmic_stdlib642.jpg}\\
{\footnotesize \textbf{Example 642~:} \texttt{image.jpg --noise\_hurl {\comma}}}
\end{center}

\subsection{\emph{pixelize\index{pixelize}} }\vspace*{-0.7em}
~\\\textbf{\Cb{Arguments: }}\begin{flushleft}
{\small \Cb{\hspace*{0.5cm}$\bullet$~~\texttt{\_scale\_x$>$0{\comma}\_scale\_y$>$0{\comma}\_scale\_z$>$0}}}\end{flushleft}
Pixelize selected images with specified scales.
\begin{flushleft}\Cc{\textbf{Default values}:\\~\\\hspace*{0.5cm}{\small $\bullet$~~\texttt{'scale\_x=20'} and \texttt{'scale\_y=scale\_z=scale\_x'.}}}\end{flushleft}
\begin{center}\includegraphics[keepaspectratio=true,height=6cm,width=\textwidth]{img/gmic_stdlib643.jpg}\\
{\footnotesize \textbf{Example 643~:} \texttt{image.jpg --pixelize {\comma}}}
\end{center}

\subsection{\emph{scanlines\index{scanlines}} }\vspace*{-0.7em}
~\\\textbf{\Cb{Arguments: }}\begin{flushleft}
{\small \Cb{\hspace*{0.5cm}$\bullet$~~\texttt{\_amplitude{\comma}\_bandwidth{\comma}\_shape=\{ 0=bloc ~$|$~ 1=triangle ~$|$~ 2=sine \-~$|$~ 3=sine+ ~$|$~ 4=random \}{\comma}\_angle{\comma}\_offset}}}\end{flushleft}
Apply ripple deformation on selected images.
\begin{flushleft}\Cc{\textbf{Default values}:\\~\\\hspace*{0.5cm}{\small $\bullet$~~\texttt{'amplitude=60'{\comma} 'bandwidth=2'{\comma} 'shape=0'{\comma} 'angle=0'} and \texttt{'offset=0'.}}}\end{flushleft}
\begin{center}\includegraphics[keepaspectratio=true,height=6cm,width=\textwidth]{img/gmic_stdlib644.jpg}\\
{\footnotesize \textbf{Example 644~:} \texttt{image.jpg --ripple {\comma}}}
\end{center}

\subsection{\emph{shade\_stripes\index{shade\_stripes}} }\vspace*{-0.7em}
~\\\textbf{\Cb{Arguments: }}\begin{flushleft}
{\small \Cb{\hspace*{0.5cm}$\bullet$~~\texttt{\_frequency$>$=0{\comma}\_direction=\{ 0=horizontal ~$|$~ 1=vertical \}{\comma}\_dark\-ness$>$=0{\comma}\_lightness$>$=0}}}\end{flushleft}
Add shade stripes to selected images.
\begin{flushleft}\Cc{\textbf{Default values}:\\~\\\hspace*{0.5cm}{\small $\bullet$~~\texttt{'frequency=5'{\comma} 'direction=1'{\comma} 'darkness=0.8'} and \texttt{'lightness=2'.}}}\end{flushleft}
\begin{center}\includegraphics[keepaspectratio=true,height=6cm,width=\textwidth]{img/gmic_stdlib645.jpg}\\
{\footnotesize \textbf{Example 645~:} \texttt{image.jpg --shade\_stripes 30}}
\end{center}

\subsection{\emph{shadow\_patch\index{shadow\_patch}} }\vspace*{-0.7em}
~\\\textbf{\Cb{Arguments: }}\begin{flushleft}
{\small \Cb{\hspace*{0.5cm}$\bullet$~~\texttt{\_opacity$>$=0}}}\end{flushleft}
Add shadow patches to selected images.
\begin{flushleft}\Cc{\textbf{Default value}:\\~\\\hspace*{0.5cm}{\small $\bullet$~~\texttt{'opacity=0.7'.}}}\end{flushleft}
\begin{center}\includegraphics[keepaspectratio=true,height=6cm,width=\textwidth]{img/gmic_stdlib646.jpg}\\
{\footnotesize \textbf{Example 646~:} \texttt{image.jpg --shadow\_patch 0.4}}
\end{center}

\subsection{\emph{spread\index{spread}} }\vspace*{-0.7em}
~\\\textbf{\Cb{Arguments: }}\begin{flushleft}
{\small \Cb{\hspace*{0.5cm}$\bullet$~~\texttt{\_dx$>$=0{\comma}\_dy$>$=0{\comma}\_dz$>$=0}}}\end{flushleft}
Spread pixel values of selected images randomly along x{\comma}y and z.
\begin{flushleft}\Cc{\textbf{Default values}:\\~\\\hspace*{0.5cm}{\small $\bullet$~~\texttt{'dx=3'{\comma} 'dy=dx'} and \texttt{'dz=0'.}}}\end{flushleft}
\begin{center}\includegraphics[keepaspectratio=true,height=6cm,width=\textwidth]{img/gmic_stdlib647.jpg}\\
{\footnotesize \textbf{Example 647~:} \texttt{image.jpg --spread 3}}
\end{center}

\subsection{\emph{stripes\_y\index{stripes\_y}} }\vspace*{-0.7em}
~\\\textbf{\Cb{Arguments: }}\begin{flushleft}
{\small \Cb{\hspace*{0.5cm}$\bullet$~~\texttt{\_frequency$>$=0}}}\end{flushleft}
Add vertical stripes to selected images.
\begin{flushleft}\Cc{\textbf{Default value}:\\~\\\hspace*{0.5cm}{\small $\bullet$~~\texttt{'frequency=10'.}}}\end{flushleft}
\begin{center}\includegraphics[keepaspectratio=true,height=6cm,width=\textwidth]{img/gmic_stdlib648.jpg}\\
{\footnotesize \textbf{Example 648~:} \texttt{image.jpg --stripes\_y {\comma}}}
\end{center}

\subsection{\emph{texturize\_canvas\index{texturize\_canvas}} }\vspace*{-0.7em}
~\\\textbf{\Cb{Arguments: }}\begin{flushleft}
{\small \Cb{\hspace*{0.5cm}$\bullet$~~\texttt{\_amplitude$>$=0{\comma}\_fibrousness$>$=0{\comma}\_emboss\_level$>$=0}}}\end{flushleft}
Add paint canvas texture to selected images.
\begin{flushleft}\Cc{\textbf{Default values}:\\~\\\hspace*{0.5cm}{\small $\bullet$~~\texttt{'amplitude=20'{\comma} 'fibrousness=3'} and \texttt{'emboss\_level=0.6'.}}}\end{flushleft}
\begin{center}\includegraphics[keepaspectratio=true,height=6cm,width=\textwidth]{img/gmic_stdlib649.jpg}\\
{\footnotesize \textbf{Example 649~:} \texttt{image.jpg --texturize\_canvas {\comma}}}
\end{center}

\subsection{\emph{texturize\_paper\index{texturize\_paper}} }\vspace*{-0.7em}
Add paper texture to selected images.
\begin{center}\includegraphics[keepaspectratio=true,height=6cm,width=\textwidth]{img/gmic_stdlib650.jpg}\\
{\footnotesize \textbf{Example 650~:} \texttt{image.jpg --texturize\_paper}}
\end{center}

\subsection{\emph{vignette\index{vignette}} }\vspace*{-0.7em}
~\\\textbf{\Cb{Arguments: }}\begin{flushleft}
{\small \Cb{\hspace*{0.5cm}$\bullet$~~\texttt{\_strength$>$=0{\comma}0$<$=\_radius\_min$<$=100{\comma}0$<$=\_radius\_max$<$=100}}}\end{flushleft}
Add vignette effect to selected images.
\begin{flushleft}\Cc{\textbf{Default values}:\\~\\\hspace*{0.5cm}{\small $\bullet$~~\texttt{'strength=100'{\comma} 'radius\_min=70'} and \texttt{'radius\_max=90'.}}}\end{flushleft}
\begin{center}\includegraphics[keepaspectratio=true,height=6cm,width=\textwidth]{img/gmic_stdlib651.jpg}\\
{\footnotesize \textbf{Example 651~:} \texttt{image.jpg --vignette {\comma}}}
\end{center}

\subsection{\emph{watermark\_visible\index{watermark\_visible}} }\vspace*{-0.7em}
~\\\textbf{\Cb{Arguments: }}\begin{flushleft}
{\small \Cb{\hspace*{0.5cm}$\bullet$~~\texttt{\_text{\comma}0$<$\_opacity$<$1{\comma}\_size$>$0{\comma}\_angle{\comma}\_mode=\{ 0=remove ~$|$~ 1=add \}\-{\comma}\_smoothness$>$=0}}}\end{flushleft}
Add or remove a visible watermark on selected images (value range must be [0{\comma}255]).
\begin{flushleft}\Cc{\textbf{Default values}:\\~\\\hspace*{0.5cm}{\small $\bullet$~~\texttt{'text=(c) G'MIC'{\comma} 'opacity=0.3'{\comma} 'size=53'{\comma} 'angle=25'{\comma} 'mode=1'} and \texttt{'smoothness=0'.}}}\end{flushleft}
\begin{center}\includegraphics[keepaspectratio=true,height=6cm,width=\textwidth]{img/gmic_stdlib652.jpg}\\
{\footnotesize \textbf{Example 652~:} \texttt{image.jpg --watermark\_visible {\comma}0.7}}
\end{center}
\section{Blending and fading}


\subsection{\emph{blend\index{blend}} }\vspace*{-0.7em}
~\\\textbf{\Cb{Arguments: }}\begin{flushleft}
{\small \Cb{\hspace*{0.5cm}$\bullet$~~\texttt{[layer]{\comma}blending\_mode{\comma}0$<$=\_opacity$<$=1{\comma}\_selection\_is=\{ 0=base-\-layers ~$|$~ 1=top-layers \}}}}~~~\\
{\small \Cb{\hspace*{0.5cm}$\bullet$~~\texttt{blending\_mode{\comma}0$<$=\_opacity$<$=1}}}\end{flushleft}
Blend selected G{\comma}GA{\comma}RGB or RGBA images by specified layer or blend all selected images together{\comma} using specified blending mode.
~\\'blending\_mode' can be \{ add ~$|$~ alpha ~$|$~ and ~$|$~ average ~$|$~ blue ~$|$~ burn ~$|$~ darken ~$|$~ difference ~$|$~
divide ~$|$~ dodge ~$|$~ edges ~$|$~ exclusion ~$|$~ freeze ~$|$~ grainextract ~$|$~ grainmerge ~$|$~ green ~$|$~ hardlight ~$|$~
hardmix ~$|$~ hue ~$|$~ interpolation ~$|$~ lighten ~$|$~ lightness ~$|$~ linearburn ~$|$~ linearlight ~$|$~ luminance ~$|$~
multiply ~$|$~ negation ~$|$~ or ~$|$~ overlay ~$|$~ pinlight ~$|$~ red ~$|$~ reflect ~$|$~ saturation ~$|$~ seamless ~$|$~ seamless\_mixed ~$|$~ screen ~$|$~
shapeareamax ~$|$~ shapeareamax0 ~$|$~ shapeareamin ~$|$~ shapeareamin0 ~$|$~ shapeaverage ~$|$~ shapeaverage0 ~$|$~ shapemedian ~$|$~ shapemedian0 ~$|$~
shapemin ~$|$~ shapemin0 ~$|$~ shapemax ~$|$~ shapemax0 ~$|$~ softburn ~$|$~ softdodge ~$|$~ softlight ~$|$~ stamp ~$|$~ subtract ~$|$~ value ~$|$~ vividlight ~$|$~ xor \}.
\begin{flushleft}\Cc{\textbf{Default values}:\\~\\\hspace*{0.5cm}{\small $\bullet$~~\texttt{'blending\_mode=alpha'{\comma} 'opacity=1'} and \texttt{'selection\_is=0'.}}}\end{flushleft}
\begin{center}\includegraphics[keepaspectratio=true,height=6cm,width=\textwidth]{img/gmic_stdlib653.jpg}\\
{\footnotesize \textbf{Example 653~:} \texttt{image.jpg --drop\_shadow {\comma} resize2dy[-1] 200 rotate[-1] 20 --blend alpha display\_rgba[-2]}}
\\\includegraphics[keepaspectratio=true,height=6cm,width=\textwidth]{img/gmic_stdlib654.jpg}\\
{\footnotesize \textbf{Example 654~:} \texttt{image.jpg testimage2d \{w\}{\comma}\{h\} blend overlay}}
\\\includegraphics[keepaspectratio=true,height=6cm,width=\textwidth]{img/gmic_stdlib655.jpg}\\
{\footnotesize \textbf{Example 655~:} \texttt{command "ex : \$""=arg repeat \$""\% --blend[0{\comma}1] \$\{arg\{\$$>$+1\}\} text\_outline[-1] Mode:\textbackslash " \textbackslash "\$\{arg\{\$$>$+1\}\}{\comma}2{\comma}2{\comma}23{\comma}2{\comma}1{\comma}255 done" image.jpg testimage2d \{w\}{\comma}\{h\} ex add{\comma}alpha{\comma}and{\comma}average{\comma}blue{\comma}burn{\comma}darken}}
\\\includegraphics[keepaspectratio=true,height=6cm,width=\textwidth]{img/gmic_stdlib656.jpg}\\
{\footnotesize \textbf{Example 656~:} \texttt{command "ex : \$""=arg repeat \$""\% --blend[0{\comma}1] \$\{arg\{\$$>$+1\}\} text\_outline[-1] Mode:\textbackslash " \textbackslash "\$\{arg\{\$$>$+1\}\}{\comma}2{\comma}2{\comma}23{\comma}2{\comma}1{\comma}255 done" image.jpg testimage2d \{w\}{\comma}\{h\} ex difference{\comma}divide{\comma}dodge{\comma}exclusion{\comma}freeze{\comma}grainextract{\comma}grainmerge}}
\\\includegraphics[keepaspectratio=true,height=6cm,width=\textwidth]{img/gmic_stdlib657.jpg}\\
{\footnotesize \textbf{Example 657~:} \texttt{command "ex : \$""=arg repeat \$""\% --blend[0{\comma}1] \$\{arg\{\$$>$+1\}\} text\_outline[-1] Mode:\textbackslash " \textbackslash "\$\{arg\{\$$>$+1\}\}{\comma}2{\comma}2{\comma}23{\comma}2{\comma}1{\comma}255 done" image.jpg testimage2d \{w\}{\comma}\{h\} ex green{\comma}hardlight{\comma}hardmix{\comma}hue{\comma}interpolation{\comma}lighten{\comma}lightness}}
\\\includegraphics[keepaspectratio=true,height=6cm,width=\textwidth]{img/gmic_stdlib658.jpg}\\
{\footnotesize \textbf{Example 658~:} \texttt{command "ex : \$""=arg repeat \$""\% --blend[0{\comma}1] \$\{arg\{\$$>$+1\}\} text\_outline[-1] Mode:\textbackslash " \textbackslash "\$\{arg\{\$$>$+1\}\}{\comma}2{\comma}2{\comma}23{\comma}2{\comma}1{\comma}255 done" image.jpg testimage2d \{w\}{\comma}\{h\} ex linearburn{\comma}linearlight{\comma}luminance{\comma}multiply{\comma}negation{\comma}or{\comma}overlay}}
\\\includegraphics[keepaspectratio=true,height=6cm,width=\textwidth]{img/gmic_stdlib659.jpg}\\
{\footnotesize \textbf{Example 659~:} \texttt{command "ex : \$""=arg repeat \$""\% --blend[0{\comma}1] \$\{arg\{\$$>$+1\}\} text\_outline[-1] Mode:\textbackslash " \textbackslash "\$\{arg\{\$$>$+1\}\}{\comma}2{\comma}2{\comma}23{\comma}2{\comma}1{\comma}255 done" image.jpg testimage2d \{w\}{\comma}\{h\} ex pinlight{\comma}red{\comma}reflect{\comma}saturation{\comma}screen{\comma}shapeaverage{\comma}softburn}}
\\\includegraphics[keepaspectratio=true,height=6cm,width=\textwidth]{img/gmic_stdlib660.jpg}\\
{\footnotesize \textbf{Example 660~:} \texttt{command "ex : \$""=arg repeat \$""\% --blend[0{\comma}1] \$\{arg\{\$$>$+1\}\} text\_outline[-1] Mode:\textbackslash " \textbackslash "\$\{arg\{\$$>$+1\}\}{\comma}2{\comma}2{\comma}23{\comma}2{\comma}1{\comma}255 done" image.jpg testimage2d \{w\}{\comma}\{h\} ex softdodge{\comma}softlight{\comma}stamp{\comma}subtract{\comma}value{\comma}vividlight{\comma}xor}}
\end{center}

\subsection{\emph{blend\_edges\index{blend\_edges}} }\vspace*{-0.7em}
~\\\textbf{\Cb{Arguments: }}\begin{flushleft}
{\small \Cb{\hspace*{0.5cm}$\bullet$~~\texttt{smoothness[\%]$>$=0}}}\end{flushleft}
Blend selected images togethers using 'edges' mode.
\begin{center}\includegraphics[keepaspectratio=true,height=6cm,width=\textwidth]{img/gmic_stdlib661.jpg}\\
{\footnotesize \textbf{Example 661~:} \texttt{image.jpg testimage2d \{w\}{\comma}\{h\} --blend\_edges 0.8}}
\end{center}

\subsection{\emph{blend\_fade\index{blend\_fade}} }\vspace*{-0.7em}
~\\\textbf{\Cb{Arguments: }}\begin{flushleft}
{\small \Cb{\hspace*{0.5cm}$\bullet$~~\texttt{[fading\_shape]}}}\end{flushleft}
Blend selected images together using specified fading shape.
\begin{center}\includegraphics[keepaspectratio=true,height=6cm,width=\textwidth]{img/gmic_stdlib662.jpg}\\
{\footnotesize \textbf{Example 662~:} \texttt{image.jpg testimage2d \{w\}{\comma}\{h\} 100\%{\comma}100\%{\comma}1{\comma}1{\comma}'cos(y/10)' normalize[-1] 0{\comma}1 --blend\_fade[0{\comma}1] [2]}}
\end{center}

\subsection{\emph{blend\_median\index{blend\_median}} }\vspace*{-0.7em}
Blend selected images together using 'median' mode.
\begin{center}\includegraphics[keepaspectratio=true,height=6cm,width=\textwidth]{img/gmic_stdlib663.jpg}\\
{\footnotesize \textbf{Example 663~:} \texttt{image.jpg testimage2d \{w\}{\comma}\{h\} --mirror[0] y --blend\_median}}
\end{center}

\subsection{\emph{blend\_seamless\index{blend\_seamless}} }\vspace*{-0.7em}
~\\\textbf{\Cb{Arguments: }}\begin{flushleft}
{\small \Cb{\hspace*{0.5cm}$\bullet$~~\texttt{\_is\_mixed\_mode=\{ 0 ~$|$~ 1 \}{\comma}\_inner\_fading[\%]$>$=0{\comma}\_outer\_fading[\%\-]$>$=0}}}\end{flushleft}
Blend selected images using a seamless blending mode (Poisson-based).
\begin{flushleft}\Cc{\textbf{Default values}:\\~\\\hspace*{0.5cm}{\small $\bullet$~~\texttt{'is\_mixed=0'{\comma} 'inner\_fading=0'} and \texttt{'outer\_fading=100\%'.}}}\end{flushleft}


\subsection{\emph{fade\_diamond\index{fade\_diamond}} }\vspace*{-0.7em}
~\\\textbf{\Cb{Arguments: }}\begin{flushleft}
{\small \Cb{\hspace*{0.5cm}$\bullet$~~\texttt{0$<$=\_start$<$=100{\comma}0$<$=\_end$<$=100}}}\end{flushleft}
Create diamond fading from selected images.
\begin{flushleft}\Cc{\textbf{Default values}:\\~\\\hspace*{0.5cm}{\small $\bullet$~~\texttt{'start=80'} and \texttt{'end=90'.}}}\end{flushleft}
\begin{center}\includegraphics[keepaspectratio=true,height=6cm,width=\textwidth]{img/gmic_stdlib664.jpg}\\
{\footnotesize \textbf{Example 664~:} \texttt{image.jpg testimage2d \{w\}{\comma}\{h\} --fade\_diamond 80{\comma}85}}
\end{center}

\subsection{\emph{fade\_linear\index{fade\_linear}} }\vspace*{-0.7em}
~\\\textbf{\Cb{Arguments: }}\begin{flushleft}
{\small \Cb{\hspace*{0.5cm}$\bullet$~~\texttt{\_angle{\comma}0$<$=\_start$<$=100{\comma}0$<$=\_end$<$=100}}}\end{flushleft}
Create linear fading from selected images.
\begin{flushleft}\Cc{\textbf{Default values}:\\~\\\hspace*{0.5cm}{\small $\bullet$~~\texttt{'angle=45'{\comma} 'start=30'} and \texttt{'end=70'.}}}\end{flushleft}
\begin{center}\includegraphics[keepaspectratio=true,height=6cm,width=\textwidth]{img/gmic_stdlib665.jpg}\\
{\footnotesize \textbf{Example 665~:} \texttt{image.jpg testimage2d \{w\}{\comma}\{h\} --fade\_linear 45{\comma}48{\comma}52}}
\end{center}

\subsection{\emph{fade\_radial\index{fade\_radial}} }\vspace*{-0.7em}
~\\\textbf{\Cb{Arguments: }}\begin{flushleft}
{\small \Cb{\hspace*{0.5cm}$\bullet$~~\texttt{0$<$=\_start$<$=100{\comma}0$<$=\_end$<$=100}}}\end{flushleft}
Create radial fading from selected images.
\begin{flushleft}\Cc{\textbf{Default values}:\\~\\\hspace*{0.5cm}{\small $\bullet$~~\texttt{'start=30'} and \texttt{'end=70'.}}}\end{flushleft}
\begin{center}\includegraphics[keepaspectratio=true,height=6cm,width=\textwidth]{img/gmic_stdlib666.jpg}\\
{\footnotesize \textbf{Example 666~:} \texttt{image.jpg testimage2d \{w\}{\comma}\{h\} --fade\_radial 30{\comma}70}}
\end{center}

\subsection{\emph{fade\_x\index{fade\_x}} }\vspace*{-0.7em}
~\\\textbf{\Cb{Arguments: }}\begin{flushleft}
{\small \Cb{\hspace*{0.5cm}$\bullet$~~\texttt{0$<$=\_start$<$=100{\comma}0$<$=\_end$<$=100}}}\end{flushleft}
Create horizontal fading from selected images.
\begin{flushleft}\Cc{\textbf{Default values}:\\~\\\hspace*{0.5cm}{\small $\bullet$~~\texttt{'start=30'} and \texttt{'end=70'.}}}\end{flushleft}
\begin{center}\includegraphics[keepaspectratio=true,height=6cm,width=\textwidth]{img/gmic_stdlib667.jpg}\\
{\footnotesize \textbf{Example 667~:} \texttt{image.jpg testimage2d \{w\}{\comma}\{h\} --fade\_x 30{\comma}70}}
\end{center}

\subsection{\emph{fade\_y\index{fade\_y}} }\vspace*{-0.7em}
~\\\textbf{\Cb{Arguments: }}\begin{flushleft}
{\small \Cb{\hspace*{0.5cm}$\bullet$~~\texttt{0$<$=\_start$<$=100{\comma}0$<$=\_end$<$=100}}}\end{flushleft}
Create vertical fading from selected images.
\begin{flushleft}\Cc{\textbf{Default values}:\\~\\\hspace*{0.5cm}{\small $\bullet$~~\texttt{'start=30'} and \texttt{'end=70'.}}}\end{flushleft}
\begin{center}\includegraphics[keepaspectratio=true,height=6cm,width=\textwidth]{img/gmic_stdlib668.jpg}\\
{\footnotesize \textbf{Example 668~:} \texttt{image.jpg testimage2d \{w\}{\comma}\{h\} --fade\_y 30{\comma}70}}
\end{center}

\subsection{\emph{fade\_z\index{fade\_z}} }\vspace*{-0.7em}
~\\\textbf{\Cb{Arguments: }}\begin{flushleft}
{\small \Cb{\hspace*{0.5cm}$\bullet$~~\texttt{0$<$=\_start$<$=100{\comma}0$<$=\_end$<$=100}}}\end{flushleft}
Create transversal fading from selected images.
\begin{flushleft}\Cc{\textbf{Default values}:\\~\\\hspace*{0.5cm}{\small $\bullet$~~\texttt{'start=30'} and \texttt{'end=70'.}}}\end{flushleft}


\subsection{\emph{sub\_alpha\index{sub\_alpha}} }\vspace*{-0.7em}
~\\\textbf{\Cb{Arguments: }}\begin{flushleft}
{\small \Cb{\hspace*{0.5cm}$\bullet$~~\texttt{[base\_image]{\comma}\_opacity\_gain$>$=1}}}\end{flushleft}
Compute the minimal alpha-channel difference (opposite of alpha blending) between the selected images and the specified base image.
~\\The alpha difference A-B is defined as the image having minimal opacity{\comma} such that alpha\_blend(B{\comma}A-B) = A.
\begin{flushleft}\Cc{\textbf{Default value}:\\~\\\hspace*{0.5cm}{\small $\bullet$~~\texttt{'opacity\_gain=1'.}}}\end{flushleft}
\begin{center}\includegraphics[keepaspectratio=true,height=6cm,width=\textwidth]{img/gmic_stdlib669.jpg}\\
{\footnotesize \textbf{Example 669~:} \texttt{image.jpg testimage2d \{w\}{\comma}\{h\} --sub\_alpha[0] [1] display\_rgba}}
\end{center}
\section{Image sequences and videos}


\subsection{\emph{animate\index{animate}} }\vspace*{-0.7em}
~\\\textbf{\Cb{Arguments: }}\begin{flushleft}
{\small \Cb{\hspace*{0.5cm}$\bullet$~~\texttt{filter\_name{\comma}"param1\_start{\comma}...{\comma}paramN\_start"{\comma}"param1\_end{\comma}...{\comma}\-paramN\_end"{\comma}nb\_frames$>$=0{\comma}\_output\_frames=\{ 0 ~$|$~ 1 \}{\comma}\_output\_fi\-lename}}}~~~\\
{\small \Cb{\hspace*{0.5cm}$\bullet$~~\texttt{delay$>$0}}}\end{flushleft}
Animate filter from starting parameters to ending parameters or animate selected images
in a display window.
\begin{flushleft}\Cc{\textbf{Default value}:\\~\\\hspace*{0.5cm}{\small $\bullet$~~\texttt{'delay=30'.}}}\end{flushleft}
\begin{center}\includegraphics[keepaspectratio=true,height=6cm,width=\textwidth]{img/gmic_stdlib670.jpg}\\
{\footnotesize \textbf{Example 670~:} \texttt{image.jpg animate flower{\comma}"0{\comma}3"{\comma}"20{\comma}8"{\comma}9}}
\end{center}

\subsection{\emph{apply\_camera\index{apply\_camera}} }\vspace*{-0.7em}
~\\\textbf{\Cb{Arguments: }}\begin{flushleft}
{\small \Cb{\hspace*{0.5cm}$\bullet$~~\texttt{\_"command"{\comma}\_camera\_index$>$=0{\comma}\_skip\_frames$>$=0{\comma}\_output\_filename}}}\end{flushleft}
Apply specified command on live camera stream{\comma} and display it on display window [0].
\begin{flushleft}\Cc{\textbf{Default values}:\\~\\\hspace*{0.5cm}{\small $\bullet$~~\texttt{'command=""'{\comma} 'camera\_index=0' (default camera){\comma} 'skip\_frames=0'} and \texttt{'output\_filename=""'.}}}\end{flushleft}


\subsection{\emph{apply\_files\index{apply\_files}} }\vspace*{-0.7em}
~\\\textbf{\Cb{Arguments: }}\begin{flushleft}
{\small \Cb{\hspace*{0.5cm}$\bullet$~~\texttt{"filename\_pattern"{\comma}\_"command"{\comma}\_first\_frame$>$=0{\comma}\_last\_frame=\{ \-$>$=0 ~$|$~ -1=last \}{\comma}\_frame\_step$>$=1{\comma}\_output\_filename}}}\end{flushleft}
Apply a G'MIC command on specified input image files{\comma} in a streamed way.
~\\If a display window is opened{\comma} rendered frames are displayed in it during processing.
~\\The output filename may have extension '.avi' (saved as a video){\comma} or any other usual image file extension (saved as a sequence of images).
\begin{flushleft}\Cc{\textbf{Default values}:\\~\\\hspace*{0.5cm}{\small $\bullet$~~\texttt{'command=(undefined)'{\comma} 'first\_frame=0'{\comma} 'last\_frame=-1'{\comma} 'frame\_step=1'} and \texttt{'output\_filename=(undefined)'.}}}\end{flushleft}


\subsection{\emph{apply\_video\index{apply\_video}} }\vspace*{-0.7em}
~\\\textbf{\Cb{Arguments: }}\begin{flushleft}
{\small \Cb{\hspace*{0.5cm}$\bullet$~~\texttt{video\_filename{\comma}\_"command"{\comma}\_first\_frame$>$=0{\comma}\_last\_frame=\{ $>$=0 \-~$|$~ -1=last \}{\comma}\_frame\_step$>$=1{\comma}\_output\_filename}}}\end{flushleft}
Apply a G'MIC command on all frames of the specified input video file{\comma} in a streamed way.
~\\If a display window is opened{\comma} rendered frames are displayed in it during processing.
~\\The output filename may have extension '.avi' (saved as a video){\comma} or any other usual image file extension (saved as a sequence of images).
\begin{flushleft}\Cc{\textbf{Default values}:\\~\\\hspace*{0.5cm}{\small $\bullet$~~\texttt{'first\_frame=0'{\comma} 'last\_frame=-1'{\comma} 'frame\_step=1'} and \texttt{'output\_filename=(undefined)'.}}}\end{flushleft}


\subsection{\emph{average\_files\index{average\_files}} }\vspace*{-0.7em}
~\\\textbf{\Cb{Arguments: }}\begin{flushleft}
{\small \Cb{\hspace*{0.5cm}$\bullet$~~\texttt{"filename\_pattern"{\comma}\_first\_frame$>$=0{\comma}\_last\_frame=\{ $>$=0 ~$|$~ -1=la\-st \}{\comma}\_frame\_step$>$=1{\comma}\_output\_filename}}}\end{flushleft}
Average specified input image files{\comma} in a streamed way.
~\\If a display window is opened{\comma} rendered frames are displayed in it during processing.
~\\The output filename may have extension '.avi' (saved as a video){\comma} or any other usual image file extension (saved as a sequence of images).
\begin{flushleft}\Cc{\textbf{Default values}:\\~\\\hspace*{0.5cm}{\small $\bullet$~~\texttt{'first\_frame=0'{\comma} 'last\_frame=-1'{\comma} 'frame\_step=1'} and \texttt{'output\_filename=(undefined)'.}}}\end{flushleft}


\subsection{\emph{average\_video\index{average\_video}} }\vspace*{-0.7em}
~\\\textbf{\Cb{Arguments: }}\begin{flushleft}
{\small \Cb{\hspace*{0.5cm}$\bullet$~~\texttt{video\_filename{\comma}\_first\_frame$>$=0{\comma}\_last\_frame=\{ $>$=0 ~$|$~ -1=last \}\-{\comma}\_frame\_step$>$=1{\comma}\_output\_filename}}}\end{flushleft}
Average frames of specified input video file{\comma} in a streamed way.
~\\If a display window is opened{\comma} rendered frames are displayed in it during processing.
~\\The output filename may have extension '.avi' (saved as a video){\comma} or any other usual image file extension (saved as a sequence of images).
\begin{flushleft}\Cc{\textbf{Default values}:\\~\\\hspace*{0.5cm}{\small $\bullet$~~\texttt{'first\_frame=0'{\comma} 'last\_frame=-1'{\comma} 'frame\_step=1'} and \texttt{'output\_filename=(undefined)'.}}}\end{flushleft}


\subsection{\emph{fade\_files\index{fade\_files}} }\vspace*{-0.7em}
~\\\textbf{\Cb{Arguments: }}\begin{flushleft}
{\small \Cb{\hspace*{0.5cm}$\bullet$~~\texttt{"filename\_pattern"{\comma}\_nb\_inner\_frames$>$0{\comma}\_first\_frame$>$=0{\comma}\_last\_\-frame=\{ $>$=0 ~$|$~ -1=last \}{\comma}\_frame\_step$>$=1{\comma}\_output\_filename}}}\end{flushleft}
Generate a temporal fading from specified input image files{\comma} in a streamed way.
~\\If a display window is opened{\comma} rendered frames are displayed in it during processing.
~\\The output filename may have extension 'avi' (saved as a video){\comma} or any other usual image file extension (saved as a sequence of images).
\begin{flushleft}\Cc{\textbf{Default values}:\\~\\\hspace*{0.5cm}{\small $\bullet$~~\texttt{'nb\_inner\_frames=10'{\comma} 'first\_frame=0'{\comma} 'last\_frame=-1'{\comma} 'frame\_step=1'} and \texttt{'output\_filename=(undefined)'.}}}\end{flushleft}


\subsection{\emph{fade\_video\index{fade\_video}} }\vspace*{-0.7em}
~\\\textbf{\Cb{Arguments: }}\begin{flushleft}
{\small \Cb{\hspace*{0.5cm}$\bullet$~~\texttt{video\_filename{\comma}\_nb\_inner\_frames$>$0{\comma}\_first\_frame$>$=0{\comma}\_last\_fram\-e=\{ $>$=0 ~$|$~ -1=last \}{\comma}\_frame\_step$>$=1{\comma}\_output\_filename}}}\end{flushleft}
Create a temporal fading sequence from specified input video file{\comma} in a streamed way.
~\\If a display window is opened{\comma} rendered frames are displayed in it during processing.
\begin{flushleft}\Cc{\textbf{Default values}:\\~\\\hspace*{0.5cm}{\small $\bullet$~~\texttt{'nb\_inner\_frames=10'{\comma} 'first\_frame=0'{\comma} 'last\_frame=-1'{\comma} 'frame\_step=1'} and \texttt{'output\_filename=(undefined)'.}}}\end{flushleft}


\subsection{\emph{files2video\index{files2video}} }\vspace*{-0.7em}
~\\\textbf{\Cb{Arguments: }}\begin{flushleft}
{\small \Cb{\hspace*{0.5cm}$\bullet$~~\texttt{"filename\_pattern"{\comma}\_output\_filename{\comma}\_fps$>$0{\comma}\_codec}}}\end{flushleft}
Convert several files into a single video file.
\begin{flushleft}\Cc{\textbf{Default values}:\\~\\\hspace*{0.5cm}{\small $\bullet$~~\texttt{'output\_filename=output.avi'{\comma} 'fps=25'} and \texttt{'codec=mp4v'.}}}\end{flushleft}


\subsection{\emph{median\_files\index{median\_files}} }\vspace*{-0.7em}
~\\\textbf{\Cb{Arguments: }}\begin{flushleft}
{\small \Cb{\hspace*{0.5cm}$\bullet$~~\texttt{"filename\_pattern"{\comma}\_first\_frame$>$=0{\comma}\_last\_frame=\{ $>$=0 ~$|$~ -1=la\-st \}{\comma}\_frame\_step$>$=1{\comma}\_frame\_rows[\%]$>$=1{\comma}\_is\_fast\_approximation\-=\{ 0 ~$|$~ 1 \}}}}\end{flushleft}
Compute the median frame of specified input image files{\comma} in a streamed way.
~\\If a display window is opened{\comma} rendered frame is displayed in it during processing.
\begin{flushleft}\Cc{\textbf{Default values}:\\~\\\hspace*{0.5cm}{\small $\bullet$~~\texttt{'first\_frame=0'{\comma} 'last\_frame=-1'{\comma} 'frame\_step=1'{\comma} 'frame\_rows=20\%'} and \texttt{'is\_fast\_approximation=0'.}}}\end{flushleft}


\subsection{\emph{median\_video\index{median\_video}} }\vspace*{-0.7em}
~\\\textbf{\Cb{Arguments: }}\begin{flushleft}
{\small \Cb{\hspace*{0.5cm}$\bullet$~~\texttt{video\_filename{\comma}\_first\_frame$>$=0{\comma}\_last\_frame=\{ $>$=0 ~$|$~ -1=last \}\-{\comma}\_frame\_step$>$=1{\comma}\_frame\_rows[\%]$>$=1{\comma}\_is\_fast\_approximation=\{ 0\- ~$|$~ 1 \}}}}\end{flushleft}
Compute the median of all frames of an input video file{\comma} in a streamed way.
~\\If a display window is opened{\comma} rendered frame is displayed in it during processing.
\begin{flushleft}\Cc{\textbf{Default values}:\\~\\\hspace*{0.5cm}{\small $\bullet$~~\texttt{'first\_frame=0'{\comma} 'last\_frame=-1'{\comma} 'frame\_step=1'{\comma} 'frame\_rows=100\%'} and \texttt{'is\_fast\_approximation=1'.}}}\end{flushleft}


\subsection{\emph{morph\index{morph}} }\vspace*{-0.7em}
~\\\textbf{\Cb{Arguments: }}\begin{flushleft}
{\small \Cb{\hspace*{0.5cm}$\bullet$~~\texttt{nb\_inner\_frames$>$=1{\comma}\_smoothness$>$=0{\comma}\_precision$>$=0}}}\end{flushleft}
Create morphing sequence between selected images.
\begin{flushleft}\Cc{\textbf{Default values}:\\~\\\hspace*{0.5cm}{\small $\bullet$~~\texttt{'smoothness=0.1'} and \texttt{'precision=4'.}}}\end{flushleft}
\begin{center}\includegraphics[keepaspectratio=true,height=6cm,width=\textwidth]{img/gmic_stdlib671.jpg}\\
{\footnotesize \textbf{Example 671~:} \texttt{image.jpg --rotate 20{\comma}1{\comma}1{\comma}50\%{\comma}50\% morph 9}}
\end{center}

\subsection{\emph{morph\_files\index{morph\_files}} }\vspace*{-0.7em}
~\\\textbf{\Cb{Arguments: }}\begin{flushleft}
{\small \Cb{\hspace*{0.5cm}$\bullet$~~\texttt{"filename\_pattern"{\comma}\_nb\_inner\_frames$>$0{\comma}\_smoothness$>$=0{\comma}\_precis\-ion$>$=0{\comma}\_first\_frame$>$=0{\comma}\_last\_frame=\{ $>$=0 ~$|$~ -1=last \}{\comma}\_frame\_\-step$>$=1{\comma}\_output\_filename}}}\end{flushleft}
Generate a temporal morphing from specified input image files{\comma} in a streamed way.
~\\If a display window is opened{\comma} rendered frames are displayed in it during processing.
~\\The output filename may have extension '.avi' (saved as a video){\comma} or any other usual image file extension (saved as a sequence of images).
\begin{flushleft}\Cc{\textbf{Default values}:\\~\\\hspace*{0.5cm}{\small $\bullet$~~\texttt{'nb\_inner\_frames=10'{\comma} 'smoothness=0.1'{\comma} 'precision=4'{\comma} 'first\_frame=0'{\comma} 'last\_frame=-1'{\comma} 'frame\_step=1'} and \texttt{'output\_filename=(undefined)'.}}}\end{flushleft}


\subsection{\emph{morph\_video\index{morph\_video}} }\vspace*{-0.7em}
~\\\textbf{\Cb{Arguments: }}\begin{flushleft}
{\small \Cb{\hspace*{0.5cm}$\bullet$~~\texttt{video\_filename{\comma}\_nb\_inner\_frames$>$0{\comma}\_smoothness$>$=0{\comma}\_precision$>$\-=0{\comma}\_first\_frame$>$=0{\comma}\_last\_frame=\{ $>$=0 ~$|$~ -1=last \}{\comma}\_frame\_step\-$>$=1{\comma}\_output\_filename}}}\end{flushleft}
Generate a temporal morphing from specified input video file{\comma} in a streamed way.
~\\If a display window is opened{\comma} rendered frames are displayed in it during processing.
~\\The output filename may have extension '.avi' (saved as a video){\comma} or any other usual image file extension (saved as a sequence of images).
\begin{flushleft}\Cc{\textbf{Default values}:\\~\\\hspace*{0.5cm}{\small $\bullet$~~\texttt{'nb\_inner\_frames=10'{\comma} 'smoothness=0.1'{\comma} 'precision=4'{\comma} 'first\_frame=0'{\comma} 'last\_frame=-1'{\comma} 'frame\_step=1'} and \texttt{'output\_filename=(undefined)'.}}}\end{flushleft}


\subsection{\emph{register\_nonrigid\index{register\_nonrigid}} }\vspace*{-0.7em}
~\\\textbf{\Cb{Arguments: }}\begin{flushleft}
{\small \Cb{\hspace*{0.5cm}$\bullet$~~\texttt{[destination]{\comma}\_smoothness$>$=0{\comma}\_precision$>$0{\comma}\_nb\_scale$>$=0}}}\end{flushleft}
Register selected source images with specified destination image{\comma} using non-rigid warp.
\begin{flushleft}\Cc{\textbf{Default values}:\\~\\\hspace*{0.5cm}{\small $\bullet$~~\texttt{'smoothness=0.2'{\comma} 'precision=6'} and \texttt{'nb\_scale=0(auto)'.}}}\end{flushleft}
\begin{center}\includegraphics[keepaspectratio=true,height=6cm,width=\textwidth]{img/gmic_stdlib672.jpg}\\
{\footnotesize \textbf{Example 672~:} \texttt{image.jpg --rotate 20{\comma}1{\comma}1{\comma}50\%{\comma}50\% --register\_nonrigid[0] [1]}}
\end{center}

\subsection{\emph{register\_rigid\index{register\_rigid}} }\vspace*{-0.7em}
~\\\textbf{\Cb{Arguments: }}\begin{flushleft}
{\small \Cb{\hspace*{0.5cm}$\bullet$~~\texttt{[destination]{\comma}\_smoothness$>$=0{\comma}\_boundary\_conditions=\{ 0=dirich\-let ~$|$~ 1=neumann ~$|$~ 2=periodic ~$|$~ 3=mirror \}}}}\end{flushleft}
Register selected source images with specified destination image{\comma} using rigid warp (shift).
\begin{flushleft}\Cc{\textbf{Default values}:\\~\\\hspace*{0.5cm}{\small $\bullet$~~\texttt{'smoothness=1'} and \texttt{'boundary\_conditions=0'.}}}\end{flushleft}
\begin{center}\includegraphics[keepaspectratio=true,height=6cm,width=\textwidth]{img/gmic_stdlib673.jpg}\\
{\footnotesize \textbf{Example 673~:} \texttt{image.jpg --shift 30{\comma}20 --register\_rigid[0] [1]}}
\end{center}

\subsection{\emph{transition\index{transition}} }\vspace*{-0.7em}
~\\\textbf{\Cb{Arguments: }}\begin{flushleft}
{\small \Cb{\hspace*{0.5cm}$\bullet$~~\texttt{[transition\_shape]{\comma}nb\_added\_frames$>$=0{\comma}100$>$=shading$>$=0{\comma}\_singl\-e\_frame\_only=\{ -1=disabled ~$|$~ $>$=0 \}}}}\end{flushleft}
Generate a transition sequence between selected images.
\begin{flushleft}\Cc{\textbf{Default values}:\\~\\\hspace*{0.5cm}{\small $\bullet$~~\texttt{'shading=0'} and \texttt{'single\_frame\_only=-1'.}}}\end{flushleft}
\begin{center}\includegraphics[keepaspectratio=true,height=6cm,width=\textwidth]{img/gmic_stdlib674.jpg}\\
{\footnotesize \textbf{Example 674~:} \texttt{image.jpg --mirror c 100\%{\comma}100\% plasma[-1] 1{\comma}1{\comma}6 transition[0{\comma}1] [2]{\comma}5}}
\end{center}

\subsection{\emph{transition3d\index{transition3d}} }\vspace*{-0.7em}
~\\\textbf{\Cb{Arguments: }}\begin{flushleft}
{\small \Cb{\hspace*{0.5cm}$\bullet$~~\texttt{\_nb\_frames$>$=2{\comma}\_nb\_xtiles$>$0{\comma}\_nb\_ytiles$>$0{\comma}\_axis\_x{\comma}\_axis\_y{\comma}\_axi\-s\_z{\comma}\_is\_antialias=\{ 0 ~$|$~ 1 \}}}}\end{flushleft}
Create 3d transition sequence between selected consecutive images.
~\\'axis\_x'{\comma} 'axis\_y' and 'axis\_z' can be set as mathematical expressions{\comma} depending on 'x' and 'y'.
\begin{flushleft}\Cc{\textbf{Default values}:\\~\\\hspace*{0.5cm}{\small $\bullet$~~\texttt{'nb\_frames=10'{\comma} 'nb\_xtiles=nb\_ytiles=3'{\comma} 'axis\_x=1'{\comma} 'axis\_y=1'{\comma} 'axis\_z=0'} and \texttt{'is\_antialias=1'.}}}\end{flushleft}
\begin{center}\includegraphics[keepaspectratio=true,height=6cm,width=\textwidth]{img/gmic_stdlib675.jpg}\\
{\footnotesize \textbf{Example 675~:} \texttt{image.jpg --blur 5 transition3d 9 display\_rgba}}
\end{center}

\subsection{\emph{video2files\index{video2files}} }\vspace*{-0.7em}
~\\\textbf{\Cb{Arguments: }}\begin{flushleft}
{\small \Cb{\hspace*{0.5cm}$\bullet$~~\texttt{input\_filename{\comma}\_output\_filename{\comma}\_first\_frame$>$=0{\comma}\_last\_frame=\-\{ $>$=0 ~$|$~ -1=last \}{\comma}\_frame\_step$>$=1}}}\end{flushleft}
Split specified input video file into image files{\comma} one for each frame.
~\\First and last frames as well as step between frames can be specified.
\begin{flushleft}\Cc{\textbf{Default values}:\\~\\\hspace*{0.5cm}{\small $\bullet$~~\texttt{'output\_filename=frame.png'{\comma} 'first\_frame=0'{\comma} 'last\_frame=-1'} and \texttt{'frame\_step=1'.}}}\end{flushleft}

\section{PINK-library operators}


\subsection{\emph{output\_pink3d\index{output\_pink3d}} }\vspace*{-0.7em}
~\\\textbf{\Cb{Arguments: }}\begin{flushleft}
{\small \Cb{\hspace*{0.5cm}$\bullet$~~\texttt{filename}}}\end{flushleft}
Save selected images as P5-coded PPM files (PINK extension for 3d volumetric images).


\subsection{\emph{pink\index{pink}} }\vspace*{-0.7em}
Pink wrapper name{\comma}p1{\comma}...{\comma}pn (requires the PINK library to be installed).
~\\(http://pinkhq.com/)
prepares input{\comma} calls external "name input p1...pn output" and reads output (/tmp)
\begin{center}\includegraphics[keepaspectratio=true,height=6cm,width=\textwidth]{img/gmic_stdlib676.jpg}\\
{\footnotesize \textbf{Example 676~:} \texttt{image.jpg --pink asfr{\comma}5 pink[0] asf{\comma}5}}
\\\includegraphics[keepaspectratio=true,height=6cm,width=\textwidth]{img/gmic_stdlib677.jpg}\\
{\footnotesize \textbf{Example 677~:} \texttt{image.jpg --blur 2 pink maxima{\comma}4}}
\end{center}

\subsection{\emph{pink\_grayskel\index{pink\_grayskel}} }\vspace*{-0.7em}
~\\\textbf{\Cb{Arguments: }}\begin{flushleft}
{\small \Cb{\hspace*{0.5cm}$\bullet$~~\texttt{\_connectivity=\{ 4 ~$|$~ 8 ~$|$~ 6 ~$|$~ 26 \}{\comma} \_lambda=0}}}\end{flushleft}
(http://pinkhq.com/doxygen/grayskel\_8c.html)
~\\Grayscale homotopic skeleton (requires the PINK library to be installed).
\begin{flushleft}\Cc{\textbf{Default values}:\\~\\\hspace*{0.5cm}{\small $\bullet$~~\texttt{'connectivity=4'} and \texttt{'lambda=0'.}}}\end{flushleft}
\begin{center}\includegraphics[keepaspectratio=true,height=6cm,width=\textwidth]{img/gmic_stdlib678.jpg}\\
{\footnotesize \textbf{Example 678~:} \texttt{image.jpg --pink\_grayskel {\comma} --pink\_grayskel[0] {\comma}10 --pink\_grayskel[0] {\comma}100 append\_tiles 2}}
\end{center}

\subsection{\emph{pink\_heightmaxima\index{pink\_heightmaxima}} }\vspace*{-0.7em}
~\\\textbf{\Cb{Arguments: }}\begin{flushleft}
{\small \Cb{\hspace*{0.5cm}$\bullet$~~\texttt{\_connectivity=\{ 4 ~$|$~ 8 ~$|$~ 6 ~$|$~ 26 \}{\comma}\_height=1}}}\end{flushleft}
(http://pinkhq.com/doxygen/heightmaxima\_8c.html)
~\\Heightmaxima filtering (requires the PINK library to be installed).
\begin{flushleft}\Cc{\textbf{Default values}:\\~\\\hspace*{0.5cm}{\small $\bullet$~~\texttt{'connectivity=4'} and \texttt{'height=1'.}}}\end{flushleft}
\begin{center}\includegraphics[keepaspectratio=true,height=6cm,width=\textwidth]{img/gmic_stdlib679.jpg}\\
{\footnotesize \textbf{Example 679~:} \texttt{image.jpg --blur 2 --pink\_heightminima {\comma}15 --pink\_heightmaxima[0{\comma}1] {\comma}15 -[-3{\comma}-1] -[-3{\comma}-1] keep[-1{\comma}-2]}}
\end{center}

\subsection{\emph{pink\_heightminima\index{pink\_heightminima}} }\vspace*{-0.7em}
~\\\textbf{\Cb{Arguments: }}\begin{flushleft}
{\small \Cb{\hspace*{0.5cm}$\bullet$~~\texttt{\_connectivity=\{ 4 ~$|$~ 8 ~$|$~ 6 ~$|$~ 26 \}{\comma}\_height=1}}}\end{flushleft}
(http://pinkhq.com/doxygen/heightminima\_8c.html)
~\\Heightminima filtering (requires the PINK library to be installed).
\begin{flushleft}\Cc{\textbf{Default values}:\\~\\\hspace*{0.5cm}{\small $\bullet$~~\texttt{'connectivity=4'} and \texttt{'height=1'.}}}\end{flushleft}
\begin{center}\includegraphics[keepaspectratio=true,height=6cm,width=\textwidth]{img/gmic_stdlib680.jpg}\\
{\footnotesize \textbf{Example 680~:} \texttt{image.jpg --blur 2 --pink\_heightminima {\comma}15 --pink\_heightmaxima[0{\comma}1] {\comma}15 -[-3{\comma}-1] -[-3{\comma}-1] keep[-1{\comma}-2]}}
\end{center}

\subsection{\emph{pink\_htkern\index{pink\_htkern}} }\vspace*{-0.7em}
~\\\textbf{\Cb{Arguments: }}\begin{flushleft}
{\small \Cb{\hspace*{0.5cm}$\bullet$~~\texttt{\_connectivity=\{ 4 ~$|$~ 8 ~$|$~ 6 ~$|$~ 26 \}{\comma} \_type=\{""~$|$~u\}}}}\end{flushleft}
(http://pinkhq.com/doxygen/htkern\_8c.html)
~\\(http://pinkhq.com/doxygen/htkernu\_8c.html)
~\\Grayscale ultimate homotopic thinning/thickening without condition (requires the PINK library to be installed).
\begin{flushleft}\Cc{\textbf{Default values}:\\~\\\hspace*{0.5cm}{\small $\bullet$~~\texttt{'connectivity=4'} and \texttt{'type=""'.}}}\end{flushleft}
\begin{center}\includegraphics[keepaspectratio=true,height=6cm,width=\textwidth]{img/gmic_stdlib681.jpg}\\
{\footnotesize \textbf{Example 681~:} \texttt{image.jpg --pink\_htkern {\comma}u --pink\_htkern[0] {\comma} ---[-1{\comma}-2] remove[0]}}
\end{center}

\subsection{\emph{pink\_lvkern\index{pink\_lvkern}} }\vspace*{-0.7em}
~\\\textbf{\Cb{Arguments: }}\begin{flushleft}
{\small \Cb{\hspace*{0.5cm}$\bullet$~~\texttt{\_connectivity=\{ 4 ~$|$~ 8 ~$|$~ 6 ~$|$~ 26 \}{\comma} \_type=\{""~$|$~u\}}}}\end{flushleft}
(http://pinkhq.com/doxygen/lvkern\_8c.html)
~\\(http://pinkhq.com/doxygen/lvkernu\_8c.html)
~\\Grayscale ultimate leveling thinning/thickening without condition (requires the PINK library to be installed).
\begin{flushleft}\Cc{\textbf{Default values}:\\~\\\hspace*{0.5cm}{\small $\bullet$~~\texttt{'connectivity=4'} and \texttt{'type=""'.}}}\end{flushleft}
\begin{center}\includegraphics[keepaspectratio=true,height=6cm,width=\textwidth]{img/gmic_stdlib682.jpg}\\
{\footnotesize \textbf{Example 682~:} \texttt{image.jpg pink\_lvkern {\comma}u}}
\end{center}

\subsection{\emph{pink\_reg\_minima\index{pink\_reg\_minima}} }\vspace*{-0.7em}
~\\\textbf{\Cb{Arguments: }}\begin{flushleft}
{\small \Cb{\hspace*{0.5cm}$\bullet$~~\texttt{\_connectivity=\{ 4 ~$|$~ 8 ~$|$~ 6 ~$|$~ 26 \}}}}\end{flushleft}
(http://pinkhq.com/doxygen/minima\_8c.html)
~\\Regional minima (requires the PINK library to be installed).
\begin{flushleft}\Cc{\textbf{Default values}:\\~\\\hspace*{0.5cm}{\small $\bullet$~~\texttt{'connectivity=4'.}}}\end{flushleft}
\begin{center}\includegraphics[keepaspectratio=true,height=6cm,width=\textwidth]{img/gmic_stdlib683.jpg}\\
{\footnotesize \textbf{Example 683~:} \texttt{image.jpg --blur 2 pink\_reg\_minima {\comma}}}
\end{center}

\subsection{\emph{pink\_skelcurv\index{pink\_skelcurv}} }\vspace*{-0.7em}
~\\\textbf{\Cb{Arguments: }}\begin{flushleft}
{\small \Cb{\hspace*{0.5cm}$\bullet$~~\texttt{\_prio=\{0~$|$~1~$|$~2~$|$~3~$|$~4~$|$~8~$|$~6~$|$~26\}{\comma}\_connectivity=\{ 4 ~$|$~ 8 ~$|$~ 6 ~$|$~ 26 \}{\comma}\_i\-nhibit=\{""\}}}}\end{flushleft}
(http://pinkhq.com/doxygen/skelcurv\_8c.html)
~\\Curvilinear binary skeleton guided by a priority function or image (requires the PINK library to be installed).
\begin{flushleft}\Cc{\textbf{Default values}:\\~\\\hspace*{0.5cm}{\small $\bullet$~~\texttt{'prio=0'{\comma} 'connectivity=4'} and \texttt{'inhibit=""'.}}}\end{flushleft}
\begin{center}\includegraphics[keepaspectratio=true,height=6cm,width=\textwidth]{img/gmic_stdlib684.jpg}\\
{\footnotesize \textbf{Example 684~:} \texttt{image.jpg threshold 50\% \{w\}{\comma}\{h\} fill[-1] 'if(x$>$w/2{\comma}255{\comma}0)' tp=\$\{-path\_tmp\} output[-1] \$\{tp\}/inhibit.pgm remove[-1] --pink\_skelcurv[0] {\comma} --pink\_skelcurv[0] {\comma}{\comma}\$\{tp\}/inhibit.pgm exec "rm "\$\{tp\}"/inhibit.pgm"}}
\\\includegraphics[keepaspectratio=true,height=6cm,width=\textwidth]{img/gmic_stdlib685.jpg}\\
{\footnotesize \textbf{Example 685~:} \texttt{image.jpg threshold 50\% --pink\_skelcurv {\comma} --pink\_skelcurv[-2] {\comma}8}}
\end{center}

\subsection{\emph{pink\_skelend\index{pink\_skelend}} }\vspace*{-0.7em}
~\\\textbf{\Cb{Arguments: }}\begin{flushleft}
{\small \Cb{\hspace*{0.5cm}$\bullet$~~\texttt{\_connectivity=\{ 4 ~$|$~ 8 ~$|$~ 6 ~$|$~ 26 \}{\comma}\_n=0}}}\end{flushleft}
(http://pinkhq.com/doxygen/skelend\_8c.html)
~\\Homotopic skeleton of a 2d or 3d binary image with dynamic detection of end points (requires the PINK library to be installed).
\begin{flushleft}\Cc{\textbf{Default values}:\\~\\\hspace*{0.5cm}{\small $\bullet$~~\texttt{'connectivity=4'} and \texttt{'n=0'.}}}\end{flushleft}
\begin{center}\includegraphics[keepaspectratio=true,height=6cm,width=\textwidth]{img/gmic_stdlib686.jpg}\\
{\footnotesize \textbf{Example 686~:} \texttt{image.jpg threshold 50\% --pink\_skelend {\comma} --pink\_skelend[-2] {\comma}-1}}
\end{center}

\subsection{\emph{pink\_skeleton\index{pink\_skeleton}} }\vspace*{-0.7em}
~\\\textbf{\Cb{Arguments: }}\begin{flushleft}
{\small \Cb{\hspace*{0.5cm}$\bullet$~~\texttt{\_prio=\{0~$|$~1~$|$~2~$|$~3~$|$~4~$|$~8~$|$~6~$|$~26\}{\comma}\_connectivity=\{ 4 ~$|$~ 8 ~$|$~ 6 ~$|$~ 26 \}{\comma}\_i\-nhibit=\{""\}}}}\end{flushleft}
(http://pinkhq.com/doxygen/skeleton\_8c.html)
~\\Ultimate binary skeleton guided by a priority image (requires the PINK library to be installed).
\begin{flushleft}\Cc{\textbf{Default values}:\\~\\\hspace*{0.5cm}{\small $\bullet$~~\texttt{'prio=0'{\comma} 'connectivity=4'} and \texttt{'inhibit=""'.}}}\end{flushleft}
\begin{center}\includegraphics[keepaspectratio=true,height=6cm,width=\textwidth]{img/gmic_stdlib687.jpg}\\
{\footnotesize \textbf{Example 687~:} \texttt{image.jpg threshold 50\% --pink\_skeleton[-1] {\comma}}}
\end{center}

\subsection{\emph{pink\_skelpar\index{pink\_skelpar}} }\vspace*{-0.7em}
~\\\textbf{\Cb{Arguments: }}\begin{flushleft}
{\small \Cb{\hspace*{0.5cm}$\bullet$~~\texttt{\_algorithm=\{0...29\}{\comma}\_nsteps=\_1{\comma}\_inhibit=""}}}\end{flushleft}
(http://pinkhq.com/doxygen/skelpar\_8c.html)
~\\Parallel binary skeleton (requires the PINK library to be installed).
\begin{flushleft}\Cc{\textbf{Default values}:\\~\\\hspace*{0.5cm}{\small $\bullet$~~\texttt{'algorithm=4'{\comma} 'nsteps=-1'} and \texttt{'inhibit=""'.}}}\end{flushleft}
\begin{center}\includegraphics[keepaspectratio=true,height=6cm,width=\textwidth]{img/gmic_stdlib688.jpg}\\
{\footnotesize \textbf{Example 688~:} \texttt{image.jpg threshold 50\% --pink\_skelpar[-1] 0 --pink\_skelpar[-1] 2}}
\end{center}

\subsection{\emph{pink\_wshed\index{pink\_wshed}} }\vspace*{-0.7em}
~\\\textbf{\Cb{Arguments: }}\begin{flushleft}
{\small \Cb{\hspace*{0.5cm}$\bullet$~~\texttt{\_connectivity=\{ 4 ~$|$~ 8 ~$|$~ 6 ~$|$~ 26 \}{\comma}\_inverse=\{ 0 ~$|$~ 1 \}{\comma}\_height=\-0}}}\end{flushleft}
(http://pinkhq.com/doxygen/wshedtopo\_8c.html)
~\\Watershed (requires the PINK library to be installed).
\begin{flushleft}\Cc{\textbf{Default values}:\\~\\\hspace*{0.5cm}{\small $\bullet$~~\texttt{'connectivity=4'{\comma} 'inverse=0'} and \texttt{'height=0'.}}}\end{flushleft}
\begin{center}\includegraphics[keepaspectratio=true,height=6cm,width=\textwidth]{img/gmic_stdlib689.jpg}\\
{\footnotesize \textbf{Example 689~:} \texttt{image.jpg --pink\_wshed {\comma}1{\comma}5 pink\_wshed[0] {\comma}{\comma}5}}
\end{center}
\section{Convenience functions}


\subsection{\emph{alert\index{alert}} }\vspace*{-0.7em}
~\\\textbf{\Cb{Arguments: }}\begin{flushleft}
{\small \Cb{\hspace*{0.5cm}$\bullet$~~\texttt{\_title{\comma}\_message{\comma}\_label\_button1{\comma}\_label\_button2{\comma}...}}}\end{flushleft}
Display an alert box and wait for user's choice.
~\\If a single image is in the selection{\comma} it is used as an icon for the alert box.
\begin{flushleft}\Cc{\textbf{Default values}:\\~\\\hspace*{0.5cm}{\small $\bullet$~~\texttt{'title=[G'MIC Alert]'} and \texttt{'message=This is an alert box.'.}}}\end{flushleft}


\subsection{\emph{arg\index{arg}} }\vspace*{-0.7em}
~\\\textbf{\Cb{Arguments: }}\begin{flushleft}
{\small \Cb{\hspace*{0.5cm}$\bullet$~~\texttt{n$>$=1{\comma}\_arg1{\comma}...{\comma}\_argN}}}\end{flushleft}
Return the n-th argument of the specified argument list.


\subsection{\emph{arg2var\index{arg2var}} }\vspace*{-0.7em}
~\\\textbf{\Cb{Arguments: }}\begin{flushleft}
{\small \Cb{\hspace*{0.5cm}$\bullet$~~\texttt{variable\_name{\comma}argument\_1{\comma}...{\comma}argument\_N}}}\end{flushleft}
For each i in [1...N]{\comma} set 'variable\_name\$i=argument\_i'.
~\\The variable name should be global to make this command useful (i.e. starts by an underscore).


\subsection{\emph{autocrop\_coords\index{autocrop\_coords}} }\vspace*{-0.7em}
~\\\textbf{\Cb{Arguments: }}\begin{flushleft}
{\small \Cb{\hspace*{0.5cm}$\bullet$~~\texttt{value1{\comma}value2{\comma}... ~$|$~ auto}}}\end{flushleft}
Return coordinates (x0{\comma}y0{\comma}z0{\comma}x1{\comma}y1{\comma}z1) of the autocrop that could be performed on the latest of the selected images.
\begin{flushleft}\Cc{\textbf{Default value}:\\~\\\hspace*{0.5cm}{\small $\bullet$~~\texttt{'auto'}}}\end{flushleft}


\subsection{\emph{average\_color\index{average\_color}} }\vspace*{-0.7em}
Return the average color of the latest of the selected images.


\subsection{\emph{basename\index{basename}} }\vspace*{-0.7em}
~\\\textbf{\Cb{Arguments: }}\begin{flushleft}
{\small \Cb{\hspace*{0.5cm}$\bullet$~~\texttt{file\_path{\comma}\_variable\_name\_for\_folder}}}\end{flushleft}
Return the basename of a file path{\comma} and opt. its folder location.
~\\When specified 'variable\_name\_for\_folder' must starts by an underscore
~\\(global variable accessible from calling function).


\subsection{\emph{bin\index{bin}} }\vspace*{-0.7em}
~\\\textbf{\Cb{Arguments: }}\begin{flushleft}
{\small \Cb{\hspace*{0.5cm}$\bullet$~~\texttt{binary\_int1{\comma}...}}}\end{flushleft}
Print specified binary integers into their octal{\comma} decimal{\comma} hexadecimal and string representations.


\subsection{\emph{bin2dec\index{bin2dec}} }\vspace*{-0.7em}
~\\\textbf{\Cb{Arguments: }}\begin{flushleft}
{\small \Cb{\hspace*{0.5cm}$\bullet$~~\texttt{binary\_int1{\comma}...}}}\end{flushleft}
Convert specified binary integers into their decimal representations.


\subsection{\emph{dec\index{dec}} }\vspace*{-0.7em}
~\\\textbf{\Cb{Arguments: }}\begin{flushleft}
{\small \Cb{\hspace*{0.5cm}$\bullet$~~\texttt{decimal\_int1{\comma}...}}}\end{flushleft}
Print specified decimal integers into their binary{\comma} octal{\comma} hexadecimal and string representations.


\subsection{\emph{dec2str\index{dec2str}} }\vspace*{-0.7em}
~\\\textbf{\Cb{Arguments: }}\begin{flushleft}
{\small \Cb{\hspace*{0.5cm}$\bullet$~~\texttt{decimal\_int1{\comma}...}}}\end{flushleft}
Convert specifial decimal integers into its string representation.


\subsection{\emph{dec2bin\index{dec2bin}} }\vspace*{-0.7em}
~\\\textbf{\Cb{Arguments: }}\begin{flushleft}
{\small \Cb{\hspace*{0.5cm}$\bullet$~~\texttt{decimal\_int1{\comma}...}}}\end{flushleft}
Convert specified decimal integers into their binary representations.


\subsection{\emph{dec2hex\index{dec2hex}} }\vspace*{-0.7em}
~\\\textbf{\Cb{Arguments: }}\begin{flushleft}
{\small \Cb{\hspace*{0.5cm}$\bullet$~~\texttt{decimal\_int1{\comma}...}}}\end{flushleft}
Convert specified decimal integers into their hexadecimal representations.


\subsection{\emph{dec2oct\index{dec2oct}} }\vspace*{-0.7em}
~\\\textbf{\Cb{Arguments: }}\begin{flushleft}
{\small \Cb{\hspace*{0.5cm}$\bullet$~~\texttt{decimal\_int1{\comma}...}}}\end{flushleft}
Convert specified decimal integers into their octal representations.


\subsection{\emph{fact\index{fact}} }\vspace*{-0.7em}
~\\\textbf{\Cb{Arguments: }}\begin{flushleft}
{\small \Cb{\hspace*{0.5cm}$\bullet$~~\texttt{value}}}\end{flushleft}
Return the factorial of the specified value.


\subsection{\emph{fibonacci\index{fibonacci}} }\vspace*{-0.7em}
~\\\textbf{\Cb{Arguments: }}\begin{flushleft}
{\small \Cb{\hspace*{0.5cm}$\bullet$~~\texttt{N$>$=0}}}\end{flushleft}
Return the Nth number of the Fibonacci sequence.


\subsection{\emph{file\_mv\index{file\_mv}} }\vspace*{-0.7em}
~\\\textbf{\Cb{Arguments: }}\begin{flushleft}
{\small \Cb{\hspace*{0.5cm}$\bullet$~~\texttt{filename\_src{\comma}filename\_dest}}}\end{flushleft}
Rename or move a file from a location \$1 to another location \$2.


\subsection{\emph{file\_rand\index{file\_rand}} }\vspace*{-0.7em}
Return a random filename for storing temporary data.


\subsection{\emph{file\_rm\index{file\_rm}} }\vspace*{-0.7em}
~\\\textbf{\Cb{Arguments: }}\begin{flushleft}
{\small \Cb{\hspace*{0.5cm}$\bullet$~~\texttt{filename}}}\end{flushleft}
Delete a file.


\subsection{\emph{filename\index{filename}} }\vspace*{-0.7em}
~\\\textbf{\Cb{Arguments: }}\begin{flushleft}
{\small \Cb{\hspace*{0.5cm}$\bullet$~~\texttt{filename{\comma}\_number1{\comma}\_number2{\comma}...{\comma}\_numberN}}}\end{flushleft}
Return a filename numbered with specified indices.


\subsection{\emph{files\index{files}} (+)}\vspace*{-0.7em}
~\\\textbf{\Cb{Arguments: }}\begin{flushleft}
{\small \Cb{\hspace*{0.5cm}$\bullet$~~\texttt{\_mode{\comma}path}}}\end{flushleft}
Return the list of files and/or subfolders from specified path.
~\\'path' can be eventually a matching pattern.
~\\'mode' can be \{ 0=files only ~$|$~ 1=folders only ~$|$~ 2=files + folders \}.
~\\Add '3' to 'mode' to return full paths instead of filenames only.
\begin{flushleft}\Cc{\textbf{Default value}:\\~\\\hspace*{0.5cm}{\small $\bullet$~~\texttt{'mode=5'.}}}\end{flushleft}


\subsection{\emph{fitratio\_wh\index{fitratio\_wh}} }\vspace*{-0.7em}
~\\\textbf{\Cb{Arguments: }}\begin{flushleft}
{\small \Cb{\hspace*{0.5cm}$\bullet$~~\texttt{min\_width{\comma}min\_height{\comma}ratio\_wh}}}\end{flushleft}
Return a 2d size 'width{\comma}height' which is bigger than 'min\_width{\comma}min\_height' and has the specified w/h ratio.


\subsection{\emph{fitscreen\index{fitscreen}} }\vspace*{-0.7em}
~\\\textbf{\Cb{Arguments: }}\begin{flushleft}
{\small \Cb{\hspace*{0.5cm}$\bullet$~~\texttt{width{\comma}height{\comma}\_depth{\comma}\_minimal\_size[\%]{\comma}\_maximal\_size[\%]}}}\end{flushleft}
Return the 'ideal' size WxH for a window intended to display an image of specified size on screen.
\begin{flushleft}\Cc{\textbf{Default values}:\\~\\\hspace*{0.5cm}{\small $\bullet$~~\texttt{'depth=1'{\comma} 'minimal\_size=128'} and \texttt{'maximal\_size=85\%'.}}}\end{flushleft}


\subsection{\emph{fps\index{fps}} }\vspace*{-0.7em}
Return the number of time this function is called per second{\comma} or -1 if this info is not yet available.
~\\Useful to display the framerate when displaying animations.


\subsection{\emph{gcd\index{gcd}} }\vspace*{-0.7em}
~\\\textbf{\Cb{Arguments: }}\begin{flushleft}
{\small \Cb{\hspace*{0.5cm}$\bullet$~~\texttt{a{\comma}b}}}\end{flushleft}
Return the GCD (greatest common divisor) between a and b.


\subsection{\emph{hex\index{hex}} }\vspace*{-0.7em}
~\\\textbf{\Cb{Arguments: }}\begin{flushleft}
{\small \Cb{\hspace*{0.5cm}$\bullet$~~\texttt{hexadecimal\_int1{\comma}...}}}\end{flushleft}
Print specified hexadecimal integers into their binary{\comma} octal{\comma} decimal and string representations.


\subsection{\emph{hex2dec\index{hex2dec}} }\vspace*{-0.7em}
~\\\textbf{\Cb{Arguments: }}\begin{flushleft}
{\small \Cb{\hspace*{0.5cm}$\bullet$~~\texttt{hexadecimal\_int1{\comma}...}}}\end{flushleft}
Convert specified hexadecimal integers into their decimal representations.


\subsection{\emph{hex2img\index{hex2img}} }\vspace*{-0.7em}
~\\\textbf{\Cb{Arguments: }}\begin{flushleft}
{\small \Cb{\hspace*{0.5cm}$\bullet$~~\texttt{"hexadecimal\_string"}}}\end{flushleft}
Insert new image 1xN at the end of the list with values specified by the given hexadecimal-encoded string.


\subsection{\emph{hex2str\index{hex2str}} }\vspace*{-0.7em}
~\\\textbf{\Cb{Arguments: }}\begin{flushleft}
{\small \Cb{\hspace*{0.5cm}$\bullet$~~\texttt{hexadecimal\_string}}}\end{flushleft}
Convert specified hexadecimal string into a string.


\subsection{\emph{img2hex\index{img2hex}} }\vspace*{-0.7em}
Return representation of last image as an hexadecimal-encoded string.
~\\Input image must have values that are integers in [0{\comma}255].


\subsection{\emph{img2str\index{img2str}} }\vspace*{-0.7em}
Return the content of the latest of the selected image as a special G'MIC input string.


\subsection{\emph{img2text\index{img2text}} }\vspace*{-0.7em}
~\\\textbf{\Cb{Arguments: }}\begin{flushleft}
{\small \Cb{\hspace*{0.5cm}$\bullet$~~\texttt{\_line\_separator}}}\end{flushleft}
Return text contained in a multi-line image.
\begin{flushleft}\Cc{\textbf{Default value}:\\~\\\hspace*{0.5cm}{\small $\bullet$~~\texttt{'line\_separator= '.}}}\end{flushleft}


\subsection{\emph{img82hex\index{img82hex}} }\vspace*{-0.7em}
Convert selected 8bits-valued vectors into their hexadecimal representations (ascii-encoded).


\subsection{\emph{hex2img8\index{hex2img8}} }\vspace*{-0.7em}
Convert selected hexadecimal representations (ascii-encoded) into 8bits-valued vectors.


\subsection{\emph{is\_3d\index{is\_3d}} }\vspace*{-0.7em}
Return 1 if all of the selected image are 3d objects{\comma} 0 otherwise.


\subsection{\emph{is\_ext\index{is\_ext}} }\vspace*{-0.7em}
~\\\textbf{\Cb{Arguments: }}\begin{flushleft}
{\small \Cb{\hspace*{0.5cm}$\bullet$~~\texttt{filename{\comma}\_extension}}}\end{flushleft}
Return 1 if specified filename has a given extensioin.


\subsection{\emph{is\_image\_arg\index{is\_image\_arg}} }\vspace*{-0.7em}
~\\\textbf{\Cb{Arguments: }}\begin{flushleft}
{\small \Cb{\hspace*{0.5cm}$\bullet$~~\texttt{string}}}\end{flushleft}
Return 1 if specified string looks like '[ind]'.


\subsection{\emph{is\_pattern\index{is\_pattern}} }\vspace*{-0.7em}
~\\\textbf{\Cb{Arguments: }}\begin{flushleft}
{\small \Cb{\hspace*{0.5cm}$\bullet$~~\texttt{string}}}\end{flushleft}
Return 1 if specified string looks like a drawing pattern '0x......'.


\subsection{\emph{is\_percent\index{is\_percent}} }\vspace*{-0.7em}
~\\\textbf{\Cb{Arguments: }}\begin{flushleft}
{\small \Cb{\hspace*{0.5cm}$\bullet$~~\texttt{string}}}\end{flushleft}
Return 1 if specified string ends with a '\%'{\comma} 0 otherwise.


\subsection{\emph{is\_videofilename\index{is\_videofilename}} }\vspace*{-0.7em}
Return 1 if extension of specified filename is typical from video files.


\subsection{\emph{is\_windows\index{is\_windows}} }\vspace*{-0.7em}
Return 1 if current computer OS is Windows{\comma} 0 otherwise.


\subsection{\emph{math\_lib\index{math\_lib}} }\vspace*{-0.7em}
Return string that defines a set of several useful macros for the embedded math evaluator.
\begin{center}\includegraphics[keepaspectratio=true,height=6cm,width=\textwidth]{img/gmic_stdlib690.jpg}\\
{\footnotesize \textbf{Example 690~:} \texttt{image.jpg eval \$\{-math\_lib\}" for (i=0{\comma} i$<$1000{\comma} ++i{\comma} draw\_line(\%0{\comma}0.5*[w{\comma}h]{\comma}u([w{\comma}h]){\comma}0.5{\comma}u([255{\comma}255{\comma}255])));"}}
\end{center}

\subsection{\emph{mad\index{mad}} }\vspace*{-0.7em}
Return the MAD (Maximum Absolute Deviation) of the last selected image.
~\\The MAD is defined as MAD = med\_i~$|$~x\_i-med\_j(x\_j)~$|$~


\subsection{\emph{max\_w\index{max\_w}} }\vspace*{-0.7em}
Return the maximal width between selected images.


\subsection{\emph{max\_h\index{max\_h}} }\vspace*{-0.7em}
Return the maximal height between selected images.


\subsection{\emph{max\_d\index{max\_d}} }\vspace*{-0.7em}
Return the maximal depth between selected images.


\subsection{\emph{max\_s\index{max\_s}} }\vspace*{-0.7em}
Return the maximal spectrum between selected images.


\subsection{\emph{max\_wh\index{max\_wh}} }\vspace*{-0.7em}
Return the maximal wxh size of selected images.


\subsection{\emph{max\_whd\index{max\_whd}} }\vspace*{-0.7em}
Return the maximal wxhxd size of selected images.


\subsection{\emph{max\_whds\index{max\_whds}} }\vspace*{-0.7em}
Return the maximal wxhxdxs size of selected images.


\subsection{\emph{med\index{med}} }\vspace*{-0.7em}
Return the median value of the last selected image.


\subsection{\emph{median\_color\index{median\_color}} }\vspace*{-0.7em}
Return the median color value of the last selected image.


\subsection{\emph{min\_w\index{min\_w}} }\vspace*{-0.7em}
Return the minimal width between selected images.


\subsection{\emph{min\_h\index{min\_h}} }\vspace*{-0.7em}
Return the minimal height between selected images.


\subsection{\emph{min\_d\index{min\_d}} }\vspace*{-0.7em}
Return the minimal depth between selected images.


\subsection{\emph{min\_s\index{min\_s}} }\vspace*{-0.7em}
Return the minimal s size of selected images.


\subsection{\emph{min\_wh\index{min\_wh}} }\vspace*{-0.7em}
Return the minimal wxh size of selected images.


\subsection{\emph{min\_whd\index{min\_whd}} }\vspace*{-0.7em}
Return the minimal wxhxd size of selected images.


\subsection{\emph{min\_whds\index{min\_whds}} }\vspace*{-0.7em}
Return the minimal wxhxdxs size of selected images.


\subsection{\emph{normalize\_filename\index{normalize\_filename}} }\vspace*{-0.7em}
~\\\textbf{\Cb{Arguments: }}\begin{flushleft}
{\small \Cb{\hspace*{0.5cm}$\bullet$~~\texttt{filename}}}\end{flushleft}
Return a "normalized" version of the specified filename{\comma} without spaces and capital letters.


\subsection{\emph{oct\index{oct}} }\vspace*{-0.7em}
~\\\textbf{\Cb{Arguments: }}\begin{flushleft}
{\small \Cb{\hspace*{0.5cm}$\bullet$~~\texttt{octal\_int1{\comma}...}}}\end{flushleft}
Print specified octal integers into their binary{\comma} decimal{\comma} hexadecimal and string representations.


\subsection{\emph{oct2dec\index{oct2dec}} }\vspace*{-0.7em}
~\\\textbf{\Cb{Arguments: }}\begin{flushleft}
{\small \Cb{\hspace*{0.5cm}$\bullet$~~\texttt{octal\_int1{\comma}...}}}\end{flushleft}
Convert specified octal integers into their decimal representations.


\subsection{\emph{padint\index{padint}} }\vspace*{-0.7em}
~\\\textbf{\Cb{Arguments: }}\begin{flushleft}
{\small \Cb{\hspace*{0.5cm}$\bullet$~~\texttt{number{\comma}\_size$>$0}}}\end{flushleft}
Return a integer with 'size' digits (eventually left-padded with '0').


\subsection{\emph{path\_gimp\index{path\_gimp}} }\vspace*{-0.7em}
Return a path to store GIMP configuration files for one user (whose value is OS-dependent).


\subsection{\emph{path\_tmp\index{path\_tmp}} }\vspace*{-0.7em}
Return a path to store temporary files (whose value is OS-dependent).


\subsection{\emph{reset\index{reset}} }\vspace*{-0.7em}
Reset global parameters of the interpreter environment.


\subsection{\emph{RGB\index{RGB}} }\vspace*{-0.7em}
Return a random int-valued RGB color.


\subsection{\emph{RGBA\index{RGBA}} }\vspace*{-0.7em}
Return a random int-valued RGBA color.


\subsection{\emph{std\_noise\index{std\_noise}} }\vspace*{-0.7em}
Return the estimated noise standard deviation of the last selected image.


\subsection{\emph{str\index{str}} }\vspace*{-0.7em}
~\\\textbf{\Cb{Arguments: }}\begin{flushleft}
{\small \Cb{\hspace*{0.5cm}$\bullet$~~\texttt{string}}}\end{flushleft}
Print specified string into its binary{\comma} octal{\comma} decimal and hexadecimal representations.


\subsection{\emph{str2hex\index{str2hex}} }\vspace*{-0.7em}
~\\\textbf{\Cb{Arguments: }}\begin{flushleft}
{\small \Cb{\hspace*{0.5cm}$\bullet$~~\texttt{string}}}\end{flushleft}
Convert specified string into a sequence of hexadecimal values.


\subsection{\emph{stresc\index{stresc}} }\vspace*{-0.7em}
~\\\textbf{\Cb{Arguments: }}\begin{flushleft}
{\small \Cb{\hspace*{0.5cm}$\bullet$~~\texttt{val1{\comma}...{\comma}valN}}}\end{flushleft}
Return escaped string from specified ascii codes.


\subsection{\emph{strcat\index{strcat}} }\vspace*{-0.7em}
~\\\textbf{\Cb{Arguments: }}\begin{flushleft}
{\small \Cb{\hspace*{0.5cm}$\bullet$~~\texttt{"string1"{\comma}"string2"{\comma}...}}}\end{flushleft}
Return the concatenation of all strings passed as arguments.


\subsection{\emph{strcmp\index{strcmp}} }\vspace*{-0.7em}
~\\\textbf{\Cb{Arguments: }}\begin{flushleft}
{\small \Cb{\hspace*{0.5cm}$\bullet$~~\texttt{"string1"{\comma}"string2"{\comma}\_nb\_characters$>$=0}}}\end{flushleft}
Return '1' if the two specified strings are equals{\comma} '0' otherwise.
~\\If 'nb\_characters' is specified{\comma} the comparison is done only for the 'nb\_characters' first characters.


\subsection{\emph{strcontains\index{strcontains}} }\vspace*{-0.7em}
~\\\textbf{\Cb{Arguments: }}\begin{flushleft}
{\small \Cb{\hspace*{0.5cm}$\bullet$~~\texttt{string1{\comma}string2}}}\end{flushleft}
Return 1 if the first string contains the second one.


\subsection{\emph{strlen\index{strlen}} }\vspace*{-0.7em}
~\\\textbf{\Cb{Arguments: }}\begin{flushleft}
{\small \Cb{\hspace*{0.5cm}$\bullet$~~\texttt{string1}}}\end{flushleft}
Return the length of specified string argument.


\subsection{\emph{strreplace\index{strreplace}} }\vspace*{-0.7em}
~\\\textbf{\Cb{Arguments: }}\begin{flushleft}
{\small \Cb{\hspace*{0.5cm}$\bullet$~~\texttt{string{\comma}search{\comma}replace}}}\end{flushleft}
Search and replace substrings in an input string.


\subsection{\emph{strlowercase\index{strlowercase}} }\vspace*{-0.7em}
~\\\textbf{\Cb{Arguments: }}\begin{flushleft}
{\small \Cb{\hspace*{0.5cm}$\bullet$~~\texttt{string}}}\end{flushleft}
Return a lower-case version of the specified string.


\subsection{\emph{strvar\index{strvar}} }\vspace*{-0.7em}
~\\\textbf{\Cb{Arguments: }}\begin{flushleft}
{\small \Cb{\hspace*{0.5cm}$\bullet$~~\texttt{string}}}\end{flushleft}
Return a simplified version of the specified string{\comma} that can be used as a variable name.


\subsection{\emph{strver\index{strver}} }\vspace*{-0.7em}
~\\\textbf{\Cb{Arguments: }}\begin{flushleft}
{\small \Cb{\hspace*{0.5cm}$\bullet$~~\texttt{\_version}}}\end{flushleft}
Return the specified version number of the G'MIC interpreter{\comma} as a string.
\begin{flushleft}\Cc{\textbf{Default value}:\\~\\\hspace*{0.5cm}{\small $\bullet$~~\texttt{'version=\$\_version'.}}}\end{flushleft}


\subsection{\emph{tic\index{tic}} }\vspace*{-0.7em}
Initialize tic-toc timer.
~\\Use it in conjunction with '-toc'.


\subsection{\emph{toc\index{toc}} }\vspace*{-0.7em}
Display elapsed time of the tic-toc timer since the last call to '-tic'.
~\\This command returns the elapsed time in the status value.
~\\Use it in conjunction with '-tic'.

\section{Other interactive commands}


\subsection{\emph{demo\index{demo}} }\vspace*{-0.7em}
~\\\textbf{\Cb{Arguments: }}\begin{flushleft}
{\small \Cb{\hspace*{0.5cm}$\bullet$~~\texttt{\_run\_in\_parallel=\{ 0=no ~$|$~ 1=yes ~$|$~ 2=auto \}}}}\end{flushleft}
Show a menu to select and view all G'MIC interactive demos.


\subsection{\emph{x\_2048\index{x\_2048}} }\vspace*{-0.7em}
Launch the 2048 game.


\subsection{\emph{x\_blobs\index{x\_blobs}} }\vspace*{-0.7em}
Launch the blobs editor.


\subsection{\emph{x\_bouncing\index{x\_bouncing}} }\vspace*{-0.7em}
Launch the bouncing balls demo.


\subsection{\emph{x\_color\_curves\index{x\_color\_curves}} }\vspace*{-0.7em}
~\\\textbf{\Cb{Arguments: }}\begin{flushleft}
{\small \Cb{\hspace*{0.5cm}$\bullet$~~\texttt{\_colorspace=\{ rgb ~$|$~ cmy ~$|$~ cmyk ~$|$~ hsi ~$|$~ hsl ~$|$~ hsv ~$|$~ lab ~$|$~ lch\- ~$|$~ ycbcr ~$|$~ last \}}}}\end{flushleft}
Apply color curves on selected RGB[A] images{\comma} using an interactive window.
~\\Set 'colorspace' to 'last' to apply last defined color curves without opening interactive windows.
\begin{flushleft}\Cc{\textbf{Default value}:\\~\\\hspace*{0.5cm}{\small $\bullet$~~\texttt{'colorspace=rgb'.}}}\end{flushleft}


\subsection{\emph{x\_colorize\index{x\_colorize}} }\vspace*{-0.7em}
~\\\textbf{\Cb{Arguments: }}\begin{flushleft}
{\small \Cb{\hspace*{0.5cm}$\bullet$~~\texttt{\_is\_lineart=\{ 0 ~$|$~ 1 \}{\comma}\_max\_resolution=\{ 0 ~$|$~ $>$=128 \}{\comma}\_multich\-annels\_output=\{ 0 ~$|$~ 1 \}{\comma}\_[palette1]{\comma}\_[palette2]{\comma}\_[grabber1]}}}\end{flushleft}
Colorized selected B\&W images{\comma} using an interactive window.
~\\When $>$0{\comma} argument 'max\_resolution' defines the maximal image resolution used in the interactive window.
\begin{flushleft}\Cc{\textbf{Default values}:\\~\\\hspace*{0.5cm}{\small $\bullet$~~\texttt{'is\_lineart=1'{\comma} 'max\_resolution=1024'} and \texttt{'multichannels\_output=0'.}}}\end{flushleft}


\subsection{\emph{x\_connect4\index{x\_connect4}} }\vspace*{-0.7em}
Launch the Connect Four game.


\subsection{\emph{x\_fire\index{x\_fire}} }\vspace*{-0.7em}
Launch the fire effect demo.


\subsection{\emph{x\_fireworks\index{x\_fireworks}} }\vspace*{-0.7em}
Launch the fireworks demo.


\subsection{\emph{x\_fisheye\index{x\_fisheye}} }\vspace*{-0.7em}
Launch the fish-eye effect demo.


\subsection{\emph{x\_fourier\index{x\_fourier}} }\vspace*{-0.7em}
Launch the fourier filtering demo.


\subsection{\emph{x\_grab\_color\index{x\_grab\_color}} }\vspace*{-0.7em}
~\\\textbf{\Cb{Arguments: }}\begin{flushleft}
{\small \Cb{\hspace*{0.5cm}$\bullet$~~\texttt{\_variable\_name}}}\end{flushleft}
Open a color grabber widget from the first selected image.
~\\Argument 'variable\_name' specifies the variable that contains the selected color values at any time.
~\\Assigning '-1' to it forces the interactive window to close.
\begin{flushleft}\Cc{\textbf{Default values}:\\~\\\hspace*{0.5cm}{\small $\bullet$~~\texttt{'variable\_name=xgc\_variable'.}}}\end{flushleft}


\subsection{\emph{x\_histogram\index{x\_histogram}} }\vspace*{-0.7em}
Launch the histogram demo.


\subsection{\emph{x\_hough\index{x\_hough}} }\vspace*{-0.7em}
Launch the hough transform demo.


\subsection{\emph{x\_jawbreaker\index{x\_jawbreaker}} }\vspace*{-0.7em}
~\\\textbf{\Cb{Arguments: }}\begin{flushleft}
{\small \Cb{\hspace*{0.5cm}$\bullet$~~\texttt{0$<$\_width$<$20{\comma}0$<$\_height$<$20{\comma}0$<$\_balls$<$=8}}}\end{flushleft}
Launch the Jawbreaker game.


\subsection{\emph{x\_landscape\index{x\_landscape}} }\vspace*{-0.7em}
Launch the virtual landscape demo.


\subsection{\emph{x\_life\index{x\_life}} }\vspace*{-0.7em}
Launch the game of life.


\subsection{\emph{x\_light\index{x\_light}} }\vspace*{-0.7em}
Launch the light effect demo.


\subsection{\emph{x\_mandelbrot\index{x\_mandelbrot}} }\vspace*{-0.7em}
~\\\textbf{\Cb{Arguments: }}\begin{flushleft}
{\small \Cb{\hspace*{0.5cm}$\bullet$~~\texttt{\_julia=\{ 0 ~$|$~ 1 \}{\comma}\_c0r{\comma}\_c0i}}}\end{flushleft}
Launch Mandelbrot/Julia explorer.


\subsection{\emph{x\_mask\_color\index{x\_mask\_color}} }\vspace*{-0.7em}
~\\\textbf{\Cb{Arguments: }}\begin{flushleft}
{\small \Cb{\hspace*{0.5cm}$\bullet$~~\texttt{\_colorspace=\{ all ~$|$~ rgb ~$|$~ lrgb ~$|$~ ycbcr ~$|$~ lab ~$|$~ lch ~$|$~ hsv ~$|$~ h\-si ~$|$~ hsl ~$|$~ cmy ~$|$~ cmyk ~$|$~ yiq \}{\comma}\_spatial\_tolerance$>$=0{\comma}\_color\_t\-olerance$>$=0}}}\end{flushleft}
Interactively select a color{\comma} and add an alpha channel containing the corresponding color mask.
~\\Argument 'colorspace' refers to the color metric used to compute color similarities{\comma} and can be basically one of \{ rgb ~$|$~ lrgb ~$|$~ ycbcr ~$|$~ lab ~$|$~ lch ~$|$~ hsv ~$|$~ hsi ~$|$~ hsl ~$|$~ cmy ~$|$~ cmyk ~$|$~ yiq \}.
~\\You can also select one one particular channel of this colorspace{\comma} by setting 'colorspace' as 'colorspace\_channel' (e.g. 'hsv\_h' for the hue).
\begin{flushleft}\Cc{\textbf{Default values}:\\~\\\hspace*{0.5cm}{\small $\bullet$~~\texttt{'colorspace=all'{\comma} 'spatial\_tolerance=5'} and \texttt{'color\_tolerance=5'.}}}\end{flushleft}


\subsection{\emph{x\_metaballs3d\index{x\_metaballs3d}} }\vspace*{-0.7em}
Launch the 3d metaballs demo.


\subsection{\emph{x\_minesweeper\index{x\_minesweeper}} }\vspace*{-0.7em}
~\\\textbf{\Cb{Arguments: }}\begin{flushleft}
{\small \Cb{\hspace*{0.5cm}$\bullet$~~\texttt{8$<$=\_width=$<$20{\comma}8$<$=\_height$<$=20}}}\end{flushleft}
Launch the Minesweeper game.


\subsection{\emph{x\_minimal\_path\index{x\_minimal\_path}} }\vspace*{-0.7em}
Launch the minimal path demo.


\subsection{\emph{x\_pacman\index{x\_pacman}} }\vspace*{-0.7em}
Launch pacman game.


\subsection{\emph{x\_paint\index{x\_paint}} }\vspace*{-0.7em}
Launch the interactive painter.


\subsection{\emph{x\_plasma\index{x\_plasma}} }\vspace*{-0.7em}
Launch the plasma effect demo.


\subsection{\emph{x\_quantize\_rgb\index{x\_quantize\_rgb}} }\vspace*{-0.7em}
~\\\textbf{\Cb{Arguments: }}\begin{flushleft}
{\small \Cb{\hspace*{0.5cm}$\bullet$~~\texttt{\_nbcolors$>$=2}}}\end{flushleft}
Launch the RGB color quantization demo.


\subsection{\emph{x\_reflection3d\index{x\_reflection3d}} }\vspace*{-0.7em}
Launch the 3d reflection demo.


\subsection{\emph{x\_rubber3d\index{x\_rubber3d}} }\vspace*{-0.7em}
Launch the 3d rubber object demo.


\subsection{\emph{x\_segment\index{x\_segment}} }\vspace*{-0.7em}
~\\\textbf{\Cb{Arguments: }}\begin{flushleft}
{\small \Cb{\hspace*{0.5cm}$\bullet$~~\texttt{\_max\_resolution=\{ 0 ~$|$~ $>$=128 \}}}}\end{flushleft}
Segment foreground from background in selected opaque RGB images{\comma} interactively.
~\\Return RGBA images with binary alpha-channels.
\begin{flushleft}\Cc{\textbf{Default value}:\\~\\\hspace*{0.5cm}{\small $\bullet$~~\texttt{'max\_resolution=1024'.}}}\end{flushleft}


\subsection{\emph{x\_select\_color\index{x\_select\_color}} }\vspace*{-0.7em}
~\\\textbf{\Cb{Arguments: }}\begin{flushleft}
{\small \Cb{\hspace*{0.5cm}$\bullet$~~\texttt{\_variable\_name}}}\end{flushleft}
Display a RGB or RGBA color selector.
~\\Argument 'variable\_name' specifies the variable that contains the selected color values (as R{\comma}G{\comma}B{\comma}[A]) at any time.
~\\Its value specifies the initial selected color. Assigning '-1' to it forces the interactive window to close.
\begin{flushleft}\Cc{\textbf{Default value}:\\~\\\hspace*{0.5cm}{\small $\bullet$~~\texttt{'variable\_name=xsc\_variable'.}}}\end{flushleft}


\subsection{\emph{x\_select\_function1d\index{x\_select\_function1d}} }\vspace*{-0.7em}
~\\\textbf{\Cb{Arguments: }}\begin{flushleft}
{\small \Cb{\hspace*{0.5cm}$\bullet$~~\texttt{\_variable\_name{\comma}\_background\_curve\_R{\comma}\_background\_curve\_G{\comma}\_back\-ground\_curve\_B}}}\end{flushleft}
Open an interactive window{\comma} where the user can defined its own 1d function.
~\\If an image is selected{\comma} it is used to display additional information :
- The first row defines the values of a background curve displayed on the window (e.g. an histogram).
- The 2nd{\comma} 3rd and 4th rows define the R{\comma}G{\comma}B color components displayed beside the X and Y axes.
~\\Argument 'variable\_name' specifies the variable that contains the selected function keypoints at any time.
~\\Assigning '-1' to it forces the interactive window to close.
\begin{flushleft}\Cc{\textbf{Default values}:\\~\\\hspace*{0.5cm}{\small $\bullet$~~\texttt{'variable\_name=xsf\_variable'{\comma} 'background\_curve\_R=220'{\comma} 'background\_curve\_G=background\_curve\_B=background\_curve\_T'.}}}\end{flushleft}


\subsection{\emph{x\_select\_palette\index{x\_select\_palette}} }\vspace*{-0.7em}
~\\\textbf{\Cb{Arguments: }}\begin{flushleft}
{\small \Cb{\hspace*{0.5cm}$\bullet$~~\texttt{\_variable\_name{\comma}\_number\_of\_columns=\{ 0=auto ~$|$~ $>$0 \}}}}\end{flushleft}
Open a RGB or RGBA color selector widget from a palette.
~\\The palette is given as a selected image.
~\\Argument 'variable\_name' specifies the variable that contains the selected color values (as R{\comma}G{\comma}B{\comma}[A]) at any time.
~\\Assigning '-1' to it forces the interactive window to close.
\begin{flushleft}\Cc{\textbf{Default values}:\\~\\\hspace*{0.5cm}{\small $\bullet$~~\texttt{'variable\_name=xsp\_variable'} and \texttt{'number\_of\_columns=2'.}}}\end{flushleft}


\subsection{\emph{x\_shadebobs\index{x\_shadebobs}} }\vspace*{-0.7em}
Launch the shade bobs demo.


\subsection{\emph{x\_spline\index{x\_spline}} }\vspace*{-0.7em}
Launch spline curve editor.


\subsection{\emph{x\_starfield3d\index{x\_starfield3d}} }\vspace*{-0.7em}
Launch the 3d starfield demo.


\subsection{\emph{x\_tetris\index{x\_tetris}} }\vspace*{-0.7em}
Launch tetris game.


\subsection{\emph{x\_tictactoe\index{x\_tictactoe}} }\vspace*{-0.7em}
Launch tic-tac-toe game.


\subsection{\emph{x\_waves\index{x\_waves}} }\vspace*{-0.7em}
Launch the image waves demo.


\subsection{\emph{x\_whirl\index{x\_whirl}} }\vspace*{-0.7em}
~\\\textbf{\Cb{Arguments: }}\begin{flushleft}
{\small \Cb{\hspace*{0.5cm}$\bullet$~~\texttt{\_opacity$>$=0}}}\end{flushleft}
Launch the fractal whirls demo.
\begin{flushleft}\Cc{\textbf{Default values}:\\~\\\hspace*{0.5cm}{\small $\bullet$~~\texttt{'opacity=0.2'.}}}\end{flushleft}

\section{Command shortcuts}
$\bullet$~'\texttt{\Ca{h}}' ~is equivalent to~~'\texttt{\Ca{help}}'.\\
$\bullet$~'\texttt{\Ca{m}}' (+)~is equivalent to~~'\texttt{\Ca{command}}'.\\
$\bullet$~'\texttt{\Ca{d}}' (+)~is equivalent to~~'\texttt{\Ca{display}}'.\\
$\bullet$~'\texttt{\Ca{d0}}' ~is equivalent to~~'\texttt{\Ca{display0}}'.\\
$\bullet$~'\texttt{\Ca{d3d}}' (+)~is equivalent to~~'\texttt{\Ca{display3d}}'.\\
$\bullet$~'\texttt{\Ca{da}}' ~is equivalent to~~'\texttt{\Ca{display\_array}}'.\\
$\bullet$~'\texttt{\Ca{dfft}}' ~is equivalent to~~'\texttt{\Ca{display\_fft}}'.\\
$\bullet$~'\texttt{\Ca{dg}}' ~is equivalent to~~'\texttt{\Ca{display\_graph}}'.\\
$\bullet$~'\texttt{\Ca{dh}}' ~is equivalent to~~'\texttt{\Ca{display\_histogram}}'.\\
$\bullet$~'\texttt{\Ca{dp}}' ~is equivalent to~~'\texttt{\Ca{display\_parallel}}'.\\
$\bullet$~'\texttt{\Ca{dp0}}' ~is equivalent to~~'\texttt{\Ca{display\_parallel0}}'.\\
$\bullet$~'\texttt{\Ca{dq}}' ~is equivalent to~~'\texttt{\Ca{display\_quiver}}'.\\
$\bullet$~'\texttt{\Ca{drgba}}' ~is equivalent to~~'\texttt{\Ca{display\_rgba}}'.\\
$\bullet$~'\texttt{\Ca{dt}}' ~is equivalent to~~'\texttt{\Ca{display\_tensors}}'.\\
$\bullet$~'\texttt{\Ca{dw}}' ~is equivalent to~~'\texttt{\Ca{display\_warp}}'.\\
$\bullet$~'\texttt{\Ca{e}}' (+)~is equivalent to~~'\texttt{\Ca{echo}}'.\\
$\bullet$~'\texttt{\Ca{i}}' (+)~is equivalent to~~'\texttt{\Ca{input}}'.\\
$\bullet$~'\texttt{\Ca{ig}}' ~is equivalent to~~'\texttt{\Ca{input\_glob}}'.\\
$\bullet$~'\texttt{\Ca{o}}' (+)~is equivalent to~~'\texttt{\Ca{output}}'.\\
$\bullet$~'\texttt{\Ca{on}}' ~is equivalent to~~'\texttt{\Ca{outputn}}'.\\
$\bullet$~'\texttt{\Ca{op}}' ~is equivalent to~~'\texttt{\Ca{outputp}}'.\\
$\bullet$~'\texttt{\Ca{ow}}' ~is equivalent to~~'\texttt{\Ca{outputw}}'.\\
$\bullet$~'\texttt{\Ca{ox}}' ~is equivalent to~~'\texttt{\Ca{outputx}}'.\\
$\bullet$~'\texttt{\Ca{p}}' (+)~is equivalent to~~'\texttt{\Ca{print}}'.\\
$\bullet$~'\texttt{\Ca{sh}}' (+)~is equivalent to~~'\texttt{\Ca{shared}}'.\\
$\bullet$~'\texttt{\Ca{sp}}' ~is equivalent to~~'\texttt{\Ca{sample}}'.\\
$\bullet$~'\texttt{\Ca{up}}' ~is equivalent to~~'\texttt{\Ca{update}}'.\\
$\bullet$~'\texttt{\Ca{v}}' (+)~is equivalent to~~'\texttt{\Ca{verbose}}'.\\
$\bullet$~'\texttt{\Ca{w}}' (+)~is equivalent to~~'\texttt{\Ca{window}}'.\\
$\bullet$~'\texttt{\Ca{k}}' (+)~is equivalent to~~'\texttt{\Ca{keep}}'.\\
$\bullet$~'\texttt{\Ca{mv}}' (+)~is equivalent to~~'\texttt{\Ca{move}}'.\\
$\bullet$~'\texttt{\Ca{nm}}' (+)~is equivalent to~~'\texttt{\Ca{name}}'.\\
$\bullet$~'\texttt{\Ca{nms}}' ~is equivalent to~~'\texttt{\Ca{names}}'.\\
$\bullet$~'\texttt{\Ca{rm}}' (+)~is equivalent to~~'\texttt{\Ca{remove}}'.\\
$\bullet$~'\texttt{\Ca{rv}}' (+)~is equivalent to~~'\texttt{\Ca{reverse}}'.\\
$\bullet$~'\texttt{\Ca{+}}' (+)~is equivalent to~~'\texttt{\Ca{add}}'.\\
$\bullet$~'\texttt{\Ca{\&}}' (+)~is equivalent to~~'\texttt{\Ca{and}}'.\\
$\bullet$~'\texttt{\Ca{$<$$<$}}' (+)~is equivalent to~~'\texttt{\Ca{bsl}}'.\\
$\bullet$~'\texttt{\Ca{$>$$>$}}' (+)~is equivalent to~~'\texttt{\Ca{bsr}}'.\\
$\bullet$~'\texttt{\Ca{/}}' (+)~is equivalent to~~'\texttt{\Ca{div}}'.\\
$\bullet$~'\texttt{\Ca{==}}' (+)~is equivalent to~~'\texttt{\Ca{eq}}'.\\
$\bullet$~'\texttt{\Ca{$>$=}}' (+)~is equivalent to~~'\texttt{\Ca{ge}}'.\\
$\bullet$~'\texttt{\Ca{$>$}}' (+)~is equivalent to~~'\texttt{\Ca{gt}}'.\\
$\bullet$~'\texttt{\Ca{$<$=}}' (+)~is equivalent to~~'\texttt{\Ca{le}}'.\\
$\bullet$~'\texttt{\Ca{$<$}}' (+)~is equivalent to~~'\texttt{\Ca{lt}}'.\\
$\bullet$~'\texttt{\Ca{m/}}' (+)~is equivalent to~~'\texttt{\Ca{mdiv}}'.\\
$\bullet$~'\texttt{\Ca{\%}}' (+)~is equivalent to~~'\texttt{\Ca{mod}}'.\\
$\bullet$~'\texttt{\Ca{m*}}' (+)~is equivalent to~~'\texttt{\Ca{mmul}}'.\\
$\bullet$~'\texttt{\Ca{*}}' (+)~is equivalent to~~'\texttt{\Ca{mul}}'.\\
$\bullet$~'\texttt{\Ca{!=}}' (+)~is equivalent to~~'\texttt{\Ca{neq}}'.\\
$\bullet$~'\texttt{\Ca{~$|$~}}' (+)~is equivalent to~~'\texttt{\Ca{or}}'.\\
$\bullet$~'\texttt{\Ca{\textasciicircum }}' (+)~is equivalent to~~'\texttt{\Ca{pow}}'.\\
$\bullet$~'\texttt{\Ca{-}}' (+)~is equivalent to~~'\texttt{\Ca{sub}}'.\\
$\bullet$~'\texttt{\Ca{c}}' (+)~is equivalent to~~'\texttt{\Ca{cut}}'.\\
$\bullet$~'\texttt{\Ca{f}}' (+)~is equivalent to~~'\texttt{\Ca{fill}}'.\\
$\bullet$~'\texttt{\Ca{ir}}' ~is equivalent to~~'\texttt{\Ca{inrange}}'.\\
$\bullet$~'\texttt{\Ca{n}}' (+)~is equivalent to~~'\texttt{\Ca{normalize}}'.\\
$\bullet$~'\texttt{\Ca{=}}' (+)~is equivalent to~~'\texttt{\Ca{set}}'.\\
$\bullet$~'\texttt{\Ca{ac}}' ~is equivalent to~~'\texttt{\Ca{apply\_channels}}'.\\
$\bullet$~'\texttt{\Ca{fc}}' ~is equivalent to~~'\texttt{\Ca{fill\_color}}'.\\
$\bullet$~'\texttt{\Ca{a}}' (+)~is equivalent to~~'\texttt{\Ca{append}}'.\\
$\bullet$~'\texttt{\Ca{z}}' (+)~is equivalent to~~'\texttt{\Ca{crop}}'.\\
$\bullet$~'\texttt{\Ca{r}}' (+)~is equivalent to~~'\texttt{\Ca{resize}}'.\\
$\bullet$~'\texttt{\Ca{rr2d}}' ~is equivalent to~~'\texttt{\Ca{resize\_ratio2d}}'.\\
$\bullet$~'\texttt{\Ca{r2dx}}' ~is equivalent to~~'\texttt{\Ca{resize2dx}}'.\\
$\bullet$~'\texttt{\Ca{r2dy}}' ~is equivalent to~~'\texttt{\Ca{resize2dy}}'.\\
$\bullet$~'\texttt{\Ca{r3dx}}' ~is equivalent to~~'\texttt{\Ca{resize3dx}}'.\\
$\bullet$~'\texttt{\Ca{r3dy}}' ~is equivalent to~~'\texttt{\Ca{resize3dy}}'.\\
$\bullet$~'\texttt{\Ca{r3dz}}' ~is equivalent to~~'\texttt{\Ca{resize3dz}}'.\\
$\bullet$~'\texttt{\Ca{s}}' (+)~is equivalent to~~'\texttt{\Ca{split}}'.\\
$\bullet$~'\texttt{\Ca{y}}' (+)~is equivalent to~~'\texttt{\Ca{unroll}}'.\\
$\bullet$~'\texttt{\Ca{b}}' (+)~is equivalent to~~'\texttt{\Ca{blur}}'.\\
$\bullet$~'\texttt{\Ca{g}}' (+)~is equivalent to~~'\texttt{\Ca{gradient}}'.\\
$\bullet$~'\texttt{\Ca{j}}' (+)~is equivalent to~~'\texttt{\Ca{image}}'.\\
$\bullet$~'\texttt{\Ca{j3d}}' (+)~is equivalent to~~'\texttt{\Ca{object3d}}'.\\
$\bullet$~'\texttt{\Ca{t}}' (+)~is equivalent to~~'\texttt{\Ca{text}}'.\\
$\bullet$~'\texttt{\Ca{to}}' ~is equivalent to~~'\texttt{\Ca{text\_outline}}'.\\
$\bullet$~'\texttt{\Ca{+3d}}' (+)~is equivalent to~~'\texttt{\Ca{add3d}}'.\\
$\bullet$~'\texttt{\Ca{c3d}}' ~is equivalent to~~'\texttt{\Ca{center3d}}'.\\
$\bullet$~'\texttt{\Ca{col3d}}' (+)~is equivalent to~~'\texttt{\Ca{color3d}}'.\\
$\bullet$~'\texttt{\Ca{/3d}}' (+)~is equivalent to~~'\texttt{\Ca{div3d}}'.\\
$\bullet$~'\texttt{\Ca{db3d}}' (+)~is equivalent to~~'\texttt{\Ca{double3d}}'.\\
$\bullet$~'\texttt{\Ca{f3d}}' (+)~is equivalent to~~'\texttt{\Ca{focale3d}}'.\\
$\bullet$~'\texttt{\Ca{l3d}}' (+)~is equivalent to~~'\texttt{\Ca{light3d}}'.\\
$\bullet$~'\texttt{\Ca{m3d}}' (+)~is equivalent to~~'\texttt{\Ca{mode3d}}'.\\
$\bullet$~'\texttt{\Ca{md3d}}' (+)~is equivalent to~~'\texttt{\Ca{moded3d}}'.\\
$\bullet$~'\texttt{\Ca{*3d}}' (+)~is equivalent to~~'\texttt{\Ca{mul3d}}'.\\
$\bullet$~'\texttt{\Ca{n3d}}' ~is equivalent to~~'\texttt{\Ca{normalize3d}}'.\\
$\bullet$~'\texttt{\Ca{o3d}}' (+)~is equivalent to~~'\texttt{\Ca{opacity3d}}'.\\
$\bullet$~'\texttt{\Ca{p3d}}' (+)~is equivalent to~~'\texttt{\Ca{primitives3d}}'.\\
$\bullet$~'\texttt{\Ca{rv3d}}' (+)~is equivalent to~~'\texttt{\Ca{reverse3d}}'.\\
$\bullet$~'\texttt{\Ca{r3d}}' (+)~is equivalent to~~'\texttt{\Ca{rotate3d}}'.\\
$\bullet$~'\texttt{\Ca{sl3d}}' (+)~is equivalent to~~'\texttt{\Ca{specl3d}}'.\\
$\bullet$~'\texttt{\Ca{ss3d}}' (+)~is equivalent to~~'\texttt{\Ca{specs3d}}'.\\
$\bullet$~'\texttt{\Ca{s3d}}' (+)~is equivalent to~~'\texttt{\Ca{split3d}}'.\\
$\bullet$~'\texttt{\Ca{-3d}}' (+)~is equivalent to~~'\texttt{\Ca{sub3d}}'.\\
$\bullet$~'\texttt{\Ca{t3d}}' (+)~is equivalent to~~'\texttt{\Ca{texturize3d}}'.\\
$\bullet$~'\texttt{\Ca{ap}}' ~is equivalent to~~'\texttt{\Ca{apply\_parallel}}'.\\
$\bullet$~'\texttt{\Ca{apc}}' ~is equivalent to~~'\texttt{\Ca{apply\_parallel\_channels}}'.\\
$\bullet$~'\texttt{\Ca{apo}}' ~is equivalent to~~'\texttt{\Ca{apply\_parallel\_overlap}}'.\\
$\bullet$~'\texttt{\Ca{endl}}' (+)~is equivalent to~~'\texttt{\Ca{endlocal}}'.\\
$\bullet$~'\texttt{\Ca{x}}' (+)~is equivalent to~~'\texttt{\Ca{exec}}'.\\
$\bullet$~'\texttt{\Ca{l}}' (+)~is equivalent to~~'\texttt{\Ca{local}}'.\\
$\bullet$~'\texttt{\Ca{q}}' (+)~is equivalent to~~'\texttt{\Ca{quit}}'.\\
$\bullet$~'\texttt{\Ca{u}}' (+)~is equivalent to~~'\texttt{\Ca{status}}'.\\
$\bullet$~'\texttt{\Ca{frame}}' ~is equivalent to~~'\texttt{\Ca{frame\_xy}}'.\\
~\\\section{Examples of use}
\small
\begin{lstlisting}[escapechar=§]
 '§\aftergroup\Ccg§gmic§\aftergroup\Ccn§' is a generic image processing tool which can be used in a wide variety of situations. 
 The few examples below illustrate possible uses of this tool: 
 
  - View a list of images: 
     §\aftergroup\Ccb§gmic file1.bmp file2.jpeg§\aftergroup\Ccn§ 
 
  - Convert an image file: 
     §\aftergroup\Ccb§gmic input.bmp output output.jpg§\aftergroup\Ccn§ 
 
  - Create a volumetric image from a movie sequence: 
     §\aftergroup\Ccb§gmic input.mpg append z output output.hdr§\aftergroup\Ccn§ 
 
  - Compute image gradient norm: 
     §\aftergroup\Ccb§gmic input.bmp gradient_norm§\aftergroup\Ccn§ 
 
  - Denoise a color image: 
     §\aftergroup\Ccb§gmic image.jpg denoise 30,10 output denoised.jpg§\aftergroup\Ccn§ 
 
  - Compose two images using overlay layer blending: 
     §\aftergroup\Ccb§gmic image1.jpg image2.jpg blend overlay output blended.jpg§\aftergroup\Ccn§ 
 
  - Evaluate a mathematical expression: 
     §\aftergroup\Ccb§gmic echo "cos(pi/4)^2+sin(pi/4)^2={cos(pi/4)^2+sin(pi/4)^2}"§\aftergroup\Ccn§ 
 
  - Plot a 2d function: 
     §\aftergroup\Ccb§gmic 1000,1,1,2 fill "X=3*(x-500)/500;X^2*sin(3*X^2)+if(c==0,u(0,-1),cos(X*10))" plot§\aftergroup\Ccn§ 
 
  - Plot a 3d elevated function in random colors: 
     §\aftergroup\Ccb§gmic 128,128,1,3,"u(0,255)" plasma 10,3 blur 4 sharpen 10000 \
      elevation3d[-1] "'X=(x-64)/6;Y=(y-64)/6;100*exp(-(X^2+Y^2)/30)*abs(cos(X)*sin(Y))'"§\aftergroup\Ccn§ 
 
  - Plot the isosurface of a 3d volume: 
     §\aftergroup\Ccb§gmic mode3d 5 moded3d 5 double3d 0 isosurface3d "'x^2+y^2+abs(z)^abs(4*cos(x*y*z*3))'",3§\aftergroup\Ccn§ 
 
  - Render a §\aftergroup\Ccg§G'MIC§\aftergroup\Ccn§ 3d logo: 
     §\aftergroup\Ccb§gmic 0 text G\'MIC,0,0,53,1,1,1,1 expand_xy 10,0 blur 1 normalize 0,100 --plasma 0.4 add \
      blur 1 elevation3d -0.1 moded3d 4§\aftergroup\Ccn§ 
 
  - Generate a 3d ring of torii: 
     §\aftergroup\Ccb§gmic repeat 20 torus3d 15,2 color3d[-1] "{u(60,255)},{u(60,255)},{u(60,255)}" \
      *3d[-1] 0.5,1 if "{$>%2}" rotate3d[-1] 0,1,0,90 endif add3d[-1] 70 add3d \
      rotate3d 0,0,1,18 done moded3d 3 mode3d 5 double3d 0§\aftergroup\Ccn§ 
 
  - Create a vase from a 3d isosurface: 
     §\aftergroup\Ccb§gmic moded3d 4 isosurface3d "'x^2+2*abs(y/2)*sin(2*y)^2+z^2-3',0" sphere3d 1.5 \
      sub3d[-1] 0,5 plane3d 15,15 rotate3d[-1] 1,0,0,90 center3d[-1] add3d[-1] 0,3.2 \
      color3d[-1] 180,150,255 color3d[-2] 128,255,0 color3d[-3] 255,128,0 add3d§\aftergroup\Ccn§ 
 
  - Display filtered webcam stream: 
     §\aftergroup\Ccb§gmic apply_camera \"--mirror x --mirror y add div 4\"§\aftergroup\Ccn§ 
 
  - Launch a set of §\aftergroup\Ccg§G'MIC§\aftergroup\Ccn§ interactive demos: 
     §\aftergroup\Ccb§gmic demo§\aftergroup\Ccn§ 

\end{lstlisting}
\normalsize
 
\printindex 
~\\$\square$~End of document. 

\end{document}